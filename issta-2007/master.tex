%% -*- TeX-master: t -*-

% \documentclass{acm_proc_article-sp}
\documentclass{sig-alternate}

% remove this before creating camera-ready final copy!
\setlength{\overfullrule}{5pt}

\newcommand{\thetitle}[0]{Statistical Debugging Using Compound Boolean Predicates}
\newcommand{\wraptitle}[0]{Statistical Debugging Using \\ Compound Boolean Predicates}

\pagenumbering{arabic}

%%%%%%%%%%%%%%%%%%%%%%%%%%%%%%%%%%%%%%%%%%%%%%%%%%%%%%%%%%%%%%%%%%%%%%%%
%%
%% standard texmf packages
%%

% font selection
\usepackage[T1]{fontenc}
\usepackage{courier}
\usepackage[scaled]{helvet}
\usepackage{mathptmx}

\usepackage{booktabs}
\usepackage{flushend}
\usepackage{graphicx}
\usepackage{hyphenat}
\usepackage{nicefrac}
\usepackage{paralist}
\usepackage{subfig}
\usepackage[dvips]{thumbpdf}
\usepackage{xspace}

\usepackage[
bookmarks,
breaklinks,
letterpaper,
pdftitle={\thetitle},
pdfauthor={Piramanayagam Arumuga Nainar, Ting Chen, Jake Rosin, and Ben Liblit},
% pdfsubject={D.2.4 [Software Engineering]: Software/Program Verification -- statistical methods; D.2.5 [Software Engineering]: Testing and Debugging -- debugging aids, distributed debugging, monitors, tracing; I.5.2 [Pattern Recognition]: Design Methodology -- feature evaluation and selection},
% pdfkeywords={bug isolation, random sampling, invariants, feature selection, statistical debugging},
]{hyperref}

\captionsetup[table]{position=top}


%%%%%%%%%%%%%%%%%%%%%%%%%%%%%%%%%%%%%%%%%%%%%%%%%%%%%%%%%%%%%%%%%%%%%%%%
%%
%% unique to this report
%%

\newcommand{\figurename}[0]{Figure}
\newcommand{\tablename}[0]{Table}

\newdef{defn}{Definition}
\newcommand{\defnautorefname}[0]{Definition}

\newcommand{\Importance}[0]{\mathit{Importance}\xspace}
\newcommand{\Increase}[0]{\mathit{Increase}\xspace}
\newcommand{\NumF}[0]{\mathit{NumF}\xspace}
\newcommand{\effort}[0]{\mathit{effort}\xspace}

\newcommand{\obs}[2]{#1(\text{$#2$ obs})}
\newcommand{\obsFalse}[2]{#1(\text{$\bar{#2}$})}

\newcommand{\true}[0]{\mathit{true}\xspace}
\newcommand{\false}[0]{\mathit{false}\xspace}
\newcommand{\unknown}[0]{\mathit{unknown}\xspace}

\newcommand{\ub}[1]{\uparrow#1}
\newcommand{\lb}[1]{\downarrow#1}

\newcommand{\prog}[1]{\texttt{#1}}
\newcommand{\func}[1]{\texttt{#1}}

\renewcommand{\sectionautorefname}[0]{Section}
\renewcommand{\subsectionautorefname}[0]{\sectionautorefname}
\renewcommand{\subsubsectionautorefname}[0]{\sectionautorefname}
\newcommand{\subtableautorefname}[0]{\tableautorefname}

\newcommand{\Max}[2]{\textbf{max(}#1, #2\textbf{)}}
\newcommand{\Min}[2]{\textbf{min(}#1, #2\textbf{)}}

%%%%%%%%%%%%%%%%%%%%%%%%%%%%%%%%%%%%%%%%%%%%%%%%%%%%%%%%%%%%%%%%%%%%%%%%
%%
%% front matter
%%

\title{%
  \wraptitle%
  \thanks{This research was supported in part by AFOSR Grant
    FA9550-07-1-0210 and NSF Grant CCF-0621487.}}

\newcommand{\mailto}[1]{\href{mailto:#1@cs.wisc.edu}{#1}}

\numberofauthors{1}
\author{\alignauthor
  Piramanayagam Arumuga Nainar \qquad Ting Chen \qquad Jake Rosin \qquad Ben Liblit \\
  \affaddr{Computer Sciences Department} \\
  \affaddr{University of Wisconsin--Madison} \\
  \email{\{\mailto{arumuga},\mailto{tchen},\mailto{rosin},\mailto{liblit}\}@cs.wisc.edu}}

%%%%%%%%%%%%%%%%%%%%%%%%%%%%%%%%%%%%%%%%%%%%%%%%%%%%%%%%%%%%%%%%%%%%%%%%
%%
%%  document body
%%

\begin{document}

\conferenceinfo{ISSTA'07,} {July 9--12, 2007, London, England, United Kingdom.}
\CopyrightYear{2007}
\crdata{978-1-59593-734-6/07/0007} 

\maketitle

\begin{abstract}
Cooperative Bug Isolation (CBI) is a technique to find bugs in programs that analyzes data collected from program executions.  At each program point, CBI identifies boolean expressions, called as predicates, to be instrumented.  We augment CBI's bug predictive ability by combining these predicates using logical operators (conjunction and disjunction).  The motivation is that a complex predicate will provide more information to the programmer by narrowing down possible program states.  Our experimental results show that the scores of complex predicates are usually higher or atleast comparable to single predicates.  This makes them good indicators of bugs.  We discuss a new metric that uses program structure to quantify the usefulness of complex predicates.  Using this metric, we could eliminate a large number of spurious predicates from consideration.  Finally we discuss the effect of sparse random sampling on the usefulness of complex predicates.

\end{abstract}

\category{D.2.4}{Software Engineering}{Software/Program Verification}[statistical methods]
\category{D.2.5}{Software Engineering}{Testing and Debugging}[debugging aids, distributed debugging, monitors, tracing]
\category{I.5.2}{Pattern Recognition}{Design Methodology}[feature evaluation and selection]

\terms{Experimentation, Reliability}

\keywords{statistical bug isolation,
          three-valued logic,
          debugging effort metrics,
          dynamic feedback analysis
          }

% -*- TeX-master: "master" -*-

\section{Introduction}

Statistical debugging improves software quality by identifying program (mis)behaviors that are highly predictive of subsequent program failure.  As embodied in the Cooperative Bug Isolation Project (\emph{CBI}) \cite{Liblit:2004:CBI}, these techniques find bugs in programs by analyzing reports collected from software executing in the hands of end users.  First the CBI instrumenting compiler injects extra code that evaluates simple Boolean expressions (called \emph{predicates}) at various program points.  Predicates are designed to capture potentially interesting program behaviors such as results of function calls, directions of branches, or values of variables.  Upon termination of an instrumented program, a feedback report is generated that records how often each predicate was observed, and how often each was both observed and found to be true.  Given many such feedback reports, e.g., from a large user community, statistical debugging is used to find predicates that are predictive of failure \cite{Liblit:2005:SSBI,Zheng:2006:SDSIMB}.  These predicates are then ranked and presented to the developer.

CBI gathers execution reports by using valuable CPU cycles at end user machines.  It is essential to make those cycles worthwhile by extracting every bit of useful information from them.  However, current statistical analysis algorithms consider predicates in isolation from one another \cite{Liblit:2005:SSBI,Zheng:2006:SDSIMB}.  They overlook potentially useful relations between predicates.  Predicates are expressions involving program variables at different program points and hence may be related by control and data dependences.  We propose to capture these relations by building \emph{complex predicates} from the set of currently instrumented predicates (which we refer to as \emph{simple predicates}).  Since predicates are Boolean expressions, they are combined using logical operators (such as conjunction and disjunction).  We construct complex predicates and include them in the input to the statistical analysis algorithms.

There are two approaches to combine predicates using logical operators:
\begin{enumerate}
\item Change the instrumenting compiler to explicitly monitor each complex predicate at run time.
\item Estimate the value of each complex predicate from the values of its components.
\end{enumerate}

The first approach will yield a precise value but needs significant modifications to existing infrastructure.  The second approach will be less precise (as described later) but requires only few modifications to existing infrastructure (and none to the instrumenting compiler).  In the present work, we implement the second approach, which serves as a proof of concept for using complex predicates, as well as a justification for a future attempt at incorporating them into the compiler.

The remainder of this paper is organized as follows.  \autoref{sec-background} reviews statistical debugging and motivates the present effort to find complex failure predictors.  \autoref{sec-complex-preds} gives a precise definition of complex predicates and discusses the trade-offs in our implementation.  \autoref{sec-metrics} defines two metrics to evaluate the usefulness of a complex predicate.  \autoref{sec-qual} discusses two case studies that demonstrate the usefulness of complex predicates.

\autoref{sec-experiments} presents the results of experiments conducted on a large suite of buggy test programs, including an assessment of the effect of sparse random sampling on complex predicates.  Sparse random sampling is a technique used by CBI to reduce the runtime overhead of instrumentation.  \autoref{sec-related-work} discusses related work and \autoref{sec-conclusion} concludes.

% LocalWords:  mis preds qual

% -*- TeX-master: "master" -*-

\section{Background}
\label{sec-background}
CBI uses lightweight instrumentation to collect feedback reports that contain truth values of predicates in an execution as well as the outcome (e.g., crash or non-crash) of the execution.  A large number of these reports are collected and analyzed using statistical debugging techniques.  These techniques identify \emph{bug predictors}: predicates that, when true, herald failure due to a specific bug.  Bug predictors indicate areas of the code related to program failure and so provide information useful when correcting program faults.  Feedback reports can be collected from deployed software in the hands of end users, who may encounter bugs not identified in program testing.  CBI can therefore be used to monitor software after its release and help direct program patches by identifying bugs as they manifest in the field.

\subsection{Finding Bug Predictors}
\label{sec-elimalg}
The feedback report for a particular program execution is formed as a bit-vector, with two bits for each predicate (observed and true), and one final bit representing success or failure.  If generated in experimental or testing conditions these feedback reports are likely to be complete; when instrumented code is distributed to end users predicates are usually sampled infrequently to reduce computational overhead.  Previous experiments \cite{Liblit:2003:BIRPS} have determined that sampling rates of \nicefrac{1}{100} to \nicefrac{1}{1,000} are most realistic for deployed use.

Using these reports, CBI assigns a score to all available predicates and identifies the single best predictor among them (this step is described in detail in \autoref{sec-scoring}).  It is assumed that this predictor corresponds to one important bug, though other bugs may remain.    This top predictor is recorded, and then all feedback reports where it was true are removed from consideration on the assumption that fixing the corresponding bug will change the behavior of runs in which the predictor originally appeared.  The next best predictor among the remaining reports is then identified, recorded, and removed in the same manner.  This iterative process repeats until either no undiagnosed failed runs remain, or no more failure-predictive predicates can be found.

This process of iterative elimination maps each predictor to a set of program runs.  Ideally each such set corresponds to the expression of a distinct bug; unfortunately this is not always the case.  Due to the statistical nature of the analysis, along with incomplete feedback reports resulting from sparse sampling rates, a single bug could be predicted by several top-ranked predicates, and predictors for less prevalent bugs may not be found at all.

The output of the analysis is a list of predicates that had the highest score during each iteration of the elimination algorithm, as well as a complete list of predicates and their scores before any elimination is performed.  These lists may be used by a programmer to identify areas of the program related to faulty behavior.  Liblit et al.\ employed this method to discover previously unknown bugs in several widely-used applications \cite{Liblit:2003:BIRPS,Liblit:2005:SSBI}.

CBI output can alternately be used as input to an automated analysis tool, such as \textsc{BTrace} \cite{Lal:2006:POPAD}. \textsc{BTrace} finds the shortest control- and dataflow-feasible path in the program that visits a given set of bug predictors.  This analysis allows a programmer to examine the fault-predicting behavior even if the connection to a bug is not easily identifiable, or if the predictors are numerous or complex enough to overwhelm a programmer examining them directly.

\subsection{Scoring Predicates}
\label{sec-scoring}
This section provides a brief overview of the numeric scores used to identify the best predictor from a set of predicates.  For a detailed discussion on this topic, the reader should refer to Liblit et al.\ \cite{Liblit:2005:SSBI}.  A good predictor should be both \emph{sensitive}: should account for many failed runs and \emph{specific}: should not mis-predict failure in successful runs.  Assigning scores based on sensitivity will result in \emph{super-bug predictors}, which include failures from more than one bug.  Super-bug predictors are highly non-deterministic, since they are not specific to any single cause of failure, and rarely provide useful debugging information.  Scoring predicates based on specificity instead results in \emph{sub-bug predictors}.  A sub-bug predictor accounts for a portion of the failures caused by a bug, but not all.  Unlike super-bug predictors, sub-bug predictors that account for a significant sub-set of failures can be useful in debugging, although perfect predictors are of course preferred.  Sensitivity and specificity are balanced using a numeric $\Importance$ score computed as follows.

The truth values of a predicate $p$ from all the runs can be aggregated into four values:

\begin{enumerate}
\item $\obs{S}{p}$ and $\obs{F}{p}$, respectively the number of successful and failed runs in which the value of $p$ was evaluated.
\item $S(p)$ and $F(p)$, respectively the number of successful and failed runs in which the value of $p$ was evaluated and was found to be true.
\end{enumerate}

Using these values, two scores of bug relevance are calculated:
\begin{description}
\item[Sensitivity:] $\log{(F(p))} / \log{(\NumF)}$ where $\NumF$ is the total number of failing runs.  A good predictor must predict a large number of failing runs.
\item[Specificity:] $\Increase(p)$.  The amount by which $p$ being true increases the probability of failure over simply reaching the line where $p$ is defined.  It is computed as follows:

  \begin{equation}
    \label{eqn1}
    \Increase(p) \equiv
    \frac{F(p)}{S(p) + F(p)}
    -
    \frac{\obs{F}{p}}{\obs{S}{p} + \obs{F}{p}}
  \end{equation}

\end{description}

Taking the harmonic mean combines these two scores, selecting out predicates that are both highly sensitive and highly specific:
\begin{equation}
\label{eqn2}
\Importance(p) \equiv
\frac{2}{%
  \frac{1}{\Increase(p)}
  +
  \frac{1}{log(F(p)) / log(\NumF)}}
\end{equation}

\subsection{Expected Benefits of Complex Predicates}
A single predicate can be thought of as partitioning the space of all runs into two subspaces: those satisfying the predicate and those not.  The more closely these partitions match the subspaces where the bug is and is not expressed, the better the predicate is as a bug predictor.  If a bug has a cause which corresponds well to a simple predicate then a simple analysis is sufficient, but analysis of more complex bugs will produce only super- and sub-bug predictors.

A richer family of predicates can describe more complex shapes within the space of runs.  This allows good predictors for bugs with more complicated causes.  Some bugs may have causes connected to simple predicates, but that no single predicate can accurately predict.  Complex predicates formed from these simpler ones would be more accurate predictors than any component predicate.  \emph{Partial predictors} are predicates that predict some aspect of a bug that is necessary, but not sufficient, for program failure.  Partial predictors and sub-bug predictors are two classes of simple predicates which can be combined into more accurate predictors.

A partial predictor will correctly partition all (or most) expressions of the bug, but would also predict the bug in a large number of runs where it did not occur.  Because partial predictors are highly non-deterministic with respect to the bug, they are likely to be outscored by a sub- or super-bug predictor.  Partial predictors can be improved by eliminating false positives - this can be accomplished by taking a conjunction with a predicate that captures another aspect of the bug.  The case study presented in \autoref{sec-exif} describes a bug best predicted by a conjunction involving a partial predictor.

Sub-bug predictors correctly partition some expressions of a bug, but not all.  They are useful in identifying a bug because, though they do not predict the bug in a general sense, they are extremely good predictors of some special case where the bug is expressed.  Combining two such predictors with a disjunction will reduce false negatives and result in a predicate that correctly partitions more manifestations of the bug; combine enough special cases in this manner and the resulting predicate will predict the bug in the general case.  It is important to note that the analysis may find a disjunction of predictors of individual bugs as a predictor for the whole set of failures.  This it is not as problematic as it seems: for such a disjunction to be high-ranked each component predicate must be a good predictor for a specific bug, providing useful information on all bugs involved.

The bug predictors that result from combining simple predicates can be conjoined or disjoined again, eliminating false positives and false negatives to approach a perfect predictor.  This process can continue, eventually finding a good predictor for any bug that can be expressed in terms of the simple predicates measured during the construction of the feedback reports.  Even if some aspect of the bug is uncovered by the simple predicates, it's likely that a sub-bug predictor may still be constructed.  The introduction of complex predicates to CBI analysis greatly increases the number of shapes that can be described within the set of runs, thereby increasing the chances of finding an accurate predictor for a bug.

% LocalWords:  BTrace mis

\section{Complex Predicates}
\label{sec-complex-preds}
This section gives a precise definition of complex predicates and discusses the trade-offs in our implementation.

A complex predicate $C$ is defined as $C = \phi(p_1, p_2, \ldots p_k)$ where $p_1, p_2, \ldots p_k$ are simple predicates and $\phi$ is a function that can be computed using only the logical operators $and$ and $or$.  The operator $not$ is not required because any propositional formula can be written in conjunctive normal form (CNF) in which the $not$ operator appears only before the literals.  And by design, the negation of every simple predicate $P$ is also a predicate.

For a predicate $P$ and a run $R$, $R(P)$ is true if $P$ was observed to be true atleast once during run $R$.  Similarly we could define $R(C)$ as follows:
\begin{defn}
\label{dfn1}
For a complex predicate $C = \phi(p_1, p_2, \ldots p_k)$, $R(C)$ = 1 iff at some point during the execution of the program, $C$ was observed to be true.
\end{defn}

The difficulty with this notion of the complex predicate is that $C$ must be explicitly monitored during the program execution.  For example, if $C_1 = p_1 \wedge p_2$ then $R(p_1)$ = 1 and $R(p_2)$ = 1 does not imply that $R(C)$ = 1.  $p_1$ and $p_2$ may be true at different stages of execution but never true at the same time.  However, as discussed earlier, explicitly monitoring $C$ requires significant changes to existing infrastructure.  In order to be able to estimate the value of $C$ from its components, we adapt a less precise definition as follows:
\begin{defn}
\label{dfn2}
For a complex predicate $C = \phi(p_1, p_2, \ldots p_k)$, $R(C)$ = 1 iff $\phi(R(p_1), R(p_2), \ldots R(p_k))$ = $true$
\end{defn}

In other words, we consider that $R$ as distributive over $\phi$.  This can lead to false positives, because $R(C)$ may be computed to 1 when it is actually 0, but no false negatives.  The impact of this assumption on the score of $C$ may be either positive or negative depending on whether $R$ failed or succeeded.

There are $2^{2^N}$ boolean functions of $N$ predicates ~\cite{MathWorld:BoolFuncs}.  This is a prohibitively large number considering that there can be hundreds of simple predicates.  To reduce the complexity, we consider only functions of two predicates.  There are $2^{2^2}$ = 16 such functions and their arguments can be chosen from $N$ predicates in $^NC_2 = \frac{N(N-1)}{2}$ ways.  Out of the 16 boolean functions of two variables, we restrict only to conjunction ($and$) and disjunction ($or$) since other functions are more complex and cannot be used effectively by the programmer.  Other functions can be easily included in our implementation once their truth table (discussed later) are derived correctly.  Thus, we evaluate only $2. ^NC_2 = N(N-1)$ functions for building complex predicates.  If $R$ is the set of runs being analysed then the time complexity required to build complex predicates is $|R|N(N-1) = O(|R|N^2)$

Even after imposing many constraints, the number of complex predicates is quadratic in the number of simple predicates.  A large number of complex predicates formed by this procedure are likely to be useless in the analysis of the program.  A complex predicate that has a lower score than its components is useless.  The component (simple) predicate with a higher score is a better predictor of failure, and so the complex predicate adds nothing to the analysis.

\begin{defn}
\label{dfn3}
A complex predicate $C = \phi(p_1, p_2, \ldots p_k)$ is ``interesting'' if $Importance(C) > Importance(p_i)$ for $i \in \{1, 2, \ldots k\}$
\end{defn}

In the case where the complex predicate has the same score as the component predictor with higher score, the simpler one is preferable.  Definition ~\ref{dfn3} is for the general case and as explained earlier, we explore only for $k = 2$ and $\phi \in \{\vee, \wedge\}$.  Keeping only interesting combinations of predicates reduces the memory burden of storing them, and helps ensure the utility of a complex predicate that passes inspection.

\subsection{Pruning}
Forming a complex predicate from two components is a nontrivial task, requiring a conjunction or disjunction for each program run.  After this computation is complete the score of the newly formed predicate can be calculated, potentially labeling it uninteresting.  In such a case the effort to form the predicate has been wasted.  This provides the motivation to prune combinations early based on an estimate of their resulting scores.  An upper bound for a predicate's score can be determined by maximizing $F(P)$ and $Increase$ under constraints based on the propositional operation.  The score of the predicate, being a harmonic mean of these two terms, will likewise be maximized.

--figure of Increase--

--subsection: pruning disjunctions--

--figure of the terms for a disjunction--

Disjunction in this context can be considered the union of the set where $P1$ and that where $P2$.  The size of the resulting set is maximized when the two do not overlap, and that is the assumption made when calculating an upper bound on $F(P)$.  $S(P)$ is minimized by making the opposite assumption - that one is a subset of the other.  The size of the union is thus the size of the superset.

The second term in $Increase$ can only reduce the result, and so it is ignored in calculating an upper bound.

--subsection: pruning conjunctions--

--figure of the terms for a conjunction--

Conjunction can be regarded as set intersection.  Maximizing $F(P)$ requires that one of the intersecting sets is a subset of the other, making the size of the intersection the size of the smaller operand.  A minimal $S(P)$ is found when the component sets are nonoverlapping; the union of two such sets is empty.

If the second term is ignored, as with disjunctions, the upper bound of a conjoined predicate's $Increase$ score is 1.  This is the maximum $Increase$ possible, reducing the likelihood of a conjoined predicate being pruned to almost zero.  The second term is therefore used for conjunctions to drop the upper bound to a more useful level.

$F(P Observed)$ is minimized by assuming that $P$ was observed only in the (failed) runs where it was true.  Maximizing $S(P Observed)$ is more difficult.  Recall that a conjunction can be considered \textit{Observed} if either component is \textit{Observed False}, or both are \textit{Observed True}.  The intersection of successful runs where a predicate is \textit{Observed False} is maximized if the sets are nonoverlapping, while the intersection where they are \textit{Observed True} is maximized if one is a subset of the other.  By assuming both cases meet that criteria $S(P Observed)$ can be maximized.  Its value is the sum of successful runs where $P1 Observed$ and those where $P2 Observed False$, formed as the difference between $S(P2 Observed)$ and $S(P2)$.  Since the result may differ depending on which predicate is chosen as $P1$ the larger result is used.


% -*- TeX-master: "master" -*-

\section{Usability Metrics}
\label{sec-metrics}
In our experiments, we often observe hundreds of complex predicates with similar or even identical high scores.  The redundancy elimination algorithm will select the top predicate randomly from all those tied for the top score; a human programmer finding a predicate to use in debugging is likely to make a similar choice.  $\Importance$ measures predictive power, so all high-scoring predicates should be good bug predictors.  However, even a perfect predictor may be difficult for a programmer to use when finding and fixing a bug.

Debugging using a simple predicate is aided by understanding the connection between the predicate and the bug it predicts.  For a complex predicate, the programmer must also understand the connection between its components.  Given a set of complex predicates with similar high scores, those that can be easily understood by a human are preferable.  In this section we propose two metrics for selecting understandable predicates from a large set of high-scoring predictors.  Only predicates selected by the metrics are presented to the user.  (If predicate data is to analyzed by an automated tool it may not be advantageous to employ these metrics.)  Both metrics use criteria unrelated to a predicate's $\Importance$ score, making them orthogonal to pruning as discussed in \autoref{sec-pruning}.

\subsection{Effort}

The first metric models the debugging effort required from the programmer to find a direct connection between component predicates.  We adapt the distance metric of Cleve and Zeller \cite{1062522} for this purpose.  In this metric, the score of a predicate is the fraction of code that can be ignored while searching for the bug.  We use a similar metric called \emph{effort} for a complex predicate.

\begin{defn}
\label{def-effort}
The effort required by a programmer while using a complex predicate $\phi(p_1, p_2)$ is inversely proportional to the smaller fraction of code ignored in a breadth-first bidirectional search for $p_1$ from $p_2$ and vice-versa.
\end{defn}

The idea behind this metric is that the larger the distance between the two predicates, the greater the effort required to understand their relationship.  Also, if a large number of other branches are seen during the search, the programmer should keep track of these dependencies too.  Per Cleve and Zeller \cite{1062522}, we use the program dependence graph (PDG) to model the program rather than the source code.  We perform a breadth-first search starting from $p_1$ until $p_2$ is reached and count the total number of vertices visited during the search. The fraction of code covered is the ratio of the number of visited vertices to the total number of PDG vertices. 

\subsection{Correlation}

The second metric is to consider the correlation between the two predicates.  Two predicates may be easily reached from one another without having an apparent connection - a complex predicate formed from them would provide little help to the programmer.  On the other hand, two predicates which are both affected by some shared area of code may have a connection which a programmer can easily discern.  The correlation between two predicates is defined based on the program dependency graph.  Given a single predicate $p$, we define its \textit{predecessor set} as the set of vertices in the PDG that can influence $p$.

\begin{defn}
\label{dfn5}
The correlation between two predicates of a complex predicate is defined as the number of vertices in the intersection of the two predecessor sets.
\end{defn}

The idea behind this metric is that a larger intersection between the $predecessor\ sets$ means it is possible that they are closely related.  We expect correlation to mitigate the issue of disjunctive predicates raised in \autoref{sec-background}, namely that the disjunction of predictors for two separate bugs will be scored very highly.  The predictors of two unrelated program faults are likely to reside in different areas of the program, and therefore the intersection of their predecessor sets would be smaller than two related predictors for the same bug, which are likely to be in closer proximity.

\subsection{Proactive and Retroactive Pruning}

%Changed ``higher'' to ``better'' in the second-to-last sentence below, since what we actually want for metric 1 is a *lower* result.
The above two metrics can be applied both \emph{proactively} and \emph{reactively}.  Proactive use of the metrics removes complex predicates whose metric values fall below a certain threshold of usefulness.  This eliminates them from being computed and hence improves performance.  Reactive use of the metrics retains all the predicates but breaks ties by giving higher ranks to those with better values for the metrics. This is desirable if neither computing time nor space is a concern.

% -*- TeX-master: "master" -*-

\section{Case Studies}
\label{sec-qual}
This section discusses two cases where complex predicates prove to be useful.  The first study is about a memory access bug in \prog{exif} 0.6.9, an open source image manipulation program.  A complex predicate is useful in increasing the score of an extremely useful bug predictor.  The second study uses an input validation bug in \prog{ccrypt} 1.2 to explain how complex predicates can be used to identify partial predictors automatically.

\subsection{\large\textbf{\prog{exif}}}
\label{sec-exif}

\prog{exif} 0.6.9 crashes while manipulating a thumbnail in a Canon image.  The bug is in function \texttt{exif\_mnote\_data\_canon\_load} in the module handling Canon images.  The following is a snippet from said function:
\begin{quote}
\small
\begin{verbatim}
for (i = 0; i < c; i++) {
    ...
    n->count = i + 1;
    ...
    if (o + s > buf_size) return;    // (a)
    ...
    n->entries[i].data = malloc(s);  // (b)
    ...
}
\end{verbatim}
\end{quote}

If the condition \texttt{o + s > buf\_size} is true on line (a), then the allocation of memory to the pointer \texttt{n->entries[i].data} on line (b) is skipped.  The program crashes when other code reads from \texttt{n->entries[i].data} without checking if the pointer is valid.  This is an example of a non-deterministic bug as the program succeeds as long as the uninitialized pointer is not accessed somewhere else.

We generated 1,000 runs of the program using randomly generated command line arguments and input images randomly selected from a set of Canon and non-Canon images.  There are 934 successful executions and 66 crashes.  Applying the redundancy elimination algorithm with only simple predicates produces two predicates that account for all failed runs as shown in \autoref{tab:tbl1}.  Studying the source code of the program does not show any obvious relation between the two predictors and the cause of failure.  Even though the second predictor is present in the crashing function it is a comparison between two unrelated variables: the loop iterator \texttt{i} and the size of the data stored in the traversed array \texttt{s}.  Also it is $\true$ in only 31 of the 66 failures.

\begin{table*}
\caption{Results for \prog{exif} with only simple predicates}
\label{tab:tbl1}
\centering
\scriptsize
\begin{tabular}{lllll}
\toprule
Score & Predicate & Function & File:Line \\
\midrule
0.704974 & $\text{new value of len} == \text{old value of len}$ & \func{jpeg\_data\_load\_data} & jpeg-data.c:224 \\
0.395001 & $\text{i} == \text{s}$ & \func{exif\_mnote\_data\_canon\_save} & exif-mnote-data-canon.c:176 \\
\bottomrule
\end{tabular}
\end{table*}

The analysis assigns a very low score of 0.0191528 to the predicate $p_1$: \texttt{o + s > buf\_size} despite the fact that it captures the exact source of the uninitialized pointer.  Because the bug is non-deterministic, $p_1$ is also $\true$ in 335 runs that succeeded, making $p_1$ a partial predictor.  Including complex predicates in the analysis produces one complex predicate shown in \autoref{tab:tbl2}.  (The second row is the second component of a complex predicate, which is a conjunction as indicated by the keyword \emph{and} at the start.)  Conjunction of $p_1$ with the second predicate $p_2$: \texttt{offset < len} eliminates all false positives and thereby earns a very high score.  This is an example of how a conjunction can improve the score of a partial predictor.  $p_2$ is in function \texttt{exif\_data\_load\_data} that calls \texttt{exif\_mnote\_data\_canon\_load} indirectly.  It is possible that $p_2$ is another partial predictor, capturing another condition that drives the bug to cause a crash.  If it does, it has to be a deep relationship as we could not find such a relation even after spending a couple of hours trying to understand the source code.  However this does not reduce the importance of this result as the conjunction has a very high score compared to $p_1$ and $p_2$ individually.

\begin{table*}
\caption{Results for \prog{exif} with complex predicates}
\label{tab:tbl2}
\centering
\scriptsize
\begin{tabular}{lllll}
\toprule
Score & Predicate & Function & File:Line \\
\midrule
0.941385 & $\text{o} + \text{s} > \text{buf\_size}$ is TRUE & \func{exif\_mnote\_data\_canon\_load} & exif-mnote-data-canon.c:237 \\
         & \emph{and} $\text{offset} < \text{len}$ & \func{exif\_data\_load\_data} & exif-data.c:644 \\
\bottomrule
\end{tabular}
\end{table*}

At the point where the uninitialized pointer is actually used, a hypothetical predicate $p_3$: \texttt{n->entries[i].data == 0} ought to be a perfect bug predictor.  However, the CBI instrumenting compiler does not actually instrument this condition or any direct equivalent.  Furthermore, this assumes that \texttt{n->entries[i].data} is zero-initialized even when \texttt{exif\_mnote\_data\_canon\_load} returns early without filling in this field.  Predicate $p_1$ provides critical additional information, as it identifies the initial trigger (skipping the \texttt{malloc}) that sets the stage for eventual failure (use of an uninitialized pointer).  Thus one role for complex predicates is to capture those program behaviors, like $p_1$, that are necessary but not sufficient preconditions for failure.

% Threat taxonomy discussion:
% Placing the threats listed below in the taxonomy is difficult because the case studies are more anecdotal than experimental.  IV and DVs can be identified, but there are many possible DV definitions, which potentially shift threats from one tier to another.
% The IV is obvious - whether analysis included complex predicates.
% DV is harder to identify.  Possibilities include: (1) Whether p_1 was top-ranked (or tied for top rank).  (2) The score of the best predicate involving p_1.  (3) The score of the top-ranked predicate.  (4) How useful the top-ranked predicate is in identifying the bug (hard to quantify).
% All of the above are at some point relevant in the above discussion (and there are many more possibilities).  Depending on which is regarded as the 'true' DV the below are threats to a different tier of validity.  What we can do is determine which tiers they definitely don't belong to.
% Conclusion validity: Obviously the analysis gave different results when complex preds were considered.  No threats to conclusion validity.
% Internal validity: threats would have to interfere with the conclusion that the differing results were caused by complex pred. analysis and not some other factor.  Simple and complex pred analyses were both run on the exact same data set.  Assuming there weren't any result-altering bugs in our code there are no threats to internal validity, since the IV was literally the only thing that differed between cases.  Altering the test suite to produce the desired bug doesn't affect internal validity because both the simple analysis and the complex pred analysis got the exact same fixed input.
% Construct validity: this one has some merit.  Input data was fixed to produce the particular bug being investigated.  We didn't try different sets of input to make sure p_1 can be identified under different input conditions.  It's possible that the fixed input altered the results we would have seen.  (Construct validity is ~'can the results you saw be generalized to an identical construct,' e.g. the same analysis run on the same program)
% External validity: This is probably the most likely.  The experimental input was fixed to emphasize the predicate we wanted high-ranked.  End users won't do this.  Additionally, searched the predicate list with a particular pred in mind.  To perform the same analysis on a similar but not identical program we would need to preidentify another predicate to examine (as opposed to one being immediately noticeable in the analysis).
% I'm of the opinion that most of the threats below are threats to external validity, but I'm not sure.  A compelling argument can be made for construct validity.  I'm splitting the difference and just calling them 'threats.'

\paragraph{Threats to validity}
There are a few threats to the validity of the above experiment.  Firstly, \prog{exif} 0.6.9 had two other bugs and we had to manually remove command line arguments that trigger those bugs.  Secondly, the bug studied here is very rare.  In order to get sufficient failed executions, we downscaled the input images by selecting many Canon images (that cause the bug) and some other images (both Canon and non-Canon) that do not trigger the bug.  These two changes introduce some bias into the scores of some predicates.  For example, a high score is assigned to the predicate \func{remove\_thumbnail} \texttt{is true} that corresponds to the \texttt{---remove-thumbnail} command-line flag.  However a subjective evaluation of the predicates in \autoref{tab:tbl1} and \autoref{tab:tbl2} shows that their scores are not affected by any bias introduced by the test suite.  Another threat is that the analysis with complex predicates produces many of other predictors tied with the predicate listed in \autoref{tab:tbl2} for the highest score (0.941385).  Because of this the predicate is not selected by the redundancy elimination algorithm, and we had to scan the list of all predicates to identify it.  This is not a real threat but is an instance of the numerous complex predicates problem discussed in \autoref{sec-metrics}.

\subsection{\large\textbf{\prog{ccrypt}}}
\label{sec-ccrypt}
\prog{ccrypt} 1.2 contains a known bug that can cause a crash on certain user-input - when an \texttt{EOF} is entered at the confirmation prompt when overwriting an existing file.  Entering \texttt{EOF} in other contexts does not cause failure, however, and an examination of the source code quickly reveals why:
\begin{quote}
\small
\begin{verbatim}
/* read a yes/no response from the user */
int prompt(void) {
  ...
  line = xreadline(fin, cmd.name);    // (a)
  return (!strcmp(line, "y") ||
     !strcmp(line, "yes"));
}

char *xreadline(FILE *fin, char *myname) {
  ...
  res = fgets(buf, INITSIZE, fin);
  if (res==NULL) {                    // (b)
    free(buf);
    return NULL;
  }
  ...
  return buf;
}
\end{verbatim}
\end{quote}

Calls to \func{xreadline}, the function used to get user-input, can return \texttt{NULL} under some circumstances.  In most cases the value is checked before being dereferenced; in \func{prompt} however it is used immediately after the call on line (a).  \func{xreadline} returning \texttt{NULL} in \func{prompt} should thus be a perfect predictor of failure, occurring in no successful runs and in every failure related to this bug.  The branch taken on line (b) in \prog{xreadline} is important as well, serving as the moment failure in \prog{prompt} becomes inevitable.  This branch is only taken when the user enters \texttt{EOF} on the command line.  In mapping the cause of failure, a programmer without a clear understanding of the code is likely to spend time tracking the user-entered \texttt{EOF} through \prog{xreadline} to the \texttt{NULL} dereference in \prog{prompt}, requiring either a visual inspection of the source or use of an interactive debugger.  Knowledge of the connection between program events such as these is necessary to make good debugging decisions, e.g., adding a \texttt{NULL} check to \prog{prompt} versus ensuring \prog{xreadline} always returns a valid pointer.  Automated bug analysis should ideally reveal as much of this chain of causation to the programmer as possible.

We generated 1,000 runs of \prog{ccrypt}, again using randomly selected command line arguments.  Input files include images and text archived from the online documentation of a remote desktop display system.  There are 658 successful executions and 342 crashes.  All failing runs crash due to the \texttt{NULL} dereference described above - no other bugs were visible to our test suite.

\begin{table*}
\caption{Results for \prog{ccrypt} with only simple predicates}
\label{tab:tbl3}
\centering
\scriptsize
\begin{tabular}{lllllll}
\toprule
Score & True Successes & False Successes & Predicate & Function & File:Line \\
\midrule
0.431678 & 0 & 342 & $\text{xreadline} == \text{0}$ & \func{prompt} & traverse.c:122 \\
0.385597 & 200 & 342 & $\text{res} == \text{(char *)0}$ & \func{xreadline} & xalloc.c:43 \\
\bottomrule
\end{tabular}
\end{table*}

An initial analysis of only simple predicates finds $p_1:$ \texttt{xreadline == 0} as the top predictor of failure: true in no successes and all 342 failed runs, verifying our assumptions.  The related predicate $p_2$: \texttt{res == (char *)0} scores substantially lower, appearing in all failures but a large number of successes.  $p_2$'s reported score is low enough that without knowledge of the nature of the bug a programmer would be likely to overlook its significance, and because of its relationship to $p_1$ it is removed by the redundancy elimination algorithm.  More importantly, traditional CBI analysis reveals no connection between the two predictors to the programmer, despite the fact that $p_2$, a necessary but not sufficient condition for failure, is subordinate to $p_1$ in predicting a crash.

\begin{table*}
\caption{Results for \prog{ccrypt} with complex predicates}
\label{tab:tbl4}
\centering
\scriptsize
\begin{tabular}{lllllll}
\toprule
Score & True Successes & False Successes & Predicate & Function & File:Line \\
\midrule
0.72814 & 0 & 342 & $\text{xreadline} == \text{0}$ & \func{prompt} & traverse.c:12 \\
	&   &     & \emph{and} $\text{res} == \text{(char *)0}$ & \func{xreadline} & xalloc.c:43 \\
\bottomrule
\end{tabular}
\end{table*}

When complex predicates are included in the analysis, a conjunction of $p_1$ and $p_2$ is among the top predictors.  This provides little help in finding the bug, which is easily identified by traditional CBI analysis, but it does reveal the nature of $p_2$ as a partial predictor.  The conjunction $p_1 \wedge p_2$ is observed in more successful runs than $p_1$ alone, but is true in the same number of successes and failures.  That $p_1$ can be conjoined with $p_2$ without affecting $p_1$'s predictive power demonstrates a connection between the two predicates, in this case suggesting that $p_1 \implies p_2$.

This implication is detectable because the experiment is run using complete data collection.  Results taken using sparse sampling rates would have made this detection impossible, given the likelihood of $p_2$ being unobserved in a run where $p_1$ was true.

This result provides evidence that complex predicate analysis can automatically group related predicates in ways traditional CBI analysis does not, including the discovery partial, sub-bug, and perfect predictor hierarchies and implications.  Grouping related predictors statistically provides insight into program structure and execution features that can be used in debugging.  This example reiterates that complex predicates can collaborate with tools like \textsc{BTrace} that produce an execution trace from a set of predicates.  Cooperative Bug Isolation can therefore utilize techniques that previously required detailed execution information by generating a facsimile from statistical data.

\paragraph{Threats to validity}

The version of \prog{ccrypt} used in this experiment had only one bug visible to our test suite.  The statistically demonstrated relationship between $p_1$ and $p_2$ is discovered in the absence of predictors for other bugs, which may affect the results.  Intuitively an unrelated bug would cause faults in different program runs, allowing the analysis to distinguish between unrelated sets of predictors, but we have not demonstrated this.  Further experimentation is needed to determine if this analysis retains this power in the face of multiple bugs.  Additionally the predictor $p_1 \wedge p_2$, though it scores highly, is not top-ranked, and in fact is discarded by the redundancy elimination algorithm due to the large number of similar predicates.  Knowledge of the code and the component predicates is necessary to distinguish it as important.  This once again demonstrates the need for techniques to effectively filter through large numbers of complex predicates, as discussed in \autoref{sec-metrics}.

% LocalWords:  ccrypt mnote buf tbl lllll len jpeg pred preidentify xreadline
% LocalWords:  lllllll src BTrace

%% -*- LaTeX -*-

\begin{table*}
\centering

\begin{tabular}{|l|r|r|r|r|r|r|r|r|r|}
\hline
Number of: & \multicolumn{2}{c|}{runs}  & sites  & 
\multicolumn{2}{c|}{branch predicates} & \multicolumn{2}{c|}{return 
predicates} & \multicolumn{2}{c|}{scalar predicates}\\ \cline{2-3} \cline{5-6} \cline{7-8} \cline{9-10}
           & successful & failing       &        & original & retained & 
original & retained & original & retained \\
\hline
\hline
ccrypt     &  3605      &  1033         &    570 &      0 &          0 &     
3420 &        6 &        0 &        0 \\
\hline
bc         &  3530      &   860         &  13442 &      0 &          0 &        
0 &        0 &    80652 &      156 \\
\hline
moss       & 28519      &  3352         &  35223 &   4170 &         33 &     
2964 &       11 &   195864 &     3322 \\
\hline
rhythmbox  & 21015      &  1873         & 145242 &   6836 &         14 &    
50574 &       21 &   800370 &      406 \\
\hline
\end{tabular}
\caption{Run, site, predicate, and retention counts for each of the experiments.}
\label{tab:exps}
\end{table*}



In this section we present the results of applying the algorithm
described in \autoref{sec:algorithm} in five case
studies.  \autoref{tab:exps} shows summary statistics for each of the
experiments.  In each study we ran the programs on about 32,000 random
inputs.  The number of instrumentation sites varies with the size of
the program, as does the number of predicates those instrumentation
sites yield.  Our algorithm is very effective in reducing the number
of predicates the user must examine.  For example, in the case of
\rhythmbox an initial set of 857,384 predicates is reduced to 537 by the $\increase(P) > 0$
test, a reduction of 99.9\%.  The elimination algorithm then yields 15 predicates, a further
reduction of 97\%.  The other case studies show a similar reduction in the number of
predicates by 3-4 orders of magnitude.

The results we discuss are all on sampled data.  Sampling creates
additional challenges that must be faced by our algorithm.  Assume $P_1$ and $P_2$ are
equivalent bug predictors and both are sampled at a rate of
$\nicefrac{1}{100}$ and both are reached once per run.  Then even though
$P_1$ and $P_2$ are equivalent, they will be observed in nearly disjoint
sets of runs and treated as close to independent by the elimination
algorithm.

To address this problem, we set the sampling rates of predicates to be
inversely proportional to their frequency of execution.  Based on a
training set of 1,000 executions, we set the sampling rate of each predicate so
as to obtain an expected 100 samples of each predicate in subsequent program
executions.  On the low end, the sampling rate is clamped to a minimum of $\nicefrac{1}{100}$; if the site is expected to be reached fewer than 100 times the sampling rate is set at 1.0.
Thus, rarely executed code has a
much higher sampling rate than very frequently executed code.  (A
similar strategy has been pursued for similar reasons in related work \cite{chil04}.)  We
have validated this approach by comparing the results for each
experiment with results obtained with no sampling at all (i.e., the
sampling rate of all predicates set to 100\%).  The results are
identical except for the \rhythmbox and \moss experiments, where we
judge the differences to be minor: sometimes a different but logically
equivalent predicate is chosen, the ranking of predictors of different
bugs is slightly different, or one or the other version has a few
extra, weak predictors at the tail end of the list.

\subsection{A Validation Experiment}

To validate our algorithm we first performed an experiment in which we
knew the set of bugs in advance.  We added nine bugs to \moss, a
widely used service for detecting plagiarism in software
\cite{Schleimer:2003:WLA}.  Six of these were previously discovered
and repaired bugs in \moss that we reintroduced.  The other three were
variations on three of the original bugs, to see if our algorithm could
discriminate between pairs of bugs with very similar behavior but
distinct causes.  The nature of the eight crashing bugs varies: four
buffer overruns, a null file pointer dereference in certain cases, a
missing end-of-list check in the traversal of a hash table bucket, a missing
out-of-memory check, and a violation of a subtle invariant that must be maintained between two
parts of a complex data structure.  In addition, some of these bugs
are non-deterministic any may not even crash when they should.

The ninth bug---incorrect handling of comments in some cases---only
causes incorrect output, not a crash.  We include this bug in our
experiment in order to show that bugs other than crashing bugs can 
also be isolated using our techniques, provided there is some 
way, whether by automatic self-checking or human inspection, to recognize
failing runs.  In particular, for our experiment we also ran a correct 
version of \moss{} and compared the output of the two versions. 
This oracle provides a labeling of runs as ``success'' or ``failure,'' 
and the resulting labels are treated identically by our program as
those based on program crashes.

\begin{table*}
\centering

\begin{tabular}{|l|r|r|r|r|r|r|r|r|r|}
\hline
Number of: & \multicolumn{2}{c|}{runs}  & sites  & 
\multicolumn{2}{c|}{branch predicates} & \multicolumn{2}{c|}{return 
predicates} & \multicolumn{2}{c|}{scalar predicates}\\ \cline{2-3} \cline{5-6} \cline{7-8} \cline{9-10}
           & successful & failing       &        & original & retained & 
original & retained & original & retained \\
\hline
\hline
ccrypt     &  3605      &  1033         &    570 &      0 &          0 &     
3420 &        6 &        0 &        0 \\
\hline
bc         &  3530      &   860         &  13442 &      0 &          0 &        
0 &        0 &    80652 &      156 \\
\hline
moss       & 28519      &  3352         &  35223 &   4170 &         33 &     
2964 &       11 &   195864 &     3322 \\
\hline
rhythmbox  & 21015      &  1873         & 145242 &   6836 &         14 &    
50574 &       21 &   800370 &      406 \\
\hline
\end{tabular}
\caption{Run, site, predicate, and retention counts for each of the experiments.}
\label{tab:exps}
\end{table*}



\autoref{tab:mossdilute} shows the results of the experiment.  The
predicates listed were selected by the elimination algorithm in the
order shown.  The first column is the initial bug thermometer for each
predicate, showing the \context{} and \increase{} scores before
elimination is performed. The fourth column is the \termdef{effective}
bug thermometer, showing the \context{} and \increase{} scores for a
predicate $P$ at the time $P$ is selected (i.e., when it is the
top-ranked predicate).  Thus the effective thermometer reflects the
cumulative diluting effect of redundancy elimination for all
predicates selected before this one.

As part of the experiment we separately recorded the exact set of
bugs that actually occurred in each run.
The columns at the far right of \autoref{tab:mossdilute} show, for
each selected predicate and for each bug, the number of failing runs in which
both the selected predicate is observed to be true and the bug occurs.
Note that while each
predicate has a very strong spike at one bug, indicating it is a
strong predictor of that bug, there are always some runs with other
bugs present.  For example, the top-ranked predicate, which is
overwhelmingly a predictor of bug \#5, also includes some runs where
bugs \#3, \#4, and \#9 occurred.  This situation is not the result of
misclassification of failing runs by our algorithm.  As observed in
\autoref{sec:introduction}, more than one bug may occur in a run.
It simply happens that in some runs bugs \#5 and \#3 both occur (to
pick just one possible combination).

A particularly interesting case of this phenomenon is bug \#7, one of
the buffer overruns.  Bug \#7 is not strongly predicted by any
predicate on the list but in fact occurs in at least a few of the
failing runs of most predicates.  We have examined the runs of bug \#7
in detail and found that the only failing runs involving bug \#7 also
trigger at least one other bug.  That is, even when the bug \#7 overrun
happens, it never causes incorrect output or a crash
in any run.  Bug \#8, another overrun, is not even shown because the
overrun is never triggered in our data (its column would be all
0's).\footnote{Bug \#8 was originally found by a code inspection.}
There is no way our algorithm can find causes of bugs that do not
occur, but recall that part of our purpose in sampling user executions
is to get an accurate picture of the most important bugs.  It is
consistent with this goal that if a bug never causes a problem, it is
not only not worth fixing, it is not even worth reporting.

The other bugs all have strong predictors on the list.  In fact,
the top eight predicates have exactly one predictor for each of the seven
bugs that occur, with the exception of bug \#1, which has one very
strong sub-bug predictor in the second spot and another predictor
in the sixth position.  Notice that even the rarest bug, bug \#2,
which occurs more than an order of magnitude less frequently than
the most common bug, is identified immediately after the last of
the other bugs.\footnote{The peculiar eighth predicate, \texttt{f < f},
says that after an assignment the new value of \texttt{f} is less than
the old value of \texttt{f}.}  Furthermore, we have verified by hand that
the selected predicates would, in our judgment, lead an engineer to
the cause of the bug. Overall, the elimination algorithm does an excellent
job of listing separate causes of each of the bugs in order of priority,
with very little redundancy.

Below the eighth position there are no new bugs to report and every
predicate is correlated with predicates higher on the list.  Even
without the columns of numbers at the right it is easy to spot the
eighth position as the natural cutoff.  Keep in mind that the length
of the thermometer is on a log scale, hence changes in larger
magnitudes may appear less evident.  Notice that the initial and
effective thermometers for the first eight predicates are essentially
identical.  Only the predicate at position six is noticeably
different, indicating that this predicate is somewhat affected by a
predicate listed earlier (specifically, its companion sub-bug
predictor at position two).  However, all of the predicates below the
eighth line have very different initial and effective thermometers
(either many fewer failing runs, or much more non-deterministic, or
both) showing that these predicates are strongly affected by
higher-ranked predicates.

The visualizations presented thus far have a drawback illustrated by
the \moss\ experiment: It is not easy to identify the predicates to
which a predicate is closely related.  Such a feature would be useful
in confirming whether two selected predicates represent different bugs
or are in fact related to the same bug.  We do have a measure of how
strongly $P$ implies another predicate $P'$: How does removing the
runs where $\report{P} = 1$ affect the importance of $P'$?  The more
closely related $P$ and $P'$ are, the more $P'$'s importance drops
when $P$'s failing runs are removed.  In the interactive version of
our analysis tools, each predicate $P$ in the final, ranked list of
links to an \termdef{affinity list} of all
predicates ranked by how much $P$ causes their ranking score to
decrease.

\subsection{Additional Experiments}

We briefly report here on experiments with additional applications
containing both known and unknown bugs.  Complete analysis results for
all experiments may be browsed interactively at
\url{http://www.cs.berkeley.edu/~liblit/pldi-2005}.

\subsubsection{\ccrypt}

\view{\ccrypt}{ccrypt}

We analyzed \ccrypt 1.2, which has a known input validation bug.  The
results are shown in \autoref{tab:views-ccrypt}.  Our algorithm
reports two predictors, both of which point directly to the single bug.
It is easy to discover that the two predictors are for the same bug;
the first predicate is listed first in the second predicate's affinity
list, indicating the first predicate is a sub-bug predictor associated
with the second predicate.

\subsubsection{\bc}

\view{\bc}{bc}

GNU \bc 1.06 has a previously reported buffer overrun.  Our results
are shown in \autoref{tab:views-bc}.  The outcome is the same as for
\ccrypt: two predicates are retained by elimination, and the second
predicate lists the first predicate at the top of its affinity list,
indicating that the first predicate is a sub-bug predictor of the second.
Both predicates point to the cause of the overrun.  This bug causes a
crash long after the overrun occurs and there is no useful information
on the stack at the point of the crash to assist in isolating this
bug.

\subsubsection{\exif}

\view{\exif}{exif}

\autoref{tab:views-exif} shows results for \exif 0.6.9, an open source
image processing program.  Each of the three predicates is a predictor
of a distinct and previously unknown crashing bug.  It took less than
20 minutes of work to find and verify the cause of each of the bugs
using these predicates and the additional highly correlated predicates
on their affinity lists.

To illustrate how statistical debugging is used in practice, we
use the last of these three failure predictors as an example, and
describe how it enabled us to
effectively isolate the cause of one of the bugs.  Failed runs
exhibiting \texttt{o + s > buf\_size} show the following unique stack
trace at the point of termination:
\begin{quote}
  \small
\begin{verbatim}
main
  exif_data_save_data
    exif_data_save_data_content
      exif_data_save_data_content
        exif_data_save_data_entry
          exif_mnote_data_save
            exif_mnote_data_canon_save
              memcpy
\end{verbatim}
\end{quote}
The code in the vicinity of this crash site is as follows:
\begin{quote}
\begin{verbatim}
// snippet of exif_mnote_data_canon_save
for (i = 0; i < n->count; i++) {
    ...
    memcpy(*buf + doff,             (c)
           n->entries[i].data, s);
    ...
}
\end{verbatim}
\end{quote}
This stack trace alone provides little insight into the cause of the
bug.  However, our algorithm highlights \texttt{o + s > buf\_size} in
function \texttt{exif\_mnote\_data\_canon\_load} as a strong bug
predictor.  Thus, a quick inspection of the source code leads us to
construct the following call sequence:
\begin{quote}
  \small
\begin{verbatim}
main
  exif_loader_get_data
    exif_data_load_data
      exif_mnote_data_canon_load
  exif_data_save_data
    exif_data_save_data_content
      exif_data_save_data_content
        exif_data_save_data_entry
          exif_mnote_data_save
            exif_mnote_data_canon_save
              memcpy
\end{verbatim}
\end{quote}
The code in the vicinity of the predicate \texttt{o + s > buf\_size} is as follows:
\begin{quote}
\begin{verbatim}
// snippet of exif_mnote_data_canon_load
for (i = 0; i < c; i++) {
    ...
    n->count = i + 1;
    ...
    if (o + s > buf_size) return;    (a)
    ...
    n->entries[i].data = malloc(s);  (b)
    ...
}
\end{verbatim}
\end{quote}
It is apparent from the above code snippets and the
call sequence that whenever the predicate \texttt{o + s > buf\_size} is true,
%%
\begin{itemize}
\item the function \texttt{exif\_mnote\_data\_canon\_load} returns on
  line \texttt{(a)}, thereby skipping the call to \texttt{malloc} on
  line \texttt{(b)} and thus leaving \texttt{n->entries[i]->data}
  uninitialized for some value of \texttt{i}, and

\item the function \texttt{exif\_mnote\_data\_canon\_save} passes the
  uninitialized \texttt{n->entries[i]->data} to \texttt{memcpy} on line \texttt{(c)}, which reads it and eventually crashes.
\end{itemize}

In summary, our algorithm enabled us to effectively isolate the causes
of several previously unknown bugs in source code unfamiliar to us in
a small amount of time and without any explicit specification beyond
``the program shouldn't crash.''

\subsubsection{\rhythmbox}

\begingroup
\setlength{\segunit}{10pt}
\view[\tiny]{\rhythmbox}{rhythmbox}
\endgroup

\autoref{tab:views-rhythmbox} shows our results for \rhythmbox 0.6.5,
an interactive, graphical, open source music player.  \rhythmbox is a
complex, multi-threaded, event-driven system, written using a library
providing object-oriented primitives in C.  Event-driven systems use
event queues; each event performs some computation and possibly adds
more events to some queues.  We know of no static analysis today that
can analyze event-driven systems accurately, because no static
analysis is currently capable of analyzing the heap-allocated event
queues with sufficient precision.  Stack inspection is also of
limited utility in analyzing event-driven systems, as the stack in the
main event loop is unchanging and all of the interesting state is in
the queues.

We isolated two distinct bugs in \rhythmbox.  The first predicate led
us to the discovery of a race condition.  The second predicate was not
useful directly, but we were able to isolate the bug using the
predicates in its affinity list.  This second bug revealed what turned
out to be a very common incorrect pattern of accessing the underlying
object library (recall \autoref{sec:introduction}).  \rhythmbox
developers confirmed the bugs and enthusiastically applied patches
within a few days, in part because we could quantify the bugs as
important crashing bugs.  It required several hours to isolate each of
the two bugs (and there are additional bugs represented in the
predictors that we did not isolate) in part because \rhythmbox is
complex and in part because the bugs were violations of subtle heap
invariants which are not directly captured by our current
instrumentation schemes.  Note, however, that we could not have even
begun to understand these bugs without the information provided by our
tool.  We intend to explore schemes that track predicates on heap
structure in future work.

\subsection{Comparison with Logistic Regression}
\label{sec:comparison}

\begin{table}
\nocaptionrule
\caption{Results of logistic regression for \moss}
\label{tab:logregression}
\centering
\small
\begin{tabular}{ll}
  \toprule
  Coefficient & Predicate \\
  \midrule
  0.769379 & \verb|(p + passage_index)->last_line < 4| \\
  0.686149 & \verb|(p + passage_index)->first_line < i| \\
  0.675982 & \verb|i > 20| \\
  0.671991 & \verb|i > 26| \\
  0.619479 & \verb|(p + passage_index)->last_line < i| \\
  0.600712 & \verb|i > 23| \\
  0.591044 & \verb|(p + passage_index)->last_line == next| \\
  0.567753 & \verb|i > 22| \\
  0.544829 & \verb|i > 25| \\
  0.536122 & \verb|i > 28| \\
  \bottomrule
\end{tabular}
\end{table}

In earlier work
we used $\ell_1$-regularized logistic regression
to rank the predicates by their
failure-prediction strength \cite{PLDI`03*141,NIPS2003_AP05}.
Logistic regression uses linearly weighted
combinations of predicates to classify a trial run as successful or
failed.  Regularized logistic regression incorporates a penalty
forcing most coefficients to be set to zero, thereby
selecting only the most important predicates.  The output is a set of
coefficients for predicates giving the best overall prediction.

A weakness of logistic regression for our application is that it seeks
to cover the set of failing runs without regard to the orthogonality
of the selected predicates (i.e., whether they represent distinct
bugs).  This problem can be seen in \autoref{tab:logregression},
which gives the top ten predicates selected by logistic regression
for \moss.  The striking fact is that all selected predicates are
either sub-bug or super-bug predictors.  The predicates beginning with
\texttt{p + \ldots} are all sub-bug predictors of bug \#1 (see
\autoref{tab:mossdilute}).  The predicates \texttt{i > \ldots} are
super-bug predictors: \texttt{i} is the length of the command line and
the predicates say program crashes are more likely for long command
lines (recall \autoref{sec:introduction}).

The prevalence of super-bug predictors on the list shows the
difficulty of making use of the penalty term.  Limiting the number of
predicates that can be selected via a penalty has the effect of
encouraging regularized logistic regression to choose super-bug predictors, as
these cover more failing runs at the expense of poorer predictive
power compared to predictors of individual bugs.  On the other hand,
the sub-bug predictors are chosen based on their excellent prediction
power of those small subsets of failed runs.
%%Relaxing the penalty
%%allows logistic regression to add more predicates to improve its
%%prediction, but the sub-bug predictors apparently are favored.

%% LocalWords:  exps mossdilute ccrypt bc exif buf mnote rhythmbox
%% LocalWords:  logregression

% -*- TeX-master: "master" -*-

\section{Related Work}
\label{sec-related-work}
Daikon \cite{ErnstCGN2001:TSE} detects invariants in a program by observing multiple program runs.  Invariants are predicates generated using operators like sum, max, etc.\ to combine program variables and collection (e.g., array) objects.  Daikon is intended for many uses beyond bug isolation, and so it monitors a much larger set of predicates than CBI\@.  This makes scalable complex predicate generation more difficult.  However, Dodoo et al.\ \cite{ErnstDRAFT} have successfully extended the work to generate implications from the simpler, measured predicates.  Dodoo et al.\ alternate clustering and invariant detection to find invariant implications over a set of program runs.  The initial clustering is performed using the $k$-means algorithm \cite{jain99data}, with program runs represented as normalized vectors of scalar variable values.  Since CBI represents run information as bit-vectors this technique can be applied essentially unchanged.

Daikon's implication generation extends its vocabulary of possible invariants.  CBI's focus is detection of bug predictors, which under sparse sampling conditions can rarely be identified as invariant.  Additionally, the existence of an implication is of questionable value in this project; the implication revealed in \autoref{sec-ccrypt} is an interesting and potentially useful side-effect of our analysis, but only because it involves identified bug predictors.  The approach described in this paper is better suited to the goals and analysis techniques of CBI\@.  There are no known attempts to use Daikon under sparse sampling conditions.

DIDUCE \cite{581377} detects invariant bits of program values during an initial training phase.  During the checking phase, DIDUCE reports each invariant violation as it occurs, then relaxes the invariant to accept the new value.  Unlike Daikon and CBI, DIDUCE tightly couples data collection and evaluation.  Because of this coupling, neither our nor Daikon's offline style of predicate generation is readily combined with DIDUCE's framework.

SOBER \cite{1081753} is a statistical debugging tool similar to CBI\@.  Where CBI considers only whether a predicate was ever observed true during an execution, SOBER estimates the likelihood of it being true at any given evaluation.  SOBER data is a probability vector, with each value representing the estimated chance of a simple predicate being true when observed.  The similarity in collected data means that similar techniques for complex predicate generation are applicable.  The three-valued logic described in \autoref{sec-tvl} could be replaced with joint-probability when generating conjunctions; De Morgan's law can be applied to generate disjunctions.  Our usability metrics can be used on the resulting data.  There are no known experiments using SOBER under sparse sampling conditions.  Complex predicate generation removes a key advantage of SOBER: predicate scores result directly from the number of actual predicate evaluations.  Complex predicates generated by this technique are never truly evaluated, so their probability values would have little connection to actual program execution.  Whether this would affect their usefulness is unknown.

Jones and Harrold \cite{1101949} discuss a fault localization technique using statement coverage as predicates and weighted failure rate as the scoring metric.  The ideas discussed in our paper, including the pruning techniques, can be applied directly to this technique.  Jones et al.\ \cite{DBLP:journals/ivs/JonesOH04} also explore visualization of program-execution data, such as failure data.  Compound predicates relate behavior at multiple program points, and therefore may be difficult to visualize.  Presenting compound predicates in a way that programmers can readily understand remains an open problem.

Haran et al.\ \cite{haran05TCEDS} analyze data from deployed software to classify executions as \emph{success} or \emph{failure}.  They use tree based classifiers and association rules to model ``failure signals.''  Tree based classifiers can encode both conjunctions and disjunctions whereas association rules cannot encode disjunctions when limited to a constant size.



% LocalWords:  Dodoo Daikon's ccrypt DIDUCE DIDUCE's tvl

% -*- TeX-master: "master" -*-
%       *         *         *         *         *         *         *         *

\section{Conclusions and Future Work}
\label{sec-conclusion}
We have demonstrated that compound Boolean predicates are useful
predictors of bugs.  Our experiments show qualitative and quantitative
evidence that statistical debugging techniques can be effectively
applied to complex predicates, and that the resulting analysis
provides improved results.  We describe two methods of eliminating
predicate combinations from consideration, making the task of
computing complex predicates more feasible.  The first employs
three-valued logic to estimate set sizes and thereby estimate the
upper bound of the score of a complex predicate.  The second uses
distances in program dependence graphs to quantify the programmer
effort involved in understanding complex predicates.

These techniques help the statistical debugging analysis scale up to
handle the large number of candidate predicates we consider.  However,
using the analysis results in debugging can require sifting through a
large number of less useful predicates that also pass automated
inspection.  Further shrinking this list while retaining useful
predictors remains an important open problem.  Most identified bug
predictors redundantly describe the same small set of program
failures.  Thus the bi-clustering algorithm of Zheng et al.\
\cite{Zheng:2006:SDSIMB} may be promising as it was designed to handle
multiple predictors for the same bug.  Automated analyses which
further process predictor lists, such as \textsc{BTrace}
\cite{Lal:2006:POPAD}, may also benefit from the richer diagnostic
language offered by the work presented here.

% LocalWords:  BTrace

% -*- TeX-master: "report" -*-

\section*{Acknowledgments}

We would like to thank Anne Mulhern for her insightful comments on an earlier draft of this paper.


\bibliographystyle{abbrv}
\bibliography{local}

\end{document}
