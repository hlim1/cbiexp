
\begin{table*}
\centering

\begin{tabular}{|l|r|r|r|r|r|r|r|r|r|}
\hline
Number of: & \multicolumn{2}{c|}{runs}  & sites  & 
\multicolumn{2}{c|}{branch predicates} & \multicolumn{2}{c|}{return 
predicates} & \multicolumn{2}{c|}{scalar predicates}\\ \cline{2-3} \cline{5-6} \cline{7-8} \cline{9-10}
           & successful & failing       &        & original & retained & 
original & retained & original & retained \\
\hline
\hline
ccrypt     &  3605      &  1033         &    570 &      0 &          0 &     
3420 &        6 &        0 &        0 \\
\hline
bc         &  3530      &   860         &  13442 &      0 &          0 &        
0 &        0 &    80652 &      156 \\
\hline
moss       & 28519      &  3352         &  35223 &   4170 &         33 &     
2964 &       11 &   195864 &     3322 \\
\hline
rhythmbox  & 21015      &  1873         & 145242 &   6836 &         14 &    
50574 &       21 &   800370 &      406 \\
\hline
\end{tabular}
\caption{Run, site, predicate, and retention counts for each of the experiments.}
\label{tab:exps}
\end{table*}



In this section we compare our new algorithm with our previous
approaches on the programs {\tt ccrypt} and {\tt bc}.
\Autoref{tab:exps} shows raw counts for all four experiments of the
successful and unsuccessful runs, number of instrumentation sites in
the program, the number of predicates for each
instrumentation scheme, and the number of predicates retained by the
test $\increase() > 0$.   Keep in mind that multiple predicates are
derived from each instrumentation site.  For example, each branches
site yields two predicates from two counters, while each scalar pairs
site yields six predicates from three counters (see \Autoref{sec:background}).

The sampling rate for all of our experiments is \nicefrac{1}{1};
i.e., we sampled every predicate every time it was reached.  An
offline downsampling program allows us generate sparser samples from this
full data; thus, we were able to use the same set of runs to test the
effect of different sampling rates on the results.  

We analyzed {\tt ccrypt} version 1.2, which has a known bug \cite{Selinger:2003:cqual}.  At one point in the code the program attempts to write a file.  If the file already exists, the user is presented with a prompt asking for confirmation that the file should be written; if the user input is EOF (i.e., the user hits {\tt Enter} without typing anything) the program will crash.  We used only the returns
instrumentation scheme for this experiment. The top
two entries of the report are (most information is elided for
brevity):
\begin{verbatim}
Predicate            Context    Increase   Failure
xreadline() == 0     .67        .33        1.00
file_exists() > 0    .35        .32         .67
\end{verbatim}

These are the same two predicates identified using a a simple process-of-elimination strategy in our previous work
\cite{PLDI`03*141}.  Our new algorithm, however, provides enough information to also see the causal chain that
leads to the crash: if the file exists, the probability of failure increases to .67, and if in addition the
read returns EOF, the probability of failure increases to 1.00.\footnote{Note that the context and increase scores do not always sum to the failure score due to roundoff error.}   A property of our algorithm is that 
whenever a causal sequence of instrumented predicates must be true for a bug to occur, each predicate along the
sequence shows a distinct step up in the probability of failure.

Note also that the report is very short; only 6 predicates out of more than 3,000 are retained by the test $\increase() > 0$.
Uniformly throughout our experiments the returns and branches reports are short enough that all predicates can
easily be examined by hand; we discuss the larger scalar-pairs reports below.

Our second experiment isolates a buffer overrun in version 1.06 of GNU's {\tt bc}.  In our
previous work, which used an algorithm based on logistic regression, all of the top
predictors showed that the variable {\tt indx} was bigger than some other value on the first line of the following
code fragment:
\begin{verbatim}
for(; indx < v_count; indx++)
      arrays[index] = NULL;
\end{verbatim}
The problem is indeed that {\tt indx} gets too big, resulting in an overrun of {\tt arrays}.  The bug
is that {\tt v\_count} is the wrong loop bound; it should be {\tt a\_count} instead.  

Using our new algorithm, the top 11 predicates (sorted by descending
$\increase()$ score) involve predicates within a few lines of the line
with the actual overrun.  However, these predicates also predict only
20-100 failing runs, a small fraction of the number of failing runs
for {\tt bc} in \Autoref{tab:exps}.  We have observed that this
phenomenon is fairly common: if there are enough failures caused by a
bug, some predicates will predict failure for small subsets of the failure set.

We have found it useful, especially when examining the larger
scalar-pairs reports, to be able to distinguish whether a
highly-ranked predicate largely captures an entire bug or just an aspect of a bug.  To
this end, we compute correlation coefficient matrices, showing the correlation coefficient between each pair of retained predicates; 
%% each retained predicate with every other retained predicate; 
intuitively, the
correlations show how often the predicates co-occur in the same failing run.  High correlations
indicate that the two predicates are very likely caused by the same bug.

In this experiment, a moment's work shows that the top-ranked predicates are highly
correlated with other predicates that are observed in many more of the
failing runs.\footnote{Our system hyperlinks each predicate to a
rank-ordered list of correlated predicates; it is literally the work
of a moment to discover this fact.}  Changing the rank order to list
the predicates with a large number of failing runs first, the first
group of predicates mostly involve {\tt indx} being larger than
something else on the first line of the code fragment given above.
Also in this cluster is the predicate {\tt v\_count > a\_count} (which is sampled at an assignment
to {\tt a\_count} a few lines before the loop), as
expected.

In summary, our new algorithm performed at least as well as either of
our previous proposals on both a deterministic and a non-deterministic
bug.     Furthermore, additional information, computed by our new
approach but not by our old ones, is also useful in understanding the
sources of bugs.







