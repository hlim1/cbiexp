
This section describes the results of applying the algorithm discussed
in \Autoref{sec:algorithm} to the experiment described in
\Autoref{sec:experiments:setup}.  We briefly summarize the results.  
The algorithm identified a cause that would be useful to a programmer
examining these results for 7 of the 9 bugs.  For the 2 other bugs,
one of the bugs (bug \#7) occurred in our experiment but never caused
the program to fail; the other missing bug (bug \#8) was never
triggered at all.  For 5 of the 7 bugs that led to program failures,
the causes of the bugs were ranked very high in the initial results.  The
remaining 2 bugs had causes that were ranked very near the top once
we removed the other 5 bugs we ``found'' with our tool from the
experiment.  A surprise was that we found a bug \#10; the results
showed us a previously undiscovered bug in \moss.

We performed 31,996 random runs of \moss.  Of these, 123 runs were
discarded because they produced no report (see
\Autoref{sec:experiments:setup}), giving us 31,873 runs to work with.

For this experiment we enabled all three instrumentation strategies
described in \Autoref{sec:background}:

\begin{description}
\item[branches:] 2,085 instrumentation sites.  Each site has two
  counters and yields two predicates, for 4,170 branch predicates over
  all.
  
\item[returns:] 494 instrumentation sites.  Each site has three
  counters and yields six predicates, for 2,964 return predicates over
  all.
  
\item[scalar-pairs:] 32,644 instrumentation sites.  Each site has
  three counters and yields six predicates, for 195,864 scalar-pair
  predicates over all.
\end{description}

Thus, the input to our analysis is 31,873 feedback reports, each of which records the
status of over 200,000 predicates.\footnote{The reader may wonder
whether it is practical to actually generate a report on 200,000
predicates on a client machine and then upload it to a central server
for analysis.  The answer is definitely yes.  These reports are mostly
zeroes and so compress extremely well, resulting in uploaded files in
the range of 10-50K.}  The sampling rate for all runs in this
experiment was \nicefrac{1}{1}; i.e., we sampled every predicate every time it was
reached.  We used a separate downsampling program to generate sparser
samples from this full data; thus, we were able to use the same set of
runs to test the effect of different sampling rates on the results.

The time to run our algorithm on all the runs is about seven minutes on
a fast machine.
Step one of our algorithm eliminates almost all of the predicates.  At
\nicefrac{1}{100} sampling, the numbers of predicates with a positive
$\increase(\ldots)$ score are

\begin{itemize}
\item 51 branch predicates;

\item 16 return predicates;

\item 8,672 scalar-pair predicates.
\end{itemize}
About 99\% of all predicates in the the branch and returns
instrumentation strategies are eliminated by step (1) of our
algorithm.  About 96\% of the predicates in the scalar-pairs strategy are
eliminated, but clearly this list is too long to process by hand,
even for the highly-ranked predicates.  We noticed that the cause of the
redundancy is that many closely related predicates are reported for the
same line number.  Often there is a logically strongest predicate that can be
selected; for example, if the system observes that $\tt x \leq y$ and $\tt x < y$
are both correlated with failure, the predicate $\tt x \leq y$ is clearly redundant.  
For our experiment, we simply reported only one predicate per line,
and then browsed the entire list of other predicates on that line if it 
appeared interesting.  Retaining one predicate per line gave us a list
of 186 lines to examine (i.e., only 186 distinct lines in the program had
predicates that were correlated with failure).

\subsection{The Bugs}

Because the branches instrumentation and the returns instrumentation
yield high quality predicates, we began by examining those two lists.  We expect this is
what a programmer or QA engineer examining these results would do as well.

\begin{figure}
\centering
\begin{small}
  \begin{BVerbatim}[gobble=4]
    0.94 Inc, 1.00 Cr, 0.06 Co, process_file_pass2, line 5523
    0.80 Inc, 1.00 Cr, 0.20 Co, handle_options, line 5742
    0.80 Inc, 1.00 Cr, 0.20 Co, string2lang, line 4366
    0.78 Inc, 1.00 Cr, 0.22 Co, handle_options, line 5789
  \end{BVerbatim}
\end{small}

\caption{The top four entries from the branches report.}
\label{fig-report}
\end{figure}

The branches report has 4 predicates with a 1.0 crash rating, shown in
Figure~\ref{fig-report}.  Each line of the report lists the increase,
crash, and context scores of a predicate, as well as the function name
and line number where it occurs.  We have dropped some fields of the
report (such as the text of the predicate itself and the number of
failing and successful runs in which the predicate is observed) to
avoid cluttering the figure.

The first predicate listed does not immediately suggest what the cause
of failure might be. By looking at the failure log, we see that this predicate
is highly correlated with a subset of the runs in which bug \#6 is
triggered; an engineer with a deep understanding of how \moss\ works
might be able to figure out the bug from this predicate alone.
\placeholder{Need to confirm with Mayur that in fact this predicate is bug \#6.}
The next three predicates are obvious ``hits''. The second predicate
says that when comment matching is turned on, the program is
guaranteed to fail; this is bug \#1, and the predicate points directly
to the fact that something is wrong in the comment matching code.
Recall that bug \#1 is non-deterministic; why then is the crash score 1.0?
The discrepancy is the result of sampling error.  In the full data set
(with \nicefrac{1}{1} sampling) the crash measure for this predicate is 0.88.  The
next predicate marks the test for whether \moss\ is analyzing Lisp
programs; the crash score of 1.0 tells us that whenever a Lisp program is
processed, \moss\ crashes.  The last predicate in
Figure~\ref{fig-report} is a much more obvious cause of 
bug \#6; it reveals that whenever the amount of memory to use is set
via a command line option, the program will crash.\footnote{The full data set shows that bug \#6 is non-deterministic with respect to this predicate, so once again the certainty that the program will crash is the result of sampling error.}  All three of these predicates illustrate the potential of our method to
help pinpoint the root cause of a crash rather than just where the crash
happened.  Each of these predicates immediately suggests
what test case to try to reproduce the bug. 

The 9th listed cause in the branches report says that whenever the
user supplies the command line option to write a database, the chance
of failure jumps to 62\% (bug \#2), and the fourteenth ranked cause
says that whenever a database is read the chance of failure is 55\%
(bug \#3).  All the other causes between the 5th and 21st in the branches report are predicates that also implicate one of bugs 1, 2, 3, 5, or 6.  Below
the 22nd listed cause the $\increase(\ldots)$ score for the predicates is
only 1\%; we did not examine these predicates.

The returns report is also quite interesting.  Recall that this
instrumentation scheme samples the return values of functions (whether
they are negative, positive, or zero).  The first two entries in this
report are the results of string compares that point directly to bug
\#5.  The third entry is the {\tt open} call in the function that writes
a database; the predicate shows that when this function returns {\tt NULL},
the program is guaranteed to crash (i.e., 1.0 crash score).  This is the
cause of bug \#2.

The fourth entry of the full data (not the \nicefrac{1}{100} downsampled data)
returns report is very interesting.  This predicate shows that one of
the file read operations in the function that loads a database can
fail, and when it does \moss\ itself fails with probability 0.64.
It turns out that \moss\ was written assuming that the databases it
loaded would never be corrupt, because it does no checking to ensure
that the file reads that load databases succeed.  This is a previously
unknown bug in \moss.  It is a bit surprising that it was detected,
because a run is only labeled a failure in our experiment if the
reference \moss\ and the buggy \moss\ differ in their outcome, and
this bug is present in both.  Thus, this bug could only be detected
when another one of the introduced bugs was also triggered in the same
run and caused the buggy version of \moss\ to fail in a different way.
This explains the very low $\increase(\ldots)$ score for this
predicate (the chance of failure only increases by 11\% when this bug
occurs) and why it was not observed in the \nicefrac{1}{100} downsampled data.\footnote{With enough runs it would be observed at any sampling rate, but the anomaly that
the reference and buggy versions of \moss\ share this bug means that likely
many more runs are needed for \nicefrac{1}{100} sampling than we did for this experiment.}

Bugs \#4 and \#9 have no causes in either the branches or the returns
reports, as they are not correlated with any branch nor correlated
with the result of any function call.  In the scalar-pairs report they
are ranked 26th (for bug \#4) and 52nd (for bug \#9).  In both cases
all of the predicates reported on the line reported were obvious
indicators for the bug, so it did not matter in this case which
predicate was selected as the representative for the report.  The
predicates ranked higher than these two are either more obscure,
but more deterministic, indicators for bug \#4 or for one of the other five observed bugs.
Given the emphasis on determinism in our ranking function, bug \#9 could not be listed
higher, as it is the most non-deterministic bug in our data set with a
$\crash(\ldots)$ score of 0.72.

Finally, we compared the reports generated from \nicefrac{1}{100} sampled data
with the reports generated from \nicefrac{1}{1} sampled data.  The reports were
similar, but not identical.  In particular the ordering of the
predicates was slightly different and a few predicates with relatively
few observations or a low $\increase(\ldots)$ score, and thus very
sensitive to whether one or two successful or failing runs were
observed or not, appeared on one list and not the other.

\placeholder{ We need to comment on the ``big picture'' with our results,
and observe the relative weakness of scalar pairs compared to the other two.
We should also revisit the two bugs that did not cause failure and remind the
reader that part of our goal is to triage bugs, and bugs that don't occur
are triaged out of existence.}

\subsection{Analysis of Predicate Elimination}
\placeholder{Mayur writes this}

\subsection{Effect of Sampling on Convergence}
\placeholder{Alice writes this}

%% LocalWords:  downsampling lang downsampled
