
This section describes the results of applying the algorithm discussed
in \Autoref{sec:algorithm} to the experiment described in
\Autoref{sec:experiments:setup}.  We briefly summarize the results.  
The algorithm identified a cause that would be useful to a programmer
examining these results for 7 of the 9 bugs.  For the 2 other bugs,
one of the bugs (bug \#7) occurred in our experiment but never caused
the program to fail; the other missing bug (bug \#8) was never
triggered at all.  For 5 of the 7 bugs that led to program failures,
the causes of the bugs were ranked very high in the initial results.  The
remaining 2 bugs had causes that were ranked very near the top once
we removed the other 5 bugs we ``found'' with our tool from the
experiment.  A surprise was that we found a bug \#10; the results
showed us a previously undiscovered bug in \moss.

We performed 31,996 random runs of
\moss\footnote{Which, not coincidentally, turns out to be the maximum
number of files and subdirectories permitted in a Linux directory.
The results of each run is saved in its own subdirectory.}  Of these,
123 runs were discarded because they produced no report (see \Autoref{sec:experiments:setup}),
giving us 31,873 runs to work with.  

For this experiment we used three different instrumentation strategies, which our
instrumentor automatically combined in a single executable:
\begin{itemize}
\item {\em branches}: For each if statement {\tt if (P) \ldots} we sample whether {\tt P} is true or false.
Two predicates are tracked for each {\tt if} ($\tt P$ and $\tt \neg P$).  There were 2,085 such conditional branches
in \moss, yielding 4,170 predicates.

\item {\em returns}: For each procedure we sample whether its return value $\tt r$ is positive, negative, or zero.
These three counters give rise to six predicates in the analysis: $\tt r = 0$, $\tt \neq 0$, $\tt r < 0$, $\tt r \leq 0$, $\tt r > 0$, and $\tt r \geq 0$.
There were 494 return sites in \moss, yielding 2,964 predicates.

\item {\em scalar-pairs}: At each assignment {\tt x = \ldots} and every other variable with the same type as {\tt x} that is in scope,
we sample whether $\tt x < y$, $\tt x = y$, or $\tt x > y$.  These
three counters also give rise to six predicates that are examined in
the analysis: $\tt x = y$, $\tt x \neq y$, $\tt x < y$, $\tt x \leq
y$, $\tt x > y$, and $\tt x \geq y$.  This instrumentation strategy 
also includes assignments to struct fields of the form {\tt foo.x =
\ldots}.  Furthermore, in addition to comparing the left-hand side of
an assignment with same-typed variables, the left-hand side is also compared against
same-typed manifest constants that appear in the program.  There are
32,644 such combinations of assignments and variables or constants
to compare in \moss, yielding 195,864 predicates in the analysis.
\end{itemize}

Thus, the input to our analysis is 31,873, each of which records the
status of over 200,000 predicates.\footnote{The reader may wonder
whether it is practical to actually generate a report on 200,000
predicates on a client machine and then upload it to a central server
for analysis.  The answer is definitely yes.  These reports are mostly
zeroes and so compress extremely well, resulting in uploaded files in
the range of 10-50K.}  The sampling rate for all runs in this
experiment was 1/1; i.e., we sampled every predicate every time it was
reached.  We used a separate downsampling program to generate sparser
samples from this full data; thus, we were able to use the same set of
runs to test the effect of different sampling rates on the results.

The time to run our algorithm on all the runs is about seven minutes on
a fast machine.
Step one of our algorithm eliminates almost all of the predicates.  At
1/100 sampling, the numbers of predicates with a positive
$\increase(\ldots)$ score are

\begin{itemize}
\item 51 branch predicates;

\item 16 returns predicates;

\item 8,672 scalar-pairs predicates.
\end{itemize}
\placeholder{We need to replace the 8,672 number with the number of distinct sites that survived step 1 in the 1/100 downsampled data.  This is the fairest number to report.}
About 99\% of all predicates in the the branch and returns
instrumentation strategies are eliminated by step (1) of our
algorithm, and the lists are short enough that almost no ranking is
needed.  About 96\% of the predicates in the scalar-pairs strategy are
eliminated, but clearly ranking is still important because of the
large number of predicates that remain.

\subsection{The Bugs}

Because the branches instrumentation and the returns instrumentation
have short lists and, in our experience, yield high quality
predicates, we began by examining those two lists.  We expect this is
what a programmer or QA engineer examining these results would do as well.

\begin{figure*}
\begin{centering}
\begin{verbatim}
0.94 Inc, 1.00 Cr, 0.06 Co, process_file_pass2, line 5523
0.80 Inc, 1.00 Cr, 0.20 Co, handle_options, line 5742
0.80 Inc, 1.00 Cr, 0.20 Co, string2lang, line 4366
0.78 Inc, 1.00 Cr, 0.22 Co, handle_options, line 5789
\end{verbatim}

\caption{The top four entries from the branches report.}
\label{fig-report}
\end{figure*}

The branches report has 4 predicates with a 1.0 crash rating, shown
in Figure~\ref{fig-report}.  Each line of the report lists the
increase, crash, and context scores of a predicate, as well as the 
function name and line number where it occurs.  We have dropped some fields of the report (such as the text of the predicate itself and the number of failing and successful runs in which the predicate is observed) to avoid cluttering the
figure.

The first predicate listed does not immediately suggest what the cause
of failure might be. By looking at the failure log, we see that this predicate
is highly correlated with a subset of the runs in which bug \#6 is
triggered; an engineer with a deep understanding of how \moss\ works
might be able to figure out the bug from this predicate alone.
\placeholder{Need to confirm with Mayur that in fact this predicate is bug \#6.}
The next three predicates are obvious ``hits''. The second predicate
says that when comment matching is turned on, the program is
guaranteed to fail; this is bug \#1, and the predicate points directly
to the fact that something is wrong in the comment matching code.
Recall that bug \#1 is non-deterministic; why then is the crash score 1.0?
The discrepancy is the result of sampling error.  In the full data set
(with 1/1 sampling) the crash measure for this predicate is .88.  The
next predicate marks the test for whether \moss\ is analyzing Lisp
programs; the crash score of 1.0 tells us that whenever a Lisp program is
processed, \moss\ crashes.  The last predicate in
Figure~\ref{fig-report} is a much more obvious cause of 
bug \#6; it reveals that whenever the amount of memory to use is set
via a command line option, the program will crash.\footnote{The full data set shows that bug \#6 is non-deterministic with respect to this predicate, so once again the certainty that the program will crash is the result of sampling error.}  All three of these predicates illustrate the potential of our method to
help pinpoint the root cause of a crash rather than just where the crash
happened.  Each of these predicates immediately suggests
what test case to try to reproduce the bug. 

The ninth listed cause in the branches report says that whenever the
user supplies the command line option to write a database, the chance
of failure jumps to 62\% (bug \#2), and the fourteenth ranked cause
says that whenever a database is read the chance of failure is 55\%
(bug \#3).  All the other causes between the 5th and 21st in the branches report are predicates that also implicate one of bugs 1, 2, 3, 5, or 6.  Below
the 22nd listed cause the $\increase(\ldots)$ score for the predicates is
only 1\%; we did not examine these predicates.

The returns report is also quite interesting.  Recall that this
instrumentation scheme samples the return values of functions (whether
they are negative, positive, or zero).  The first two entries in this
report are the results of string compares that point directly to bug
\#5.  The third entry is the {\tt open} call in the function that writes
a database; the predicate shows that when this function returns {\tt NULL},
the program is guaranteed to crash (i.e., 1.0 crash score).  This is the
cause of bug \#2.

The fourth entry of the full data (not the 1/100 downsampled data)
returns report is very interesting.  This predicate shows that one of
the file read operations in the function that loads a database can
fail, and when it does the program itself fails with probability .64.
It turns out that \moss\ was written assuming that the databases it
loaded would never be corrupt, because it does no checking to ensure
that the file reads that load databases succeed.  This is a previously
unknown bug in \moss.  It is a bit surprising that it was detected,
because a run is only labeled a failure in our experiment if the
reference \moss\ and the buggy \moss\ differ in their outcome, and
this bug is present in both.  Thus, this bug could only be detected
when another one of the introduced bugs was also triggered in the same
run and caused the buggy version of \moss\ to fail in a different way.
This explains the very low $\increase(\ldots)$ score for this
predicate (the chance of failure only increases by 11\% when this bug
occurs) and why it was not observed in the 1/100 downsampled data.\footnote{With enough runs it would be observed at any sampling rate, but the anomaly that
the reference and buggy versions of \moss\ share this bug means that likely
many more runs are needed for 1/100 sampling than we did for this experiment.}

Bugs \#4 and \#9 have no causes in either the branches or the returns
reports, as they are not correlated with any branch nor correlated
with the result of any function call.  In the initial scalar-pairs
report the highest-ranked cause of these bugs that we could find was
in position 85 (for bug \#4) and 289 (for bug \#9).
\placeholder{These locations are far too pessimistic, as many of these
predicates are not unique; we should remove the redundant ones and
then see where these predicates are, but if we don't have time it just
us makes us look worse than necessary to report these numbers. - Alex}
Even though these predicates are highly ranked in a list of thousands
of scalar-pairs predicates, it is likely that the cause of bug \#9, at
least, would be overlooked in this report.

The problem is that there are many highly ranked scalar-pairs
predicates that point to the other bugs besides
\#4 and \#9.  To verify this, we simulated what would happen once bugs
\#1, \#2, \#3, \#5, and \#6 were fixed by removing all failing runs where
these bugs occurred from our data and rerunning the algorithm.  Once
we did that, the obvious cause of bug \#4 was 7th on the list in the
scalar-pairs report, and the obvious cause of bug \#9 was listed 28th,
with the other causes between 1 and 28 being other predicates that
were correlated with (and in some cases more deterministic than) bug
\#4.

Finally, we compared the reports generated from 1/100 sampled data
with the reports generated from 1/1 sampled data.  The reports were
similar, but not identical.  In particular the ordering of the
predicates was slightly different and a few predicates with relatively
few observations or a low $\increase(\ldots)$ score, and thus very
sensitive to whether one or two successful or failing runs were
observed or not, appeared on one list and not the other.

\placeholder{ We need to comment on the ``big picture'' with our results,
and observe the relative weakness of scalar pairs compared to the other two.}

\subsection{Analysis of Predicate Elimination}
\placeholder{Mayur writes this}

\subsection{Effect of Sampling on Convergence}
\placeholder{Alice writes this}





