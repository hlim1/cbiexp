%% -*- LaTeX -*-

\begin{table*}
\centering

\begin{tabular}{|l|r|r|r|r|r|r|r|r|r|}
\hline
Number of: & \multicolumn{2}{c|}{runs}  & sites  & 
\multicolumn{2}{c|}{branch predicates} & \multicolumn{2}{c|}{return 
predicates} & \multicolumn{2}{c|}{scalar predicates}\\ \cline{2-3} \cline{5-6} \cline{7-8} \cline{9-10}
           & successful & failing       &        & original & retained & 
original & retained & original & retained \\
\hline
\hline
ccrypt     &  3605      &  1033         &    570 &      0 &          0 &     
3420 &        6 &        0 &        0 \\
\hline
bc         &  3530      &   860         &  13442 &      0 &          0 &        
0 &        0 &    80652 &      156 \\
\hline
moss       & 28519      &  3352         &  35223 &   4170 &         33 &     
2964 &       11 &   195864 &     3322 \\
\hline
rhythmbox  & 21015      &  1873         & 145242 &   6836 &         14 &    
50574 &       21 &   800370 &      406 \\
\hline
\end{tabular}
\caption{Run, site, predicate, and retention counts for each of the experiments.}
\label{tab:exps}
\end{table*}



In this section we present the results of applying the algorithm
described in \autoref{sec:algorithm} in five case
studies.  \autoref{tab:exps} shows summary statistics for each of the
experiments.  In each study we ran the programs on about 32,000 random
inputs.  The number of instrumentation sites varies with the size of
the program, as does the number of predicates those instrumentation
sites yield.  Our algorithm is very effective in reducing the number
of predicates the user must examine.  For example, in the case of
\rhythmbox an initial set of 857,384 predicates is reduced to 537 by the $\increase(P) > 0$
test, a reduction of 99.9\%.  The elimination algorithm then yields 15 predicates, a further
reduction of 97\%.  The other case studies show a similar reduction in the number of
predicates by 3-4 orders of magnitude.

The results we discuss are all on sampled data.  Sampling creates
additional challenges that must be faced by our algorithm.  Assume $P_1$ and $P_2$ are
equivalent bug predictors and both are sampled at a rate of
$\nicefrac{1}{100}$ and both are reached once per run.  Then even though
$P_1$ and $P_2$ are equivalent, they will be observed in nearly disjoint
sets of runs and treated as close to independent by the elimination
algorithm.

To address this problem, we set the sampling rates of predicates to be
inversely proportional to their frequency of execution.  Based on a
training set of 1,000 executions, we set the sampling rate of each predicate so
as to obtain an expected 100 samples of each predicate in subsequent program
executions.  On the low end, the sampling rate is clamped to a minimum of $\nicefrac{1}{100}$; if the site is expected to be reached fewer than 100 times the sampling rate is set at 1.0.
Thus, rarely executed code has a
much higher sampling rate than very frequently executed code.  (A
similar strategy has been pursued for similar reasons in related work \cite{chil04}.)  We
have validated this approach by comparing the results for each
experiment with results obtained with no sampling at all (i.e., the
sampling rate of all predicates set to 100\%).  The results are
identical except for the \rhythmbox and \moss experiments, where we
judge the differences to be minor: sometimes a different but logically
equivalent predicate is chosen, the ranking of predictors of different
bugs is slightly different, or one or the other version has a few
extra, weak predictors at the tail end of the list.

\subsection{A Validation Experiment}

To validate our algorithm we first performed an experiment in which we
knew the set of bugs in advance.  We added nine bugs to \moss, a
widely used service for detecting plagiarism in software
\cite{Schleimer:2003:WLA}.  Six of these were previously discovered
and repaired bugs in \moss that we reintroduced.  The other three were
variations on three of the original bugs, to see if our algorithm could
discriminate between pairs of bugs with very similar behavior but
distinct causes.  The nature of the eight crashing bugs varies: four
buffer overruns, a null file pointer dereference in certain cases, a
missing end-of-list check in the traversal of a hash table bucket, a missing
out-of-memory check, and a violation of a subtle invariant that must be maintained between two
parts of a complex data structure.  In addition, some of these bugs
are non-deterministic and may not even crash when they should.

The ninth bug---incorrect comment handling in some cases---only
causes incorrect output, not a crash.  We include this bug in our
experiment in order to show that bugs other than crashing bugs can 
also be isolated using our techniques, provided there is some 
way, whether by automatic self-checking or human inspection, to recognize
failing runs.  In particular, for our experiment we also ran a correct 
version of \moss{} and compared the output of the two versions. 
This oracle provides a labeling of runs as ``success'' or ``failure,'' 
and the resulting labels are treated identically by our program as
those based on program crashes.

\begin{table*}
\centering

\begin{tabular}{|l|r|r|r|r|r|r|r|r|r|}
\hline
Number of: & \multicolumn{2}{c|}{runs}  & sites  & 
\multicolumn{2}{c|}{branch predicates} & \multicolumn{2}{c|}{return 
predicates} & \multicolumn{2}{c|}{scalar predicates}\\ \cline{2-3} \cline{5-6} \cline{7-8} \cline{9-10}
           & successful & failing       &        & original & retained & 
original & retained & original & retained \\
\hline
\hline
ccrypt     &  3605      &  1033         &    570 &      0 &          0 &     
3420 &        6 &        0 &        0 \\
\hline
bc         &  3530      &   860         &  13442 &      0 &          0 &        
0 &        0 &    80652 &      156 \\
\hline
moss       & 28519      &  3352         &  35223 &   4170 &         33 &     
2964 &       11 &   195864 &     3322 \\
\hline
rhythmbox  & 21015      &  1873         & 145242 &   6836 &         14 &    
50574 &       21 &   800370 &      406 \\
\hline
\end{tabular}
\caption{Run, site, predicate, and retention counts for each of the experiments.}
\label{tab:exps}
\end{table*}



\autoref{tab:mossdilute} shows the results of the experiment.  The
predicates listed were selected by the elimination algorithm in the
order shown.  The first column is the initial bug thermometer for each
predicate, showing the \context{} and \increase{} scores before
elimination is performed. The fourth column is the \termdef{effective}
bug thermometer, showing the \context{} and \increase{} scores for a
predicate $P$ at the time $P$ is selected (i.e., when it is the
top-ranked predicate).  Thus the effective thermometer reflects the
cumulative diluting effect of redundancy elimination for all
predicates selected before this one.

As part of the experiment we separately recorded the exact set of
bugs that actually occurred in each run.
The columns at the far right of \autoref{tab:mossdilute} show, for
each selected predicate and for each bug, the actual number of failing
runs in which both the selected predicate is observed to be true and
the bug occurs.
Note that while each
predicate has a very strong spike at one bug, indicating it is a
strong predictor of that bug, there are always some runs with other
bugs present.  For example, the top-ranked predicate, which is
overwhelmingly a predictor of bug \#5, also includes some runs where
bugs \#3, \#4, and \#9 occurred.  This situation is not the result of
misclassification of failing runs by our algorithm.  As observed in
\autoref{sec:introduction}, more than one bug may occur in a run.
It simply happens that in some runs bugs \#5 and \#3 both occur (to
pick just one possible combination).

A particularly interesting case of this phenomenon is bug \#7, one of
the buffer overruns.  Bug \#7 is not strongly predicted by any
predicate on the list but occurs in at least a few of the
failing runs of most predicates.  We have examined the runs of bug \#7
in detail and found that the failing runs involving bug \#7 also
trigger at least one other bug.  That is, even when the bug \#7 overrun
occurs, it never causes incorrect output or a crash
in any run.  Bug \#8, another overrun, is not even shown because the
overrun is never triggered in our data (its column would be all
0's).\footnote{Bug \#8 was originally found by a code inspection.}
There is no way our algorithm can find causes of bugs that do not
occur, but recall that part of our purpose in sampling user executions
is to get an accurate picture of the most important bugs.  It is
consistent with this goal that if a bug never causes a problem, it is
not only not worth fixing, it is not even worth reporting.

The other bugs all have strong predictors on the list.  In fact,
the top eight predicates have exactly one predictor for each of the seven
bugs that occur, with the exception of bug \#1, which has one very
strong sub-bug predictor in the second spot and another predictor
in the sixth position.  Notice that even the rarest bug, bug \#2,
which occurs more than an order of magnitude less frequently than
the most common bug, is identified immediately after the last of
the other bugs.\footnote{The peculiar eighth predicate, \texttt{f < f},
says that after an assignment the new value of \texttt{f} is less than
the old value of \texttt{f}.}  Furthermore, we have verified by hand that
the selected predicates would, in our judgment, lead an engineer to
the cause of the bug. Overall, the elimination algorithm does an excellent
job of listing separate causes of each of the bugs in order of priority,
with very little redundancy.

Below the eighth position there are no new bugs to report and every
predicate is correlated with predicates higher on the list.  Even
without the columns of numbers at the right it is easy to spot the
eighth position as the natural cutoff.  Keep in mind that the length
of the thermometer is on a log scale, hence changes in larger
magnitudes may appear less evident.  Notice that the initial and
effective thermometers for the first eight predicates are essentially
identical.  Only the predicate at position six is noticeably
different, indicating that this predicate is somewhat affected by a
predicate listed earlier (specifically, its companion sub-bug
predictor at position two).  However, all of the predicates below the
eighth line have very different initial and effective thermometers
(either many fewer failing runs, or much more non-deterministic, or
both) showing that these predicates are strongly affected by
higher-ranked predicates.

The visualizations presented thus far have a drawback illustrated by
the \moss\ experiment: It is not easy to identify the predicates to
which a predicate is closely related.  Such a feature would be useful
in confirming whether two selected predicates represent different bugs
or are in fact related to the same bug.  We do have a measure of how
strongly $P$ implies another predicate $P'$: How does removing the
runs where $\report{P} = 1$ affect the importance of $P'$?  The more
closely related $P$ and $P'$ are, the more $P'$'s importance drops
when $P$'s failing runs are removed.  In the interactive version of
our analysis tools, each predicate $P$ in the final, ranked list of
links to an \termdef{affinity list} of all
predicates ranked by how much $P$ causes their ranking score to
decrease.

\subsection{Additional Experiments}

We briefly report here on experiments with additional applications
containing both known and unknown bugs.  Complete analysis results for
all experiments may be browsed interactively at
\url{http://www.cs.berkeley.edu/~liblit/pldi-2005}.

\subsubsection{\ccrypt}

\begin{view}{\ccrypt}{ccrypt}
  0.836 & \bugometer{3.20003}{0.176135}{0.819603}{0.004262}{0} & 0.836 & \bugometer{3.20003}{0.176135}{0.819603}{0.004262}{0} & \verb|files[filesindex].language > 16| & 0 & 0 & 28 & 54 & 1585 & 0 & 0 & 68 \\
0.820 & \bugometer{2.8893}{0.115811}{0.877037}{0.005862}{0.00129} & 0.838 & \bugometer{2.8893}{0.115811}{0.877037}{0.005862}{0.00129} & \verb|((*(fi + i)))->this.last_line == 1| & 774 & 0 & 17 & 0 & 0 & 0 & 18 & 2 \\
0.803 & \bugometer{2.85612}{0.130935}{0.851155}{0.008161}{0.009749} & 0.842 & \bugometer{2.85612}{0.107337}{0.874878}{0.008036}{0.009749} & \verb|token_index > 500| & 31 & 0 & 16 & 711 & 0 & 0 & 0 & 47 \\
0.748 & \bugometer{2.74819}{0.114434}{0.774373}{0.023693}{0.0875} & 0.797 & \bugometer{2.7332}{0.0820414}{0.802985}{0.0244006}{0.090573} & \verb|(p + passage_index)->last_token <= filesbase| & 28 & 2 & 508 & 0 & 0 & 0 & 1 & 29 \\
0.722 & \bugometer{2.52244}{0.175776}{0.783215}{0.016985}{0.024024} & 0.812 & \bugometer{2.48714}{0.0685383}{0.887332}{0.0180707}{0.026059} & \verb|__result == 0 is TRUE| & 16 & 0 & 0 & 9 & 19 & 291 & 0 & 13 \\
0.801 & \bugometer{2.91698}{0.117265}{0.829939}{0.014055}{0.038741} & 0.803 & \bugometer{2.58092}{0.0648828}{0.823005}{0.0281222}{0.08399} & \verb|config.match_comment is TRUE| & 791 & 2 & 23 & 1 & 0 & 5 & 11 & 41 \\
0.688 & \bugometer{3.00346}{0.114176}{0.622208}{0.026513}{0.237103} & 0.774 & \bugometer{2.9722}{0.0426556}{0.675659}{0.0279544}{0.253731} & \verb|i == yy_last_accepting_state| & 55 & 0 & 21 & 0 & 0 & 3 & 7 & 769 \\
0.674 & \bugometer{2.15836}{0.167018}{0.813616}{0.019366}{0} & 0.799 & \bugometer{2.14613}{0.134873}{0.846851}{0.018276}{0} & \verb|f < f| & 3 & 144 & 2 & 2 & 0 & 0 & 0 & 5 \\
0.802 & \bugometer{2.8014}{0.130883}{0.865253}{0.003864}{0} & 0.672 & \bugometer{1.39794}{0.0144433}{0.984092}{0.0014647}{0} & \verb|files[fileid].size < token_index| & 31 & 0 & 10 & 633 & 0 & 0 & 0 & 40 \\
0.648 & \bugometer{2.67761}{0.11462}{0.616239}{0.038049}{0.231092} & 0.627 & \bugometer{2.19866}{0.0113816}{0.546359}{0.0751704}{0.367089} & \verb|passage_index == 293| & 27 & 3 & 8 & 0 & 0 & 0 & 2 & 366 \\
0.820 & \bugometer{2.89042}{0.116206}{0.876638}{0.005869}{0.001287} & 0.628 & \bugometer{1.20412}{0.00441909}{0.994416}{0.00116491}{0} & \verb|((*(fi + i)))->other.last_line == yyleng| & 776 & 0 & 16 & 0 & 0 & 0 & 18 & 1 \\
0.625 & \bugometer{2.51055}{0.111797}{0.610649}{0.046073}{0.231481} & 0.594 & \bugometer{1.61278}{0.00661276}{0.618031}{0.13145424}{0.243902} & \verb|min_index == 64| & 24 & 1 & 7 & 0 & 0 & 1 & 1 & 249 \\
0.819 & \bugometer{2.88762}{0.117597}{0.87521}{0.005898}{0.001295} & 0.519 & \bugometer{0.90309}{0.00289622}{0.996159}{0.00094478}{0} & \verb|((*(fi + i)))->this.last_line == yy_start| & 771 & 0 & 18 & 0 & 0 & 0 & 19 & 0 \\
0.489 & \bugometer{2.70586}{0.532437}{0.372044}{0.034495}{0.061024} & 0.476 & \bugometer{1.5563}{0.0500894}{0.430855}{0.1579446}{0.361111} & \verb|(passages + i)->fileid == 52| & 24 & 0 & 477 & 14 & 24 & 0 & 1 & 14 \\
0.666 & \bugometer{3.13066}{0.114929}{0.572709}{0.024427}{0.287935} & 0.461 & \bugometer{2.24797}{0.00443167}{0.33547}{0.07252733}{0.587571} & \verb|passage_index == 25| & 60 & 5 & 27 & 0 & 0 & 4 & 10 & 962 \\
0.836 & \bugometer{3.19976}{0.176195}{0.819543}{0.004262}{0} & 0.446 & \bugometer{0.69897}{0.00422009}{0.994974}{0.00080591}{0} & \verb|strcmp > 0| & 0 & 0 & 28 & 54 & 1584 & 0 & 0 & 68 \\
0.743 & \bugometer{2.99034}{0.143387}{0.707495}{0.021306}{0.127812} & 0.401 & \bugometer{2.25768}{0.00387753}{0.274786}{0.06940247}{0.651934} & \verb|i > 500| & 32 & 2 & 18 & 853 & 54 & 0 & 0 & 53 \\
0.812 & \bugometer{3.12613}{0.130952}{0.796744}{0.013216}{0.059088} & 0.347 & \bugometer{1.8451}{0.00152788}{0.243364}{0.11225112}{0.642857} & \verb|token_sequence[token_index].val >= 100| & 1250 & 3 & 28 & 38 & 0 & 15 & 19 & 65 \\
0.640 & \bugometer{2.80754}{0.104214}{0.582085}{0.034885}{0.278816} & 0.232 & \bugometer{0.845098}{0.00052404}{0.204298}{0.36660696}{0.428571} & \verb|i == 50| & 27 & 0 & 11 & 0 & 0 & 1 & 4 & 463 \\
0.668 & \bugometer{3.12743}{0.114601}{0.575334}{0.024457}{0.285608} & 0.197 & \bugometer{1.20412}{0.00036305}{0.137417}{0.23721995}{0.625} & \verb|passage_index == 19| & 59 & 5 & 28 & 0 & 0 & 4 & 10 & 958 \\
0.497 & \bugometer{1.27875}{0.0714847}{0.914362}{0.0141533}{0} & 0.000 & \bugometer{0}{0.00176211}{0.995798}{0.00243989}{0} & \verb|bytes <= filesbase| & 1 & 0 & 19 & 0 & 0 & 0 & 0 & 1 \\

\end{view}

We analyzed \ccrypt 1.2, which has a known input validation bug.  The
results are shown in \autoref{tab:views-ccrypt}.  Our algorithm
reports two predictors, both of which point directly to the single bug.
It is easy to discover that the two predictors are for the same bug;
the first predicate is listed first in the second predicate's affinity
list, indicating the first predicate is a sub-bug predictor associated
with the second predicate.

\subsubsection{\bc}

\begin{view}{\bc}{bc}
  0.836 & \bugometer{3.20003}{0.176135}{0.819603}{0.004262}{0} & 0.836 & \bugometer{3.20003}{0.176135}{0.819603}{0.004262}{0} & \verb|files[filesindex].language > 16| & 0 & 0 & 28 & 54 & 1585 & 0 & 0 & 68 \\
0.820 & \bugometer{2.8893}{0.115811}{0.877037}{0.005862}{0.00129} & 0.838 & \bugometer{2.8893}{0.115811}{0.877037}{0.005862}{0.00129} & \verb|((*(fi + i)))->this.last_line == 1| & 774 & 0 & 17 & 0 & 0 & 0 & 18 & 2 \\
0.803 & \bugometer{2.85612}{0.130935}{0.851155}{0.008161}{0.009749} & 0.842 & \bugometer{2.85612}{0.107337}{0.874878}{0.008036}{0.009749} & \verb|token_index > 500| & 31 & 0 & 16 & 711 & 0 & 0 & 0 & 47 \\
0.748 & \bugometer{2.74819}{0.114434}{0.774373}{0.023693}{0.0875} & 0.797 & \bugometer{2.7332}{0.0820414}{0.802985}{0.0244006}{0.090573} & \verb|(p + passage_index)->last_token <= filesbase| & 28 & 2 & 508 & 0 & 0 & 0 & 1 & 29 \\
0.722 & \bugometer{2.52244}{0.175776}{0.783215}{0.016985}{0.024024} & 0.812 & \bugometer{2.48714}{0.0685383}{0.887332}{0.0180707}{0.026059} & \verb|__result == 0 is TRUE| & 16 & 0 & 0 & 9 & 19 & 291 & 0 & 13 \\
0.801 & \bugometer{2.91698}{0.117265}{0.829939}{0.014055}{0.038741} & 0.803 & \bugometer{2.58092}{0.0648828}{0.823005}{0.0281222}{0.08399} & \verb|config.match_comment is TRUE| & 791 & 2 & 23 & 1 & 0 & 5 & 11 & 41 \\
0.688 & \bugometer{3.00346}{0.114176}{0.622208}{0.026513}{0.237103} & 0.774 & \bugometer{2.9722}{0.0426556}{0.675659}{0.0279544}{0.253731} & \verb|i == yy_last_accepting_state| & 55 & 0 & 21 & 0 & 0 & 3 & 7 & 769 \\
0.674 & \bugometer{2.15836}{0.167018}{0.813616}{0.019366}{0} & 0.799 & \bugometer{2.14613}{0.134873}{0.846851}{0.018276}{0} & \verb|f < f| & 3 & 144 & 2 & 2 & 0 & 0 & 0 & 5 \\
0.802 & \bugometer{2.8014}{0.130883}{0.865253}{0.003864}{0} & 0.672 & \bugometer{1.39794}{0.0144433}{0.984092}{0.0014647}{0} & \verb|files[fileid].size < token_index| & 31 & 0 & 10 & 633 & 0 & 0 & 0 & 40 \\
0.648 & \bugometer{2.67761}{0.11462}{0.616239}{0.038049}{0.231092} & 0.627 & \bugometer{2.19866}{0.0113816}{0.546359}{0.0751704}{0.367089} & \verb|passage_index == 293| & 27 & 3 & 8 & 0 & 0 & 0 & 2 & 366 \\
0.820 & \bugometer{2.89042}{0.116206}{0.876638}{0.005869}{0.001287} & 0.628 & \bugometer{1.20412}{0.00441909}{0.994416}{0.00116491}{0} & \verb|((*(fi + i)))->other.last_line == yyleng| & 776 & 0 & 16 & 0 & 0 & 0 & 18 & 1 \\
0.625 & \bugometer{2.51055}{0.111797}{0.610649}{0.046073}{0.231481} & 0.594 & \bugometer{1.61278}{0.00661276}{0.618031}{0.13145424}{0.243902} & \verb|min_index == 64| & 24 & 1 & 7 & 0 & 0 & 1 & 1 & 249 \\
0.819 & \bugometer{2.88762}{0.117597}{0.87521}{0.005898}{0.001295} & 0.519 & \bugometer{0.90309}{0.00289622}{0.996159}{0.00094478}{0} & \verb|((*(fi + i)))->this.last_line == yy_start| & 771 & 0 & 18 & 0 & 0 & 0 & 19 & 0 \\
0.489 & \bugometer{2.70586}{0.532437}{0.372044}{0.034495}{0.061024} & 0.476 & \bugometer{1.5563}{0.0500894}{0.430855}{0.1579446}{0.361111} & \verb|(passages + i)->fileid == 52| & 24 & 0 & 477 & 14 & 24 & 0 & 1 & 14 \\
0.666 & \bugometer{3.13066}{0.114929}{0.572709}{0.024427}{0.287935} & 0.461 & \bugometer{2.24797}{0.00443167}{0.33547}{0.07252733}{0.587571} & \verb|passage_index == 25| & 60 & 5 & 27 & 0 & 0 & 4 & 10 & 962 \\
0.836 & \bugometer{3.19976}{0.176195}{0.819543}{0.004262}{0} & 0.446 & \bugometer{0.69897}{0.00422009}{0.994974}{0.00080591}{0} & \verb|strcmp > 0| & 0 & 0 & 28 & 54 & 1584 & 0 & 0 & 68 \\
0.743 & \bugometer{2.99034}{0.143387}{0.707495}{0.021306}{0.127812} & 0.401 & \bugometer{2.25768}{0.00387753}{0.274786}{0.06940247}{0.651934} & \verb|i > 500| & 32 & 2 & 18 & 853 & 54 & 0 & 0 & 53 \\
0.812 & \bugometer{3.12613}{0.130952}{0.796744}{0.013216}{0.059088} & 0.347 & \bugometer{1.8451}{0.00152788}{0.243364}{0.11225112}{0.642857} & \verb|token_sequence[token_index].val >= 100| & 1250 & 3 & 28 & 38 & 0 & 15 & 19 & 65 \\
0.640 & \bugometer{2.80754}{0.104214}{0.582085}{0.034885}{0.278816} & 0.232 & \bugometer{0.845098}{0.00052404}{0.204298}{0.36660696}{0.428571} & \verb|i == 50| & 27 & 0 & 11 & 0 & 0 & 1 & 4 & 463 \\
0.668 & \bugometer{3.12743}{0.114601}{0.575334}{0.024457}{0.285608} & 0.197 & \bugometer{1.20412}{0.00036305}{0.137417}{0.23721995}{0.625} & \verb|passage_index == 19| & 59 & 5 & 28 & 0 & 0 & 4 & 10 & 958 \\
0.497 & \bugometer{1.27875}{0.0714847}{0.914362}{0.0141533}{0} & 0.000 & \bugometer{0}{0.00176211}{0.995798}{0.00243989}{0} & \verb|bytes <= filesbase| & 1 & 0 & 19 & 0 & 0 & 0 & 0 & 1 \\

\end{view}

GNU \bc 1.06 has a previously reported buffer overrun.  Our results
are shown in \autoref{tab:views-bc}.  The outcome is the same as for
\ccrypt: two predicates are retained by elimination, and the second
predicate lists the first predicate at the top of its affinity list,
indicating that the first predicate is a sub-bug predictor of the second.
Both predicates point to the cause of the overrun.  This bug causes a
crash long after the overrun occurs and there is no useful information
on the stack at the point of the crash to assist in isolating this
bug.

\subsubsection{\exif}

\begin{view}{\exif}{exif}
  0.836 & \bugometer{3.20003}{0.176135}{0.819603}{0.004262}{0} & 0.836 & \bugometer{3.20003}{0.176135}{0.819603}{0.004262}{0} & \verb|files[filesindex].language > 16| & 0 & 0 & 28 & 54 & 1585 & 0 & 0 & 68 \\
0.820 & \bugometer{2.8893}{0.115811}{0.877037}{0.005862}{0.00129} & 0.838 & \bugometer{2.8893}{0.115811}{0.877037}{0.005862}{0.00129} & \verb|((*(fi + i)))->this.last_line == 1| & 774 & 0 & 17 & 0 & 0 & 0 & 18 & 2 \\
0.803 & \bugometer{2.85612}{0.130935}{0.851155}{0.008161}{0.009749} & 0.842 & \bugometer{2.85612}{0.107337}{0.874878}{0.008036}{0.009749} & \verb|token_index > 500| & 31 & 0 & 16 & 711 & 0 & 0 & 0 & 47 \\
0.748 & \bugometer{2.74819}{0.114434}{0.774373}{0.023693}{0.0875} & 0.797 & \bugometer{2.7332}{0.0820414}{0.802985}{0.0244006}{0.090573} & \verb|(p + passage_index)->last_token <= filesbase| & 28 & 2 & 508 & 0 & 0 & 0 & 1 & 29 \\
0.722 & \bugometer{2.52244}{0.175776}{0.783215}{0.016985}{0.024024} & 0.812 & \bugometer{2.48714}{0.0685383}{0.887332}{0.0180707}{0.026059} & \verb|__result == 0 is TRUE| & 16 & 0 & 0 & 9 & 19 & 291 & 0 & 13 \\
0.801 & \bugometer{2.91698}{0.117265}{0.829939}{0.014055}{0.038741} & 0.803 & \bugometer{2.58092}{0.0648828}{0.823005}{0.0281222}{0.08399} & \verb|config.match_comment is TRUE| & 791 & 2 & 23 & 1 & 0 & 5 & 11 & 41 \\
0.688 & \bugometer{3.00346}{0.114176}{0.622208}{0.026513}{0.237103} & 0.774 & \bugometer{2.9722}{0.0426556}{0.675659}{0.0279544}{0.253731} & \verb|i == yy_last_accepting_state| & 55 & 0 & 21 & 0 & 0 & 3 & 7 & 769 \\
0.674 & \bugometer{2.15836}{0.167018}{0.813616}{0.019366}{0} & 0.799 & \bugometer{2.14613}{0.134873}{0.846851}{0.018276}{0} & \verb|f < f| & 3 & 144 & 2 & 2 & 0 & 0 & 0 & 5 \\
0.802 & \bugometer{2.8014}{0.130883}{0.865253}{0.003864}{0} & 0.672 & \bugometer{1.39794}{0.0144433}{0.984092}{0.0014647}{0} & \verb|files[fileid].size < token_index| & 31 & 0 & 10 & 633 & 0 & 0 & 0 & 40 \\
0.648 & \bugometer{2.67761}{0.11462}{0.616239}{0.038049}{0.231092} & 0.627 & \bugometer{2.19866}{0.0113816}{0.546359}{0.0751704}{0.367089} & \verb|passage_index == 293| & 27 & 3 & 8 & 0 & 0 & 0 & 2 & 366 \\
0.820 & \bugometer{2.89042}{0.116206}{0.876638}{0.005869}{0.001287} & 0.628 & \bugometer{1.20412}{0.00441909}{0.994416}{0.00116491}{0} & \verb|((*(fi + i)))->other.last_line == yyleng| & 776 & 0 & 16 & 0 & 0 & 0 & 18 & 1 \\
0.625 & \bugometer{2.51055}{0.111797}{0.610649}{0.046073}{0.231481} & 0.594 & \bugometer{1.61278}{0.00661276}{0.618031}{0.13145424}{0.243902} & \verb|min_index == 64| & 24 & 1 & 7 & 0 & 0 & 1 & 1 & 249 \\
0.819 & \bugometer{2.88762}{0.117597}{0.87521}{0.005898}{0.001295} & 0.519 & \bugometer{0.90309}{0.00289622}{0.996159}{0.00094478}{0} & \verb|((*(fi + i)))->this.last_line == yy_start| & 771 & 0 & 18 & 0 & 0 & 0 & 19 & 0 \\
0.489 & \bugometer{2.70586}{0.532437}{0.372044}{0.034495}{0.061024} & 0.476 & \bugometer{1.5563}{0.0500894}{0.430855}{0.1579446}{0.361111} & \verb|(passages + i)->fileid == 52| & 24 & 0 & 477 & 14 & 24 & 0 & 1 & 14 \\
0.666 & \bugometer{3.13066}{0.114929}{0.572709}{0.024427}{0.287935} & 0.461 & \bugometer{2.24797}{0.00443167}{0.33547}{0.07252733}{0.587571} & \verb|passage_index == 25| & 60 & 5 & 27 & 0 & 0 & 4 & 10 & 962 \\
0.836 & \bugometer{3.19976}{0.176195}{0.819543}{0.004262}{0} & 0.446 & \bugometer{0.69897}{0.00422009}{0.994974}{0.00080591}{0} & \verb|strcmp > 0| & 0 & 0 & 28 & 54 & 1584 & 0 & 0 & 68 \\
0.743 & \bugometer{2.99034}{0.143387}{0.707495}{0.021306}{0.127812} & 0.401 & \bugometer{2.25768}{0.00387753}{0.274786}{0.06940247}{0.651934} & \verb|i > 500| & 32 & 2 & 18 & 853 & 54 & 0 & 0 & 53 \\
0.812 & \bugometer{3.12613}{0.130952}{0.796744}{0.013216}{0.059088} & 0.347 & \bugometer{1.8451}{0.00152788}{0.243364}{0.11225112}{0.642857} & \verb|token_sequence[token_index].val >= 100| & 1250 & 3 & 28 & 38 & 0 & 15 & 19 & 65 \\
0.640 & \bugometer{2.80754}{0.104214}{0.582085}{0.034885}{0.278816} & 0.232 & \bugometer{0.845098}{0.00052404}{0.204298}{0.36660696}{0.428571} & \verb|i == 50| & 27 & 0 & 11 & 0 & 0 & 1 & 4 & 463 \\
0.668 & \bugometer{3.12743}{0.114601}{0.575334}{0.024457}{0.285608} & 0.197 & \bugometer{1.20412}{0.00036305}{0.137417}{0.23721995}{0.625} & \verb|passage_index == 19| & 59 & 5 & 28 & 0 & 0 & 4 & 10 & 958 \\
0.497 & \bugometer{1.27875}{0.0714847}{0.914362}{0.0141533}{0} & 0.000 & \bugometer{0}{0.00176211}{0.995798}{0.00243989}{0} & \verb|bytes <= filesbase| & 1 & 0 & 19 & 0 & 0 & 0 & 0 & 1 \\

\end{view}

\autoref{tab:views-exif} shows results for \exif 0.6.9, an open source
image processing program.  Each of the three predicates is a predictor
of a distinct and previously unknown crashing bug.  It took less than
20 minutes of work to find and verify the cause of each of the bugs
using these predicates and the additional highly correlated predicates
on their affinity lists.  All bugs have been confirmed as valid by
\exif project developers.

To illustrate how statistical debugging is used in practice, we
use the last of these three failure predictors as an example, and
describe how it enabled us to
effectively isolate the cause of one of the bugs.  Failed runs
exhibiting \texttt{o + s > buf\_size} show the following unique stack
trace at the point of termination:
\begin{quote}
  \small
\begin{verbatim}
main
  exif_data_save_data
    exif_data_save_data_content
      exif_data_save_data_content
        exif_data_save_data_entry
          exif_mnote_data_save
            exif_mnote_data_canon_save
              memcpy
\end{verbatim}
\end{quote}
The code in the vicinity of this crash site is as follows:
\begin{quote}
\begin{verbatim}
// snippet of exif_mnote_data_canon_save
for (i = 0; i < n->count; i++) {
    ...
    memcpy(*buf + doff,             (c)
           n->entries[i].data, s);
    ...
}
\end{verbatim}
\end{quote}
This stack trace alone provides little insight into the cause of the
bug.  However, our algorithm highlights \texttt{o + s > buf\_size} in
function \texttt{exif\_mnote\_data\_canon\_load} as a strong bug
predictor.  Thus, a quick inspection of the source code leads us to
construct the following call sequence:
\begin{quote}
  \small
\begin{verbatim}
main
  exif_loader_get_data
    exif_data_load_data
      exif_mnote_data_canon_load
  exif_data_save_data
    exif_data_save_data_content
      exif_data_save_data_content
        exif_data_save_data_entry
          exif_mnote_data_save
            exif_mnote_data_canon_save
              memcpy
\end{verbatim}
\end{quote}
The code in the vicinity of the predicate \texttt{o + s > buf\_size} is as follows:
\begin{quote}
\begin{verbatim}
// snippet of exif_mnote_data_canon_load
for (i = 0; i < c; i++) {
    ...
    n->count = i + 1;
    ...
    if (o + s > buf_size) return;    (a)
    ...
    n->entries[i].data = malloc(s);  (b)
    ...
}
\end{verbatim}
\end{quote}
It is apparent from the above code snippets and the
call sequence that whenever the predicate \texttt{o + s > buf\_size} is true,
%%
\begin{itemize}
\item the function \texttt{exif\_mnote\_data\_canon\_load} returns on
  line \texttt{(a)}, thereby skipping the call to \texttt{malloc} on
  line \texttt{(b)} and thus leaving \texttt{n->entries[i]->data}
  uninitialized for some value of \texttt{i}, and

\item the function \texttt{exif\_mnote\_data\_canon\_save} passes the
  uninitialized \texttt{n->entries[i]->data} to \texttt{memcpy} on line \texttt{(c)}, which reads it and eventually crashes.
\end{itemize}

In summary, our algorithm enabled us to effectively isolate the causes
of several previously unknown bugs in source code unfamiliar to us in
a small amount of time and without any explicit specification beyond
``the program shouldn't crash.''

\subsubsection{\rhythmbox}

\begingroup
\setlength{\segunit}{10pt}
\begin{view}[\tiny]{\rhythmbox}{rhythmbox}
  0.836 & \bugometer{3.20003}{0.176135}{0.819603}{0.004262}{0} & 0.836 & \bugometer{3.20003}{0.176135}{0.819603}{0.004262}{0} & \verb|files[filesindex].language > 16| & 0 & 0 & 28 & 54 & 1585 & 0 & 0 & 68 \\
0.820 & \bugometer{2.8893}{0.115811}{0.877037}{0.005862}{0.00129} & 0.838 & \bugometer{2.8893}{0.115811}{0.877037}{0.005862}{0.00129} & \verb|((*(fi + i)))->this.last_line == 1| & 774 & 0 & 17 & 0 & 0 & 0 & 18 & 2 \\
0.803 & \bugometer{2.85612}{0.130935}{0.851155}{0.008161}{0.009749} & 0.842 & \bugometer{2.85612}{0.107337}{0.874878}{0.008036}{0.009749} & \verb|token_index > 500| & 31 & 0 & 16 & 711 & 0 & 0 & 0 & 47 \\
0.748 & \bugometer{2.74819}{0.114434}{0.774373}{0.023693}{0.0875} & 0.797 & \bugometer{2.7332}{0.0820414}{0.802985}{0.0244006}{0.090573} & \verb|(p + passage_index)->last_token <= filesbase| & 28 & 2 & 508 & 0 & 0 & 0 & 1 & 29 \\
0.722 & \bugometer{2.52244}{0.175776}{0.783215}{0.016985}{0.024024} & 0.812 & \bugometer{2.48714}{0.0685383}{0.887332}{0.0180707}{0.026059} & \verb|__result == 0 is TRUE| & 16 & 0 & 0 & 9 & 19 & 291 & 0 & 13 \\
0.801 & \bugometer{2.91698}{0.117265}{0.829939}{0.014055}{0.038741} & 0.803 & \bugometer{2.58092}{0.0648828}{0.823005}{0.0281222}{0.08399} & \verb|config.match_comment is TRUE| & 791 & 2 & 23 & 1 & 0 & 5 & 11 & 41 \\
0.688 & \bugometer{3.00346}{0.114176}{0.622208}{0.026513}{0.237103} & 0.774 & \bugometer{2.9722}{0.0426556}{0.675659}{0.0279544}{0.253731} & \verb|i == yy_last_accepting_state| & 55 & 0 & 21 & 0 & 0 & 3 & 7 & 769 \\
0.674 & \bugometer{2.15836}{0.167018}{0.813616}{0.019366}{0} & 0.799 & \bugometer{2.14613}{0.134873}{0.846851}{0.018276}{0} & \verb|f < f| & 3 & 144 & 2 & 2 & 0 & 0 & 0 & 5 \\
0.802 & \bugometer{2.8014}{0.130883}{0.865253}{0.003864}{0} & 0.672 & \bugometer{1.39794}{0.0144433}{0.984092}{0.0014647}{0} & \verb|files[fileid].size < token_index| & 31 & 0 & 10 & 633 & 0 & 0 & 0 & 40 \\
0.648 & \bugometer{2.67761}{0.11462}{0.616239}{0.038049}{0.231092} & 0.627 & \bugometer{2.19866}{0.0113816}{0.546359}{0.0751704}{0.367089} & \verb|passage_index == 293| & 27 & 3 & 8 & 0 & 0 & 0 & 2 & 366 \\
0.820 & \bugometer{2.89042}{0.116206}{0.876638}{0.005869}{0.001287} & 0.628 & \bugometer{1.20412}{0.00441909}{0.994416}{0.00116491}{0} & \verb|((*(fi + i)))->other.last_line == yyleng| & 776 & 0 & 16 & 0 & 0 & 0 & 18 & 1 \\
0.625 & \bugometer{2.51055}{0.111797}{0.610649}{0.046073}{0.231481} & 0.594 & \bugometer{1.61278}{0.00661276}{0.618031}{0.13145424}{0.243902} & \verb|min_index == 64| & 24 & 1 & 7 & 0 & 0 & 1 & 1 & 249 \\
0.819 & \bugometer{2.88762}{0.117597}{0.87521}{0.005898}{0.001295} & 0.519 & \bugometer{0.90309}{0.00289622}{0.996159}{0.00094478}{0} & \verb|((*(fi + i)))->this.last_line == yy_start| & 771 & 0 & 18 & 0 & 0 & 0 & 19 & 0 \\
0.489 & \bugometer{2.70586}{0.532437}{0.372044}{0.034495}{0.061024} & 0.476 & \bugometer{1.5563}{0.0500894}{0.430855}{0.1579446}{0.361111} & \verb|(passages + i)->fileid == 52| & 24 & 0 & 477 & 14 & 24 & 0 & 1 & 14 \\
0.666 & \bugometer{3.13066}{0.114929}{0.572709}{0.024427}{0.287935} & 0.461 & \bugometer{2.24797}{0.00443167}{0.33547}{0.07252733}{0.587571} & \verb|passage_index == 25| & 60 & 5 & 27 & 0 & 0 & 4 & 10 & 962 \\
0.836 & \bugometer{3.19976}{0.176195}{0.819543}{0.004262}{0} & 0.446 & \bugometer{0.69897}{0.00422009}{0.994974}{0.00080591}{0} & \verb|strcmp > 0| & 0 & 0 & 28 & 54 & 1584 & 0 & 0 & 68 \\
0.743 & \bugometer{2.99034}{0.143387}{0.707495}{0.021306}{0.127812} & 0.401 & \bugometer{2.25768}{0.00387753}{0.274786}{0.06940247}{0.651934} & \verb|i > 500| & 32 & 2 & 18 & 853 & 54 & 0 & 0 & 53 \\
0.812 & \bugometer{3.12613}{0.130952}{0.796744}{0.013216}{0.059088} & 0.347 & \bugometer{1.8451}{0.00152788}{0.243364}{0.11225112}{0.642857} & \verb|token_sequence[token_index].val >= 100| & 1250 & 3 & 28 & 38 & 0 & 15 & 19 & 65 \\
0.640 & \bugometer{2.80754}{0.104214}{0.582085}{0.034885}{0.278816} & 0.232 & \bugometer{0.845098}{0.00052404}{0.204298}{0.36660696}{0.428571} & \verb|i == 50| & 27 & 0 & 11 & 0 & 0 & 1 & 4 & 463 \\
0.668 & \bugometer{3.12743}{0.114601}{0.575334}{0.024457}{0.285608} & 0.197 & \bugometer{1.20412}{0.00036305}{0.137417}{0.23721995}{0.625} & \verb|passage_index == 19| & 59 & 5 & 28 & 0 & 0 & 4 & 10 & 958 \\
0.497 & \bugometer{1.27875}{0.0714847}{0.914362}{0.0141533}{0} & 0.000 & \bugometer{0}{0.00176211}{0.995798}{0.00243989}{0} & \verb|bytes <= filesbase| & 1 & 0 & 19 & 0 & 0 & 0 & 0 & 1 \\

\end{view}
\endgroup

\autoref{tab:views-rhythmbox} shows our results for \rhythmbox 0.6.5,
an interactive, graphical, open source music player.  \rhythmbox is a
complex, multi-threaded, event-driven system, written using a library
providing object-oriented primitives in C.  Event-driven systems use
event queues; each event performs some computation and possibly adds
more events to some queues.  We know of no static analysis today that
can analyze event-driven systems accurately, because no static
analysis is currently capable of analyzing the heap-allocated event
queues with sufficient precision.  Stack inspection is also of
limited utility in analyzing event-driven systems, as the stack in the
main event loop is unchanging and all of the interesting state is in
the queues.

We isolated two distinct bugs in \rhythmbox.  The first predicate led
us to the discovery of a race condition.  The second predicate was not
useful directly, but we were able to isolate the bug using the
predicates in its affinity list.  This second bug revealed what turned
out to be a very common incorrect pattern of accessing the underlying
object library (recall \autoref{sec:introduction}).  \rhythmbox
developers confirmed the bugs and enthusiastically applied patches
within a few days, in part because we could quantify the bugs as
important crashing bugs.  It required several hours to isolate each of
the two bugs (and there are additional bugs represented in the
predictors that we did not isolate) because they were violations of subtle heap
invariants which are not directly captured by our current
instrumentation schemes.  Note, however, that we could not have even
begun to understand these bugs without the information provided by our
tool.  We intend to explore schemes that track predicates on heap
structure in future work.

\subsection{How many runs do we need?}
\label{sec:numruns}

\begin{table*}
\nocaptionrule
\caption{Minimum number of runs needed}
\label{tab:numruns}
\small
\centering
\begin{tabular}{lrrrrrrrr}
  \toprule
              &          & \multicolumn{7}{c}{Bug \#} \\
              \cmidrule(lr){3-9}
             & \mbox{\em runs} & \#1 & \#2 & \#3 & \#4 & \#5 & \#6 & \#9 \\
  \midrule
  \moss       & $F(P)$ & 18 & 10 & 32 & 12 & 21 & 11 & 20 \\
              &$N$   & 500 & 3000 & 2000 & 800 & 300 & 1000 & 600 \\ \midrule
  \ccrypt     & $F(P)$ & 26 \\
		& $N$ & 200 \\ \midrule
  \bc         & $F(P)$ & 40 \\
		& $N$ & 200 \\ \midrule
  \rhythmbox  & $F(P)$ & 22 & 35 \\
		& $N$ & 300 & 100 \\ \midrule
  \exif       & $F(P)$ & 28 & 12 & 13 \\
		& $N$ & 2000 & 300 & 21000 \\
  \bottomrule
\end{tabular}
\end{table*}

%% LocalWords:  lrrrrrrrr lr


Recall that we used about 32,000 runs in each of the five case studies.
We studied the minimum number of runs needed in each case study in order
to yield the same results as those obtained using 32,000 runs.
For each bug $B_i$,
we computed the minimum number of runs $N_i$ such that the initial $Importance$ of
predictor $P_i$ of bug $B_i$ computed using $N_i$ runs is roughly the same as that
computed using 32,000 runs.
Specifically, let $Importance_N(P)$ denote the initial $Importance$ of predicate $P$
computed using $N$ runs.
Then, we computed the minimum number of runs $N_i$ such that for predictor
$P_i$ of bug $B_i$, $Importance_{32,000}(P_i) - Importance_{N_i}(P_i) < 0.2$.

Table~\ref{tab:numruns} presents the results of our analysis.
The second column shows the maximum $N_i$ over all bugs detected in each case
study.
Intuitively, it is the minimum number of runs needed in each case
study in order to yield the same results as those obtained using
32,000 runs.
The reason we need 21,000 runs in the \exif experiment is that only
21 failing runs out of 33,000 runs share bug \#3 as the cause of failure.
If we exclude bug \#3, then just 2000 runs suffice.
The third column contains entries of the form $B_i : F_i$, where $F_i$ denotes
the number of failing runs in which predictor $P_i$ of bug $B_i$ is true
when $N_i$ runs are used.
Notice that we hardly need between 10 and 40 failing runs to share a bug
as their cause of failure in order for the predictor of that bug to be reported by
our algorithm.


\subsection{Comparison with Logistic Regression}
\label{sec:comparison}

\begin{table}
\nocaptionrule
\caption{Results of logistic regression for \moss}
\label{tab:logregression}
\centering
\scriptsize
\begin{tabular}{ll}
  \toprule
  Coefficient & Predicate \\
  \midrule
  0.769379 & \verb|(p + passage_index)->last_line < 4| \\
  0.686149 & \verb|(p + passage_index)->first_line < i| \\
  0.675982 & \verb|i > 20| \\
  0.671991 & \verb|i > 26| \\
  0.619479 & \verb|(p + passage_index)->last_line < i| \\
  0.600712 & \verb|i > 23| \\
  0.591044 & \verb|(p + passage_index)->last_line == next| \\
  0.567753 & \verb|i > 22| \\
  0.544829 & \verb|i > 25| \\
  0.536122 & \verb|i > 28| \\
  \bottomrule
\end{tabular}
\end{table}

In earlier work
we used $\ell_1$-regularized logistic regression
to rank the predicates by their
failure-prediction strength \cite{PLDI`03*141,NIPS2003_AP05}.
Logistic regression uses linearly weighted
combinations of predicates to classify a trial run as successful or
failed.  Regularized logistic regression incorporates a penalty
forcing most coefficients to be set to zero, thereby
selecting only the most important predicates.  The output is a set of
coefficients for predicates giving the best overall prediction.

A weakness of logistic regression for our application is that it seeks
to cover the set of failing runs without regard to the orthogonality
of the selected predicates (i.e., whether they represent distinct
bugs).  This problem can be seen in \autoref{tab:logregression},
which gives the top ten predicates selected by logistic regression
for \moss.  The striking fact is that all selected predicates are
either sub-bug or super-bug predictors.  The predicates beginning with
\texttt{p + \ldots} are all sub-bug predictors of bug \#1 (see
\autoref{tab:mossdilute}).  The predicates \texttt{i > \ldots} are
super-bug predictors: \texttt{i} is the length of the command line and
the predicates say program crashes are more likely for long command
lines (recall \autoref{sec:introduction}).

The prevalence of super-bug predictors on the list shows the
difficulty of making use of the penalty term.  Limiting the number of
predicates that can be selected via a penalty has the effect of
encouraging regularized logistic regression to choose super-bug predictors, as
these cover more failing runs at the expense of poorer predictive
power compared to predictors of individual bugs.  On the other hand,
the sub-bug predictors are chosen based on their excellent prediction
power of those small subsets of failed runs.
%%Relaxing the penalty
%%allows logistic regression to add more predicates to improve its
%%prediction, but the sub-bug predictors apparently are favored.

%% LocalWords:  exps mossdilute ccrypt bc exif buf mnote rhythmbox
%% LocalWords:  logregression
