


While we have targeted our algorithm at finding bugs, there are other
possible applications, and there are variations
on and extensions of the basic approach we have presented that may
prove useful.  In this section we briefly discuss some of these
possibilities.

While we have focused on bug finding, the same ideas
can be used to isolate predictors of any program event.  For example,
we could potentially look for early predictors of when the program
will raise an exception, send a message on the network, write to disk,
or suspend itself.  Furthermore, it is interesting to consider
applications in which the predictors are used on-line by the running
program; for example, knowing that a strong predictor of program
failure has become true may enable preemptive action (see
Section~\ref{sec:related-work}).

There are also variations on the specific algorithm we have proposed
that are worth investigating.  For example, we have chosen to discard
all the runs where $R(P) = 1$ when $P$ is selected by the
elimination algorithm, but there are at least three natural choices:
\begin{enumerate}
\item When $P$ is selected, discard all runs where $R(P) = 1$.

\item When $P$ is selected, discard only the failing runs where $R(P) = 1$.

\item When $P$ is selected, relabel all failing runs where $R(P) = 1$ as successful runs.
\end{enumerate}

We have already given the intuition for (1), our current choice.  For
(2), the idea is that whatever the bug is, it is not present in the
successful runs and thus retaining all successful runs is more
representative of correct program behavior.  Proposal (3) goes one step
farther, asserting that even the failing runs will look mostly the same
once the bug is fixed, and the best approximation to a program without the
bug is simply that the failing runs are now successful runs.

On a more technical level, the three proposals differ in how much code
coverage they preserve.  By discarding no runs, proposal (3) preserves
all the code paths that were executed in the original
runs, while proposal (1) discards the most runs and so
potentially renders more paths unreached in the runs that
remain.  This difference in paths preserved translates into differences
in the \crash\ and \context\ scores of predicates under the different schemes.
In fact, it is possible to prove that for a predicate $P$ and its complement
$\neg P$, that when predicate $P$ is selected by the elimination algorithm,
then 
\[ \increase_3(\neg P) \geq \increase_2(\neg P) \geq \increase_1(\neg P) = 0 \]
where the subscripts indicate which scheme for discarding runs is used and
assuming all the quantities are defined.
Thus, scheme (1) is the most conservative, in the sense that only one of $P$
or $\neg P$ can have positive predictive power, while scheme (3) potentially allows more predictors to have positive \increase\ scores.


