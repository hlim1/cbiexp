%% -*- TeX-master: t -*-

%% \documentclass{acm_proc_article-sp}
\documentclass{sig-alternate}


%%%%%%%%%%%%%%%%%%%%%%%%%%%%%%%%%%%%%%%%%%%%%%%%%%%%%%%%%%%%%%%%%%%%%%%%
%%
%% standard texmf packages
%%

\usepackage{amsmath}
\usepackage{ifthen}
\usepackage{nicefrac}
\usepackage{fancyvrb}
\usepackage{subfigure}
\usepackage{xspace}

\usepackage[bookmarks, pdftitle={Statistical Debugging in the Presence of
  Multiple Errors}, pdfauthor={Ben Liblit, Mayur Naik, Alice X.
  Zheng, Alex Aiken, and Michael I.  Jordan}, pdfsubject={D.2.4
  [Software Engineering]: Software/Program Verification -- statistical
  methods; D.2.5 [Software Engineering]: Testing and Debugging --
  debugging aids, distributed debugging, monitors, tracing; I.5.2
  [Pattern Recognition]: Design Methodology -- feature evaluation and
  selection}, pdfkeywords={bug isolation, random sampling, invariants,
  feature selection, statistical debugging},
pdfpagemode=UseOutlines]{hyperref}

\ifthenelse{\isundefined{\pdfoutput}}{}{\usepackage{thumbpdf}}


%%%%%%%%%%%%%%%%%%%%%%%%%%%%%%%%%%%%%%%%%%%%%%%%%%%%%%%%%%%%%%%%%%%%%%%%
%%
%% unique to this paper
%%

\usepackage{Autoref}
\newcommand{\subfigureautorefname}[0]{Figure}

%% assorted handy macros
\newcommand{\moss}{\textsc{Moss}\xspace}
\newcommand{\rhythmbox}{\textsc{Rhythmbox}\xspace}
\newcommand{\termdef}[1]{\textit{#1}}
\newcommand{\prob}{\mbox{\textit{Prob}}}
\newcommand{\fail}{\mbox{\textit{Crash}}}
\newcommand{\crash}{\mbox{\textit{Failure}}}
\newcommand{\context}{\mbox{\textit{Context}}}
\newcommand{\increase}{\mbox{\textit{Increase}}}
\renewcommand{\H}{{\mathcal{H}}}


%%%%%%%%%%%%%%%%%%%%%%%%%%%%%%%%%%%%%%%%%%%%%%%%%%%%%%%%%%%%%%%%%%%%%%%%
%%
%% front matter
%%

\title{Statistical Debugging in the Presence of Multiple Errors
  %%
  \thanks{This research was supported in part by NASA Grant No.\
    NAG2-1210; NSF Grant Nos.\ EIA-9802069, CCR-0085949, ACI-9619020,
    and IIS-9988642; DOE Prime Contract No.\ W-7405-ENG-48 through
    Memorandum Agreement No.\ B504962 with LLNL; and DARPA ARO-MURI 
    ACCLIMATE DAAD-19-02-1-0383.  The information
    presented here does not necessarily reflect the position or the
    policy of the Government and no official endorsement should be
    inferred.}}
\numberofauthors{3}

\makeatletter
\newcommand*{\eecsMark}[0]{\@fnsymbol{2}}
\newcommand*{\statMark}[0]{\@fnsymbol{3}}
\newcommand*{\stanMark}[0]{\@fnsymbol{4}}
\makeatother
\newcommand*{\eecs}[0]{\textsuperscript{\eecsMark}}
\newcommand*{\stat}[0]{\textsuperscript{\statMark}}
\newcommand*{\both}[0]{\textsuperscript{\eecsMark, \statMark}}
\newcommand*{\stan}[0]{\textsuperscript{\stanMark}}

\newcommand{\moreauthors}[0]{\end{tabular}\\\vspace{-.5\baselineskip}\begin{tabular}{c}}

\author{
  \alignauthor Ben Liblit \eecs \\
  \alignauthor Mayur Naik \stan \\
  \alignauthor Alice X.\ Zheng \eecs \\
  \moreauthors
  \global\multiply\auwidth by 3
  \global\divide\auwidth by 2
  \alignauthor Alex Aiken \stan \\
  \alignauthor Michael I.\ Jordan \both
  \moreauthors
  \alignauthor
  \affaddr{\eecs Department of Electrical \\ Engineering and Computer Science} \\
  \affaddr{\stat Department of Statistics} \\
  \affaddr{University of California, Berkeley} \\
  \affaddr{Berkeley, CA 94720-1776}
  \alignauthor
  \affaddr{\stan Computer Science Department} \\
  \affaddr{353 Serra Mall} \\
  \affaddr{Stanford University} \\
  \affaddr{Stanford CA 94305-9025}
}

\bibliographystyle{abbrv}


%%%%%%%%%%%%%%%%%%%%%%%%%%%%%%%%%%%%%%%%%%%%%%%%%%%%%%%%%%%%%%%%%%%%%%%%
%%
%%  document body
%%

\begin{document}

\conferenceinfo{PLDI'04,}{June 9--11, 2004, Washington, DC, USA.}
\CopyrightYear{2004}
%% \crdata{}
\maketitle

\begin{abstract}
  We present a statistical debugging algorithm that operates on very
  sparsely sampled data drawn from large numbers of user runs.  By
  identifying program behaviors that significantly increase the
  likelihood of failure, our technique helps guide software engineers
  to the most significant flaws in an application.  The approach has
  connections to statistical hypothesis testing and is validated
  using several case studies.
\end{abstract}


%\category{D.2.4}{Software Engineering}{Software/Program
%  Verification}[statistical methods]
%%
%\category{D.2.5}{Software Engineering}{Testing and
%  Debugging}[debugging aids, distributed debugging, monitors, tracing]
%%
%\category{I.5.2}{Pattern Recognition}{Design Methodology}[feature
%  evaluation and selection]

%\terms{Experimentation, Reliability}

%\keywords{bug isolation, random sampling, invariants, feature
%  selection, statistical debugging}


\section{Introduction}
\label{sec:introduction}

It is a fact of life that most software ships with many known and
unknown bugs.  Unfortunately, in most cases it is impractical to delay
releasing code until the last bug has been fixed.  For example, the
Mozilla web browser project has a ``topcrash'' software QA team
specifically dedicated to identifying and tracking the top forty most
common crash bugs.  This cut-off is not because anyone believes that
there are only forty bugs; it is an acknowledgment that engineering
resources are finite and so the most important bugs should be fixed
first.

Our vision is to use \termdef{statistical debugging} to improve
software quality after code is deployed.  Specially instrumented
programs monitor their own behavior and send feedback reports to a
central collection server.  Monitoring is both sparse and random, so
complete information is never available about any single run.
However, monitoring is also lightweight and therefore practical to
deploy to user communities numbering in the thousands or millions.
Statistical debugging does not seek to understand the single cause of
a single failure on one machine; rather, it identifies statistical
trends that isolate bug causes across many runs.  Because the process
is driven by data from actual user executions, it reflects the reality
of how the software is used and implicitly attends to those program
behaviors that cause the most problems for the most users, most often.

In previous work we examined the problem of automatically isolating
the causes of bugs and proposed two algorithms, the most general of
which was based on a statistical technique known as {\em logistic
regression} \cite{PLDI`03*141,Zheng:2003:SDSP}.  As we worked to apply
these methods to large programs under realistic conditions, we
discovered a number of scalability problems, the most serious of which
was that the performance of the techniques declined significantly 
when the applications being analyzed had multiple bugs (see
\Autoref{sec:comparison}).  As mentioned above, real applications
have many bugs.

The first contribution of this paper is a new algorithm for isolating
multiple bugs in complex applications (see \Autoref{sec:algorithm}).  This algorithm 
is based on an entirely different family of statistical techniques than our previous work and offers
a number of improvements:
it is much more efficient, the quality of the
results is high and unaffected by increases in either the size of the program
or the number of bugs, and it naturally produces information
that shows how important (in number of program crashes) each detected
cause of a bug is.  Furthermore, the method is statistically sound, meaning that
the results computed accurately summarize the analyzed program 
executions, and the results can be directly understood in terms of program behavior.
For example, here is a portion of a report generated for \rhythmbox,
a popular Open Source jukebox:
\begin{verbatim}
Predicate: g_source_remove() > 0 
Location: line 77 of disclosure-widget.c
Context: 0.08
Increase: 0.92   
Failure: 1.00   
\end{verbatim}
This information summarizes what is observed about a particular call site of the function {\tt g\_source\_remove}.
Any execution that reaches this function call has an 8\% chance of failure (the Context score).  If, however,
the function returns a
positive value\footnote{A common C programming idiom is for functions to signal success or failure by the sign of the return value.}, the probability that the program fails is 100\%; i.e., when this predicate on this line is true the program always crashes eventually.  We say that the predicate contributes an increase in the probability of failure of 100\% - 8\% = 92\%.

It turns out that the crash happens long after the call to {\tt g\_source\_remove} completes,
but the fact that {\tt g\_source\_remove} succeeds (indicated by the positive return value) is part of what causes the bug.  Thus,
our approach can help engineers isolate bugs by showing the conditions and early events that trigger the bug, rather than just the
program state at the point of failure.  Furthermore, our algorithm also allows accurate calculations of what fraction of all failures occur when this predicate
is true; in the example above, the predicate was true in 184 of 1873 \rhythmbox\ executions that crashed in our dataset, which means that
it is involved in 9.8\% of all the failures observed.  This information allows engineers to assign accurate priorities to bug fixes.


Our second contribution is a series of case studies evaluating the effectiveness of
our technique.  The first two studies compare our algorithm to the two techniques
we previously proposed (see \Autoref{sec:revisited}).  Each of these studies focuses on a
program with a single bug; in both cases the new algorithm
performs at least as well as the old one at identifying the conditions under which the bug occurs.  

The third study involves an application with multiple known bugs (see
\Autoref{sec:experiments:results}).  Knowing the bugs makes this a controlled study, 
in the sense that we are able to determine whether the algorithm fails to recognize any of the
bugs.  The results show that the algorithm correctly
isolates strong predictors for each of the bugs that occurred in our
experiment, even though the algorithm has no knowledge of how many
bugs are in the program or which bug caused any individual failure.
In addition, our algorithm isolated another bug in this application that was previously unknown.

We also use this third case study to investigate a speculative
application of our techniques.  Our algorithm relies only on the
ability to label the outcome of a program as ``correct'' or
``incorrect''.  Furthermore, because our techniques are statistical,
the labeling of correct/incorrect results need not be perfect; our
algorithm can tolerate noise in the labeling.  These features open up
the possibility of having a single approach that can detect a wide
range of program errors from low-level, crashing bugs to high-level
logic bugs that do not crash the program but simply cause it to
produce incorrect output, provided only that there is some way to
label program outcomes reasonably accurately.  Detecting incorrect
output can be done in a number of ways.  Certainly program crashes are
incorrect (e.g., in C programs), as are unhandled top-level exceptions
(e.g., in Java programs).  Internal program checks for consistency,
such as low-level assert statements and higher-level input validation
checks between untrusting components, can also be used to label
results as correct or incorrect.  Finally (and this is the speculative
part) users could also provide feedback on whether the output of a
program execution appears correct or not, allowing the automatic
isolation of bugs that are not otherwise observed by a crash or
internal consistency check.  The third study includes some subtle,
non-crashing bugs that our system is able to isolate correctly.


In the fourth study we apply our techniques to the current release
of \rhythmbox, the Open Source jukebox mentioned above (see
\Autoref{sec:rb}).  \rhythmbox\ is the largest and most complex of our
case studies, and the bugs in \rhythmbox\ are unknown to us.  Thus far
we have successfully isolated two serious crashing bugs in \rhythmbox\ 
using our approach.  At least in part because we were able to
quantify the importance of these bugs for the \rhythmbox\ developers, both
bug reports were given high priority and fixes were applied to the
main development branch within a few days.


The paper is organized as follows.  After providing some background
(\Autoref{sec:background}), we discuss our algorithm
(\Autoref{sec:algorithm}) and its advantages over our previous approach
based on logistic regression (\Autoref{sec:comparison}).  The case studies
are presented in Sections~\ref{sec:revisited}-\ref{sec:rb}.  We conclude with a
discussion of related work (\Autoref{sec:related-work}) and future
work (\Autoref{sec:conclusions}). 

\section{Background}
\label{sec:background}

This section summarizes ideas and terminology needed to to present our
algorithm.  The ideal program monitoring system would gather complete
execution traces and provide them to an engineer (or, more likely, a
tool) to mine for the causes of bugs.  However, complete tracing of
program behavior is simply impractical; no end user would accept the
required performance overhead or network bandwidth.

Instead, we use a combination of sparse random sampling, which controls
performance overhead, and client-side summarization of the data, which
limits the storage and transmission costs.  We briefly discuss
both aspects.

Random sampling is added to a program via a source-to-source transformation.
Our sampling transformation is general: any collection of
statements within (or added to) a program may be designated as
``instrumentation'' and thereby sampled instead of run
unconditionally.  That is, each time instrumentation code is reached,
a coin flip decides whether the instrumentation is executed or not.
Coin flipping is simulated in a statistically fair
manner equivalent to a Bernoulli process: each potential sample is
taken or skipped randomly and independently as the program runs.
We have found that a sampling rate of \nicefrac{1}{100} normally keeps the performance overhead
of instrumentation low, often unmeasurable.

Orthogonal to the sampling transformation is the decision about what
instrumentation to introduce and how to concisely
summarize the resultant data.  This determines what behaviors one can
possibly observe once the sampling transformation is applied.  A useful instrumentation
scheme should be fairly general but selected to capture behaviors
which are likely to be of interest when hunting for bugs.  At present
our system offers the following instrumentation schemes:

\begin{description}
\sloppy
\item[branches:] At each conditional, including implicit conditionals
such as loop tests and short-circuiting logical operators,
  count how often each branch is taken.  
%The observation is
%  made just after the predicate is evaluated but before the selected
%  branch is taken.  This scheme also applies to implicit conditionals
%  in loops and logical operators (\texttt{\&\&}, \texttt{||},
%  \texttt{?:}).  Each conditional induces one instrumentation site
%  with two counters: number of times branched true and number of times
%  branched false.

\item[returns:] In C, the
  sign of a return value is often used to encode an operation's success or failure.
  At each scalar-returning function call site, count how
  often the returned value is negative, zero, or positive.  For
  pointer-returning calls, this scheme reduces to counting
  \texttt{NULL} versus non-\texttt{NULL}.  
%The observation is made
%  just after the function returns but before the result is used by the
%  original program.  An instrumentation site is added even if the
%  source program discards the return value, as unchecked return
%  values are a common source of bugs.  Each call site induces one
%  instrumentation site with three counters: number of negative
%  returns, number of zero returns, and number of positive returns.

\item[scalar-pairs:] Many bugs
  concern boundary issues exposed by the relationship between a pair
  of variables, or between a variable and some program constant.  At
  each scalar assignment \texttt{x = \dots}, identify each
  \emph{other} same-typed in-scope variable $\mathtt{y}_i$ and each
  constant expression $\mathtt{c}_j$.  Count how often the new value
  for \texttt{x} is less than, equal to, or greater than each
  $\mathtt{y}_i$ and each $\mathtt{c}_j$.  
%The observation is made
%  after both sides of the assignment have been evaluated but
%  before the assignment takes place.  This lets us compare
%  \texttt{x} to \texttt{x} as well, effectively comparing the new and
%  old values of the left-hand side.  
Each compared-to $\mathtt{y}_i$
  or $\mathtt{c}_j$ is treated as a distinct instrumentation site;
  thus a single assignment may induce a large number of sites.  
% Each
%  such site maintains three counters: how often the value being
%  assigned is less than, equal to, or greater than the compared-to
%  variable or constant.
\end{description}

The common theme among these instrumentation schemes is that we
instrument program predicates and count the number of times that these
predicates are observed to be true or false when a sample is taken.
For the returns and scalar-pairs instrumentation schemes, the three
predicates that are tracked explicitly actually give rise to a family
of six predicates that we use in our analysis of the data.
For example, for each scalar-pairs comparison between \texttt{x} and \texttt{y}, one and
only one of the three predicates $\mathtt{x} < \mathtt{y}$, $\mathtt{x} = \mathtt{y}$, or
$\mathtt{x} > \mathtt{y}$ can be true.  Thus one observation at one
site always updates exactly one of that site's counters.  An
observation that $\mathtt{x} < \mathtt{y}$ is true is therefore equivalent to
an observation that $\mathtt{x} \geq \mathtt{y}$ is false.  Similarly, we can
infer from the data the two remaining predicates $\mathtt{x} \leq \mathtt{y}$ or
$\mathtt{x} \neq \mathtt{y}$.

Also note that because some counter is always incremented at every
sample in all three schemes, we can easily determine the number of
times a site was sampled by summing all the counters for the site
(two counters for the branches instrumentation and three counters for
the others).  Thus, we can distinguish between predicates that are
never observed during execution (all the counters at the site are 0)
and predicates that are observed but never true (the counter(s) for the predicate
is 0, but some other counter at the site is positive).

The post-execution feedback report consists of the final counter values for
each instrumentation site.  Reducing a trace to a set of counters
prevents us from reasoning about relative time ordering of events
during execution.  However, it also means that program actions early
in execution remain just as visible as those much later.  This
contrasts with traditional postmortem debugging tools which expose
only the final state of the program at the point of failure.

\section{Cause Isolation Algorithm}
\label{sec:algorithm}
This section presents our algorithm for automatically isolating
multiple bugs.  As discussed in \autoref{sec:background}, the input is
a set of feedback reports from individual program runs $R$, where
$R(P) = 1$ if predicate $P$ is observed to be true during the
execution of $R$.

The idea behind the algorithm is to simulate the iterative manner in which  human programmers
typically find and fix bugs:
\begin{enumerate}

\item Identify the cause of the most important bug ${\cal B}$.

\item Fix ${\cal B}$, and repeat.

\end{enumerate}

For our purposes, identifying the cause of a bug ${\cal B}$ means selecting a
predicate $P$ closely correlated with ${\cal B}$.  The difficulty is that we
know the set of runs that succeed and fail, but we do not know which
set of failing runs corresponds to ${\cal B}$, or even how many bugs there
are.  Thus, in the first step we must infer which predicates are most
likely to correspond to individual bugs and rank those predicates in
importance.

For the second step, while we cannot literally fix the bug
corresponding to the chosen predictor $P$, we can simulate what
happens if the bug does not occur.  We discard any run $R$ such that
$R(P) = 1$ and repeat.  Discarding all the runs where $R(P) = 1$
reduces the importance of other predictors of ${\cal B}$, allowing predicates
that predict different bugs (i.e., different sets of failing runs) to
rise to the top in subsequent iterations.

\subsection{Increase Scores}
\label{sec:increase}

We now discuss the first step, how to find the cause of the most important bug.
We break this step into two sub-steps.  First, we eliminate predicates that have no
predictive power at all; this typically reduces the number of predicates we need
to consider by two orders of magnitude (e.g., from hundreds of thousands to thousands).
Next, we rank the surviving predicates by importance (see \autoref{sec:ranking}).

Consider the following C code fragment:
\begin{quote}
\begin{verbatim}
f = ...;          (a)
if (f == NULL) {  (b)
        x = 0;    (c)
        *f;       (d)
}
\end{verbatim}
\end{quote}
Consider the predicate {\tt f == NULL} at line {\tt (b)}, which would
be captured by branches instrumentation.  Clearly
this predicate is highly correlated with failure; in fact, whenever it
is true this program inevitably crashes.\footnote{We also note that this bug could 
be detected by a simple static analysis; this example is meant to be concise rather than 
a significant application of our techniques.}   An important observation,
however, is that even a ``smoking gun'' such as {\tt f == NULL} at
line {\tt (b)} cannot be a perfect predictor of failure when there are
multiple bugs in the program---since there are other bugs, the run can fail
even if the predicate is never true in a run.

The bug in the code fragment above is \termdef{deterministic} with
respect to {\tt f == NULL}: if {\tt f == NULL} is true at line {\tt
(b)}, the program fails.  In many cases it is impossible to observe
the exact conditions causing failure; for example, buffer overrun bugs
in a C program may or may not cause the program to crash depending on
runtime system decisions about how data is laid out in memory.  Such
bugs are \termdef{non-deterministic} with respect to every predicate;
even for the best predictor $P$, it is possible
that $P$ is true and still the program terminates normally.  In the
example above, if we insert before line {\tt (d)} a valid pointer
assignment to {\tt f} controlled by a conditional that is true at
least occasionally (say via a call to read input):
\begin{quote}
\begin{verbatim}
if (read()) f = ... some valid pointer ...;
*f;
\end{verbatim}
\end{quote}
the bug becomes non-deterministic with respect to {\tt f == NULL}.

To summarize, even for a predicate $P$ that is truly the cause of a bug, we can neither assume that
when $P$ is true that
the program fails nor that when $P$ is never observed to be true  that
the program succeeds. But we can express the probability that $P$
being true implies failure.  Let $\fail$ be an atomic predicate that is
true for failing runs and false for successful runs.  Let $\prob(A | B)$ denote
the conditional probability function of the event $A$ given event $B$ .  
We want to compute:
% [[ modified by Alice]]
\[ \crash(P) \equiv \prob(\fail = 1 | P = 1) \]
%= \prob(\fail | P \mbox{ is true}) \]
for every instrumented predicate $P$ over the set of all runs.  Let $S(P)$ be the number
of successful runs in which $P$ is observed to be true, and let $F(P)$ be the number of
failing runs in which $P$ is observed to be true.  We
estimate $\crash(P)$ as:
\[ \crash(P) = \frac{F(P)}{S(P) + F(P)} \]

Notice that $\crash(P)$ is unaffected by the set of runs in which
$P$ is not observed to be true.  Thus, if $P$ is the cause of a bug, the
causes of other independent bugs do not affect $\crash(P)$.
Also note that runs in which $P$ is not observed at all (either because
the line of code on which $P$ is checked is not reached, or the line is reached
but $P$ is not sampled) have no effect on $\crash(P)$.
Finally, the definition of $\crash(P)$
generalizes the idea of deterministic and non-deterministic bugs.  A
bug is deterministic for $P$ if $\crash(P) = 1.0$ or, equivalently,
$P$ is never observed to be true in a successful run ($S(P) =
0$) and $P$ is observed to be true in at least one failing run ($F(P) > 0$).
If $\crash(P) < 1.0$ then the bug is non-deterministic, with
lower scores showing weaker correlation between the predicate and
program failure.

Now $\crash(P)$ is a useful measure, but it is not good
enough for the first step of our algorithm. To see this, consider again the
code fragment given above (in its original form, not with the
modification to make the bug non-deterministic).  At line {\tt (b)} we
have $\crash(\mbox{\tt f == NULL}) = 1.0$, so this predicate is a good
candidate for the cause of the bug.
But on line {\tt (c)} we have the surprising fact that $\crash(\mbox{\tt x == 0}) = 1.0$ as well.
To understand why, observe that the 
predicate \texttt{x == 0} is always true at line {\tt (c)} and, in
addition,
only failing runs reach this line.
Thus $S(\mbox{\tt x == 0}) = 0$, and, so long as there is at least one run that
reaches line {\tt (c)} at all, $\crash(\mbox{\tt x == 0})$ at line {\tt (c)} is 1.0.

As the predicate {\tt x == 0} at line {\tt (c)} of the example
shows, just because $\crash(P)$ is high does not
mean $P$ is the cause of a bug.  In the case of {\tt x == 0}, the
decision that eventually causes the crash is made earlier, and the
$\crash(\mbox{\tt x == 0})$ score merely reflects the fact that this
predicate is checked on a path where the program is already doomed.

A way to address this difficulty is to score a predicate not by the chance
that it implies failure, but by how much difference it makes that the predicate
is observed to be true versus simply reaching the line where the predicate is checked.
That is, on line {\tt (c)}, the probability of crashing is already 1.0 regardless
of the value of the predicate {\tt x == 0}, and thus the fact that {\tt x == 0} is
true does not increase the probability of failure at all; this coincides with
the intuition that this predicate is irrelevant to the bug.

This leads us to the following definition:
\[
%% [[modified by Alice]]
\context(P) \equiv \prob(\fail = 1 | P \mbox{ is observed})  
\]
%= \prob(\fail |  P \mbox{ observed}) 
%\end{gather*}
Now, $P \lor \lnot P$ is not the set of all runs, because we are not working in a two-valued logic.
In any given run, neither $P$ nor $\lnot P$ may be observed (because the site where this predicate is
sampled is not reached), or one may be observed, or both may be observed (because the statement is executed
multiple times and $P$ is sometimes true and sometimes false).  Thus, $\context(P)$ is the probability that
in the set of runs where the value of $P$ is observed at all, the program fails. We can compute $\context(P)$ as follows:
\[ \context(P) = \frac{F(P \lor \lnot P)}{S(P \lor \lnot P) + F(P \lor \lnot P)} \]

The interesting quantity, then, is
\begin{equation*}
 \increase(P) \equiv \crash(P) - \context(P) \label{eqn:1}
\end{equation*}
%%
which can be read as: How much does $P$ being true increase the probability of failure
over simply reaching the line where $P$ is sampled?  For example, for the predicate {\tt x == 0} on line {\tt (c)},
we have
\[\crash(\mbox{\tt x == 0}) = \context(\mbox{\tt x == 0}) = 1.0 \]
and so $\increase(\mbox{\tt x == 0}) = 0$.

A predicate $P$ with $\increase(P) \leq 0$ has no predictive power; whether it is true does not increase the
probability of failure, and we can safely discard all such predicates.
But because some $\increase(P)$ scores may be based on few observations of $P$, it is important
to attach confidence intervals to the scores.  Since $\increase()$ is a statistic, computing
its confidence interval is a well-understood problem. In our experiments we retain a predicate $P$
only if $\increase(P) > 0$ with 95\% confidence; this removes predicates from consideration that have high
increase scores but very low confidence because of few observations. 

Pruning predicates using $\increase(P) \leq 0$ has many desirable
properties.  It is easy to prove that large classes of irrelevant
predicates always have scores $\leq 0$.  For example, any predicate
that is unreachable, that is a program invariant, or that is obviously
control-dependent on a true cause is eliminated by this test.  It is
also worth pointing out that this test tends to localize bugs at
a point where the condition that causes the bug becomes true, rather than at
the crash site.  For example, in the code fragment given above, the bug is
attributed to the success of the conditional branch test {\tt f ==
NULL} on line {\tt (b)} rather than the pointer dereference on line
{\tt (d)}.  Thus, the cause of the bug discovered by the algorithm
points directly to the conditions under which the crash occurs, rather than
the line on which it occurs (which is usually available anyway in the
stack trace).

\subsection{Statistical Interpretation}
\label{sec:statisticalinterpretation}

We have explained the test $\increase(P) > 0$ using programming terminology,
but it also has a natural statistical interpretation as a simplified {\em likelihood ratio} hypothesis
test.  Consider the two classes of trial runs
of the program: failed runs $F$ and successful runs $S$.  For each
class, we can treat the predicate $P$ as a Bernoulli random variable
with heads probabilities $\pi_f(P)$ and $\pi_s(P)$, respectively, for the
two classes.  The heads
probability is the probability that the predicate is observed to be
true.  If a predicate causes a set of crashes, then $\pi_f$ should be
much bigger than $\pi_s$.  We can formulate two statistical hypotheses:
the null hypothesis $\H_0:
\pi_f \leq \pi_s$, versus the alternate hypothesis $\H_1: \pi_f > \pi_s$.  Since
$\pi_f$ and $\pi_s$ are not known, we must estimate them:
\begin{align*}
  \hat \pi_f(P) &= \frac{F(P)}{F(P \lor \lnot P)} &
  \hat \pi_s(P) &= \frac{S(P)}{S(P \lor \lnot P)}
\end{align*}

Although these proportion estimates of $\pi_f$ and $\pi_s$ approach the
actual heads probabilities as we increase the number of trial runs, they
still differ due to sampling.  With a certain probability, using these
estimates instead of the actual values results in the wrong
answer.  A \textit{likelihood ratio test} takes this uncertainty into
account, and makes use of the statistic $ Z = \frac{(\hat \pi_f - \hat
  \pi_s)}{V_{f,s}}$, where $V_{f,s}$ is a sample variance term (see
e.g., \cite{Lehmann:1986:hyptest}).  When
the data size is large, $Z$ can be approximated as a standard Gaussian
random variable.  Performed independently for each predicate $P$, the
test decides whether or not $\pi_f(P) \leq \pi_s(P)$ with a guaranteed
false-positive probability (i.e.,\ choosing $\H_1$ when $\H_0$ is true).
A necessary (but not sufficient) condition for choosing $\H_1$ is that
$\hat \pi_f(P) > \hat \pi_s(P)$.  This turns out to be
equivalent to the condition that $\increase(P) > 0$.  To see this,
let $a = F(P)$, $b = S(P)$, $c = F(P\lor\lnot P)$, and $d = S(P\lor\lnot P)$.
Then
\begin{gather*}
  \increase(P) > 0 \iff \crash(P) > \context(P) \\
  \iff \frac{a}{a+b} > \frac{c}{c+d}
  \iff a (c+d) > (a+b) c \\
  \iff ad > bc \iff \frac{a}{c} > \frac{b}{d}
  \iff \hat \pi_f(P) > \hat \pi_s(P)
\end{gather*}

%% -*- TeX-master: "../master.tex" -*-

\begin{table*}[tb]
  \nocaptionrule
  \caption{Comparison of ranking strategies for \moss without
    redundancy elimination}
  \label{tab:sorts}
  \centering

  \newenvironment{stats}[2]{%
    \begin{subtab}
      \caption{Sort descending by #1}
      \small
      \label{tab:sorts-#2}
      \begin{tabular}{l|rr@{$\:\pm\:$}rr|rrl}
        \toprule
        Thermometer & Context & \multicolumn{2}{r}{Increase} &
        S & F & F + S & Predicate \\
        \midrule}{%
        \bottomrule
      \end{tabular}
    \end{subtab}}

  \begin{stats}{$\text{F}(P)$}{failure}
    \bugometer{4.44089}{0.17602}{0.000535293}{0.006240707}{0.817204} & 0.176 & 0.007 & 0.012 & 22554 & 5045 & 27599 & \verb|files[filesindex].language &#8800; 15| \\
\bugometer{4.44108}{0.175879}{0.000600451}{0.006237549}{0.817283} & 0.176 & 0.007 & 0.012 & 22566 & 5045 & 27611 & \verb|tmp == 0 is FALSE| \\
\bugometer{4.44116}{0.175904}{0.000543038}{0.006236962}{0.817316} & 0.176 & 0.007 & 0.012 & 22571 & 5045 & 27616 & \verb|strcmp &#8800; 0| \\
\bugometer{4.36446}{0.176292}{0.000813956}{0.006562044}{0.816332} & 0.176 & 0.007 & 0.013 & 18894 & 4251 & 23145 & \verb|tmp == 0 is FALSE| \\
\bugometer{4.36408}{0.176018}{0.00077294}{0.00656006}{0.816649} & 0.176 & 0.007 & 0.013 & 18885 & 4240 & 23125 & \verb|files[filesindex].language &#8800; 14| \\
\bugometer{4.33774}{0.175829}{0.00160129}{0.00668071}{0.815889} & 0.176 & 0.008 & 0.013 & 17757 & 4007 & 21764 & \verb|filesindex &#8805; 25| \\
\bugometer{4.30501}{0.176949}{0.000967379}{0.006932621}{0.815151} & 0.177 & 0.008 & 0.014 & 16453 & 3731 & 20184 & \verb|M < M| \\
\bugometer{3.93024}{0.175776}{0.249223}{0.011356}{0.563645} & 0.176 & 0.261 & 0.023 & 4800 & 3716 & 8516 & \verb|config.winnowing_window_size &#8800; argc| \\
\multicolumn{8}{c}{\dotfill{} 2732 additional predictors follow \dotfill{}} \\

  \end{stats}

  \begin{stats}{$\text{Increase}(P)$}{increase}
    \bugometer{1.36173}{0.0652963}{0.924982}{0.0097217}{0} & 0.065 & 0.935 & 0.019 & 0 & 23 & 23 & \verb|((*(fi + i)))->this.last_token < filesbase| \\
\bugometer{1}{0.0652535}{0.924878}{0.0098685}{0} & 0.065 & 0.935 & 0.020 & 0 & 10 & 10 & \verb|((*(fi + i)))->other.last_line == last| \\
\bugometer{1.25527}{0.0713427}{0.918557}{0.0101003}{0} & 0.071 & 0.929 & 0.020 & 0 & 18 & 18 & \verb|((*(fi + i)))->other.last_line == filesbase| \\
\bugometer{1}{0.0727273}{0.917042}{0.0102307}{0} & 0.073 & 0.927 & 0.020 & 0 & 10 & 10 & \verb|((*(fi + i)))->other.last_line == yy_n_chars| \\
\bugometer{1.27875}{0.0714847}{0.914362}{0.0141533}{0} & 0.071 & 0.929 & 0.028 & 0 & 19 & 19 & \verb|bytes <= filesbase| \\
\bugometer{1.14613}{0.0751634}{0.913671}{0.0111656}{0} & 0.075 & 0.925 & 0.022 & 0 & 14 & 14 & \verb|((*(fi + i)))->other.first_line == 2| \\
\bugometer{1.07918}{0.0764302}{0.912429}{0.0111408}{0} & 0.076 & 0.924 & 0.022 & 0 & 12 & 12 & \verb|((*(fi + i)))->this.first_line < nid| \\
\bugometer{1}{0.077222}{0.911248}{0.01153}{0} & 0.077 & 0.923 & 0.023 & 0 & 10 & 10 & \verb|((*(fi + i)))->other.last_line == yy_init| \\
\multicolumn{8}{c}{\dotfill{} 2732 additional predictors follow \dotfill{}} \\

  \end{stats}

  \begin{stats}{harmonic mean}{harmonic}
    \bugometer{3.20003}{0.176135}{0.819603}{0.004262}{0} & 0.176 & 0.824 & 0.009 & 0 & 1585 & 1585 & \verb|files[filesindex].language > 16| \\
\bugometer{3.19976}{0.176195}{0.819543}{0.004262}{0} & 0.176 & 0.824 & 0.009 & 0 & 1584 & 1584 & \verb|strcmp > 0| \\
\bugometer{3.19866}{0.17595}{0.819792}{0.004258}{0} & 0.176 & 0.824 & 0.009 & 0 & 1580 & 1580 & \verb|strcmp == 0| \\
\bugometer{3.19783}{0.175935}{0.819808}{0.004257}{0} & 0.176 & 0.824 & 0.009 & 0 & 1577 & 1577 & \verb|files[filesindex].language == 17| \\
\bugometer{3.19756}{0.17599}{0.819751}{0.004259}{0} & 0.176 & 0.824 & 0.009 & 0 & 1576 & 1576 & \verb|tmp == 0 is TRUE| \\
\bugometer{3.19673}{0.175904}{0.819838}{0.004258}{0} & 0.176 & 0.824 & 0.009 & 0 & 1573 & 1573 & \verb|strcmp > 0| \\
\bugometer{2.8893}{0.115811}{0.877037}{0.005862}{0.00129} & 0.116 & 0.883 & 0.012 & 1 & 774 & 775 & \verb|((*(fi + i)))->this.last_line == 1| \\
\bugometer{2.89042}{0.116206}{0.876638}{0.005869}{0.001287} & 0.116 & 0.883 & 0.012 & 1 & 776 & 777 & \verb|((*(fi + i)))->other.last_line == yyleng| \\
\bugometer{3.10585}{0.110724}{0.818742}{0.013324}{0.05721} & 0.111 & 0.832 & 0.027 & 73 & 1203 & 1276 & \verb|config.match_comment is TRUE| \\
\bugometer{2.88649}{0.115628}{0.877204}{0.005869}{0.001299} & 0.116 & 0.883 & 0.012 & 1 & 769 & 770 & \verb|((*(fi + i)))->other.last_line == yy_start| \\
\bugometer{2.89042}{0.118305}{0.874501}{0.005907}{0.001287} & 0.118 & 0.880 & 0.012 & 1 & 776 & 777 & \verb|((*(fi + i)))->other.last_line < 2| \\
\bugometer{2.88818}{0.117597}{0.875205}{0.005904}{0.001294} & 0.118 & 0.881 & 0.012 & 1 & 772 & 773 & \verb|((*(fi + i)))->other.last_line == 1| \\
\bugometer{2.88762}{0.117597}{0.87521}{0.005898}{0.001295} & 0.118 & 0.881 & 0.012 & 1 & 771 & 772 & \verb|((*(fi + i)))->this.last_line == yy_start| \\
\bugometer{2.88649}{0.117501}{0.8753}{0.0059}{0.001299} & 0.118 & 0.881 & 0.012 & 1 & 769 & 770 & \verb|((*(fi + i)))->this.last_line < 2| \\
\bugometer{2.88874}{0.117651}{0.873345}{0.00642}{0.002584} & 0.118 & 0.880 & 0.013 & 2 & 772 & 774 & \verb|((*(fi + i)))->this.last_line == yyleng| \\
\bugometer{2.89597}{0.116671}{0.867635}{0.00807}{0.007624} & 0.117 & 0.876 & 0.016 & 6 & 781 & 787 & \verb|((*(fi + i)))->this.last_line == diff| \\
\multicolumn{8}{c}{\dotfill{} 2724 additional predictors follow \dotfill{}} \\

  \end{stats}

\end{table*}


\subsection{Ranking Predicates: Precision vs. Recall}
\label{sec:ranking}

We now turn to the question of ranking the predicates.
\autoref{tab:sorts} shows the top
predicates under different ranking schemes (explained below) for one of our
experiments.  We use
a simple {\em thermometer} to visualize the information for each
predicate.  The length of the thermometer shows the number of runs in
which the predicate was observed, plotted on a log scale (so small increases
in thermometer size indicate many more runs).  The thermometer has
a sequence of bands:
the black band on the left shows the context score;
the next lighter band shows the increase score, the next, even  lighter band (which is either not
visible or very small in all thermometers) shows the confidence
interval, and the white space at the right end of the thermometer
shows the successful runs in which the predicate was
observed to be true.  The tables show the thermometer as well as the
numbers for each of the quantities that make up the thermometer. 

The most important bug is the one that causes the greatest number
of failed runs.  This observation suggests:
\[ \importance(P) = F(P) \]
\autoref{tab:sorts-failure} shows the top predicates 
ranked by decreasing $F(P)$.\footnote{These predicates are ranked after predicates where $\increase(P) \leq 0$ are discarded.}
While the predicates in \autoref{tab:sorts-failure} are, as expected, involved
in many failing runs, these predicates are also highly 
non-deterministic, meaning they are also true in many successful runs
and are weakly correlated with bugs.  

Our experience with other
ranking strategies that emphasize the number of failed runs is also that
they select predicates involved in many failing, but also many
successful, runs.  The best of these predicates (the ones with high
$\increase()$ scores) are {\em super-bug predictors}:
predictors that include failures from more than one bug.  The
signature of super-bug predictors is that they account for a very
large number of failures (by combining the failures of multiple bugs)
but are also highly non-deterministic despite reasonably high $\increase()$
scores.

Another possibility is:
\[ \importance(P) = \increase(P) \]
\autoref{tab:sorts-increase} shows the top predicates ranked by decreasing
$\increase()$ score (note the thermometers in the different figures
are not drawn to the same scale).  These predicates do a much better
job of predicting failure; in fact, the program always fails when any
of these predicates is true. However, note that the number of failing
runs is very small.  These predicates are {\em sub-bug predictors}:
predictors for a subset of the failures caused by a bug.  Unlike
super-bug predictors, which are not useful in our experience, sub-bug
predictors that account for a significant fraction of the failures for
a bug often provide valuable clues, but still they represent special
cases and may not suggest other, more fundamental, causes of the bug.

Tables~\ref{tab:sorts-failure} and \ref{tab:sorts-increase} illustrate the difficulty of defining
``importance''.  It is helpful to consider this problem in the language of information
retrieval. We are looking for predicates with
high {\em recall} (meaning predicates that account for many failed runs) but also high {\em precision}
(meaning predicates that do not mispredict failure in many successful runs).  In statistics,
the corresponding terms are {\em selectivity} and {\em sensitivity}.  In both fields,
the standard metric is the harmonic mean of precision and recall, which disproportionately
rewards high scores in both dimensions.  In our
case, $\increase(P)$ measures precision and $F(P)$ measures recall.  Define
%%
\[ \importance(P) = \frac{2}{\frac{1}{\increase(P)} + \frac{log(\numfail)}{log(F(P))}} \]
%%
where $\numfail$ is total number of failing runs (independent of $P$)
and therefore scales recall to between zero and one.

The results are given in \autoref{tab:sorts-harmonic}.  We observe that
all of the predicates on this list indeed have both high precision and
recall, accurately describing a large number of failures. 


\subsection{Predicate Elimination}
\label{sec:elimination}

The remaining problem with the results in \autoref{tab:sorts-harmonic}
is that there is substantial redundancy; it is easy to see that several of these
predicates are related hiding other, distinct bugs that either have
fewer failed runs or more non-deterministic predictors further down the
list.  As discussed above, we use a simple recursive algorithm to eliminate
redundant predicates:
\begin{enumerate}

\item Rank predicates by $\importance()$.

\item Remove the top-ranked predicate $P$ and discard all runs $R$ (feedback reports) where  $R(P) = 1$.

\item Repeat (1) and (2) until the set of runs is empty or the set of predicates is empty.
\end{enumerate}




We can now state an easy, but important, property of this algorithm.  
\begin{lemma}
\rm
Let $P_1,\ldots,P_n$ be a set of instrumented predicates and let ${\cal B}_1,\ldots,{\cal B}_m$ be a set of bugs.  Let
\[ {\cal Z} = \bigcup_{1 \leq i \leq n} \{ R | R(P_i) = 1 \} \]
Then if for all $1 \leq j \leq m$ we have ${\cal B}_j \cap {\cal Z} \neq \emptyset$, then 
the algorithm chooses at least one predicate from the list $P_1,\ldots,P_n$ that predicts
at least one failure due to ${\cal B}_j$.
\end{lemma}

Thus, the elimination algorithm chooses at least one predicate 
predictive of each bug represented by the input set of predicates.
The other property we might like, that the algorithm chooses exactly
one predicate to represent each bug, does not hold; we shall see in
\autoref{sec:experiments} that the algorithm sometimes selects a
strong sub-bug predictor as well as a more natural predictor.  Besides
always representing each bug, the algorithm works well for two other
reasons.  First, two predicates are redundant if they predict the same
(or nearly the same) set of failing runs.  Thus, simply removing the
set of runs in which a predicate is true automatically dramatically
reduces the importance of any related predicates in the correct
proportions. Second, because elimination is
iterative, it is only necessary that $\importance()$ selects a good
predictor at each step, and not necessarily the best one; any
predicate that predicts a different set of failing runs than all
higher-ranked predicates is selected eventually.




\section{Experiments Revisited}
\label{sec:revisited}

\begin{table*}
\centering

\begin{tabular}{|l|r|r|r|r|r|r|r|r|r|}
\hline
Number of: & \multicolumn{2}{c|}{runs}  & sites  & 
\multicolumn{2}{c|}{branch predicates} & \multicolumn{2}{c|}{return 
predicates} & \multicolumn{2}{c|}{scalar predicates}\\ \cline{2-3} \cline{5-6} \cline{7-8} \cline{9-10}
           & successful & failing       &        & original & retained & 
original & retained & original & retained \\
\hline
\hline
ccrypt     &  3605      &  1033         &    570 &      0 &          0 &     
3420 &        6 &        0 &        0 \\
\hline
bc         &  3530      &   860         &  13442 &      0 &          0 &        
0 &        0 &    80652 &      156 \\
\hline
moss       & 28519      &  3352         &  35223 &   4170 &         33 &     
2964 &       11 &   195864 &     3322 \\
\hline
rhythmbox  & 21015      &  1873         & 145242 &   6836 &         14 &    
50574 &       21 &   800370 &      406 \\
\hline
\end{tabular}
\caption{Run, site, predicate, and retention counts for each of the experiments.}
\label{tab:exps}
\end{table*}



In this section we compare our new algorithm with our previous
approaches on the programs {\tt ccrypt} and {\tt bc}.
\Autoref{tab:exps} shows raw counts for all four experiments of the
successful and unsuccessful runs, number of instrumentation sites in
the program, the number of predicates for each
instrumentation scheme, and the number of predicates retained by the
test $\increase() > 0$.   Keep in mind that multiple predicates are
derived from each instrumentation site.  For example, each branches
site yields two predicates from two counters, while each scalar pairs
site yields six predicates from three counters (see \Autoref{sec:background}).

The sampling rate for all of our experiments is \nicefrac{1}{1};
i.e., we sampled every predicate every time it was reached.  An
offline downsampling program allows us generate sparser samples from this
full data; thus, we were able to use the same set of runs to test the
effect of different sampling rates on the results.  

We analyzed {\tt ccrypt} version 1.2, which has a known bug \cite{Selinger:2003:cqual}.  At one point in the code the program attempts to write a file.  If the file already exists, the user is presented with a prompt asking for confirmation that the file should be written; if the user input is EOF (i.e., the user hits {\tt Enter} without typing anything) the program will crash.  We used only the returns
instrumentation scheme for this experiment. The top
two entries of the report are (most information is elided for
brevity):
\begin{verbatim}
Predicate            Context    Increase   Failure
xreadline() == 0     .67        .33        1.00
file_exists() > 0    .35        .32         .67
\end{verbatim}

These are the same two predicates identified using a a simple process-of-elimination strategy in our previous work
\cite{PLDI`03*141}.  Our new algorithm, however, provides enough information to also see the causal chain that
leads to the crash: if the file exists, the probability of failure increases to .67, and if in addition the
read returns EOF, the probability of failure increases to 1.00.\footnote{Note that the context and increase scores do not always sum to the failure score due to roundoff error.}   A property of our algorithm is that 
whenever a causal sequence of instrumented predicates must be true for a bug to occur, each predicate along the
sequence shows a distinct step up in the probability of failure.

Note also that the report is very short; only 6 predicates out of more than 3,000 are retained by the test $\increase() > 0$.
Uniformly throughout our experiments the returns and branches reports are short enough that all predicates can
easily be examined by hand; we discuss the larger scalar-pairs reports below.

Our second experiment isolates a buffer overrun in version 1.06 of GNU's {\tt bc}.  In our
previous work, which used an algorithm based on logistic regression, all of the top
predictors showed that the variable {\tt indx} was bigger than some other value on the first line of the following
code fragment:
\begin{verbatim}
for(; indx < v_count; indx++)
      arrays[index] = NULL;
\end{verbatim}
The problem is indeed that {\tt indx} gets too big, resulting in an overrun of {\tt arrays}.  The bug
is that {\tt v\_count} is the wrong loop bound; it should be {\tt a\_count} instead.  

Using our new algorithm, the top 11 predicates (sorted by descending
$\increase()$ score) involve predicates within a few lines of the line
with the actual overrun.  However, these predicates also predict only
20-100 failing runs, a small fraction of the number of failing runs
for {\tt bc} in \Autoref{tab:exps}.  We have observed that this
phenomenon is fairly common: if there are enough failures caused by a
bug, some predicates will predict failure for small subsets of the failure set.

We have found it useful, especially when examining the larger
scalar-pairs reports, to be able to distinguish whether a
highly-ranked predicate largely captures an entire bug or just an aspect of a bug.  To
this end, we compute correlation coefficient matrices, showing the correlation coefficient between each pair of retained predicates; 
%% each retained predicate with every other retained predicate; 
intuitively, the
correlations show how often the predicates co-occur in the same failing run.  High correlations
indicate that the two predicates are very likely caused by the same bug.

In this experiment, a moment's work shows that the top-ranked predicates are highly
correlated with other predicates that are observed in many more of the
failing runs.\footnote{Our system hyperlinks each predicate to a
rank-ordered list of correlated predicates; it is literally the work
of a moment to discover this fact.}  Changing the rank order to list
the predicates with a large number of failing runs first, the first
group of predicates mostly involve {\tt indx} being larger than
something else on the first line of the code fragment given above.
Also in this cluster is the predicate {\tt v\_count > a\_count} (which is sampled at an assignment
to {\tt a\_count} a few lines before the loop), as
expected.

In summary, our new algorithm performed at least as well as either of
our previous proposals on both a deterministic and a non-deterministic
bug.     Furthermore, additional information, computed by our new
approach but not by our old ones, is also useful in understanding the
sources of bugs.










\section{A Controlled Experiment}
\label{sec:experiments:results}
%% -*- LaTeX -*-

\begin{table*}
\centering

\begin{tabular}{|l|r|r|r|r|r|r|r|r|r|}
\hline
Number of: & \multicolumn{2}{c|}{runs}  & sites  & 
\multicolumn{2}{c|}{branch predicates} & \multicolumn{2}{c|}{return 
predicates} & \multicolumn{2}{c|}{scalar predicates}\\ \cline{2-3} \cline{5-6} \cline{7-8} \cline{9-10}
           & successful & failing       &        & original & retained & 
original & retained & original & retained \\
\hline
\hline
ccrypt     &  3605      &  1033         &    570 &      0 &          0 &     
3420 &        6 &        0 &        0 \\
\hline
bc         &  3530      &   860         &  13442 &      0 &          0 &        
0 &        0 &    80652 &      156 \\
\hline
moss       & 28519      &  3352         &  35223 &   4170 &         33 &     
2964 &       11 &   195864 &     3322 \\
\hline
rhythmbox  & 21015      &  1873         & 145242 &   6836 &         14 &    
50574 &       21 &   800370 &      406 \\
\hline
\end{tabular}
\caption{Run, site, predicate, and retention counts for each of the experiments.}
\label{tab:exps}
\end{table*}



In this section we present the results of applying the algorithm
described in \autoref{sec:algorithm} in five case
studies.  \autoref{tab:exps} shows summary statistics for each of the
experiments.  In each study we ran the programs on about 32,000 random
inputs.  The number of instrumentation sites varies with the size of
the program, as does the number of predicates those instrumentation
sites yield.  Our algorithm is very effective in reducing the number
of predicates the user must examine.  For example, in the case of
\rhythmbox an initial set of 857,384 predicates is reduced to 537 by the $\increase(P) > 0$
test, a reduction of 99.9\%.  The elimination algorithm then yields 15 predicates, a further
reduction of 97\%.  The other case studies show a similar reduction in the number of
predicates by 3-4 orders of magnitude.

The results we discuss are all on sampled data.  Sampling creates
additional challenges that must be faced by our algorithm.  Assume $P_1$ and $P_2$ are
equivalent bug predictors and both are sampled at a rate of
$\nicefrac{1}{100}$ and both are reached once per run.  Then even though
$P_1$ and $P_2$ are equivalent, they will be observed in nearly disjoint
sets of runs and treated as close to independent by the elimination
algorithm.

To address this problem, we set the sampling rates of predicates to be
inversely proportional to their frequency of execution.  Based on a
training set of 1,000 executions, we set the sampling rate of each predicate so
as to obtain an expected 100 samples of each predicate in subsequent program
executions.  On the low end, the sampling rate is clamped to a minimum of $\nicefrac{1}{100}$; if the site is expected to be reached fewer than 100 times the sampling rate is set at 1.0.
Thus, rarely executed code has a
much higher sampling rate than very frequently executed code.  (A
similar strategy has been pursued for similar reasons in related work \cite{chil04}.)  We
have validated this approach by comparing the results for each
experiment with results obtained with no sampling at all (i.e., the
sampling rate of all predicates set to 100\%).  The results are
identical except for the \rhythmbox and \moss experiments, where we
judge the differences to be minor: sometimes a different but logically
equivalent predicate is chosen, the ranking of predictors of different
bugs is slightly different, or one or the other version has a few
extra, weak predictors at the tail end of the list.

\subsection{A Validation Experiment}

To validate our algorithm we first performed an experiment in which we
knew the set of bugs in advance.  We added nine bugs to \moss, a
widely used service for detecting plagiarism in software
\cite{Schleimer:2003:WLA}.  Six of these were previously discovered
and repaired bugs in \moss that we reintroduced.  The other three were
variations on three of the original bugs, to see if our algorithm could
discriminate between pairs of bugs with very similar behavior but
distinct causes.  The nature of the eight crashing bugs varies: four
buffer overruns, a null file pointer dereference in certain cases, a
missing end-of-list check in the traversal of a hash table bucket, a missing
out-of-memory check, and a violation of a subtle invariant that must be maintained between two
parts of a complex data structure.  In addition, some of these bugs
are non-deterministic any may not even crash when they should.

The ninth bug---incorrect handling of comments in some cases---only
causes incorrect output, not a crash.  We include this bug in our
experiment in order to show that bugs other than crashing bugs can 
also be isolated using our techniques, provided there is some 
way, whether by automatic self-checking or human inspection, to recognize
failing runs.  In particular, for our experiment we also ran a correct 
version of \moss{} and compared the output of the two versions. 
This oracle provides a labeling of runs as ``success'' or ``failure,'' 
and the resulting labels are treated identically by our program as
those based on program crashes.

\begin{table*}
\centering

\begin{tabular}{|l|r|r|r|r|r|r|r|r|r|}
\hline
Number of: & \multicolumn{2}{c|}{runs}  & sites  & 
\multicolumn{2}{c|}{branch predicates} & \multicolumn{2}{c|}{return 
predicates} & \multicolumn{2}{c|}{scalar predicates}\\ \cline{2-3} \cline{5-6} \cline{7-8} \cline{9-10}
           & successful & failing       &        & original & retained & 
original & retained & original & retained \\
\hline
\hline
ccrypt     &  3605      &  1033         &    570 &      0 &          0 &     
3420 &        6 &        0 &        0 \\
\hline
bc         &  3530      &   860         &  13442 &      0 &          0 &        
0 &        0 &    80652 &      156 \\
\hline
moss       & 28519      &  3352         &  35223 &   4170 &         33 &     
2964 &       11 &   195864 &     3322 \\
\hline
rhythmbox  & 21015      &  1873         & 145242 &   6836 &         14 &    
50574 &       21 &   800370 &      406 \\
\hline
\end{tabular}
\caption{Run, site, predicate, and retention counts for each of the experiments.}
\label{tab:exps}
\end{table*}



\autoref{tab:mossdilute} shows the results of the experiment.  The
predicates listed were selected by the elimination algorithm in the
order shown.  The first column is the initial bug thermometer for each
predicate, showing the \context{} and \increase{} scores before
elimination is performed. The fourth column is the \termdef{effective}
bug thermometer, showing the \context{} and \increase{} scores for a
predicate $P$ at the time $P$ is selected (i.e., when it is the
top-ranked predicate).  Thus the effective thermometer reflects the
cumulative diluting effect of redundancy elimination for all
predicates selected before this one.

As part of the experiment we separately recorded the exact set of
bugs that actually occurred in each run.
The columns at the far right of \autoref{tab:mossdilute} show, for
each selected predicate and for each bug, the number of failing runs in which
both the selected predicate is observed to be true and the bug occurs.
Note that while each
predicate has a very strong spike at one bug, indicating it is a
strong predictor of that bug, there are always some runs with other
bugs present.  For example, the top-ranked predicate, which is
overwhelmingly a predictor of bug \#5, also includes some runs where
bugs \#3, \#4, and \#9 occurred.  This situation is not the result of
misclassification of failing runs by our algorithm.  As observed in
\autoref{sec:introduction}, more than one bug may occur in a run.
It simply happens that in some runs bugs \#5 and \#3 both occur (to
pick just one possible combination).

A particularly interesting case of this phenomenon is bug \#7, one of
the buffer overruns.  Bug \#7 is not strongly predicted by any
predicate on the list but in fact occurs in at least a few of the
failing runs of most predicates.  We have examined the runs of bug \#7
in detail and found that the only failing runs involving bug \#7 also
trigger at least one other bug.  That is, even when the bug \#7 overrun
happens, it never causes incorrect output or a crash
in any run.  Bug \#8, another overrun, is not even shown because the
overrun is never triggered in our data (its column would be all
0's).\footnote{Bug \#8 was originally found by a code inspection.}
There is no way our algorithm can find causes of bugs that do not
occur, but recall that part of our purpose in sampling user executions
is to get an accurate picture of the most important bugs.  It is
consistent with this goal that if a bug never causes a problem, it is
not only not worth fixing, it is not even worth reporting.

The other bugs all have strong predictors on the list.  In fact,
the top eight predicates have exactly one predictor for each of the seven
bugs that occur, with the exception of bug \#1, which has one very
strong sub-bug predictor in the second spot and another predictor
in the sixth position.  Notice that even the rarest bug, bug \#2,
which occurs more than an order of magnitude less frequently than
the most common bug, is identified immediately after the last of
the other bugs.\footnote{The peculiar eighth predicate, \texttt{f < f},
says that after an assignment the new value of \texttt{f} is less than
the old value of \texttt{f}.}  Furthermore, we have verified by hand that
the selected predicates would, in our judgment, lead an engineer to
the cause of the bug. Overall, the elimination algorithm does an excellent
job of listing separate causes of each of the bugs in order of priority,
with very little redundancy.

Below the eighth position there are no new bugs to report and every
predicate is correlated with predicates higher on the list.  Even
without the columns of numbers at the right it is easy to spot the
eighth position as the natural cutoff.  Keep in mind that the length
of the thermometer is on a log scale, hence changes in larger
magnitudes may appear less evident.  Notice that the initial and
effective thermometers for the first eight predicates are essentially
identical.  Only the predicate at position six is noticeably
different, indicating that this predicate is somewhat affected by a
predicate listed earlier (specifically, its companion sub-bug
predictor at position two).  However, all of the predicates below the
eighth line have very different initial and effective thermometers
(either many fewer failing runs, or much more non-deterministic, or
both) showing that these predicates are strongly affected by
higher-ranked predicates.

The visualizations presented thus far have a drawback illustrated by
the \moss\ experiment: It is not easy to identify the predicates to
which a predicate is closely related.  Such a feature would be useful
in confirming whether two selected predicates represent different bugs
or are in fact related to the same bug.  We do have a measure of how
strongly $P$ implies another predicate $P'$: How does removing the
runs where $\report{P} = 1$ affect the importance of $P'$?  The more
closely related $P$ and $P'$ are, the more $P'$'s importance drops
when $P$'s failing runs are removed.  In the interactive version of
our analysis tools, each predicate $P$ in the final, ranked list of
links to an \termdef{affinity list} of all
predicates ranked by how much $P$ causes their ranking score to
decrease.

\subsection{Additional Experiments}

We briefly report here on experiments with additional applications
containing both known and unknown bugs.  Complete analysis results for
all experiments may be browsed interactively at
\url{http://www.cs.berkeley.edu/~liblit/pldi-2005}.

\subsubsection{\ccrypt}

\view{\ccrypt}{ccrypt}

We analyzed \ccrypt 1.2, which has a known input validation bug.  The
results are shown in \autoref{tab:views-ccrypt}.  Our algorithm
reports two predictors, both of which point directly to the single bug.
It is easy to discover that the two predictors are for the same bug;
the first predicate is listed first in the second predicate's affinity
list, indicating the first predicate is a sub-bug predictor associated
with the second predicate.

\subsubsection{\bc}

\view{\bc}{bc}

GNU \bc 1.06 has a previously reported buffer overrun.  Our results
are shown in \autoref{tab:views-bc}.  The outcome is the same as for
\ccrypt: two predicates are retained by elimination, and the second
predicate lists the first predicate at the top of its affinity list,
indicating that the first predicate is a sub-bug predictor of the second.
Both predicates point to the cause of the overrun.  This bug causes a
crash long after the overrun occurs and there is no useful information
on the stack at the point of the crash to assist in isolating this
bug.

\subsubsection{\exif}

\view{\exif}{exif}

\autoref{tab:views-exif} shows results for \exif 0.6.9, an open source
image processing program.  Each of the three predicates is a predictor
of a distinct and previously unknown crashing bug.  It took less than
20 minutes of work to find and verify the cause of each of the bugs
using these predicates and the additional highly correlated predicates
on their affinity lists.

To illustrate how statistical debugging is used in practice, we
use the last of these three failure predictors as an example, and
describe how it enabled us to
effectively isolate the cause of one of the bugs.  Failed runs
exhibiting \texttt{o + s > buf\_size} show the following unique stack
trace at the point of termination:
\begin{quote}
  \small
\begin{verbatim}
main
  exif_data_save_data
    exif_data_save_data_content
      exif_data_save_data_content
        exif_data_save_data_entry
          exif_mnote_data_save
            exif_mnote_data_canon_save
              memcpy
\end{verbatim}
\end{quote}
The code in the vicinity of this crash site is as follows:
\begin{quote}
\begin{verbatim}
// snippet of exif_mnote_data_canon_save
for (i = 0; i < n->count; i++) {
    ...
    memcpy(*buf + doff,             (c)
           n->entries[i].data, s);
    ...
}
\end{verbatim}
\end{quote}
This stack trace alone provides little insight into the cause of the
bug.  However, our algorithm highlights \texttt{o + s > buf\_size} in
function \texttt{exif\_mnote\_data\_canon\_load} as a strong bug
predictor.  Thus, a quick inspection of the source code leads us to
construct the following call sequence:
\begin{quote}
  \small
\begin{verbatim}
main
  exif_loader_get_data
    exif_data_load_data
      exif_mnote_data_canon_load
  exif_data_save_data
    exif_data_save_data_content
      exif_data_save_data_content
        exif_data_save_data_entry
          exif_mnote_data_save
            exif_mnote_data_canon_save
              memcpy
\end{verbatim}
\end{quote}
The code in the vicinity of the predicate \texttt{o + s > buf\_size} is as follows:
\begin{quote}
\begin{verbatim}
// snippet of exif_mnote_data_canon_load
for (i = 0; i < c; i++) {
    ...
    n->count = i + 1;
    ...
    if (o + s > buf_size) return;    (a)
    ...
    n->entries[i].data = malloc(s);  (b)
    ...
}
\end{verbatim}
\end{quote}
It is apparent from the above code snippets and the
call sequence that whenever the predicate \texttt{o + s > buf\_size} is true,
%%
\begin{itemize}
\item the function \texttt{exif\_mnote\_data\_canon\_load} returns on
  line \texttt{(a)}, thereby skipping the call to \texttt{malloc} on
  line \texttt{(b)} and thus leaving \texttt{n->entries[i]->data}
  uninitialized for some value of \texttt{i}, and

\item the function \texttt{exif\_mnote\_data\_canon\_save} passes the
  uninitialized \texttt{n->entries[i]->data} to \texttt{memcpy} on line \texttt{(c)}, which reads it and eventually crashes.
\end{itemize}

In summary, our algorithm enabled us to effectively isolate the causes
of several previously unknown bugs in source code unfamiliar to us in
a small amount of time and without any explicit specification beyond
``the program shouldn't crash.''

\subsubsection{\rhythmbox}

\begingroup
\setlength{\segunit}{10pt}
\view[\tiny]{\rhythmbox}{rhythmbox}
\endgroup

\autoref{tab:views-rhythmbox} shows our results for \rhythmbox 0.6.5,
an interactive, graphical, open source music player.  \rhythmbox is a
complex, multi-threaded, event-driven system, written using a library
providing object-oriented primitives in C.  Event-driven systems use
event queues; each event performs some computation and possibly adds
more events to some queues.  We know of no static analysis today that
can analyze event-driven systems accurately, because no static
analysis is currently capable of analyzing the heap-allocated event
queues with sufficient precision.  Stack inspection is also of
limited utility in analyzing event-driven systems, as the stack in the
main event loop is unchanging and all of the interesting state is in
the queues.

We isolated two distinct bugs in \rhythmbox.  The first predicate led
us to the discovery of a race condition.  The second predicate was not
useful directly, but we were able to isolate the bug using the
predicates in its affinity list.  This second bug revealed what turned
out to be a very common incorrect pattern of accessing the underlying
object library (recall \autoref{sec:introduction}).  \rhythmbox
developers confirmed the bugs and enthusiastically applied patches
within a few days, in part because we could quantify the bugs as
important crashing bugs.  It required several hours to isolate each of
the two bugs (and there are additional bugs represented in the
predictors that we did not isolate) in part because \rhythmbox is
complex and in part because the bugs were violations of subtle heap
invariants which are not directly captured by our current
instrumentation schemes.  Note, however, that we could not have even
begun to understand these bugs without the information provided by our
tool.  We intend to explore schemes that track predicates on heap
structure in future work.

\subsection{Comparison with Logistic Regression}
\label{sec:comparison}

\begin{table}
\nocaptionrule
\caption{Results of logistic regression for \moss}
\label{tab:logregression}
\centering
\small
\begin{tabular}{ll}
  \toprule
  Coefficient & Predicate \\
  \midrule
  0.769379 & \verb|(p + passage_index)->last_line < 4| \\
  0.686149 & \verb|(p + passage_index)->first_line < i| \\
  0.675982 & \verb|i > 20| \\
  0.671991 & \verb|i > 26| \\
  0.619479 & \verb|(p + passage_index)->last_line < i| \\
  0.600712 & \verb|i > 23| \\
  0.591044 & \verb|(p + passage_index)->last_line == next| \\
  0.567753 & \verb|i > 22| \\
  0.544829 & \verb|i > 25| \\
  0.536122 & \verb|i > 28| \\
  \bottomrule
\end{tabular}
\end{table}

In earlier work
we used $\ell_1$-regularized logistic regression
to rank the predicates by their
failure-prediction strength \cite{PLDI`03*141,NIPS2003_AP05}.
Logistic regression uses linearly weighted
combinations of predicates to classify a trial run as successful or
failed.  Regularized logistic regression incorporates a penalty
forcing most coefficients to be set to zero, thereby
selecting only the most important predicates.  The output is a set of
coefficients for predicates giving the best overall prediction.

A weakness of logistic regression for our application is that it seeks
to cover the set of failing runs without regard to the orthogonality
of the selected predicates (i.e., whether they represent distinct
bugs).  This problem can be seen in \autoref{tab:logregression},
which gives the top ten predicates selected by logistic regression
for \moss.  The striking fact is that all selected predicates are
either sub-bug or super-bug predictors.  The predicates beginning with
\texttt{p + \ldots} are all sub-bug predictors of bug \#1 (see
\autoref{tab:mossdilute}).  The predicates \texttt{i > \ldots} are
super-bug predictors: \texttt{i} is the length of the command line and
the predicates say program crashes are more likely for long command
lines (recall \autoref{sec:introduction}).

The prevalence of super-bug predictors on the list shows the
difficulty of making use of the penalty term.  Limiting the number of
predicates that can be selected via a penalty has the effect of
encouraging regularized logistic regression to choose super-bug predictors, as
these cover more failing runs at the expense of poorer predictive
power compared to predictors of individual bugs.  On the other hand,
the sub-bug predictors are chosen based on their excellent prediction
power of those small subsets of failed runs.
%%Relaxing the penalty
%%allows logistic regression to add more predicates to improve its
%%prediction, but the sub-bug predictors apparently are favored.

%% LocalWords:  exps mossdilute ccrypt bc exif buf mnote rhythmbox
%% LocalWords:  logregression



\section{Rhythmbox}
\label{sec:rb}


For our final case study we instrumented \rhythmbox, an Open Source
jukebox application.  Thus far we have isolated two serious crashing
bugs in \rhythmbox, and fixes for both bugs have been accepted and
applied to the main branch of the \rhythmbox\ source tree.  We briefly
describe the two bugs and the lessons we have learned from working on
a large application with unknown bugs.

The first bug, discssed briefly in Section~\ref{sec:introduction}, is
predicted to occur when a particular call to {\tt g\_source\_remove}
returns a positive value.  This function is part of the GTK, a graphical
object toolbox for C.
%%an object system for C.  
%%(?? "an object system for C" doesn't make sense.  -- Alice)
Objects in GTK can be referred to both
via pointers and via object IDs; {\tt g\_source\_remove}
takes an object ID as its argument.

In this case the problem is that the object ID passed as an argument
is stale---the object it belongs to was destroyed much earlier in the
prorgram, but the object ID was retained in the owning object's fields.
A section of clean-up code much later in the program dutifully calls {\tt g\_source\_remove} on the
stale ID; if the ID has been reallocated, the unfortunate object that
has that ID is then destroyed, wreaking havoc in the rest of the execution.

A strong predictor for the second bug is 
\begin{verbatim}
monkey_media_player_get_uri() == NULL 
\end{verbatim} 
This function
returns the current play source (e.g., a CD library or internet radio
station) for \rhythmbox.  The interesting thing is that this source
should never be {\tt NULL} while \rhythmbox\ is still playing.  This
observation leads to the bug.  GTK is reference-counted, and the
programmer has freed a complex data structure by removing references
and nulling fields (thus the predicate above) in the belief that the objects are freed. But there
is a race condition: one object sometimes survives long enough to
invoke callbacks into the other objects that have been reclaimed,
resulting in a crash.

These bugs were the most complex that we dealt with in 
our experiments (although one or two of the \moss\ bugs are close).  The
call stack at the point of crash for both bugs is unhelpful (see Section~\ref{sec:conclusions}).  We feel confident in saying that without the use of
our tool we could not have isolated these bugs, and we believe
 the fact that these crashing bugs have gone undiscovered in \rhythmbox\ is
good evidence that they are not easy to find without tool support.

On the other hand, it should also be clear from our descriptions of the bugs
that our techniques did not point us as directly to the bugs as one might hope.
Working only from the output of our system and having no prior knowledge of
the rather involved architecture of \rhythmbox, it took us considerable time
to understand these two bugs.  

The lesson is that while our algorithm is theoretically sound and (in
our opinion) performs well in practice, it is only as good as the
predicates chosen for instrumentation.  Our current instrumentation
strategies focus on low-level properties of the C code, while \rhythmbox\
is really written in a higher-level language, namely at the level of
the GTK API.  We have already identified a new instrumentation strategy
that makes use of the semantics of the GTK and would have led us
much more quickly to the second bug; such a strategy, while GTK-specific,
would still be widely applicable, as GTK is widely used.  We expect
that such system- or API-specific instrumentation strategies would be
needed in many cases to obtain maximum benefit from our techniques.


\section{Related Work}
\label{sec:related-work}

In this section we briefly survey related work. There is currently a great
deal of interest in applying static analysis to improve software quality.  Some,
but certainly not all, of the bugs we found in our studies could also have
been detected by known static analyses.  While we firmly believe in
the use of static analysis to find and prevent bugs, our dynamic approach has
advantages. A dynamic analysis can observe actual run-time values, which is often
better than either making a very conservative static assumption about runtime
values to be sound, or allowing some even very simple bugs to escape undetected.
Another advantage of dynamic analysis, especially one that uses actual user
executions for its data, is the ability to assign an accurate importance to each
bug.
  
The Daikon project \cite{ernst2001} monitors instrumented applications
to discover likely program invariants.  It collects extensive trace
information at run time and uses this offline to accept or reject any
of a wide variety of guessed candidate predicates.  The DIDUCE project
\cite{ICSE02*291} tests a more restricted set of predicates within the
client program, and attempts to relate state changes in candidate
predicates to manifestation of bugs.  Both projects assume complete
monitoring, such as within a controlled test environment.  Our goal is
to use lightweight partial monitoring, suitable for deployment to end
users.  We never have complete information, and therefore must use a
more statistical approach.

\termdef{Software tomography} as realized through the GAMMA system
\cite{PASTE'02*2,Orso:2003:LFDIART} shares our goal of low-overhead
distributed monitoring of deployed code.  GAMMA collects code coverage
data to support a variety of code evolution tasks.  Our
instrumentation exposes a broader family of data- and
control-dependent predicates on program behavior and uses randomized
sparse sampling to control overhead.  Our
predicates, however, give coverage information: the sum of all predicate counters at a site converges to the relative coverage of that site.

Efforts to directly apply statistical modeling principles to debugging
have met with mixed results.  Early work in this area by Burnell and
Horvitz \cite{Burnell:1995:SCM} uses program slicing in conjunction
with Bayesian belief networks to filter and rank the possible causes
for a given bug.  Empirical evaluation shows that the slicing component
alone finds 65\% of bug causes, while the probabilistic model
correctly identifies another 10\%.  This additional payoff may seem
small in light of the effort, measured in multiple
man-years, required to distill experts' often tacit knowledge into a
formal belief network.  However, the approach does illustrate one
strategy for integrating information about program structure into the
statistical modeling process.

In more recent work, Podgurski et al.\ \cite{ICSE`03*465} apply
statistical feature selection, clustering, and multivariate
visualization techniques to the task of classifying software failure
reports.  The intent is to bucket each report into an equivalence
group believed to share the same underlying cause.  Features are
derived offline from fine-grained execution traces without sampling;
this reduces the noise level of the data but greatly restricts the
instrumentation schemes that are practical to deploy outside of a
controlled testing environment.  As in our own earlier work, Podgurski
uses logistic regression to select features which are highly
predictive of failure.  
Clustering tends to identify small, tight groups of runs which do
share a single cause but which are not always maximal.  That is, one
cause may be split across several clusters.

In contrast, current
industrial practice uses stack traces to cluster failure reports into
equivalence classes.  Two crash reports showing the same stack trace,
or perhaps only the same top-of-stack function, are presumed to be two
reports of the same failure.  This works to the extent that a single
cause corresponds to a single point of failure, but our experience
with \moss\ and \rhythmbox\ suggests that this assumption may not often hold.  In \moss\ find
that only bugs 2 and 5 have truly unique ``signature'' stacks: a
crash location which is present if and only if the corresponding bug
was actually triggered.  These bugs are also our most deterministic.
Bugs 4 and 6 also have nearly unique stack signatures.
The remaining bugs are much less consistent: each stack signature is
observed after a variety of different bugs, and each triggered bug
causes failure in a variety of different stack states.  In \rhythmbox,
both bugs caused crashes so long after the bad behavior that the crash stacks
were not useful at all.

Studies that attempt real-world deployment of monitored software must
address a host of practical engineering concerns, from distribution to
installation to user support to data collection and warehousing.
Elbaum and Hardojo \cite{Elbaum:2003:DISATA} have reported on a
limited deployment of instrumented Pine binaries.  Their experiences
have helped to guide our own design of a wide public deployment of
applications with sampled instrumentation, presently underway
\cite{Liblit:2003:CBIP}.

For some highly available systems, even a single failure must be
avoided.  Once the behaviors that predict imminent failure are known,
automatic corrective measures may be able to prevent the failure from
occurring at all.  The Software Dependability Framework (SDF)
\cite{Gross:2003:PSMUST} uses the multivariate state estimation
technique to model and thereby predict impending system failures.
Instrumentation is assumed to be complete and is typically
domain-specific.

\section{Conclusions and Future Work}
\label{sec:conclusions}

We have demonstrated a practical algorithm for isolating multiple bugs
in complex software systems.  Given feedback profiles of enough runs,
overall trends emerge which can help to guide an engineer to the most
likely causes of the most common bugs.  Our approach uses lightweight,
sampled instrumentation suitable for wide scale deployment to real end
users, which means that the system also performs implicit triage: it
learns the most, most quickly, about the bugs that happen most often.
The key property of our approach is that it filters potential causes
based on the degree to which they increase the likelihood of failure.
Our algorithm appears to do a good job of isolating a wide variety of
bugs, even when multiple bugs are present simultaneously.

When hunting for bugs, the first thing an engineer wants to know is
under what circumstances the failure occurs.  Perhaps the most salient
feature of our approach is the ability to pinpoint the circumstances
under which bugs occur.

We also see several possible avenues for improvement, which we leave
as future work:
\begin{itemize}

\item The scalar-pairs instrumentation scheme induces many
predicates and finds fewer bugs than the branches and
returns instrumentation schemes.  A more specialized version of scalar-pairs
would be useful.

\item To date, we have sampled all sites at the same rate.  However,
rarely executed code, such as code that executes once on system start-up,
can be sampled at a higher rate with no impact on overall performance.
Moreover, infrequently executed code is more likely
to harbor bugs.  By observing rare events more often, we would need fewer
runs to isolate bugs.

\item The algorithm we present here analyzes every predicate independently
of every other predicate.  We believe we can do more to exploit
statistical correlations between predicates.


\end{itemize}


\bibliography{cacm1990,icse02,icse03,misc,paste02,pldi03,pods,ramss,refs}

\end{document}

%% LocalWords:  DIDUCE Burnell Horvitz Podgurski Elbaum Hardojo SDF
%% LocalWords:  topcrash cacm icse ramss pldi Podgurski's Kanduri
%% LocalWords:  McMaster Umranov Votta
