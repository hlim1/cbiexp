%% -*- TeX-master: t -*-

%% \documentclass[draft]{acm_proc_article-sp}
\documentclass[draft]{sig-alternate}


%%%%%%%%%%%%%%%%%%%%%%%%%%%%%%%%%%%%%%%%%%%%%%%%%%%%%%%%%%%%%%%%%%%%%%%%
%%
%% standard texmf packages
%%

\usepackage{amsmath}
\usepackage{booktabs}
\usepackage{graphicx}
\usepackage{ifthen}
\usepackage{nicefrac}
\usepackage{fancyvrb}
\usepackage[scrtime]{prelim2e}
\usepackage[TABTOPCAP]{subfigure}
\usepackage{xspace}

\usepackage{mathptmx}
\usepackage{times}
\DeclareMathAlphabet{\mathcal}{OMS}{cmsy}{m}{n}

\usepackage[
bookmarks,
breaklinks,
draft=false,
pdftitle={Scalable Statistical Debugging},
pdfauthor={Ben Liblit, Mayur Naik, Alice X. Zheng, Alex Aiken, and Michael I.  Jordan},
pdfsubject={D.2.4 [Software Engineering]: Software/Program Verification -- statistical methods; D.2.5 [Software Engineering]: Testing and Debugging -- debugging aids, distributed debugging, monitors, tracing; I.5.2 [Pattern Recognition]: Design Methodology -- feature evaluation and selection},
pdfkeywords={bug isolation, random sampling, invariants, feature selection, statistical debugging},
pdfpagemode=UseOutlines
]{hyperref}

\ifthenelse{\isundefined{\pdfoutput}}{}{\usepackage{thumbpdf}}

%% \VerbatimFootnotes


%%%%%%%%%%%%%%%%%%%%%%%%%%%%%%%%%%%%%%%%%%%%%%%%%%%%%%%%%%%%%%%%%%%%%%%%
%%
%% unique to this paper
%%

\usepackage{views/bug-o-meter}
\usepackage{views/view}

\renewcommand{\sectionautorefname}[0]{Section}
\renewcommand{\subsectionautorefname}[0]{\sectionautorefname}
\newcommand{\subtableautorefname}[0]{\tableautorefname}

\newtheorem{theorem}{Theorem}[section]
\newtheorem{lemma}[theorem]{Lemma}

\newcommand{\moss}{\textsc{Moss}\xspace}
\newcommand{\rhythmbox}{\textsc{Rhythmbox}\xspace}
\newcommand{\exif}{\textsc{exif}\xspace}
\newcommand{\bc}{\textsc{bc}\xspace}
\newcommand{\ccrypt}{\textsc{ccrypt}\xspace}
\newcommand{\termdef}[1]{\emph{#1}}
\newcommand{\prob}{\ensuremath{\textit{Pr}}}
\newcommand{\fail}{\ensuremath{\textit{Crash}}}
\newcommand{\crash}{\ensuremath{\textit{Failure}}}
\newcommand{\context}{\ensuremath{\textit{Context}}}
\newcommand{\increase}{\ensuremath{\textit{Increase}}}
\newcommand{\importance}{\ensuremath{\textit{Importance}}}
\newcommand{\numfail}{\ensuremath{\textit{NumF}}}
\renewcommand{\H}{{\mathcal{H}}}

\newcommand{\issue}[2][]{}


%%%%%%%%%%%%%%%%%%%%%%%%%%%%%%%%%%%%%%%%%%%%%%%%%%%%%%%%%%%%%%%%%%%%%%%%
%%
%% front matter
%%



\title{Scalable Statistical Debugging
  \renewcommand{\footnotemark}[0]{}
  \thanks{This research was supported in part by NASA Grant No.\
    NAG2-1210; NSF Grant Nos.\ EIA-9802069, CCR-0085949, ACI-9619020,
    and IIS-9988642; DOE Prime Contract No.\ W-7405-ENG-48 through
    Memorandum Agreement No.\ B504962 with LLNL; and DARPA ARO-MURI 
    ACCLIMATE DAAD-19-02-1-0383.  The information
    presented here does not necessarily reflect the position or the
    policy of the Government and no official endorsement should be
    inferred.}
}

\makeatletter
\newcommand*{\eecsMark}[0]{\@fnsymbol{1}}
\newcommand*{\statMark}[0]{\@fnsymbol{2}}
\newcommand*{\stanMark}[0]{\@fnsymbol{3}}
\makeatother
\newcommand*{\eecs}[0]{\textsuperscript{\eecsMark}}
\newcommand*{\stat}[0]{\textsuperscript{\statMark}}
\newcommand*{\both}[0]{\textsuperscript{\eecsMark, \statMark}}
\newcommand*{\stan}[0]{\textsuperscript{\stanMark}}

\newcommand{\moreauthors}[0]{\end{tabular}\\\begin{tabular}[t]{c}}

\author{%
  Ben Liblit \eecs
  \and Mayur Naik \stan
  \and Alice X.\ Zheng \eecs
  \and Alex Aiken \stan
  \and Michael I.\ Jordan \both
  \moreauthors
  \eecs Department of Electrical \\
  Engineering and Computer Science \\
  \stat Department of Statistics \\
  University of California, Berkeley \\
  Berkeley, CA 94720-1776
  \and
  \stan Computer Science Department \\
  Stanford University \\
  Stanford CA 94305-9025
}

\bibliographystyle{abbrv}


%%%%%%%%%%%%%%%%%%%%%%%%%%%%%%%%%%%%%%%%%%%%%%%%%%%%%%%%%%%%%%%%%%%%%%%%
%%
%%  document body
%%


\begin{document}

\issue[Alice]{Overall comment: font too small?}
\issue[Mike]{ I think that Section 5.1 would be better placed at the
end of Section 4.}

\maketitle

\begin{abstract}
  We present a statistical debugging algorithm that is
  robust in the face of multiple undiagnosed bugs.  The algorithm
  identifies program behaviors that significantly increase the
  likelihood of failure; these predictors reveal both the
  circumstances under which bugs occur as well as the frequencies of
  failure modes, making it easier to prioritize debugging efforts.
  Our algorithm is validated using several case studies, including finding
  previously unknown and significant crashing bugs in widely used systems.
  We compare our technique with an earlier algorithm and find it much more accurate 
  and scalable.
\end{abstract}


%\category{D.2.4}{Software Engineering}{Software/Program
%  Verification}[statistical methods]
%%
%\category{D.2.5}{Software Engineering}{Testing and
%  Debugging}[debugging aids, distributed debugging, monitors, tracing]
%%
%\category{I.5.2}{Pattern Recognition}{Design Methodology}[feature
%  evaluation and selection]

%\terms{Experimentation, Reliability}

%\keywords{bug isolation, random sampling, invariants, feature
%  selection, statistical debugging}


\section{Introduction}
\label{sec:introduction}

\issue[Alice]{Too much notation in introduction?}

This paper is about \termdef{statistical debugging}, a dynamic
analysis for identifying the causes of software failures (i.e., bugs).
Instrumented programs monitor their own behavior and send feedback
reports to a central collection server.  The instrumentation examines
program behavior during execution by sampling; complete information is
never available about any single run.  However, monitoring is also
lightweight and therefore practical to deploy to large user
communities, making it possible to gather information about many runs.
The collected data can then be analyzed for interesting trends across
all of the monitored executions.

In our approach, instrumentation consists of {\em predicates} tested
at particular program points; we defer discussing which predicates are
chosen for instrumentation to \autoref{sec:background}.  A given
program point may have many predicates that are sampled independently
during program execution when that program point is reached (i.e.,
each predicate associated with a program point may or may not be
tested each time the program point is reached).  A feedback report $R$
consists of one bit indicating whether the {\em run} of the program
succeeded or failed, as well as a bit vector with one bit for each
predicate $P$; if $P$ is observed to be true at least once during run
$R$ then $R(P) = 1$, otherwise $R(P) = 0$.\footnote{In reality, we count
the number of times $P$ is observed to be true, but the analysis of
the feedback reports only uses whether $P$ is observed to be true at
least once.}

\issue[Alice]{Comment about notation: the f(x) usually denotes some
  kind of function, do we want to treat R (a run) as a function?  I'd
  much prefer the subscript or superscript notation, which usually
  refers to an element of a vector.  How about R_P instead of R(P)?
  [I didn't make this change in the text]}
    
\issue{modified by Alice}

The usual definition of a bug refers to a programmatic element that 
unintentionally causes incorrect behavior.  We're going to abuse notation a
bit and overload its definition.  In this paper, a {\em bug} also refers to
a set of failing runs (feedback reports) ${\cal B}$
that share a cause of failure.  The meaning becomes clear in context.  The 
union of all bugs is exactly the
set of failing runs, but note that ${\cal B}_i \cap {\cal B}_j \neq
\emptyset$ in general; more than one bug can occur in some runs.

\issue[Alice]{The definition of "bug" overloads the usual definition
  of "something that causes incorrect behavior."  I understand that
  this makes the super-bug and sub-bug terminology easier to justify
  (weren't we going to use some other names for these?).  But if we're
  going to veer from the usual definition, then at least we should
  make this clear.  [text modified]}

\issue[Mike]{"A bug is a set of failing runs".  Semantically one thinks 
of a bug as someone that characterizes a program, not a set of runs.
What about calling a set of a failing runs a "bug profile"?}

A predicate $P$ is a {\em bug predictor} (or simply a {\em predictor})
of bug ${\cal B}$ if whenever $R(P) = 1$ then it is statistically
likely that $R \in {\cal B}$ (see \autoref{sec:increase}).  {\em
Statistical debugging} selects a small subset ${\cal S}$ of the set of
all instrumented predicates ${\cal P}$ such that ${\cal S}$ has
predictors of all bugs.  We also rank the predictors in ${\cal S}$
from the most to least important.  The set ${\cal S}$ and associated
metrics (see \autoref{sec:experiments}) are then available to
engineers to help speed the process of finding and fixing the most
serious bugs.

In previous work, we focused on techniques
for lightweight instrumentation and sampling of program executions, but
we also studied two preliminary algorithms for statistical debugging and
presented experimental results on medium-size applications with a
single bug \cite{PLDI`03*141,NIPS2003_AP05}.  The most general
technique we studied is {\em regularized logistic regression}, a standard
statistical procedure that tries to select a set of predicates that
best predict the outcome of every run. As we worked to apply these
methods to much larger programs under realistic conditions, we
discovered a number of serious scalability problems:

\begin{itemize}

\item For large applications the set ${\cal P}$ numbers in the hundreds of
thousands of predicates, many of which are, or are very nearly,
logically redundant.  In our experience, this redundancy in ${\cal P}$
causes logistic regression to choose highly redundant lists of
predictors ${\cal S}$.  Redundancy is
already evident in prior work \cite{PLDI`03*141} but becomes much
worse for larger programs.
%%
\issue[Alice]{Logistic regression is not a global optimization
  technique.  The fact that predictors for less common bugs are ranked
  lower by logistic regression is a consequence of doing binary
  classification when we really should be doing multi-class
  classification.  But we can't do multi-class classification since we
  don't know the underlying bug labels. [text modified]}

\item A separate difficulty is the prevalence of predicates predicting
multiple bugs.  For example, for many Unix programs a bug is more
likely to be encountered when many command line flags are given,
because the more options that are given non-default settings the more
likely unusual code paths are to be exercised.  Thus, predicates
implying a long command line may rank near the top, even though such
predicates are useless for isolating the cause of individual bugs.

\item Finally, different bugs occur at rates that differ by orders of
magnitude.  In reality, we do not know which crash is caused by which bug, 
hence we are forced to lump all the bugs together and try to learn a binary
classifier.  Thus, predictors for all but the most common bugs have relatively little
influence over the global optimum and tend to be ranked low or not included 
in ${\cal S}$ at all.

\end{itemize}

These problems with logistic regression persist across many variations
we have investigated. From this experimental work we did garner some
key technical insights.  In addition to the bug predictors we wish to
find among the instrumented predicates, there are several other kinds
of predicates.  First, nearly all predicates (often 98\% or 99\%) are
not predictive of anything.  These \termdef{non-predictors} are best identified and discarded as
quickly as possible. Among the remaining predicates that can
predict failure in some way, there are some bug predictors.
There are also \termdef{super-bug predictors}, predicates that, as
described above, predict failures due to a variety of bugs.  And there
are \termdef{sub-bug predictors}, predicates that characterize a subset of
the instances of a specific bug; these are often special cases of more
general problems.  We give the concepts of super- and sub-bug
predictors more precise technical treatment in
\autoref{sec:ranking}.

The difficulty in identifying the best bug predictors lies in not being
misled by the sub- or super-bug predictors and not being overwhelmed
by the sheer number of predicates to sift through.
This paper makes a number of contributions on these problems:

\begin{itemize}

\item We present a new algorithm for isolating multiple bugs in
complex applications (\autoref{sec:algorithm}) 
that offers significant improvements over previous work.
It scales much more gracefully in all the dimensions discussed above
and for each selected predicate $P$ it naturally yields information
that shows both how important (in number of explained program
failures) and how accurate a predictor $P$ is.

\item We validate the algorithm by a variety of experiments.  We show
improved results for previously reported experiments
\cite{PLDI`03*141}.  In a
controlled experiment we show that the algorithm is able to find a
number of known bugs in a complex application.  Lastly, we use
the algorithm to discover previously unknown serious crashing bugs in
two large and widely used open source applications.

\item We show that relatively few runs (we used 32,000) are sufficient
to isolate all of the bugs described in this paper, showing that our
approach is feasible for in-house automatic testing as well as use
post-deployment.

\item We report on the effectiveness of the current industry practice
of collecting stack traces from failing runs.  We find that across all
of our experiments, in about half the cases the stack is useful in
isolating the cause of a bug; in the other half the stack contains
essentially no information about the bug's cause.

\item Finally, we show that, in principle, it is possible for our
approach to help isolate any kind of failure, not just program
crashes.  All that is required is a way to label each run as either
``successful'' or ``unsuccessful.''

\end{itemize}

With respect to this last point, perhaps the greatest strength of our
system is its ability to automatically identify the cause of many
different kinds of bugs, including new classes of bugs that we did not
anticipate in building the tool.  By relying only on the distinction
between good and bad executions, our analysis does not require a
specification of the program properties to be analyzed.  Thus,
statistical debugging provides a complementary approach to static
analyses, which generally do require specification of the properties
to check.  Statistical debugging can identify bugs beyond the reach of
static analysis techniques and even new classes of bugs which may be
amenable to static analysis if anyone thought to check for them.

One of the bugs we found, in the \rhythmbox open source music player,
provides a good illustration of the potential for positive interaction
with static analysis.  A strong predictor of failure detected by our
algorithm revealed a previously unrecognized unsafe usage pattern of a
library's API\@.  A simple syntactic static analysis subsequently showed
more than one hundred instances of the same unsafe pattern throughout
\rhythmbox.

The paper is organized as follows.  After providing background
(\autoref{sec:background}), we discuss our algorithm
(\autoref{sec:algorithm}). The experimental results are presented in
\autoref{sec:experiments}.  We then discuss related
work (\autoref{sec:related-work}), including the advantages over our
previous approach based on logistic regression
(\autoref{sec:comparison}).


\section{Background}
\label{sec:background}

This section summarizes ideas and terminology needed to to present our
algorithm.  The ideal program monitoring system would gather complete
execution traces and provide them to an engineer (or, more likely, a
tool) to mine for the causes of bugs.  However, complete tracing of
program behavior is simply impractical; no end user would accept the
required performance overhead or network bandwidth.

Instead, we use a combination of sparse random sampling, which controls
performance overhead, and client-side summarization of the data, which
limits the storage and transmission costs.  We briefly discuss
both aspects.

Random sampling is added to a program via a source-to-source transformation.
Our sampling transformation is general: any collection of
statements within (or added to) a program may be designated as
``instrumentation'' and thereby sampled instead of run
unconditionally.
%%
\issue[Mayur]{"Our sampling transformation is general: any collection
  of stmts within (or added to) a program may be designated as
  "instrumentation" and thereby sampled instead of run conditionally"

  I didn't get the "within (or added to)" part.  I thought
  instrumented code always refers to code that is added, and not code
  that is within?

  Ben later clarified this but we need to put it into the paper:

  We can treat existing code as instrumentation to be sampled even if
  we didn't add it ourself.  assert() statements are a clear candidate
  for this sort of thing.  Our PLDI 2003 paper did this with the
  assert-like statements added by CCured.  From our perspective, that
  code is "within" the program, not "added to" it.}
%%
That is, each time instrumentation code is reached,
a coin flip decides whether the instrumentation is executed or not.
Coin flipping is simulated in a statistically fair
manner equivalent to a Bernoulli process: each potential sample is
taken or skipped randomly and independently as the program runs.
We have found that a sampling rate of \nicefrac{1}{100} normally keeps the performance overhead
of instrumentation low, often unmeasurable.
%%
\issue[Alice]{Is this true?  I thought Ben's thesis would claim otherwise.}

Orthogonal to the sampling transformation is the decision about what
instrumentation to introduce and how to concisely summarize the
resulting data.  A useful instrumentation captures behaviors likely to
be of interest when hunting for bugs.  At present our system offers
the following instrumentation schemes for C programs:

\begin{description}
\sloppy
\item[branches:] At each conditional, including implicit conditionals
such as loop tests and short-circuiting logical operators, we track two predicates
indicating whether the true or false branches were ever taken.

\item[returns:] In C, the
  sign of a return value is often used to signal success or failure.
  At each scalar-returning function call site, we track six predicates:
  whether the returned value is ever $< 0$, $\le 0$, $> 0$, $\ge 0$,
  $= 0$, or $\ne 0$.
%  For
%  pointer-returning calls, this scheme reduces to counting
%  \texttt{NULL} versus non-\texttt{NULL}.  
%The observation is made
%  just after the function returns but before the result is used by the
%  original program.  An instrumentation site is added even if the
%  source program discards the return value, as unchecked return
%  values are a common source of bugs.  Each call site induces one
%  instrumentation site with three counters: number of negative
%  returns, number of zero returns, and number of positive returns.
%%

\item[scalar-pairs:] Many bugs
  concern boundary issues in the relationship between a 
  variable and another variable or constant.  At
  each scalar assignment \texttt{x = \dots}, identify each
  same-typed in-scope variable $\mathtt{y}_i$ and each
  constant expression $\mathtt{c}_j$.  For each   $\mathtt{y}_i$ and each $\mathtt{c}_j$,  
  we track six relationships to the new value of \texttt{x}: $<, \leq, >, \geq, =, \neq$.
Each compared-to $\mathtt{y}_i$
  or $\mathtt{c}_j$ is treated as a distinct instrumentation site.
  %%
\end{description}

All predicates at a given site are updated jointly.  For
example, sampling a single negative return value would yield true
observations of the $< 0$, $\le 0$, and $\ne 0$ predicates in the
returns scheme.  Note that the logical negation of each predicate is
itself a predicate.

  These are natural properties to check and provide good coverage of a
  program's scalar values and control flow.  This set is by no means
  complete, however; in particular, it would be useful to have
  predicates on heap structures as well.

\section{Cause Isolation Algorithm}
\label{sec:algorithm}
This section presents our algorithm for automatically isolating
multiple bugs.  As discussed in \autoref{sec:background}, the input is
a set of feedback reports from individual program runs $R$, where
$R(P) = 1$ if predicate $P$ is observed to be true during the
execution of $R$.

The idea behind the algorithm is to simulate the iterative manner in which  human programmers
typically find and fix bugs:
\begin{enumerate}

\item Identify the cause of the most important bug ${\cal B}$.

\item Fix ${\cal B}$, and repeat.

\end{enumerate}

For our purposes, identifying the cause of a bug ${\cal B}$ means selecting a
predicate $P$ closely correlated with ${\cal B}$.  The difficulty is that we
know the set of runs that succeed and fail, but we do not know which
set of failing runs corresponds to ${\cal B}$, or even how many bugs there
are.  Thus, in the first step we must infer which predicates are most
likely to correspond to individual bugs and rank those predicates in
importance.

For the second step, while we cannot literally fix the bug
corresponding to the chosen predictor $P$, we can simulate what
happens if the bug does not occur.  We discard any run $R$ such that
$R(P) = 1$ and repeat.  Discarding all the runs where $R(P) = 1$
reduces the importance of other predictors of ${\cal B}$, allowing predicates
that predict different bugs (i.e., different sets of failing runs) to
rise to the top in subsequent iterations.

\subsection{Increase Scores}
\label{sec:increase}

We now discuss the first step, how to find the cause of the most important bug.
We break this step into two sub-steps.  First, we eliminate predicates that have no
predictive power at all; this typically reduces the number of predicates we need
to consider by two orders of magnitude (e.g., from hundreds of thousands to thousands).
Next, we rank the surviving predicates by importance (see \autoref{sec:ranking}).

Consider the following C code fragment:
\begin{quote}
\begin{verbatim}
f = ...;          (a)
if (f == NULL) {  (b)
        x = 0;    (c)
        *f;       (d)
}
\end{verbatim}
\end{quote}
Consider the predicate {\tt f == NULL} at line {\tt (b)}, which would
be captured by branches instrumentation.  Clearly
this predicate is highly correlated with failure; in fact, whenever it
is true this program inevitably crashes.\footnote{We also note that this bug could 
be detected by a simple static analysis; this example is meant to be concise rather than 
a significant application of our techniques.}   An important observation,
however, is that even a ``smoking gun'' such as {\tt f == NULL} at
line {\tt (b)} cannot be a perfect predictor of failure when there are
multiple bugs in the program---since there are other bugs, the run can fail
even if the predicate is never true in a run.

The bug in the code fragment above is \termdef{deterministic} with
respect to {\tt f == NULL}: if {\tt f == NULL} is true at line {\tt
(b)}, the program fails.  In many cases it is impossible to observe
the exact conditions causing failure; for example, buffer overrun bugs
in a C program may or may not cause the program to crash depending on
runtime system decisions about how data is laid out in memory.  Such
bugs are \termdef{non-deterministic} with respect to every predicate;
even for the best predictor $P$, it is possible
that $P$ is true and still the program terminates normally.  In the
example above, if we insert before line {\tt (d)} a valid pointer
assignment to {\tt f} controlled by a conditional that is true at
least occasionally (say via a call to read input):
\begin{quote}
\begin{verbatim}
if (read()) f = ... some valid pointer ...;
*f;
\end{verbatim}
\end{quote}
the bug becomes non-deterministic with respect to {\tt f == NULL}.

To summarize, even for a predicate $P$ that is truly the cause of a bug, we can neither assume that
when $P$ is true that
the program fails nor that when $P$ is never observed to be true  that
the program succeeds. But we can express the probability that $P$
being true implies failure.  Let $\fail$ be an atomic predicate that is
true for failing runs and false for successful runs.  Let $\prob(A | B)$ denote
the conditional probability function of the event $A$ given event $B$ .  
We want to compute:
% [[ modified by Alice]]
\[ \crash(P) \equiv \prob(\fail = 1 | P = 1) \]
%= \prob(\fail | P \mbox{ is true}) \]
for every instrumented predicate $P$ over the set of all runs.  Let $S(P)$ be the number
of successful runs in which $P$ is observed to be true, and let $F(P)$ be the number of
failing runs in which $P$ is observed to be true.  We
estimate $\crash(P)$ as:
\[ \crash(P) = \frac{F(P)}{S(P) + F(P)} \]

Notice that $\crash(P)$ is unaffected by the set of runs in which
$P$ is not observed to be true.  Thus, if $P$ is the cause of a bug, the
causes of other independent bugs do not affect $\crash(P)$.
Also note that runs in which $P$ is not observed at all (either because
the line of code on which $P$ is checked is not reached, or the line is reached
but $P$ is not sampled) have no effect on $\crash(P)$.
Finally, the definition of $\crash(P)$
generalizes the idea of deterministic and non-deterministic bugs.  A
bug is deterministic for $P$ if $\crash(P) = 1.0$ or, equivalently,
$P$ is never observed to be true in a successful run ($S(P) =
0$) and $P$ is observed to be true in at least one failing run ($F(P) > 0$).
If $\crash(P) < 1.0$ then the bug is non-deterministic, with
lower scores showing weaker correlation between the predicate and
program failure.

Now $\crash(P)$ is a useful measure, but it is not good
enough for the first step of our algorithm. To see this, consider again the
code fragment given above (in its original form, not with the
modification to make the bug non-deterministic).  At line {\tt (b)} we
have $\crash(\mbox{\tt f == NULL}) = 1.0$, so this predicate is a good
candidate for the cause of the bug.
But on line {\tt (c)} we have the surprising fact that $\crash(\mbox{\tt x == 0}) = 1.0$ as well.
To understand why, observe that the 
predicate \texttt{x == 0} is always true at line {\tt (c)} and, in
addition,
only failing runs reach this line.
Thus $S(\mbox{\tt x == 0}) = 0$, and, so long as there is at least one run that
reaches line {\tt (c)} at all, $\crash(\mbox{\tt x == 0})$ at line {\tt (c)} is 1.0.

As the predicate {\tt x == 0} at line {\tt (c)} of the example
shows, just because $\crash(P)$ is high does not
mean $P$ is the cause of a bug.  In the case of {\tt x == 0}, the
decision that eventually causes the crash is made earlier, and the
$\crash(\mbox{\tt x == 0})$ score merely reflects the fact that this
predicate is checked on a path where the program is already doomed.

A way to address this difficulty is to score a predicate not by the chance
that it implies failure, but by how much difference it makes that the predicate
is observed to be true versus simply reaching the line where the predicate is checked.
That is, on line {\tt (c)}, the probability of crashing is already 1.0 regardless
of the value of the predicate {\tt x == 0}, and thus the fact that {\tt x == 0} is
true does not increase the probability of failure at all; this coincides with
the intuition that this predicate is irrelevant to the bug.

This leads us to the following definition:
\[
%% [[modified by Alice]]
\context(P) \equiv \prob(\fail = 1 | P \mbox{ is observed})  
\]
%= \prob(\fail |  P \mbox{ observed}) 
%\end{gather*}
Now, $P \lor \lnot P$ is not the set of all runs, because we are not working in a two-valued logic.
In any given run, neither $P$ nor $\lnot P$ may be observed (because the site where this predicate is
sampled is not reached), or one may be observed, or both may be observed (because the statement is executed
multiple times and $P$ is sometimes true and sometimes false).  Thus, $\context(P)$ is the probability that
in the set of runs where the value of $P$ is observed at all, the program fails. We can compute $\context(P)$ as follows:
\[ \context(P) = \frac{F(P \lor \lnot P)}{S(P \lor \lnot P) + F(P \lor \lnot P)} \]

The interesting quantity, then, is
\begin{equation*}
 \increase(P) \equiv \crash(P) - \context(P) \label{eqn:1}
\end{equation*}
%%
which can be read as: How much does $P$ being true increase the probability of failure
over simply reaching the line where $P$ is sampled?  For example, for the predicate {\tt x == 0} on line {\tt (c)},
we have
\[\crash(\mbox{\tt x == 0}) = \context(\mbox{\tt x == 0}) = 1.0 \]
and so $\increase(\mbox{\tt x == 0}) = 0$.

A predicate $P$ with $\increase(P) \leq 0$ has no predictive power; whether it is true does not increase the
probability of failure, and we can safely discard all such predicates.
But because some $\increase(P)$ scores may be based on few observations of $P$, it is important
to attach confidence intervals to the scores.  Since $\increase()$ is a statistic, computing
its confidence interval is a well-understood problem. In our experiments we retain a predicate $P$
only if $\increase(P) > 0$ with 95\% confidence; this removes predicates from consideration that have high
increase scores but very low confidence because of few observations. 

Pruning predicates using $\increase(P) \leq 0$ has many desirable
properties.  It is easy to prove that large classes of irrelevant
predicates always have scores $\leq 0$.  For example, any predicate
that is unreachable, that is a program invariant, or that is obviously
control-dependent on a true cause is eliminated by this test.  It is
also worth pointing out that this test tends to localize bugs at
a point where the condition that causes the bug becomes true, rather than at
the crash site.  For example, in the code fragment given above, the bug is
attributed to the success of the conditional branch test {\tt f ==
NULL} on line {\tt (b)} rather than the pointer dereference on line
{\tt (d)}.  Thus, the cause of the bug discovered by the algorithm
points directly to the conditions under which the crash occurs, rather than
the line on which it occurs (which is usually available anyway in the
stack trace).

\subsection{Statistical Interpretation}
\label{sec:statisticalinterpretation}

We have explained the test $\increase(P) > 0$ using programming terminology,
but it also has a natural statistical interpretation as a simplified {\em likelihood ratio} hypothesis
test.  Consider the two classes of trial runs
of the program: failed runs $F$ and successful runs $S$.  For each
class, we can treat the predicate $P$ as a Bernoulli random variable
with heads probabilities $\pi_f(P)$ and $\pi_s(P)$, respectively, for the
two classes.  The heads
probability is the probability that the predicate is observed to be
true.  If a predicate causes a set of crashes, then $\pi_f$ should be
much bigger than $\pi_s$.  We can formulate two statistical hypotheses:
the null hypothesis $\H_0:
\pi_f \leq \pi_s$, versus the alternate hypothesis $\H_1: \pi_f > \pi_s$.  Since
$\pi_f$ and $\pi_s$ are not known, we must estimate them:
\begin{align*}
  \hat \pi_f(P) &= \frac{F(P)}{F(P \lor \lnot P)} &
  \hat \pi_s(P) &= \frac{S(P)}{S(P \lor \lnot P)}
\end{align*}

Although these proportion estimates of $\pi_f$ and $\pi_s$ approach the
actual heads probabilities as we increase the number of trial runs, they
still differ due to sampling.  With a certain probability, using these
estimates instead of the actual values results in the wrong
answer.  A \textit{likelihood ratio test} takes this uncertainty into
account, and makes use of the statistic $ Z = \frac{(\hat \pi_f - \hat
  \pi_s)}{V_{f,s}}$, where $V_{f,s}$ is a sample variance term (see
e.g., \cite{Lehmann:1986:hyptest}).  When
the data size is large, $Z$ can be approximated as a standard Gaussian
random variable.  Performed independently for each predicate $P$, the
test decides whether or not $\pi_f(P) \leq \pi_s(P)$ with a guaranteed
false-positive probability (i.e.,\ choosing $\H_1$ when $\H_0$ is true).
A necessary (but not sufficient) condition for choosing $\H_1$ is that
$\hat \pi_f(P) > \hat \pi_s(P)$.  This turns out to be
equivalent to the condition that $\increase(P) > 0$.  To see this,
let $a = F(P)$, $b = S(P)$, $c = F(P\lor\lnot P)$, and $d = S(P\lor\lnot P)$.
Then
\begin{gather*}
  \increase(P) > 0 \iff \crash(P) > \context(P) \\
  \iff \frac{a}{a+b} > \frac{c}{c+d}
  \iff a (c+d) > (a+b) c \\
  \iff ad > bc \iff \frac{a}{c} > \frac{b}{d}
  \iff \hat \pi_f(P) > \hat \pi_s(P)
\end{gather*}

%% -*- TeX-master: "../master.tex" -*-

\begin{table*}[tb]
  \nocaptionrule
  \caption{Comparison of ranking strategies for \moss without
    redundancy elimination}
  \label{tab:sorts}
  \centering

  \newenvironment{stats}[2]{%
    \begin{subtab}
      \caption{Sort descending by #1}
      \small
      \label{tab:sorts-#2}
      \begin{tabular}{l|rr@{$\:\pm\:$}rr|rrl}
        \toprule
        Thermometer & Context & \multicolumn{2}{r}{Increase} &
        S & F & F + S & Predicate \\
        \midrule}{%
        \bottomrule
      \end{tabular}
    \end{subtab}}

  \begin{stats}{$\text{F}(P)$}{failure}
    \bugometer{4.44089}{0.17602}{0.000535293}{0.006240707}{0.817204} & 0.176 & 0.007 & 0.012 & 22554 & 5045 & 27599 & \verb|files[filesindex].language &#8800; 15| \\
\bugometer{4.44108}{0.175879}{0.000600451}{0.006237549}{0.817283} & 0.176 & 0.007 & 0.012 & 22566 & 5045 & 27611 & \verb|tmp == 0 is FALSE| \\
\bugometer{4.44116}{0.175904}{0.000543038}{0.006236962}{0.817316} & 0.176 & 0.007 & 0.012 & 22571 & 5045 & 27616 & \verb|strcmp &#8800; 0| \\
\bugometer{4.36446}{0.176292}{0.000813956}{0.006562044}{0.816332} & 0.176 & 0.007 & 0.013 & 18894 & 4251 & 23145 & \verb|tmp == 0 is FALSE| \\
\bugometer{4.36408}{0.176018}{0.00077294}{0.00656006}{0.816649} & 0.176 & 0.007 & 0.013 & 18885 & 4240 & 23125 & \verb|files[filesindex].language &#8800; 14| \\
\bugometer{4.33774}{0.175829}{0.00160129}{0.00668071}{0.815889} & 0.176 & 0.008 & 0.013 & 17757 & 4007 & 21764 & \verb|filesindex &#8805; 25| \\
\bugometer{4.30501}{0.176949}{0.000967379}{0.006932621}{0.815151} & 0.177 & 0.008 & 0.014 & 16453 & 3731 & 20184 & \verb|M < M| \\
\bugometer{3.93024}{0.175776}{0.249223}{0.011356}{0.563645} & 0.176 & 0.261 & 0.023 & 4800 & 3716 & 8516 & \verb|config.winnowing_window_size &#8800; argc| \\
\multicolumn{8}{c}{\dotfill{} 2732 additional predictors follow \dotfill{}} \\

  \end{stats}

  \begin{stats}{$\text{Increase}(P)$}{increase}
    \bugometer{1.36173}{0.0652963}{0.924982}{0.0097217}{0} & 0.065 & 0.935 & 0.019 & 0 & 23 & 23 & \verb|((*(fi + i)))->this.last_token < filesbase| \\
\bugometer{1}{0.0652535}{0.924878}{0.0098685}{0} & 0.065 & 0.935 & 0.020 & 0 & 10 & 10 & \verb|((*(fi + i)))->other.last_line == last| \\
\bugometer{1.25527}{0.0713427}{0.918557}{0.0101003}{0} & 0.071 & 0.929 & 0.020 & 0 & 18 & 18 & \verb|((*(fi + i)))->other.last_line == filesbase| \\
\bugometer{1}{0.0727273}{0.917042}{0.0102307}{0} & 0.073 & 0.927 & 0.020 & 0 & 10 & 10 & \verb|((*(fi + i)))->other.last_line == yy_n_chars| \\
\bugometer{1.27875}{0.0714847}{0.914362}{0.0141533}{0} & 0.071 & 0.929 & 0.028 & 0 & 19 & 19 & \verb|bytes <= filesbase| \\
\bugometer{1.14613}{0.0751634}{0.913671}{0.0111656}{0} & 0.075 & 0.925 & 0.022 & 0 & 14 & 14 & \verb|((*(fi + i)))->other.first_line == 2| \\
\bugometer{1.07918}{0.0764302}{0.912429}{0.0111408}{0} & 0.076 & 0.924 & 0.022 & 0 & 12 & 12 & \verb|((*(fi + i)))->this.first_line < nid| \\
\bugometer{1}{0.077222}{0.911248}{0.01153}{0} & 0.077 & 0.923 & 0.023 & 0 & 10 & 10 & \verb|((*(fi + i)))->other.last_line == yy_init| \\
\multicolumn{8}{c}{\dotfill{} 2732 additional predictors follow \dotfill{}} \\

  \end{stats}

  \begin{stats}{harmonic mean}{harmonic}
    \bugometer{3.20003}{0.176135}{0.819603}{0.004262}{0} & 0.176 & 0.824 & 0.009 & 0 & 1585 & 1585 & \verb|files[filesindex].language > 16| \\
\bugometer{3.19976}{0.176195}{0.819543}{0.004262}{0} & 0.176 & 0.824 & 0.009 & 0 & 1584 & 1584 & \verb|strcmp > 0| \\
\bugometer{3.19866}{0.17595}{0.819792}{0.004258}{0} & 0.176 & 0.824 & 0.009 & 0 & 1580 & 1580 & \verb|strcmp == 0| \\
\bugometer{3.19783}{0.175935}{0.819808}{0.004257}{0} & 0.176 & 0.824 & 0.009 & 0 & 1577 & 1577 & \verb|files[filesindex].language == 17| \\
\bugometer{3.19756}{0.17599}{0.819751}{0.004259}{0} & 0.176 & 0.824 & 0.009 & 0 & 1576 & 1576 & \verb|tmp == 0 is TRUE| \\
\bugometer{3.19673}{0.175904}{0.819838}{0.004258}{0} & 0.176 & 0.824 & 0.009 & 0 & 1573 & 1573 & \verb|strcmp > 0| \\
\bugometer{2.8893}{0.115811}{0.877037}{0.005862}{0.00129} & 0.116 & 0.883 & 0.012 & 1 & 774 & 775 & \verb|((*(fi + i)))->this.last_line == 1| \\
\bugometer{2.89042}{0.116206}{0.876638}{0.005869}{0.001287} & 0.116 & 0.883 & 0.012 & 1 & 776 & 777 & \verb|((*(fi + i)))->other.last_line == yyleng| \\
\bugometer{3.10585}{0.110724}{0.818742}{0.013324}{0.05721} & 0.111 & 0.832 & 0.027 & 73 & 1203 & 1276 & \verb|config.match_comment is TRUE| \\
\bugometer{2.88649}{0.115628}{0.877204}{0.005869}{0.001299} & 0.116 & 0.883 & 0.012 & 1 & 769 & 770 & \verb|((*(fi + i)))->other.last_line == yy_start| \\
\bugometer{2.89042}{0.118305}{0.874501}{0.005907}{0.001287} & 0.118 & 0.880 & 0.012 & 1 & 776 & 777 & \verb|((*(fi + i)))->other.last_line < 2| \\
\bugometer{2.88818}{0.117597}{0.875205}{0.005904}{0.001294} & 0.118 & 0.881 & 0.012 & 1 & 772 & 773 & \verb|((*(fi + i)))->other.last_line == 1| \\
\bugometer{2.88762}{0.117597}{0.87521}{0.005898}{0.001295} & 0.118 & 0.881 & 0.012 & 1 & 771 & 772 & \verb|((*(fi + i)))->this.last_line == yy_start| \\
\bugometer{2.88649}{0.117501}{0.8753}{0.0059}{0.001299} & 0.118 & 0.881 & 0.012 & 1 & 769 & 770 & \verb|((*(fi + i)))->this.last_line < 2| \\
\bugometer{2.88874}{0.117651}{0.873345}{0.00642}{0.002584} & 0.118 & 0.880 & 0.013 & 2 & 772 & 774 & \verb|((*(fi + i)))->this.last_line == yyleng| \\
\bugometer{2.89597}{0.116671}{0.867635}{0.00807}{0.007624} & 0.117 & 0.876 & 0.016 & 6 & 781 & 787 & \verb|((*(fi + i)))->this.last_line == diff| \\
\multicolumn{8}{c}{\dotfill{} 2724 additional predictors follow \dotfill{}} \\

  \end{stats}

\end{table*}


\subsection{Ranking Predicates: Precision vs. Recall}
\label{sec:ranking}

We now turn to the question of ranking the predicates.
\autoref{tab:sorts} shows the top
predicates under different ranking schemes (explained below) for one of our
experiments.  We use
a simple {\em thermometer} to visualize the information for each
predicate.  The length of the thermometer shows the number of runs in
which the predicate was observed, plotted on a log scale (so small increases
in thermometer size indicate many more runs).  The thermometer has
a sequence of bands:
the black band on the left shows the context score;
the next lighter band shows the increase score, the next, even  lighter band (which is either not
visible or very small in all thermometers) shows the confidence
interval, and the white space at the right end of the thermometer
shows the successful runs in which the predicate was
observed to be true.  The tables show the thermometer as well as the
numbers for each of the quantities that make up the thermometer. 

The most important bug is the one that causes the greatest number
of failed runs.  This observation suggests:
\[ \importance(P) = F(P) \]
\autoref{tab:sorts-failure} shows the top predicates 
ranked by decreasing $F(P)$.\footnote{These predicates are ranked after predicates where $\increase(P) \leq 0$ are discarded.}
While the predicates in \autoref{tab:sorts-failure} are, as expected, involved
in many failing runs, these predicates are also highly 
non-deterministic, meaning they are also true in many successful runs
and are weakly correlated with bugs.  

Our experience with other
ranking strategies that emphasize the number of failed runs is also that
they select predicates involved in many failing, but also many
successful, runs.  The best of these predicates (the ones with high
$\increase()$ scores) are {\em super-bug predictors}:
predictors that include failures from more than one bug.  The
signature of super-bug predictors is that they account for a very
large number of failures (by combining the failures of multiple bugs)
but are also highly non-deterministic despite reasonably high $\increase()$
scores.

Another possibility is:
\[ \importance(P) = \increase(P) \]
\autoref{tab:sorts-increase} shows the top predicates ranked by decreasing
$\increase()$ score (note the thermometers in the different figures
are not drawn to the same scale).  These predicates do a much better
job of predicting failure; in fact, the program always fails when any
of these predicates is true. However, note that the number of failing
runs is very small.  These predicates are {\em sub-bug predictors}:
predictors for a subset of the failures caused by a bug.  Unlike
super-bug predictors, which are not useful in our experience, sub-bug
predictors that account for a significant fraction of the failures for
a bug often provide valuable clues, but still they represent special
cases and may not suggest other, more fundamental, causes of the bug.

Tables~\ref{tab:sorts-failure} and \ref{tab:sorts-increase} illustrate the difficulty of defining
``importance''.  It is helpful to consider this problem in the language of information
retrieval. We are looking for predicates with
high {\em recall} (meaning predicates that account for many failed runs) but also high {\em precision}
(meaning predicates that do not mispredict failure in many successful runs).  In statistics,
the corresponding terms are {\em selectivity} and {\em sensitivity}.  In both fields,
the standard metric is the harmonic mean of precision and recall, which disproportionately
rewards high scores in both dimensions.  In our
case, $\increase(P)$ measures precision and $F(P)$ measures recall.  Define
%%
\[ \importance(P) = \frac{2}{\frac{1}{\increase(P)} + \frac{log(\numfail)}{log(F(P))}} \]
%%
where $\numfail$ is total number of failing runs (independent of $P$)
and therefore scales recall to between zero and one.

The results are given in \autoref{tab:sorts-harmonic}.  We observe that
all of the predicates on this list indeed have both high precision and
recall, accurately describing a large number of failures. 


\subsection{Predicate Elimination}
\label{sec:elimination}

The remaining problem with the results in \autoref{tab:sorts-harmonic}
is that there is substantial redundancy; it is easy to see that several of these
predicates are related hiding other, distinct bugs that either have
fewer failed runs or more non-deterministic predictors further down the
list.  As discussed above, we use a simple recursive algorithm to eliminate
redundant predicates:
\begin{enumerate}

\item Rank predicates by $\importance()$.

\item Remove the top-ranked predicate $P$ and discard all runs $R$ (feedback reports) where  $R(P) = 1$.

\item Repeat (1) and (2) until the set of runs is empty or the set of predicates is empty.
\end{enumerate}




We can now state an easy, but important, property of this algorithm.  
\begin{lemma}
\rm
Let $P_1,\ldots,P_n$ be a set of instrumented predicates and let ${\cal B}_1,\ldots,{\cal B}_m$ be a set of bugs.  Let
\[ {\cal Z} = \bigcup_{1 \leq i \leq n} \{ R | R(P_i) = 1 \} \]
Then if for all $1 \leq j \leq m$ we have ${\cal B}_j \cap {\cal Z} \neq \emptyset$, then 
the algorithm chooses at least one predicate from the list $P_1,\ldots,P_n$ that predicts
at least one failure due to ${\cal B}_j$.
\end{lemma}

Thus, the elimination algorithm chooses at least one predicate 
predictive of each bug represented by the input set of predicates.
The other property we might like, that the algorithm chooses exactly
one predicate to represent each bug, does not hold; we shall see in
\autoref{sec:experiments} that the algorithm sometimes selects a
strong sub-bug predictor as well as a more natural predictor.  Besides
always representing each bug, the algorithm works well for two other
reasons.  First, two predicates are redundant if they predict the same
(or nearly the same) set of failing runs.  Thus, simply removing the
set of runs in which a predicate is true automatically dramatically
reduces the importance of any related predicates in the correct
proportions. Second, because elimination is
iterative, it is only necessary that $\importance()$ selects a good
predictor at each step, and not necessarily the best one; any
predicate that predicts a different set of failing runs than all
higher-ranked predicates is selected eventually.




\section{Experiments}
\label{sec:experiments}
%% -*- LaTeX -*-

\begin{table*}
\centering

\begin{tabular}{|l|r|r|r|r|r|r|r|r|r|}
\hline
Number of: & \multicolumn{2}{c|}{runs}  & sites  & 
\multicolumn{2}{c|}{branch predicates} & \multicolumn{2}{c|}{return 
predicates} & \multicolumn{2}{c|}{scalar predicates}\\ \cline{2-3} \cline{5-6} \cline{7-8} \cline{9-10}
           & successful & failing       &        & original & retained & 
original & retained & original & retained \\
\hline
\hline
ccrypt     &  3605      &  1033         &    570 &      0 &          0 &     
3420 &        6 &        0 &        0 \\
\hline
bc         &  3530      &   860         &  13442 &      0 &          0 &        
0 &        0 &    80652 &      156 \\
\hline
moss       & 28519      &  3352         &  35223 &   4170 &         33 &     
2964 &       11 &   195864 &     3322 \\
\hline
rhythmbox  & 21015      &  1873         & 145242 &   6836 &         14 &    
50574 &       21 &   800370 &      406 \\
\hline
\end{tabular}
\caption{Run, site, predicate, and retention counts for each of the experiments.}
\label{tab:exps}
\end{table*}



In this section we present the results of applying the algorithm
described in \autoref{sec:algorithm} in five case
studies.  \autoref{tab:exps} shows summary statistics for each of the
experiments.  In each study we ran the programs on about 32,000 random
inputs.  The number of instrumentation sites varies with the size of
the program, as does the number of predicates those instrumentation
sites yield.  Our algorithm is very effective in reducing the number
of predicates the user must examine.  For example, in the case of
\rhythmbox an initial set of 857,384 predicates is reduced to 537 by the $\increase(P) > 0$
test, a reduction of 99.9\%.  The elimination algorithm then yields 15 predicates, a further
reduction of 97\%.  The other case studies show a similar reduction in the number of
predicates by 3-4 orders of magnitude.

The results we discuss are all on sampled data.  Sampling creates
additional challenges that must be faced by our algorithm.  Assume $P_1$ and $P_2$ are
equivalent bug predictors and both are sampled at a rate of
$\nicefrac{1}{100}$ and both are reached once per run.  Then even though
$P_1$ and $P_2$ are equivalent, they will be observed in nearly disjoint
sets of runs and treated as close to independent by the elimination
algorithm.

To address this problem, we set the sampling rates of predicates to be
inversely proportional to their frequency of execution.  Based on a
training set of 1,000 executions, we set the sampling rate of each predicate so
as to obtain an expected 100 samples of each predicate in subsequent program
executions.  On the low end, the sampling rate is clamped to a minimum of $\nicefrac{1}{100}$; if the site is expected to be reached fewer than 100 times the sampling rate is set at 1.0.
Thus, rarely executed code has a
much higher sampling rate than very frequently executed code.  (A
similar strategy has been pursued for similar reasons in related work \cite{chil04}.)  We
have validated this approach by comparing the results for each
experiment with results obtained with no sampling at all (i.e., the
sampling rate of all predicates set to 100\%).  The results are
identical except for the \rhythmbox and \moss experiments, where we
judge the differences to be minor: sometimes a different but logically
equivalent predicate is chosen, the ranking of predictors of different
bugs is slightly different, or one or the other version has a few
extra, weak predictors at the tail end of the list.

\subsection{A Validation Experiment}

To validate our algorithm we first performed an experiment in which we
knew the set of bugs in advance.  We added nine bugs to \moss, a
widely used service for detecting plagiarism in software
\cite{Schleimer:2003:WLA}.  Six of these were previously discovered
and repaired bugs in \moss that we reintroduced.  The other three were
variations on three of the original bugs, to see if our algorithm could
discriminate between pairs of bugs with very similar behavior but
distinct causes.  The nature of the eight crashing bugs varies: four
buffer overruns, a null file pointer dereference in certain cases, a
missing end-of-list check in the traversal of a hash table bucket, a missing
out-of-memory check, and a violation of a subtle invariant that must be maintained between two
parts of a complex data structure.  In addition, some of these bugs
are non-deterministic any may not even crash when they should.

The ninth bug---incorrect handling of comments in some cases---only
causes incorrect output, not a crash.  We include this bug in our
experiment in order to show that bugs other than crashing bugs can 
also be isolated using our techniques, provided there is some 
way, whether by automatic self-checking or human inspection, to recognize
failing runs.  In particular, for our experiment we also ran a correct 
version of \moss{} and compared the output of the two versions. 
This oracle provides a labeling of runs as ``success'' or ``failure,'' 
and the resulting labels are treated identically by our program as
those based on program crashes.

\begin{table*}
\centering

\begin{tabular}{|l|r|r|r|r|r|r|r|r|r|}
\hline
Number of: & \multicolumn{2}{c|}{runs}  & sites  & 
\multicolumn{2}{c|}{branch predicates} & \multicolumn{2}{c|}{return 
predicates} & \multicolumn{2}{c|}{scalar predicates}\\ \cline{2-3} \cline{5-6} \cline{7-8} \cline{9-10}
           & successful & failing       &        & original & retained & 
original & retained & original & retained \\
\hline
\hline
ccrypt     &  3605      &  1033         &    570 &      0 &          0 &     
3420 &        6 &        0 &        0 \\
\hline
bc         &  3530      &   860         &  13442 &      0 &          0 &        
0 &        0 &    80652 &      156 \\
\hline
moss       & 28519      &  3352         &  35223 &   4170 &         33 &     
2964 &       11 &   195864 &     3322 \\
\hline
rhythmbox  & 21015      &  1873         & 145242 &   6836 &         14 &    
50574 &       21 &   800370 &      406 \\
\hline
\end{tabular}
\caption{Run, site, predicate, and retention counts for each of the experiments.}
\label{tab:exps}
\end{table*}



\autoref{tab:mossdilute} shows the results of the experiment.  The
predicates listed were selected by the elimination algorithm in the
order shown.  The first column is the initial bug thermometer for each
predicate, showing the \context{} and \increase{} scores before
elimination is performed. The fourth column is the \termdef{effective}
bug thermometer, showing the \context{} and \increase{} scores for a
predicate $P$ at the time $P$ is selected (i.e., when it is the
top-ranked predicate).  Thus the effective thermometer reflects the
cumulative diluting effect of redundancy elimination for all
predicates selected before this one.

As part of the experiment we separately recorded the exact set of
bugs that actually occurred in each run.
The columns at the far right of \autoref{tab:mossdilute} show, for
each selected predicate and for each bug, the number of failing runs in which
both the selected predicate is observed to be true and the bug occurs.
Note that while each
predicate has a very strong spike at one bug, indicating it is a
strong predictor of that bug, there are always some runs with other
bugs present.  For example, the top-ranked predicate, which is
overwhelmingly a predictor of bug \#5, also includes some runs where
bugs \#3, \#4, and \#9 occurred.  This situation is not the result of
misclassification of failing runs by our algorithm.  As observed in
\autoref{sec:introduction}, more than one bug may occur in a run.
It simply happens that in some runs bugs \#5 and \#3 both occur (to
pick just one possible combination).

A particularly interesting case of this phenomenon is bug \#7, one of
the buffer overruns.  Bug \#7 is not strongly predicted by any
predicate on the list but in fact occurs in at least a few of the
failing runs of most predicates.  We have examined the runs of bug \#7
in detail and found that the only failing runs involving bug \#7 also
trigger at least one other bug.  That is, even when the bug \#7 overrun
happens, it never causes incorrect output or a crash
in any run.  Bug \#8, another overrun, is not even shown because the
overrun is never triggered in our data (its column would be all
0's).\footnote{Bug \#8 was originally found by a code inspection.}
There is no way our algorithm can find causes of bugs that do not
occur, but recall that part of our purpose in sampling user executions
is to get an accurate picture of the most important bugs.  It is
consistent with this goal that if a bug never causes a problem, it is
not only not worth fixing, it is not even worth reporting.

The other bugs all have strong predictors on the list.  In fact,
the top eight predicates have exactly one predictor for each of the seven
bugs that occur, with the exception of bug \#1, which has one very
strong sub-bug predictor in the second spot and another predictor
in the sixth position.  Notice that even the rarest bug, bug \#2,
which occurs more than an order of magnitude less frequently than
the most common bug, is identified immediately after the last of
the other bugs.\footnote{The peculiar eighth predicate, \texttt{f < f},
says that after an assignment the new value of \texttt{f} is less than
the old value of \texttt{f}.}  Furthermore, we have verified by hand that
the selected predicates would, in our judgment, lead an engineer to
the cause of the bug. Overall, the elimination algorithm does an excellent
job of listing separate causes of each of the bugs in order of priority,
with very little redundancy.

Below the eighth position there are no new bugs to report and every
predicate is correlated with predicates higher on the list.  Even
without the columns of numbers at the right it is easy to spot the
eighth position as the natural cutoff.  Keep in mind that the length
of the thermometer is on a log scale, hence changes in larger
magnitudes may appear less evident.  Notice that the initial and
effective thermometers for the first eight predicates are essentially
identical.  Only the predicate at position six is noticeably
different, indicating that this predicate is somewhat affected by a
predicate listed earlier (specifically, its companion sub-bug
predictor at position two).  However, all of the predicates below the
eighth line have very different initial and effective thermometers
(either many fewer failing runs, or much more non-deterministic, or
both) showing that these predicates are strongly affected by
higher-ranked predicates.

The visualizations presented thus far have a drawback illustrated by
the \moss\ experiment: It is not easy to identify the predicates to
which a predicate is closely related.  Such a feature would be useful
in confirming whether two selected predicates represent different bugs
or are in fact related to the same bug.  We do have a measure of how
strongly $P$ implies another predicate $P'$: How does removing the
runs where $\report{P} = 1$ affect the importance of $P'$?  The more
closely related $P$ and $P'$ are, the more $P'$'s importance drops
when $P$'s failing runs are removed.  In the interactive version of
our analysis tools, each predicate $P$ in the final, ranked list of
links to an \termdef{affinity list} of all
predicates ranked by how much $P$ causes their ranking score to
decrease.

\subsection{Additional Experiments}

We briefly report here on experiments with additional applications
containing both known and unknown bugs.  Complete analysis results for
all experiments may be browsed interactively at
\url{http://www.cs.berkeley.edu/~liblit/pldi-2005}.

\subsubsection{\ccrypt}

\view{\ccrypt}{ccrypt}

We analyzed \ccrypt 1.2, which has a known input validation bug.  The
results are shown in \autoref{tab:views-ccrypt}.  Our algorithm
reports two predictors, both of which point directly to the single bug.
It is easy to discover that the two predictors are for the same bug;
the first predicate is listed first in the second predicate's affinity
list, indicating the first predicate is a sub-bug predictor associated
with the second predicate.

\subsubsection{\bc}

\view{\bc}{bc}

GNU \bc 1.06 has a previously reported buffer overrun.  Our results
are shown in \autoref{tab:views-bc}.  The outcome is the same as for
\ccrypt: two predicates are retained by elimination, and the second
predicate lists the first predicate at the top of its affinity list,
indicating that the first predicate is a sub-bug predictor of the second.
Both predicates point to the cause of the overrun.  This bug causes a
crash long after the overrun occurs and there is no useful information
on the stack at the point of the crash to assist in isolating this
bug.

\subsubsection{\exif}

\view{\exif}{exif}

\autoref{tab:views-exif} shows results for \exif 0.6.9, an open source
image processing program.  Each of the three predicates is a predictor
of a distinct and previously unknown crashing bug.  It took less than
20 minutes of work to find and verify the cause of each of the bugs
using these predicates and the additional highly correlated predicates
on their affinity lists.

To illustrate how statistical debugging is used in practice, we
use the last of these three failure predictors as an example, and
describe how it enabled us to
effectively isolate the cause of one of the bugs.  Failed runs
exhibiting \texttt{o + s > buf\_size} show the following unique stack
trace at the point of termination:
\begin{quote}
  \small
\begin{verbatim}
main
  exif_data_save_data
    exif_data_save_data_content
      exif_data_save_data_content
        exif_data_save_data_entry
          exif_mnote_data_save
            exif_mnote_data_canon_save
              memcpy
\end{verbatim}
\end{quote}
The code in the vicinity of this crash site is as follows:
\begin{quote}
\begin{verbatim}
// snippet of exif_mnote_data_canon_save
for (i = 0; i < n->count; i++) {
    ...
    memcpy(*buf + doff,             (c)
           n->entries[i].data, s);
    ...
}
\end{verbatim}
\end{quote}
This stack trace alone provides little insight into the cause of the
bug.  However, our algorithm highlights \texttt{o + s > buf\_size} in
function \texttt{exif\_mnote\_data\_canon\_load} as a strong bug
predictor.  Thus, a quick inspection of the source code leads us to
construct the following call sequence:
\begin{quote}
  \small
\begin{verbatim}
main
  exif_loader_get_data
    exif_data_load_data
      exif_mnote_data_canon_load
  exif_data_save_data
    exif_data_save_data_content
      exif_data_save_data_content
        exif_data_save_data_entry
          exif_mnote_data_save
            exif_mnote_data_canon_save
              memcpy
\end{verbatim}
\end{quote}
The code in the vicinity of the predicate \texttt{o + s > buf\_size} is as follows:
\begin{quote}
\begin{verbatim}
// snippet of exif_mnote_data_canon_load
for (i = 0; i < c; i++) {
    ...
    n->count = i + 1;
    ...
    if (o + s > buf_size) return;    (a)
    ...
    n->entries[i].data = malloc(s);  (b)
    ...
}
\end{verbatim}
\end{quote}
It is apparent from the above code snippets and the
call sequence that whenever the predicate \texttt{o + s > buf\_size} is true,
%%
\begin{itemize}
\item the function \texttt{exif\_mnote\_data\_canon\_load} returns on
  line \texttt{(a)}, thereby skipping the call to \texttt{malloc} on
  line \texttt{(b)} and thus leaving \texttt{n->entries[i]->data}
  uninitialized for some value of \texttt{i}, and

\item the function \texttt{exif\_mnote\_data\_canon\_save} passes the
  uninitialized \texttt{n->entries[i]->data} to \texttt{memcpy} on line \texttt{(c)}, which reads it and eventually crashes.
\end{itemize}

In summary, our algorithm enabled us to effectively isolate the causes
of several previously unknown bugs in source code unfamiliar to us in
a small amount of time and without any explicit specification beyond
``the program shouldn't crash.''

\subsubsection{\rhythmbox}

\begingroup
\setlength{\segunit}{10pt}
\view[\tiny]{\rhythmbox}{rhythmbox}
\endgroup

\autoref{tab:views-rhythmbox} shows our results for \rhythmbox 0.6.5,
an interactive, graphical, open source music player.  \rhythmbox is a
complex, multi-threaded, event-driven system, written using a library
providing object-oriented primitives in C.  Event-driven systems use
event queues; each event performs some computation and possibly adds
more events to some queues.  We know of no static analysis today that
can analyze event-driven systems accurately, because no static
analysis is currently capable of analyzing the heap-allocated event
queues with sufficient precision.  Stack inspection is also of
limited utility in analyzing event-driven systems, as the stack in the
main event loop is unchanging and all of the interesting state is in
the queues.

We isolated two distinct bugs in \rhythmbox.  The first predicate led
us to the discovery of a race condition.  The second predicate was not
useful directly, but we were able to isolate the bug using the
predicates in its affinity list.  This second bug revealed what turned
out to be a very common incorrect pattern of accessing the underlying
object library (recall \autoref{sec:introduction}).  \rhythmbox
developers confirmed the bugs and enthusiastically applied patches
within a few days, in part because we could quantify the bugs as
important crashing bugs.  It required several hours to isolate each of
the two bugs (and there are additional bugs represented in the
predictors that we did not isolate) in part because \rhythmbox is
complex and in part because the bugs were violations of subtle heap
invariants which are not directly captured by our current
instrumentation schemes.  Note, however, that we could not have even
begun to understand these bugs without the information provided by our
tool.  We intend to explore schemes that track predicates on heap
structure in future work.

\subsection{Comparison with Logistic Regression}
\label{sec:comparison}

\begin{table}
\nocaptionrule
\caption{Results of logistic regression for \moss}
\label{tab:logregression}
\centering
\small
\begin{tabular}{ll}
  \toprule
  Coefficient & Predicate \\
  \midrule
  0.769379 & \verb|(p + passage_index)->last_line < 4| \\
  0.686149 & \verb|(p + passage_index)->first_line < i| \\
  0.675982 & \verb|i > 20| \\
  0.671991 & \verb|i > 26| \\
  0.619479 & \verb|(p + passage_index)->last_line < i| \\
  0.600712 & \verb|i > 23| \\
  0.591044 & \verb|(p + passage_index)->last_line == next| \\
  0.567753 & \verb|i > 22| \\
  0.544829 & \verb|i > 25| \\
  0.536122 & \verb|i > 28| \\
  \bottomrule
\end{tabular}
\end{table}

In earlier work
we used $\ell_1$-regularized logistic regression
to rank the predicates by their
failure-prediction strength \cite{PLDI`03*141,NIPS2003_AP05}.
Logistic regression uses linearly weighted
combinations of predicates to classify a trial run as successful or
failed.  Regularized logistic regression incorporates a penalty
forcing most coefficients to be set to zero, thereby
selecting only the most important predicates.  The output is a set of
coefficients for predicates giving the best overall prediction.

A weakness of logistic regression for our application is that it seeks
to cover the set of failing runs without regard to the orthogonality
of the selected predicates (i.e., whether they represent distinct
bugs).  This problem can be seen in \autoref{tab:logregression},
which gives the top ten predicates selected by logistic regression
for \moss.  The striking fact is that all selected predicates are
either sub-bug or super-bug predictors.  The predicates beginning with
\texttt{p + \ldots} are all sub-bug predictors of bug \#1 (see
\autoref{tab:mossdilute}).  The predicates \texttt{i > \ldots} are
super-bug predictors: \texttt{i} is the length of the command line and
the predicates say program crashes are more likely for long command
lines (recall \autoref{sec:introduction}).

The prevalence of super-bug predictors on the list shows the
difficulty of making use of the penalty term.  Limiting the number of
predicates that can be selected via a penalty has the effect of
encouraging regularized logistic regression to choose super-bug predictors, as
these cover more failing runs at the expense of poorer predictive
power compared to predictors of individual bugs.  On the other hand,
the sub-bug predictors are chosen based on their excellent prediction
power of those small subsets of failed runs.
%%Relaxing the penalty
%%allows logistic regression to add more predicates to improve its
%%prediction, but the sub-bug predictors apparently are favored.

%% LocalWords:  exps mossdilute ccrypt bc exif buf mnote rhythmbox
%% LocalWords:  logregression




\section{Previous and Related Work}
\label{sec:related-work}

In this section we briefly survey related work. There is currently a
great deal of interest in applying static analysis to improve software
quality.  While we firmly believe in the use of static analysis to
find and prevent bugs, our dynamic approach has advantages as well.  A dynamic
analysis can observe actual run-time values, which is often better
than either making a very conservative static assumption about run-time
values for the sake of soundness, or allowing some even very simple bugs to escape
undetected.  Another advantage of dynamic analysis, especially one
that uses actual user executions for its data, is the ability to
assign an accurate importance to each bug.  Additionally, as we have shown,
a dynamic analysis that does not require an explicit specification of
the properties to check can find clues to a very wide range of errors,
including classes of errors not considered in the design of the
analysis.
  
The Daikon project \cite{ernst2001} monitors instrumented applications
to discover likely program invariants.  It collects extensive trace
information at run time and uses this offline to accept or reject any
of a wide variety of guessed candidate predicates.  The DIDUCE project
\cite{ICSE02*291} tests a more restricted set of predicates within the
client program, and attempts to relate state changes in candidate
predicates to manifestation of bugs.  Both projects assume complete
monitoring, such as within a controlled test environment.  Our goal is
to use lightweight partial monitoring, suitable for either for testing
or deployment to end users.  We never have complete information, and
therefore must use a more statistical approach.

\termdef{Software tomography} as realized through the GAMMA system
\cite{PASTE'02*2,Orso:2003:LFDIART} shares our goal of low-overhead
distributed monitoring of deployed code.  GAMMA collects code coverage
data to support a variety of code evolution tasks.  Our
instrumentation exposes a broader family of data- and
control-dependent predicates on program behavior and uses randomized
sparse sampling to control overhead.  Our
predicates do, however, give coverage information: the sum of all predicate counters at a site converges to the relative coverage of that site.

Efforts to directly apply statistical modeling principles to debugging
have met with mixed results.  Early work in this area by Burnell and
Horvitz \cite{Burnell:1995:SCM} uses program slicing in conjunction
with Bayesian belief networks to filter and rank the possible causes
for a given bug.  Empirical evaluation shows that the slicing component
alone finds 65\% of bug causes, while the probabilistic model
correctly identifies another 10\%.  This additional payoff may seem
small in light of the effort, measured in multiple
man-years, required to distill experts' often tacit knowledge into a
formal belief network.  However, the approach does illustrate one
strategy for integrating information about program structure into the
statistical modeling process.

In more recent work, Podgurski et al.\ \cite{ICSE`03*465} apply
statistical feature selection, clustering, and multivariate
visualization techniques to the task of classifying software failure
reports.  The intent is to bucket each report into an equivalence
group believed to share the same underlying cause.  Features are
derived offline from fine-grained execution traces without sampling;
this reduces the noise level of the data but greatly restricts the
instrumentation schemes that are practical to deploy outside of a
controlled testing environment.  As in our own earlier work, Podgurski
uses logistic regression to select features which are highly
predictive of failure.  
Clustering tends to identify small, tight groups of runs which do
share a single cause but which are not always maximal.  That is, one
cause may be split across several clusters.

In contrast, current
industrial practice uses stack traces to cluster failure reports into
equivalence classes.  Two crash reports showing the same stack trace,
or perhaps only the same top-of-stack function, are presumed to be two
reports of the same failure.  This works to the extent that a single
cause corresponds to a single point of failure, but our experience
with \moss, \rhythmbox, and \exif suggests that this assumption may not often hold.  In \moss,
we find that only bugs \#2 and \#5 have truly unique ``signature'' stacks: a
crash location which is present if and only if the corresponding bug
was actually triggered.  These bugs are also our most deterministic.
Bugs \#4 and \#6 also have nearly unique stack signatures.
The remaining bugs are much less consistent: each stack signature is
observed after a variety of different bugs, and each triggered bug
causes failure in a variety of different stack states.  \rhythmbox and \exif
bugs caused crashes so long after the bad behavior that the crash stacks
were not useful at all.

Studies that attempt real-world deployment of monitored software must
address a host of practical engineering concerns, from distribution to
installation to user support to data collection and warehousing.
Elbaum and Hardojo \cite{Elbaum:2003:DISATA} have reported on a
limited deployment of instrumented Pine binaries.  Their experiences
have helped to guide our own design of a wide public deployment of
applications with sampled instrumentation, presently underway
\cite{Liblit:2003:CBIP}.

For some highly available systems, even a single failure must be
avoided.  Once the behaviors that predict imminent failure are known,
automatic corrective measures may be able to prevent the failure from
occurring at all.  The Software Dependability Framework (SDF)
\cite{Gross:2003:PSMUST} uses multivariate state estimation
techniques to model and thereby predict impending system failures.
Instrumentation is assumed to be complete and is typically
domain-specific.

\issue[Alex]{There is a new Ernst paper in FSE and there was another
  one in ICSE.  I'm not sure either is really relevant, but if we are
  going to cite him we should show awareness of the more recent work.}

\issue[Alex]{I don't know what Orso has been doing other than that he
  told me they have done a study on whether code coverage in the field
  is similar to code coverage in testing; we should cite that one.
  It's on his home page, I'm pretty sure (Alessandro Orso, I think, at
  Georgia Tech).}

\issue[Mayur]{Add references to Ernst and Orso.  [I think we already
  have enough references to these folks and the related work section
  is already quite long.  Let us give this the least priority.]}

\subsection{Comparison with Logistic Regression}
\label{sec:comparison}

\begin{table}
\caption{Results of logistic regression for \moss}
\label{tab:logregression}
\centering
\small
\begin{tabular}{ll}
  \toprule
  Coefficient & Predicate \\
  \midrule
  0.769379 & \verb|p + passage_index)->last_line < 4| \\
  0.686149 & \verb|(p + passage_index)->first_line < i| \\
  0.675982 & \verb|i > 20| \\
  0.671991 & \verb|i > 26| \\
  0.619479 & \verb|(p + passage_index)->last_line < i| \\
  0.600712 & \verb|i > 23| \\
  0.591044 & \verb|(p + passage_index)->last_line == next| \\
  0.567753 & \verb|i > 22| \\
  0.544829 & \verb|i > 25| \\
  0.536122 & \verb|i > 28| \\
  \bottomrule
\end{tabular}
\end{table}

As discussed in \autoref{sec:introduction}, in earlier work 
we used $\ell_1$-regularized logistic regression
to rank the predicates by their
failure-prediction strength \cite{PLDI`03*141,NIPS2003_AP05}.  Independently others have used
logistic regression with good results to cluster program failures (without
identifying causes) related to the same bug \cite{ICSE`03*465}.  However, we have come
to believe that logistic regression has scalability problems in our application.

Logistic regression uses linearly weighted
combinations of predicates to classify a trial run as successful or
failed.  Regularized logistic regression incorporates a penalty
forcing most coefficients to be set to zero, thereby
selecting only the most important predicates.  The output is a set of
coefficients for predicates giving the best overall prediction.

A weakness of logistic regression for our application is that it seeks
to cover the set of failing runs without regard to the orthogonality
of the selected predicates (i.e., whether they represent distinct
bugs).  This problem can be seen in \autoref{tab:logregression},
which gives the top ten predicates selected by logistic regression
for \moss.  The striking fact is that all selected predicates are
either sub-bug or super-bug predictors.  The predicates beginning {\tt
p + \ldots} are all sub-bug predictors of bug \#1 (see
\autoref{tab:mossdilute}).  The predicates {\tt i > \ldots} are
super-bug predictors: {\tt i} is the length of the command line and
the predicates say program crashes are more likely for long command
lines (recall \autoref{sec:introduction}).

The prevalence of super-bug predictors on the list shows the
difficulty of making use of the penalty term.  Limiting the number of
predicates that can be selected via a penalty has the effect of
encouraging logistic regression to choose super-bug predictors, as
these cover more failing runs at the expense of poorer predictive
power compared to predictors of individual bugs.  On the other hand,
some of the predicates are apparently excellent sub-bug predictors,
and are therefore chosen over others.
%%Relaxing the penalty
%%allows logistic regression to add more predicates to improve its
%%prediction, but the sub-bug predictors apparently are favored.

\section{Conclusions}
\label{sec:conclusions}

We have demonstrated a practical, scalable algorithm for isolating multiple bugs
in complex software systems.  Our experimental results show that we can
detect a wide variety of both anticipated and unanticipated causes of failure
in realistic systems and do so with a relatively modest number of program
executions.

{\small
\bibliography{cacm1990,gcbib,icse02,icse03,misc,nips16,paste02,pldi03,pods,ramss,refs}
}
\end{document}

%% LocalWords:  DIDUCE Burnell Horvitz Podgurski Elbaum Hardojo SDF
%% LocalWords:  topcrash cacm icse ramss pldi Podgurski's Kanduri
%% LocalWords:  McMaster Umranov Votta
