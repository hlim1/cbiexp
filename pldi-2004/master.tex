%% -*- TeX-master: t -*-

%% \documentclass{acm_proc_article-sp}
\documentclass{sig-alternate}


%%%%%%%%%%%%%%%%%%%%%%%%%%%%%%%%%%%%%%%%%%%%%%%%%%%%%%%%%%%%%%%%%%%%%%%%
%%
%% standard texmf packages
%%

\usepackage{nicefrac}
\usepackage{fancyvrb}
\usepackage{xspace}

\usepackage[bookmarks, pdftitle={Bug Isolation in the Presence of
  Multiple Errors}, pdfauthor={Ben Liblit, Mayur Naik, Alice X.
  Zheng, Alex Aiken, and Michael I.  Jordan}, pdfsubject={D.2.4
  [Software Engineering]: Software/Program Verification -- statistical
  methods; D.2.5 [Software Engineering]: Testing and Debugging --
  debugging aids, distributed debugging, monitors, tracing; I.5.2
  [Pattern Recognition]: Design Methodology -- feature evaluation and
  selection}, pdfkeywords={bug isolation, random sampling, invariants,
  feature selection, statistical debugging},
pdfpagemode=UseOutlines]{hyperref}


%%%%%%%%%%%%%%%%%%%%%%%%%%%%%%%%%%%%%%%%%%%%%%%%%%%%%%%%%%%%%%%%%%%%%%%%
%%
%% unique to this paper
%%

\usepackage{Autoref}

\usepackage{amsmath}

%% assorted handy macros
\newcommand{\moss}{\textsc{Moss}\xspace}
\newcommand{\termdef}[1]{\textit{#1}}
\newcommand{\prob}{\mbox{\textit{Prob}}}
\newcommand{\fail}{\mbox{\textit{Fail}}}
\newcommand{\crash}{\mbox{\textit{Crash}}}
\newcommand{\context}{\mbox{\textit{Context}}}
\newcommand{\increase}{\mbox{\textit{Increase}}}
\renewcommand{\H}{{\mathcal{H}}}


%%%%%%%%%%%%%%%%%%%%%%%%%%%%%%%%%%%%%%%%%%%%%%%%%%%%%%%%%%%%%%%%%%%%%%%%
%%
%% remove the following before submission!
%%

\usepackage{color}
\newcommand{\placeholder}[1]{{\color[cmyk]{0,0.61,0.87,0}[#1]}}
\newcommand{\comment}[1]{}
\pagenumbering{arabic}


%%%%%%%%%%%%%%%%%%%%%%%%%%%%%%%%%%%%%%%%%%%%%%%%%%%%%%%%%%%%%%%%%%%%%%%%
%%
%% front matter
%%

\title{Bug Isolation in the Presence of Multiple Errors
  %%
  \thanks{This research was supported in part by NASA Grant No.\ 
    NAG2-1210; NSF Grant Nos.\ EIA-9802069, CCR-0085949, ACI-9619020,
    and IIS-9988642; and DOE Prime Contract No.\ W-7405-ENG-48 through
    Memorandum Agreement No.\ B504962 with LLNL.  The information
    presented here does not necessarily reflect the position or the
    policy of the Government and no official endorsement should be
    inferred.}}

\numberofauthors{3}

\makeatletter
\newcommand*{\eecsMark}[0]{\@fnsymbol{2}}
\newcommand*{\statMark}[0]{\@fnsymbol{3}}
\newcommand*{\stanMark}[0]{\@fnsymbol{4}}
\makeatother
\newcommand*{\eecs}[0]{\textsuperscript{\eecsMark}}
\newcommand*{\stat}[0]{\textsuperscript{\statMark}}
\newcommand*{\both}[0]{\textsuperscript{\eecsMark, \statMark}}
\newcommand*{\stan}[0]{\textsuperscript{\stanMark}}

\newcommand{\moreauthors}[0]{\end{tabular}\\\vspace{-.5\baselineskip}\begin{tabular}{c}}

\author{
  \alignauthor Ben Liblit \eecs \\
  \alignauthor Mayur Naik \stan \\
  \alignauthor Alice X.\ Zheng \eecs \\
  \moreauthors
  \global\multiply\auwidth by 3
  \global\divide\auwidth by 2
  \alignauthor Alex Aiken \stan \\
  \alignauthor Michael I.\ Jordan \both
  \moreauthors
  \alignauthor
  \affaddr{\eecs Department of Electrical \\ Engineering and Computer Science} \\
  \affaddr{\stat Department of Statistics} \\
  \affaddr{University of California, Berkeley} \\
  \affaddr{Berkeley, CA 94720-1776}
  \alignauthor
  \affaddr{\stan Computer Science Department} \\
  \affaddr{353 Serra Mall} \\
  \affaddr{Stanford University} \\
  \affaddr{Stanford CA 94305-9025}
}

\bibliographystyle{abbrv}


%%%%%%%%%%%%%%%%%%%%%%%%%%%%%%%%%%%%%%%%%%%%%%%%%%%%%%%%%%%%%%%%%%%%%%%%
%%
%%  document body
%%

\begin{document}

\conferenceinfo{PLDI'04,}{June 9--11, 2004, Washington, DC, USA.}
\CopyrightYear{2004}
%% \crdata{}
\maketitle

\begin{abstract}
\placeholder{Abstract needed here.}
\end{abstract}

\category{D.2.4}{Software Engineering}{Software/Program
  Verification}[statistical methods]
%%
\category{D.2.5}{Software Engineering}{Testing and
  Debugging}[debugging aids, distributed debugging, monitors, tracing]
%%
\category{I.5.2}{Pattern Recognition}{Design Methodology}[feature
  evaluation and selection]

\terms{Experimentation, Reliability}

\keywords{bug isolation, random sampling, invariants, feature
  selection, statistical debugging}


\section{Introduction}
\label{sec:introduction}

Programs are buggy.  We all know this, and yet we use them anyway.
The commercial reality is that most software ships with many known
bugs and untold numbers of bugs not yet discovered.  Software
engineers understand that it is neither practical nor feasible to
delay releasing code until the last bug has been fixed.

Statistical debugging is a body of tools and techniques that help to
improve software quality in an imperfect world.  Specially
instrumented programs monitor their own behavior and send feedback
reports to a central collection server.  Monitoring is both sparse and
random, which means that complete information is never available about
any single run.  However, monitoring is also lightweight and therefore
practical to deploy to user communities numbering in the thousands or
millions.  Statistical debugging does not seek to understand the
single cause of a single failure on one machine; rather, it identifies
broad statistical trends that isolate bug causes across many thousands
or millions of runs.  Because the process is driven by data from real
users, it implicitly attends to those program behaviors that cause the
most problems for the most users, most often.

Of course, our imperfect world is rarely so kind as to provide
programs with only a single bug.  A single complex commercial
application may have hundreds or thousands of latent flaws waiting to
be triggered.  Multiple bugs may come into play even within a single
run of an application.  Some errors may allow execution to continue
more-or-less normally, while one may eventually cause a crash.
Consider the fact that the Mozilla web browser project has a
``topcrash'' software QA team specifically dedicated to identifying
and tracking the top forty most common crash bugs.  This cut-off is
not because anyone believes that there are only forty bugs; it is an
acknowledgment that engineering resources are finite so one should
target the most important bugs first.  Fix one bug and another comes
into view.

Statistical isolation of a single bug is well understood
\cite{Zheng:2003:SDSP}.  In general terms, one looks for predicates on
program behavior which are strongly correlated with the binary outcome
of bug/no-bug, or crash/no-crash.  Both sparse sampling and
nondeterministic failure modes add noise to the process, but with a
large population of runs consistent trends emerge.  The presence of
multiple bugs complicates matters significantly, especially if we
assume that the number of distinct bugs is itself unknown.  Even if we
believe we have identified the direct cause of one bug, and even if we
had complete instead of sampled data, we should expect that the
``smoking gun'' will be absent from most failed runs, because no one
bug accounts for more than some small fraction of failures.

This paper presents an effective approach to isolating multiple bugs
in the face of such challenges.  Our discussion is organized as
follows.  \Autoref{sec:background} reviews the sampling
instrumentation infrastructure and the kind of feedback data that it
provides.  In \Autoref{sec:algorithm} we describe the main filtering
and ranking algorithm used to mine feedback data for bug causes.
\Autoref{sec:experiments:setup} defines the parameters of a case study
in multiple-bug isolation, while \Autoref{sec:experiments:results}
presents our results.  We review related work in
\Autoref{sec:related-work}, while \Autoref{sec:conclusions} concludes.

\section{Background}
\label{sec:background}

Software errors arise from unexpected interactions between inputs,
program internal state, and dynamic decisions made as the program
runs.  Software engineers reason about programs at the level of source
code statements and variables, and this is the level at which program
monitoring can expose properties of interest.  However, complete
source-level tracing of program behavior is impractical for field
deployment.  No end user would accept the performance overhead or
network bandwidth required to collect and transmit such a trace.

Instead, we use a combination of sparse random sampling and
client-side summarization.  The former controls performance overhead
while the later limits storage and transmission requirements.
However, each also adds noise and uncertainty to the resulting data.
The particulars of the sampling transformation have been reported
elsewhere \cite{PLDI`03*141}.  We highlight a few key properties here.
First, the sampling transformation is quite general: any collection of
statements within (or added to) a program may be designated as
``instrumentation'' and thereby sampled instead of run
unconditionally.  Second, samples are taken in a statistically fair
manner equivalent to a Bernoulli process: each potential sample is
taken or skipped randomly and independently as the program runs.
Lastly, the technique is highly effective at controlling performance
provided that sampling is sparse: average rates on the order of one
sample per hundred opportunities keep overhead low, often
unmeasurable.

The difficulty in using sparse sampling is that the resultant data is
both noisy and incomplete.  When \nicefrac{99}{100} of all monitored
program behaviors are omitted from any given report, it is impossible
to have a complete picture of what happened during a single run.  Some
predicate of interest may never have been true, or may simply have
been true but never recorded.  However, because sampling is fair,
large numbers of runs do converge on a truthful representation of
overall program behavior.

Orthogonal to the sampling transformation is the decision about what
instrumentation to introduce in the first place and how to concisely
summarize the resultant data.  This determines what behaviors one can
possibly observe once sampling is applied.  A useful instrumentation
scheme should be fairly general but selected to capture behaviors
which are likely to be of interest when hunting for bugs.  At present
our system offers the following instrumentation schemes:

\begin{description}
\item[branches:] Control flow is interesting.  At each conditional,
  count how often each branch is taken (equivalently, how often the
  conditional predicate is true versus false).  The observation is
  made just after the predicate is evaluated but before the selected
  branch is taken.  This scheme also applies to implicit conditionals
  in loops and logical operators (\texttt{\&\&}, \texttt{||},
  \texttt{?:}).  Each conditional induces one instrumentation site
  with two counters: number of times branched true and number of times
  branched false.

\item[returns:] Function return values are interesting.  In C, the
  sign of a result often used to encode the success or failure of some
  operation.  At each scalar-returning function call site, count how
  often the returned value is negative, zero, or positive.  For
  pointer-returning calls, this implicitly reduces to counting
  \texttt{NULL} versus non-\texttt{NULL}.  The observation is made
  just after the function returns but before the result is used by the
  original program.  An instrumentation site is added even if the
  source program discarded the return value, as unchecked return
  values are a common source of bugs.  Each call site induces one
  instrumentation site with three counters: number of negative
  returns, number of zero returns, and number of positive returns.

\item[scalar-pairs:] Values of variables are interesting.  Many bugs
  concern boundary issues exposed by the relationship between a pair
  of variables, or between a variable and some program constant.  At
  each scalar assignment \texttt{x = \dots}, identify each
  \emph{other} same-typed in-scope variable $\mathtt{y}_i$ and each
  constant expression $\mathtt{c}_j$.  Count how often the new value
  for \texttt{x} is less than, equal to, or greater than each
  $\mathtt{y}_i$ and each $\mathtt{c}_j$.  The observation is made
  after both sides of the assignment have been evaluated but just
  before the assignment itself takes place.  This lets us compare
  \texttt{x} to \texttt{x} as well, effectively comparing the new and
  old values of the left-hand side.  Each compared-to $\mathtt{y}_i$
  or $\mathtt{c}_j$ is treated as a distinct instrumentation site;
  thus a single assignment may induce a large number of sites.  Each
  such site maintains three counters: how often the value being
  assigned is less than, equal to, or greater than the compared-to
  variable or constant.
\end{description}

These schemes are quite broad; they represent a large set of wild
guesses as to what behavior may be interesting.  Nearly all of them
are wrong or irrelevant for any particular bug.  Engineers may enable
or disable any mix of schemes in a single binary, and may include or
exclude entire regions of code on a per-file or per-function basis.
For the most part, though, instrumenting an executable simply requires
switching to our instrumenting compiler.  By design the system
requires minimal human intervention at to build instrumented code.

The schemes described above share certain key properties.  In the
returns and scalar-pairs schemes, each site takes a large space of
possible observations (e.g.\ the exact value returned by a call) and
reduces it to a much smaller group of equivalence classes (e.g.\ the
sign of the returned value).  Furthermore, each site can be treated as
a set of predicates which form a complete, non-overlapping partition
of the behaviors observed by that site.  For example, at each
scalar-pairs comparison between \texttt{x} and \texttt{y}, one and
only one of $\mathtt{x} < \mathtt{y}$, $\mathtt{x} = \mathtt{y}$, or
$\mathtt{x} > \mathtt{y}$ can be true.  Thus one observation at one
site always updates exactly one of that site's counters.  An
observation that $\mathtt{x} < \mathtt{y}$ is therefore equivalent to
an observation that $\mathtt{x} \geq \mathtt{y}$.  This property is
exploited in \Autoref{sec:algorithm} to synthesize new predicates out
of existing observations.

The post-execution feedback report is a dump of the counter values for
each instrumentation site.  Reducing a trace to a set of counters
prevents us from reasoning about relative time ordering of events
during execution.  However, it also means that program actions early
in execution remain just as visible as those much later.  This
contrasts with traditional postmortem debugging tools which expose
only the final state of the program at the point of failure.

\section{Cause Isolation Algorithm}
\label{sec:algorithm}
This section presents our algorithm for automatically isolating
multiple bugs.  As discussed in \autoref{sec:background}, the input is
a set of feedback reports from individual program runs $R$, where
$R(P) = 1$ if predicate $P$ is observed to be true during the
execution of $R$.

The idea behind the algorithm is to simulate the iterative manner in which  human programmers
typically find and fix bugs:
\begin{enumerate}

\item Identify the cause of the most important bug ${\cal B}$.

\item Fix ${\cal B}$, and repeat.

\end{enumerate}

For our purposes, identifying the cause of a bug ${\cal B}$ means selecting a
predicate $P$ closely correlated with ${\cal B}$.  The difficulty is that we
know the set of runs that succeed and fail, but we do not know which
set of failing runs corresponds to ${\cal B}$, or even how many bugs there
are.  Thus, in the first step we must infer which predicates are most
likely to correspond to individual bugs and rank those predicates in
importance.

For the second step, while we cannot literally fix the bug
corresponding to the chosen predictor $P$, we can simulate what
happens if the bug does not occur.  We discard any run $R$ such that
$R(P) = 1$ and repeat.  Discarding all the runs where $R(P) = 1$
reduces the importance of other predictors of ${\cal B}$, allowing predicates
that predict different bugs (i.e., different sets of failing runs) to
rise to the top in subsequent iterations.

\subsection{Increase Scores}
\label{sec:increase}

We now discuss the first step, how to find the cause of the most important bug.
We break this step into two sub-steps.  First, we eliminate predicates that have no
predictive power at all; this typically reduces the number of predicates we need
to consider by two orders of magnitude (e.g., from hundreds of thousands to thousands).
Next, we rank the surviving predicates by importance (see \autoref{sec:ranking}).

Consider the following C code fragment:
\begin{quote}
\begin{verbatim}
f = ...;          (a)
if (f == NULL) {  (b)
        x = 0;    (c)
        *f;       (d)
}
\end{verbatim}
\end{quote}
Consider the predicate {\tt f == NULL} at line {\tt (b)}, which would
be captured by branches instrumentation.  Clearly
this predicate is highly correlated with failure; in fact, whenever it
is true this program inevitably crashes.\footnote{We also note that this bug could 
be detected by a simple static analysis; this example is meant to be concise rather than 
a significant application of our techniques.}   An important observation,
however, is that even a ``smoking gun'' such as {\tt f == NULL} at
line {\tt (b)} cannot be a perfect predictor of failure when there are
multiple bugs in the program---since there are other bugs, the run can fail
even if the predicate is never true in a run.

The bug in the code fragment above is \termdef{deterministic} with
respect to {\tt f == NULL}: if {\tt f == NULL} is true at line {\tt
(b)}, the program fails.  In many cases it is impossible to observe
the exact conditions causing failure; for example, buffer overrun bugs
in a C program may or may not cause the program to crash depending on
runtime system decisions about how data is laid out in memory.  Such
bugs are \termdef{non-deterministic} with respect to every predicate;
even for the best predictor $P$, it is possible
that $P$ is true and still the program terminates normally.  In the
example above, if we insert before line {\tt (d)} a valid pointer
assignment to {\tt f} controlled by a conditional that is true at
least occasionally (say via a call to read input):
\begin{quote}
\begin{verbatim}
if (read()) f = ... some valid pointer ...;
*f;
\end{verbatim}
\end{quote}
the bug becomes non-deterministic with respect to {\tt f == NULL}.

To summarize, even for a predicate $P$ that is truly the cause of a bug, we can neither assume that
when $P$ is true that
the program fails nor that when $P$ is never observed to be true  that
the program succeeds. But we can express the probability that $P$
being true implies failure.  Let $\fail$ be an atomic predicate that is
true for failing runs and false for successful runs.  Let $\prob(A | B)$ denote
the conditional probability function of the event $A$ given event $B$ .  
We want to compute:
% [[ modified by Alice]]
\[ \crash(P) \equiv \prob(\fail = 1 | P = 1) \]
%= \prob(\fail | P \mbox{ is true}) \]
for every instrumented predicate $P$ over the set of all runs.  Let $S(P)$ be the number
of successful runs in which $P$ is observed to be true, and let $F(P)$ be the number of
failing runs in which $P$ is observed to be true.  We
estimate $\crash(P)$ as:
\[ \crash(P) = \frac{F(P)}{S(P) + F(P)} \]

Notice that $\crash(P)$ is unaffected by the set of runs in which
$P$ is not observed to be true.  Thus, if $P$ is the cause of a bug, the
causes of other independent bugs do not affect $\crash(P)$.
Also note that runs in which $P$ is not observed at all (either because
the line of code on which $P$ is checked is not reached, or the line is reached
but $P$ is not sampled) have no effect on $\crash(P)$.
Finally, the definition of $\crash(P)$
generalizes the idea of deterministic and non-deterministic bugs.  A
bug is deterministic for $P$ if $\crash(P) = 1.0$ or, equivalently,
$P$ is never observed to be true in a successful run ($S(P) =
0$) and $P$ is observed to be true in at least one failing run ($F(P) > 0$).
If $\crash(P) < 1.0$ then the bug is non-deterministic, with
lower scores showing weaker correlation between the predicate and
program failure.

Now $\crash(P)$ is a useful measure, but it is not good
enough for the first step of our algorithm. To see this, consider again the
code fragment given above (in its original form, not with the
modification to make the bug non-deterministic).  At line {\tt (b)} we
have $\crash(\mbox{\tt f == NULL}) = 1.0$, so this predicate is a good
candidate for the cause of the bug.
But on line {\tt (c)} we have the surprising fact that $\crash(\mbox{\tt x == 0}) = 1.0$ as well.
To understand why, observe that the 
predicate \texttt{x == 0} is always true at line {\tt (c)} and, in
addition,
only failing runs reach this line.
Thus $S(\mbox{\tt x == 0}) = 0$, and, so long as there is at least one run that
reaches line {\tt (c)} at all, $\crash(\mbox{\tt x == 0})$ at line {\tt (c)} is 1.0.

As the predicate {\tt x == 0} at line {\tt (c)} of the example
shows, just because $\crash(P)$ is high does not
mean $P$ is the cause of a bug.  In the case of {\tt x == 0}, the
decision that eventually causes the crash is made earlier, and the
$\crash(\mbox{\tt x == 0})$ score merely reflects the fact that this
predicate is checked on a path where the program is already doomed.

A way to address this difficulty is to score a predicate not by the chance
that it implies failure, but by how much difference it makes that the predicate
is observed to be true versus simply reaching the line where the predicate is checked.
That is, on line {\tt (c)}, the probability of crashing is already 1.0 regardless
of the value of the predicate {\tt x == 0}, and thus the fact that {\tt x == 0} is
true does not increase the probability of failure at all; this coincides with
the intuition that this predicate is irrelevant to the bug.

This leads us to the following definition:
\[
%% [[modified by Alice]]
\context(P) \equiv \prob(\fail = 1 | P \mbox{ is observed})  
\]
%= \prob(\fail |  P \mbox{ observed}) 
%\end{gather*}
Now, $P \lor \lnot P$ is not the set of all runs, because we are not working in a two-valued logic.
In any given run, neither $P$ nor $\lnot P$ may be observed (because the site where this predicate is
sampled is not reached), or one may be observed, or both may be observed (because the statement is executed
multiple times and $P$ is sometimes true and sometimes false).  Thus, $\context(P)$ is the probability that
in the set of runs where the value of $P$ is observed at all, the program fails. We can compute $\context(P)$ as follows:
\[ \context(P) = \frac{F(P \lor \lnot P)}{S(P \lor \lnot P) + F(P \lor \lnot P)} \]

The interesting quantity, then, is
\begin{equation*}
 \increase(P) \equiv \crash(P) - \context(P) \label{eqn:1}
\end{equation*}
%%
which can be read as: How much does $P$ being true increase the probability of failure
over simply reaching the line where $P$ is sampled?  For example, for the predicate {\tt x == 0} on line {\tt (c)},
we have
\[\crash(\mbox{\tt x == 0}) = \context(\mbox{\tt x == 0}) = 1.0 \]
and so $\increase(\mbox{\tt x == 0}) = 0$.

A predicate $P$ with $\increase(P) \leq 0$ has no predictive power; whether it is true does not increase the
probability of failure, and we can safely discard all such predicates.
But because some $\increase(P)$ scores may be based on few observations of $P$, it is important
to attach confidence intervals to the scores.  Since $\increase()$ is a statistic, computing
its confidence interval is a well-understood problem. In our experiments we retain a predicate $P$
only if $\increase(P) > 0$ with 95\% confidence; this removes predicates from consideration that have high
increase scores but very low confidence because of few observations. 

Pruning predicates using $\increase(P) \leq 0$ has many desirable
properties.  It is easy to prove that large classes of irrelevant
predicates always have scores $\leq 0$.  For example, any predicate
that is unreachable, that is a program invariant, or that is obviously
control-dependent on a true cause is eliminated by this test.  It is
also worth pointing out that this test tends to localize bugs at
a point where the condition that causes the bug becomes true, rather than at
the crash site.  For example, in the code fragment given above, the bug is
attributed to the success of the conditional branch test {\tt f ==
NULL} on line {\tt (b)} rather than the pointer dereference on line
{\tt (d)}.  Thus, the cause of the bug discovered by the algorithm
points directly to the conditions under which the crash occurs, rather than
the line on which it occurs (which is usually available anyway in the
stack trace).

\subsection{Statistical Interpretation}
\label{sec:statisticalinterpretation}

We have explained the test $\increase(P) > 0$ using programming terminology,
but it also has a natural statistical interpretation as a simplified {\em likelihood ratio} hypothesis
test.  Consider the two classes of trial runs
of the program: failed runs $F$ and successful runs $S$.  For each
class, we can treat the predicate $P$ as a Bernoulli random variable
with heads probabilities $\pi_f(P)$ and $\pi_s(P)$, respectively, for the
two classes.  The heads
probability is the probability that the predicate is observed to be
true.  If a predicate causes a set of crashes, then $\pi_f$ should be
much bigger than $\pi_s$.  We can formulate two statistical hypotheses:
the null hypothesis $\H_0:
\pi_f \leq \pi_s$, versus the alternate hypothesis $\H_1: \pi_f > \pi_s$.  Since
$\pi_f$ and $\pi_s$ are not known, we must estimate them:
\begin{align*}
  \hat \pi_f(P) &= \frac{F(P)}{F(P \lor \lnot P)} &
  \hat \pi_s(P) &= \frac{S(P)}{S(P \lor \lnot P)}
\end{align*}

Although these proportion estimates of $\pi_f$ and $\pi_s$ approach the
actual heads probabilities as we increase the number of trial runs, they
still differ due to sampling.  With a certain probability, using these
estimates instead of the actual values results in the wrong
answer.  A \textit{likelihood ratio test} takes this uncertainty into
account, and makes use of the statistic $ Z = \frac{(\hat \pi_f - \hat
  \pi_s)}{V_{f,s}}$, where $V_{f,s}$ is a sample variance term (see
e.g., \cite{Lehmann:1986:hyptest}).  When
the data size is large, $Z$ can be approximated as a standard Gaussian
random variable.  Performed independently for each predicate $P$, the
test decides whether or not $\pi_f(P) \leq \pi_s(P)$ with a guaranteed
false-positive probability (i.e.,\ choosing $\H_1$ when $\H_0$ is true).
A necessary (but not sufficient) condition for choosing $\H_1$ is that
$\hat \pi_f(P) > \hat \pi_s(P)$.  This turns out to be
equivalent to the condition that $\increase(P) > 0$.  To see this,
let $a = F(P)$, $b = S(P)$, $c = F(P\lor\lnot P)$, and $d = S(P\lor\lnot P)$.
Then
\begin{gather*}
  \increase(P) > 0 \iff \crash(P) > \context(P) \\
  \iff \frac{a}{a+b} > \frac{c}{c+d}
  \iff a (c+d) > (a+b) c \\
  \iff ad > bc \iff \frac{a}{c} > \frac{b}{d}
  \iff \hat \pi_f(P) > \hat \pi_s(P)
\end{gather*}

%% -*- TeX-master: "../master.tex" -*-

\begin{table*}[tb]
  \nocaptionrule
  \caption{Comparison of ranking strategies for \moss without
    redundancy elimination}
  \label{tab:sorts}
  \centering

  \newenvironment{stats}[2]{%
    \begin{subtab}
      \caption{Sort descending by #1}
      \small
      \label{tab:sorts-#2}
      \begin{tabular}{l|rr@{$\:\pm\:$}rr|rrl}
        \toprule
        Thermometer & Context & \multicolumn{2}{r}{Increase} &
        S & F & F + S & Predicate \\
        \midrule}{%
        \bottomrule
      \end{tabular}
    \end{subtab}}

  \begin{stats}{$\text{F}(P)$}{failure}
    \bugometer{4.44089}{0.17602}{0.000535293}{0.006240707}{0.817204} & 0.176 & 0.007 & 0.012 & 22554 & 5045 & 27599 & \verb|files[filesindex].language &#8800; 15| \\
\bugometer{4.44108}{0.175879}{0.000600451}{0.006237549}{0.817283} & 0.176 & 0.007 & 0.012 & 22566 & 5045 & 27611 & \verb|tmp == 0 is FALSE| \\
\bugometer{4.44116}{0.175904}{0.000543038}{0.006236962}{0.817316} & 0.176 & 0.007 & 0.012 & 22571 & 5045 & 27616 & \verb|strcmp &#8800; 0| \\
\bugometer{4.36446}{0.176292}{0.000813956}{0.006562044}{0.816332} & 0.176 & 0.007 & 0.013 & 18894 & 4251 & 23145 & \verb|tmp == 0 is FALSE| \\
\bugometer{4.36408}{0.176018}{0.00077294}{0.00656006}{0.816649} & 0.176 & 0.007 & 0.013 & 18885 & 4240 & 23125 & \verb|files[filesindex].language &#8800; 14| \\
\bugometer{4.33774}{0.175829}{0.00160129}{0.00668071}{0.815889} & 0.176 & 0.008 & 0.013 & 17757 & 4007 & 21764 & \verb|filesindex &#8805; 25| \\
\bugometer{4.30501}{0.176949}{0.000967379}{0.006932621}{0.815151} & 0.177 & 0.008 & 0.014 & 16453 & 3731 & 20184 & \verb|M < M| \\
\bugometer{3.93024}{0.175776}{0.249223}{0.011356}{0.563645} & 0.176 & 0.261 & 0.023 & 4800 & 3716 & 8516 & \verb|config.winnowing_window_size &#8800; argc| \\
\multicolumn{8}{c}{\dotfill{} 2732 additional predictors follow \dotfill{}} \\

  \end{stats}

  \begin{stats}{$\text{Increase}(P)$}{increase}
    \bugometer{1.36173}{0.0652963}{0.924982}{0.0097217}{0} & 0.065 & 0.935 & 0.019 & 0 & 23 & 23 & \verb|((*(fi + i)))->this.last_token < filesbase| \\
\bugometer{1}{0.0652535}{0.924878}{0.0098685}{0} & 0.065 & 0.935 & 0.020 & 0 & 10 & 10 & \verb|((*(fi + i)))->other.last_line == last| \\
\bugometer{1.25527}{0.0713427}{0.918557}{0.0101003}{0} & 0.071 & 0.929 & 0.020 & 0 & 18 & 18 & \verb|((*(fi + i)))->other.last_line == filesbase| \\
\bugometer{1}{0.0727273}{0.917042}{0.0102307}{0} & 0.073 & 0.927 & 0.020 & 0 & 10 & 10 & \verb|((*(fi + i)))->other.last_line == yy_n_chars| \\
\bugometer{1.27875}{0.0714847}{0.914362}{0.0141533}{0} & 0.071 & 0.929 & 0.028 & 0 & 19 & 19 & \verb|bytes <= filesbase| \\
\bugometer{1.14613}{0.0751634}{0.913671}{0.0111656}{0} & 0.075 & 0.925 & 0.022 & 0 & 14 & 14 & \verb|((*(fi + i)))->other.first_line == 2| \\
\bugometer{1.07918}{0.0764302}{0.912429}{0.0111408}{0} & 0.076 & 0.924 & 0.022 & 0 & 12 & 12 & \verb|((*(fi + i)))->this.first_line < nid| \\
\bugometer{1}{0.077222}{0.911248}{0.01153}{0} & 0.077 & 0.923 & 0.023 & 0 & 10 & 10 & \verb|((*(fi + i)))->other.last_line == yy_init| \\
\multicolumn{8}{c}{\dotfill{} 2732 additional predictors follow \dotfill{}} \\

  \end{stats}

  \begin{stats}{harmonic mean}{harmonic}
    \bugometer{3.20003}{0.176135}{0.819603}{0.004262}{0} & 0.176 & 0.824 & 0.009 & 0 & 1585 & 1585 & \verb|files[filesindex].language > 16| \\
\bugometer{3.19976}{0.176195}{0.819543}{0.004262}{0} & 0.176 & 0.824 & 0.009 & 0 & 1584 & 1584 & \verb|strcmp > 0| \\
\bugometer{3.19866}{0.17595}{0.819792}{0.004258}{0} & 0.176 & 0.824 & 0.009 & 0 & 1580 & 1580 & \verb|strcmp == 0| \\
\bugometer{3.19783}{0.175935}{0.819808}{0.004257}{0} & 0.176 & 0.824 & 0.009 & 0 & 1577 & 1577 & \verb|files[filesindex].language == 17| \\
\bugometer{3.19756}{0.17599}{0.819751}{0.004259}{0} & 0.176 & 0.824 & 0.009 & 0 & 1576 & 1576 & \verb|tmp == 0 is TRUE| \\
\bugometer{3.19673}{0.175904}{0.819838}{0.004258}{0} & 0.176 & 0.824 & 0.009 & 0 & 1573 & 1573 & \verb|strcmp > 0| \\
\bugometer{2.8893}{0.115811}{0.877037}{0.005862}{0.00129} & 0.116 & 0.883 & 0.012 & 1 & 774 & 775 & \verb|((*(fi + i)))->this.last_line == 1| \\
\bugometer{2.89042}{0.116206}{0.876638}{0.005869}{0.001287} & 0.116 & 0.883 & 0.012 & 1 & 776 & 777 & \verb|((*(fi + i)))->other.last_line == yyleng| \\
\bugometer{3.10585}{0.110724}{0.818742}{0.013324}{0.05721} & 0.111 & 0.832 & 0.027 & 73 & 1203 & 1276 & \verb|config.match_comment is TRUE| \\
\bugometer{2.88649}{0.115628}{0.877204}{0.005869}{0.001299} & 0.116 & 0.883 & 0.012 & 1 & 769 & 770 & \verb|((*(fi + i)))->other.last_line == yy_start| \\
\bugometer{2.89042}{0.118305}{0.874501}{0.005907}{0.001287} & 0.118 & 0.880 & 0.012 & 1 & 776 & 777 & \verb|((*(fi + i)))->other.last_line < 2| \\
\bugometer{2.88818}{0.117597}{0.875205}{0.005904}{0.001294} & 0.118 & 0.881 & 0.012 & 1 & 772 & 773 & \verb|((*(fi + i)))->other.last_line == 1| \\
\bugometer{2.88762}{0.117597}{0.87521}{0.005898}{0.001295} & 0.118 & 0.881 & 0.012 & 1 & 771 & 772 & \verb|((*(fi + i)))->this.last_line == yy_start| \\
\bugometer{2.88649}{0.117501}{0.8753}{0.0059}{0.001299} & 0.118 & 0.881 & 0.012 & 1 & 769 & 770 & \verb|((*(fi + i)))->this.last_line < 2| \\
\bugometer{2.88874}{0.117651}{0.873345}{0.00642}{0.002584} & 0.118 & 0.880 & 0.013 & 2 & 772 & 774 & \verb|((*(fi + i)))->this.last_line == yyleng| \\
\bugometer{2.89597}{0.116671}{0.867635}{0.00807}{0.007624} & 0.117 & 0.876 & 0.016 & 6 & 781 & 787 & \verb|((*(fi + i)))->this.last_line == diff| \\
\multicolumn{8}{c}{\dotfill{} 2724 additional predictors follow \dotfill{}} \\

  \end{stats}

\end{table*}


\subsection{Ranking Predicates: Precision vs. Recall}
\label{sec:ranking}

We now turn to the question of ranking the predicates.
\autoref{tab:sorts} shows the top
predicates under different ranking schemes (explained below) for one of our
experiments.  We use
a simple {\em thermometer} to visualize the information for each
predicate.  The length of the thermometer shows the number of runs in
which the predicate was observed, plotted on a log scale (so small increases
in thermometer size indicate many more runs).  The thermometer has
a sequence of bands:
the black band on the left shows the context score;
the next lighter band shows the increase score, the next, even  lighter band (which is either not
visible or very small in all thermometers) shows the confidence
interval, and the white space at the right end of the thermometer
shows the successful runs in which the predicate was
observed to be true.  The tables show the thermometer as well as the
numbers for each of the quantities that make up the thermometer. 

The most important bug is the one that causes the greatest number
of failed runs.  This observation suggests:
\[ \importance(P) = F(P) \]
\autoref{tab:sorts-failure} shows the top predicates 
ranked by decreasing $F(P)$.\footnote{These predicates are ranked after predicates where $\increase(P) \leq 0$ are discarded.}
While the predicates in \autoref{tab:sorts-failure} are, as expected, involved
in many failing runs, these predicates are also highly 
non-deterministic, meaning they are also true in many successful runs
and are weakly correlated with bugs.  

Our experience with other
ranking strategies that emphasize the number of failed runs is also that
they select predicates involved in many failing, but also many
successful, runs.  The best of these predicates (the ones with high
$\increase()$ scores) are {\em super-bug predictors}:
predictors that include failures from more than one bug.  The
signature of super-bug predictors is that they account for a very
large number of failures (by combining the failures of multiple bugs)
but are also highly non-deterministic despite reasonably high $\increase()$
scores.

Another possibility is:
\[ \importance(P) = \increase(P) \]
\autoref{tab:sorts-increase} shows the top predicates ranked by decreasing
$\increase()$ score (note the thermometers in the different figures
are not drawn to the same scale).  These predicates do a much better
job of predicting failure; in fact, the program always fails when any
of these predicates is true. However, note that the number of failing
runs is very small.  These predicates are {\em sub-bug predictors}:
predictors for a subset of the failures caused by a bug.  Unlike
super-bug predictors, which are not useful in our experience, sub-bug
predictors that account for a significant fraction of the failures for
a bug often provide valuable clues, but still they represent special
cases and may not suggest other, more fundamental, causes of the bug.

Tables~\ref{tab:sorts-failure} and \ref{tab:sorts-increase} illustrate the difficulty of defining
``importance''.  It is helpful to consider this problem in the language of information
retrieval. We are looking for predicates with
high {\em recall} (meaning predicates that account for many failed runs) but also high {\em precision}
(meaning predicates that do not mispredict failure in many successful runs).  In statistics,
the corresponding terms are {\em selectivity} and {\em sensitivity}.  In both fields,
the standard metric is the harmonic mean of precision and recall, which disproportionately
rewards high scores in both dimensions.  In our
case, $\increase(P)$ measures precision and $F(P)$ measures recall.  Define
%%
\[ \importance(P) = \frac{2}{\frac{1}{\increase(P)} + \frac{log(\numfail)}{log(F(P))}} \]
%%
where $\numfail$ is total number of failing runs (independent of $P$)
and therefore scales recall to between zero and one.

The results are given in \autoref{tab:sorts-harmonic}.  We observe that
all of the predicates on this list indeed have both high precision and
recall, accurately describing a large number of failures. 


\subsection{Predicate Elimination}
\label{sec:elimination}

The remaining problem with the results in \autoref{tab:sorts-harmonic}
is that there is substantial redundancy; it is easy to see that several of these
predicates are related hiding other, distinct bugs that either have
fewer failed runs or more non-deterministic predictors further down the
list.  As discussed above, we use a simple recursive algorithm to eliminate
redundant predicates:
\begin{enumerate}

\item Rank predicates by $\importance()$.

\item Remove the top-ranked predicate $P$ and discard all runs $R$ (feedback reports) where  $R(P) = 1$.

\item Repeat (1) and (2) until the set of runs is empty or the set of predicates is empty.
\end{enumerate}




We can now state an easy, but important, property of this algorithm.  
\begin{lemma}
\rm
Let $P_1,\ldots,P_n$ be a set of instrumented predicates and let ${\cal B}_1,\ldots,{\cal B}_m$ be a set of bugs.  Let
\[ {\cal Z} = \bigcup_{1 \leq i \leq n} \{ R | R(P_i) = 1 \} \]
Then if for all $1 \leq j \leq m$ we have ${\cal B}_j \cap {\cal Z} \neq \emptyset$, then 
the algorithm chooses at least one predicate from the list $P_1,\ldots,P_n$ that predicts
at least one failure due to ${\cal B}_j$.
\end{lemma}

Thus, the elimination algorithm chooses at least one predicate 
predictive of each bug represented by the input set of predicates.
The other property we might like, that the algorithm chooses exactly
one predicate to represent each bug, does not hold; we shall see in
\autoref{sec:experiments} that the algorithm sometimes selects a
strong sub-bug predictor as well as a more natural predictor.  Besides
always representing each bug, the algorithm works well for two other
reasons.  First, two predicates are redundant if they predict the same
(or nearly the same) set of failing runs.  Thus, simply removing the
set of runs in which a predicate is true automatically dramatically
reduces the importance of any related predicates in the correct
proportions. Second, because elimination is
iterative, it is only necessary that $\importance()$ selects a good
predictor at each step, and not necessarily the best one; any
predicate that predicts a different set of failing runs than all
higher-ranked predicates is selected eventually.




\section{Experimental Design}
\label{sec:experiments:setup}
This section discusses background for the experimental results
reported in \Autoref{sec:experiments:results}.

%While all software
%experiments are difficult to do well, we have learned the hard way
%that there are some particular problems that must be addressed to do
%our experiment well.
While all software experiments are difficult to design, we have
learned the hard way that there are some particular problems that must
be addressed in order to design our experiment correctly.
Thus, this section discusses the set-up for our
experiment in some detail, especially how we have compensated for
potential sources of bias.

The basic framework of our experiment is a straightforward five step
process:
\begin{enumerate}
\item select an existing software application,
\item modify the source code to inject bugs into the program,
\item instrument the modified program,
\item gather results from a large number of runs performed with automatically generated data and
\item apply our algorithm to the results.
\end{enumerate}
We discuss each step; the reader who is not interested in
these details may wish to proceed to \Autoref{sec:experiments:results} and
use this section only for reference.

We chose \moss\ \cite{Schleimer:2003:WLA} as our benchmark program.  \moss\ is a
software plagiarism detection service\footnote{That is,
\moss\ detects copying in large sets of programs.  The typical \moss
user is a professor or teaching assistant in a programming course.}
that has been available since the late 1990's and has several thousand
users worldwide.  As such, \moss\ has many of the characteristics of
real software: it has users who depend on it, it is constantly
undergoing revision as its purpose and the environment in which it
runs evolves, and it is complex enough to be composed of several
interacting subsystems.  
From our point of view, \moss\ has the additional advantage that it
was written and is maintained by one of the authors.

The next step, injecting bugs into the software, is problematic, as
the choice of bugs to include or exclude can dramatically affect the
results.  Nearly all of the bugs were taken directly from the bug
logs for \moss.  In some cases the code had evolved since the original
bug was fixed, in which case we had to judge how to modify the
bug to inject it into the code.  We also included three bugs that
were not \moss\ bugs.  One of these is a known bug from another system
where there is an obviously analogous place to add that bug to \moss\
(see below). The other two are duplicates of two different buffer
overrun bugs in \moss.  In each case, we restored the original bug,
and then added a second, very similar buffer overrun in a different
place, the purpose being to see if our algorithm could not only detect
the overruns, but also distinguish between them.

We briefly describe the nine bugs we added to \moss:
\begin{enumerate}
\item To correctly report the location of duplicate code \moss\ must
track line numbers.  We introduced a bug that causes the number of
lines in C-style multi-line comments to be counted incorrectly.  The
bug only occurs under a special set of circumstances: the option to
match comments must be on (normally \moss\ ignores comments
completely, and that is a separate code path with no bug), the
programs involved must have C multi-line comments, and in addition the
position of these comments must ultimately affect the output.
Note that this bug is not only non-deterministic in
the sense defined in \Autoref{sec:algorithm}, it also does not
cause the program to crash; the program simply generates incorrect
output.

\item \moss\ has the option to dump its internal data structures in a
binary file format called a \termdef{database}.  We removed the check for a
null {\tt FILE} pointer in the case that the database cannot be opened
for writing.  This bug is analogous to one reported in {\tt ccrypt}
\cite{Selinger:2003:cqual}.  This is a deterministic bug, and in fact the
program crashes almost immediately after failing to open the file.

\item Loading a \moss\ database is complex, as a number of data
structures must be kept in sync.  We removed an array bounds update
in the database loading routine, so that even though a database was
loaded, the pointer to the end of one array {\tt A} was not moved to
reflect that new data had been added to the end of {\tt A}. The
program behaves normally unless a second database is loaded, at which
point the second database at least partially overwrites that portion
of the first database stored in {\tt A}.  This bug has unpredictable
effects.  Depending on what files are compared and the contents of the
databases loaded, the result might be that the program terminates with
correct output, that it terminates with incorrect output, or that it
crashes.  This was a particularly difficult bug to find originally.

\item We removed a size check that prevented users from supplying command-line arguments
that could cause the program to overrun the bounds of an array.  When
this bug is triggered the program may terminate with correct output,
terminate with incorrect output, or crash.


\item \moss\ handles Lisp programs differently from other languages;
at one time all languages where handled in the same manner, but the
others have been gradually ported to an improved algorithm.  The Lisp
processing involves a standard hash table.  We removed one of the
end-of-bucket checks, which causes a crash when the program scans to
the end of a hash bucket and tries to dereference a \texttt{NULL} pointer. 

\item For efficiency \moss\ preallocates a large area of memory for its primary data structure.
When this area of memory is filled, the program should fail
gracefully.  We removed the out-of-memory check.  The original bug
was more complex, but cannot be reproduced exactly because this
portion of the code has been revised.

\item \moss\ has a routine that scans an array for multiple copies of a data value.
We removed the limit check that prevents the code from searching past the end of the array.  This is another
buffer overrun, but of a different kind.  First, whether the overrun occurs is very data dependent and in fact it is
difficult to construct a test case by hand that triggers the bug.  Second, the routine in question only reads
past the end of the array (no memory locations are written), so it is quite likely that the program will
succeed in spite of the error.  This bug is synthetic (it never occurred in \moss) but is derived from bug \#8.

\item This is a variant on bug \#7, in another routine that deals with duplicates, but
bug \#8 occurs under an even rarer set of circumstances.  In
fact, this bug was never known to have caused a failure in \moss; it
was discovered by a code review.

\item This bug is a variant of bug \#4, but involves a different command-line argument and
a different array.
\end{enumerate}

In summary, the nine bugs are all either real bugs in \moss\ or bugs
closely related to real bugs in \moss\ or other programs.  The bugs
range from typical C coding errors (e.g., \texttt{NULL} pointer dereferences
and array overruns) to high-level violations of a system's internal
invariants (e.g., bugs \#1 and \#3).

To allow us to measure the accuracy of our techniques we also added code to \moss\ to
log when each bug was triggered.  We were careful to exclude this code from the
code that was instrumented for sampling, as predicates on the logging code would be
very highly correlated with program failures.

In the next step we ran our source-to-source instrumentor on \moss,
modified to include the bugs and the logging code.  We excluded the logging
code from instrumentation as well as code automatically generated by the tool
{\tt flex}.  While examining predicates on {\tt flex}'s internal state
would quite possibly yield some useful clues about the sources of
bugs, it is very unlikely that any programmer working on \moss\ would
be able to interpret such predicates.

After instrumenting \moss\ we ran both the buggy version and the
original version on the same random inputs. We recorded whether the buggy
version succeeded or failed by examining its exit code and, in the
case that it terminated normally, by checking whether the output of
both versions matched.  In practice, we envision that users would have
a way to give simple feedback on program executions in the case that a
program terminates normally but produces incorrect output.  We have
modeled that in our experiment by recording one bit (success or
failure) for each run based on whether its output matches that of the
unmodified \moss\ on the same inputs.

The randomly generated inputs are produced according to the following
scheme.  A different probability distribution is associated with each
command line option.  For the numeric options, the distribution gives
the probability that the option takes on a small, medium, or large
value, or is absent altogether.  For options that are essentially
enumerated types (such as the programming language used) the
distribution just includes the probability of each element of the
enumeration as well as the probability that the option is absent.
Depending on the programming language chosen, files to submit to
\moss\ are chosen randomly from a collection of thousands of C, Lisp,
and Java files.  The number of programs to submit with each run is
another probability distribution, which guarantees that there are at
least some runs with a very small number of files and some with a very
large number of files submitted.

In the process of performing the experiment we discovered several ways
in which our inputs were not as random as they could be---we had
accidentally coupled two or more choices that could be independent.
We removed the sources of coupling among input parameters that we
discovered.  

Finally, the infrastructure for this experiment was sufficiently complex
that we found it necessary to automatically discard executions that
failed in ways we could not handle.  For example, rarely the buggy version of
\moss\ would hang instead of crashing, in which case it was eventually killed
by a watchdog process we set up for that purpose.  In these cases no report
was generated, so we could not make use of the run.

The analysis of the results of applying our algorithm is the
subject of \Autoref{sec:experiments:results}.

%% LocalWords:  ccrypt


\section{Experimental Results}
\label{sec:experiments:results}
%% -*- LaTeX -*-

\begin{table*}
\centering

\begin{tabular}{|l|r|r|r|r|r|r|r|r|r|}
\hline
Number of: & \multicolumn{2}{c|}{runs}  & sites  & 
\multicolumn{2}{c|}{branch predicates} & \multicolumn{2}{c|}{return 
predicates} & \multicolumn{2}{c|}{scalar predicates}\\ \cline{2-3} \cline{5-6} \cline{7-8} \cline{9-10}
           & successful & failing       &        & original & retained & 
original & retained & original & retained \\
\hline
\hline
ccrypt     &  3605      &  1033         &    570 &      0 &          0 &     
3420 &        6 &        0 &        0 \\
\hline
bc         &  3530      &   860         &  13442 &      0 &          0 &        
0 &        0 &    80652 &      156 \\
\hline
moss       & 28519      &  3352         &  35223 &   4170 &         33 &     
2964 &       11 &   195864 &     3322 \\
\hline
rhythmbox  & 21015      &  1873         & 145242 &   6836 &         14 &    
50574 &       21 &   800370 &      406 \\
\hline
\end{tabular}
\caption{Run, site, predicate, and retention counts for each of the experiments.}
\label{tab:exps}
\end{table*}



In this section we present the results of applying the algorithm
described in \autoref{sec:algorithm} in five case
studies.  \autoref{tab:exps} shows summary statistics for each of the
experiments.  In each study we ran the programs on about 32,000 random
inputs.  The number of instrumentation sites varies with the size of
the program, as does the number of predicates those instrumentation
sites yield.  Our algorithm is very effective in reducing the number
of predicates the user must examine.  For example, in the case of
\rhythmbox an initial set of 857,384 predicates is reduced to 537 by the $\increase(P) > 0$
test, a reduction of 99.9\%.  The elimination algorithm then yields 15 predicates, a further
reduction of 97\%.  The other case studies show a similar reduction in the number of
predicates by 3-4 orders of magnitude.

The results we discuss are all on sampled data.  Sampling creates
additional challenges that must be faced by our algorithm.  Assume $P_1$ and $P_2$ are
equivalent bug predictors and both are sampled at a rate of
$\nicefrac{1}{100}$ and both are reached once per run.  Then even though
$P_1$ and $P_2$ are equivalent, they will be observed in nearly disjoint
sets of runs and treated as close to independent by the elimination
algorithm.

To address this problem, we set the sampling rates of predicates to be
inversely proportional to their frequency of execution.  Based on a
training set of 1,000 executions, we set the sampling rate of each predicate so
as to obtain an expected 100 samples of each predicate in subsequent program
executions.  On the low end, the sampling rate is clamped to a minimum of $\nicefrac{1}{100}$; if the site is expected to be reached fewer than 100 times the sampling rate is set at 1.0.
Thus, rarely executed code has a
much higher sampling rate than very frequently executed code.  (A
similar strategy has been pursued for similar reasons in related work \cite{chil04}.)  We
have validated this approach by comparing the results for each
experiment with results obtained with no sampling at all (i.e., the
sampling rate of all predicates set to 100\%).  The results are
identical except for the \rhythmbox and \moss experiments, where we
judge the differences to be minor: sometimes a different but logically
equivalent predicate is chosen, the ranking of predictors of different
bugs is slightly different, or one or the other version has a few
extra, weak predictors at the tail end of the list.

\subsection{A Validation Experiment}

To validate our algorithm we first performed an experiment in which we
knew the set of bugs in advance.  We added nine bugs to \moss, a
widely used service for detecting plagiarism in software
\cite{Schleimer:2003:WLA}.  Six of these were previously discovered
and repaired bugs in \moss that we reintroduced.  The other three were
variations on three of the original bugs, to see if our algorithm could
discriminate between pairs of bugs with very similar behavior but
distinct causes.  The nature of the eight crashing bugs varies: four
buffer overruns, a null file pointer dereference in certain cases, a
missing end-of-list check in the traversal of a hash table bucket, a missing
out-of-memory check, and a violation of a subtle invariant that must be maintained between two
parts of a complex data structure.  In addition, some of these bugs
are non-deterministic any may not even crash when they should.

The ninth bug---incorrect handling of comments in some cases---only
causes incorrect output, not a crash.  We include this bug in our
experiment in order to show that bugs other than crashing bugs can 
also be isolated using our techniques, provided there is some 
way, whether by automatic self-checking or human inspection, to recognize
failing runs.  In particular, for our experiment we also ran a correct 
version of \moss{} and compared the output of the two versions. 
This oracle provides a labeling of runs as ``success'' or ``failure,'' 
and the resulting labels are treated identically by our program as
those based on program crashes.

\begin{table*}
\centering

\begin{tabular}{|l|r|r|r|r|r|r|r|r|r|}
\hline
Number of: & \multicolumn{2}{c|}{runs}  & sites  & 
\multicolumn{2}{c|}{branch predicates} & \multicolumn{2}{c|}{return 
predicates} & \multicolumn{2}{c|}{scalar predicates}\\ \cline{2-3} \cline{5-6} \cline{7-8} \cline{9-10}
           & successful & failing       &        & original & retained & 
original & retained & original & retained \\
\hline
\hline
ccrypt     &  3605      &  1033         &    570 &      0 &          0 &     
3420 &        6 &        0 &        0 \\
\hline
bc         &  3530      &   860         &  13442 &      0 &          0 &        
0 &        0 &    80652 &      156 \\
\hline
moss       & 28519      &  3352         &  35223 &   4170 &         33 &     
2964 &       11 &   195864 &     3322 \\
\hline
rhythmbox  & 21015      &  1873         & 145242 &   6836 &         14 &    
50574 &       21 &   800370 &      406 \\
\hline
\end{tabular}
\caption{Run, site, predicate, and retention counts for each of the experiments.}
\label{tab:exps}
\end{table*}



\autoref{tab:mossdilute} shows the results of the experiment.  The
predicates listed were selected by the elimination algorithm in the
order shown.  The first column is the initial bug thermometer for each
predicate, showing the \context{} and \increase{} scores before
elimination is performed. The fourth column is the \termdef{effective}
bug thermometer, showing the \context{} and \increase{} scores for a
predicate $P$ at the time $P$ is selected (i.e., when it is the
top-ranked predicate).  Thus the effective thermometer reflects the
cumulative diluting effect of redundancy elimination for all
predicates selected before this one.

As part of the experiment we separately recorded the exact set of
bugs that actually occurred in each run.
The columns at the far right of \autoref{tab:mossdilute} show, for
each selected predicate and for each bug, the number of failing runs in which
both the selected predicate is observed to be true and the bug occurs.
Note that while each
predicate has a very strong spike at one bug, indicating it is a
strong predictor of that bug, there are always some runs with other
bugs present.  For example, the top-ranked predicate, which is
overwhelmingly a predictor of bug \#5, also includes some runs where
bugs \#3, \#4, and \#9 occurred.  This situation is not the result of
misclassification of failing runs by our algorithm.  As observed in
\autoref{sec:introduction}, more than one bug may occur in a run.
It simply happens that in some runs bugs \#5 and \#3 both occur (to
pick just one possible combination).

A particularly interesting case of this phenomenon is bug \#7, one of
the buffer overruns.  Bug \#7 is not strongly predicted by any
predicate on the list but in fact occurs in at least a few of the
failing runs of most predicates.  We have examined the runs of bug \#7
in detail and found that the only failing runs involving bug \#7 also
trigger at least one other bug.  That is, even when the bug \#7 overrun
happens, it never causes incorrect output or a crash
in any run.  Bug \#8, another overrun, is not even shown because the
overrun is never triggered in our data (its column would be all
0's).\footnote{Bug \#8 was originally found by a code inspection.}
There is no way our algorithm can find causes of bugs that do not
occur, but recall that part of our purpose in sampling user executions
is to get an accurate picture of the most important bugs.  It is
consistent with this goal that if a bug never causes a problem, it is
not only not worth fixing, it is not even worth reporting.

The other bugs all have strong predictors on the list.  In fact,
the top eight predicates have exactly one predictor for each of the seven
bugs that occur, with the exception of bug \#1, which has one very
strong sub-bug predictor in the second spot and another predictor
in the sixth position.  Notice that even the rarest bug, bug \#2,
which occurs more than an order of magnitude less frequently than
the most common bug, is identified immediately after the last of
the other bugs.\footnote{The peculiar eighth predicate, \texttt{f < f},
says that after an assignment the new value of \texttt{f} is less than
the old value of \texttt{f}.}  Furthermore, we have verified by hand that
the selected predicates would, in our judgment, lead an engineer to
the cause of the bug. Overall, the elimination algorithm does an excellent
job of listing separate causes of each of the bugs in order of priority,
with very little redundancy.

Below the eighth position there are no new bugs to report and every
predicate is correlated with predicates higher on the list.  Even
without the columns of numbers at the right it is easy to spot the
eighth position as the natural cutoff.  Keep in mind that the length
of the thermometer is on a log scale, hence changes in larger
magnitudes may appear less evident.  Notice that the initial and
effective thermometers for the first eight predicates are essentially
identical.  Only the predicate at position six is noticeably
different, indicating that this predicate is somewhat affected by a
predicate listed earlier (specifically, its companion sub-bug
predictor at position two).  However, all of the predicates below the
eighth line have very different initial and effective thermometers
(either many fewer failing runs, or much more non-deterministic, or
both) showing that these predicates are strongly affected by
higher-ranked predicates.

The visualizations presented thus far have a drawback illustrated by
the \moss\ experiment: It is not easy to identify the predicates to
which a predicate is closely related.  Such a feature would be useful
in confirming whether two selected predicates represent different bugs
or are in fact related to the same bug.  We do have a measure of how
strongly $P$ implies another predicate $P'$: How does removing the
runs where $\report{P} = 1$ affect the importance of $P'$?  The more
closely related $P$ and $P'$ are, the more $P'$'s importance drops
when $P$'s failing runs are removed.  In the interactive version of
our analysis tools, each predicate $P$ in the final, ranked list of
links to an \termdef{affinity list} of all
predicates ranked by how much $P$ causes their ranking score to
decrease.

\subsection{Additional Experiments}

We briefly report here on experiments with additional applications
containing both known and unknown bugs.  Complete analysis results for
all experiments may be browsed interactively at
\url{http://www.cs.berkeley.edu/~liblit/pldi-2005}.

\subsubsection{\ccrypt}

\view{\ccrypt}{ccrypt}

We analyzed \ccrypt 1.2, which has a known input validation bug.  The
results are shown in \autoref{tab:views-ccrypt}.  Our algorithm
reports two predictors, both of which point directly to the single bug.
It is easy to discover that the two predictors are for the same bug;
the first predicate is listed first in the second predicate's affinity
list, indicating the first predicate is a sub-bug predictor associated
with the second predicate.

\subsubsection{\bc}

\view{\bc}{bc}

GNU \bc 1.06 has a previously reported buffer overrun.  Our results
are shown in \autoref{tab:views-bc}.  The outcome is the same as for
\ccrypt: two predicates are retained by elimination, and the second
predicate lists the first predicate at the top of its affinity list,
indicating that the first predicate is a sub-bug predictor of the second.
Both predicates point to the cause of the overrun.  This bug causes a
crash long after the overrun occurs and there is no useful information
on the stack at the point of the crash to assist in isolating this
bug.

\subsubsection{\exif}

\view{\exif}{exif}

\autoref{tab:views-exif} shows results for \exif 0.6.9, an open source
image processing program.  Each of the three predicates is a predictor
of a distinct and previously unknown crashing bug.  It took less than
20 minutes of work to find and verify the cause of each of the bugs
using these predicates and the additional highly correlated predicates
on their affinity lists.

To illustrate how statistical debugging is used in practice, we
use the last of these three failure predictors as an example, and
describe how it enabled us to
effectively isolate the cause of one of the bugs.  Failed runs
exhibiting \texttt{o + s > buf\_size} show the following unique stack
trace at the point of termination:
\begin{quote}
  \small
\begin{verbatim}
main
  exif_data_save_data
    exif_data_save_data_content
      exif_data_save_data_content
        exif_data_save_data_entry
          exif_mnote_data_save
            exif_mnote_data_canon_save
              memcpy
\end{verbatim}
\end{quote}
The code in the vicinity of this crash site is as follows:
\begin{quote}
\begin{verbatim}
// snippet of exif_mnote_data_canon_save
for (i = 0; i < n->count; i++) {
    ...
    memcpy(*buf + doff,             (c)
           n->entries[i].data, s);
    ...
}
\end{verbatim}
\end{quote}
This stack trace alone provides little insight into the cause of the
bug.  However, our algorithm highlights \texttt{o + s > buf\_size} in
function \texttt{exif\_mnote\_data\_canon\_load} as a strong bug
predictor.  Thus, a quick inspection of the source code leads us to
construct the following call sequence:
\begin{quote}
  \small
\begin{verbatim}
main
  exif_loader_get_data
    exif_data_load_data
      exif_mnote_data_canon_load
  exif_data_save_data
    exif_data_save_data_content
      exif_data_save_data_content
        exif_data_save_data_entry
          exif_mnote_data_save
            exif_mnote_data_canon_save
              memcpy
\end{verbatim}
\end{quote}
The code in the vicinity of the predicate \texttt{o + s > buf\_size} is as follows:
\begin{quote}
\begin{verbatim}
// snippet of exif_mnote_data_canon_load
for (i = 0; i < c; i++) {
    ...
    n->count = i + 1;
    ...
    if (o + s > buf_size) return;    (a)
    ...
    n->entries[i].data = malloc(s);  (b)
    ...
}
\end{verbatim}
\end{quote}
It is apparent from the above code snippets and the
call sequence that whenever the predicate \texttt{o + s > buf\_size} is true,
%%
\begin{itemize}
\item the function \texttt{exif\_mnote\_data\_canon\_load} returns on
  line \texttt{(a)}, thereby skipping the call to \texttt{malloc} on
  line \texttt{(b)} and thus leaving \texttt{n->entries[i]->data}
  uninitialized for some value of \texttt{i}, and

\item the function \texttt{exif\_mnote\_data\_canon\_save} passes the
  uninitialized \texttt{n->entries[i]->data} to \texttt{memcpy} on line \texttt{(c)}, which reads it and eventually crashes.
\end{itemize}

In summary, our algorithm enabled us to effectively isolate the causes
of several previously unknown bugs in source code unfamiliar to us in
a small amount of time and without any explicit specification beyond
``the program shouldn't crash.''

\subsubsection{\rhythmbox}

\begingroup
\setlength{\segunit}{10pt}
\view[\tiny]{\rhythmbox}{rhythmbox}
\endgroup

\autoref{tab:views-rhythmbox} shows our results for \rhythmbox 0.6.5,
an interactive, graphical, open source music player.  \rhythmbox is a
complex, multi-threaded, event-driven system, written using a library
providing object-oriented primitives in C.  Event-driven systems use
event queues; each event performs some computation and possibly adds
more events to some queues.  We know of no static analysis today that
can analyze event-driven systems accurately, because no static
analysis is currently capable of analyzing the heap-allocated event
queues with sufficient precision.  Stack inspection is also of
limited utility in analyzing event-driven systems, as the stack in the
main event loop is unchanging and all of the interesting state is in
the queues.

We isolated two distinct bugs in \rhythmbox.  The first predicate led
us to the discovery of a race condition.  The second predicate was not
useful directly, but we were able to isolate the bug using the
predicates in its affinity list.  This second bug revealed what turned
out to be a very common incorrect pattern of accessing the underlying
object library (recall \autoref{sec:introduction}).  \rhythmbox
developers confirmed the bugs and enthusiastically applied patches
within a few days, in part because we could quantify the bugs as
important crashing bugs.  It required several hours to isolate each of
the two bugs (and there are additional bugs represented in the
predictors that we did not isolate) in part because \rhythmbox is
complex and in part because the bugs were violations of subtle heap
invariants which are not directly captured by our current
instrumentation schemes.  Note, however, that we could not have even
begun to understand these bugs without the information provided by our
tool.  We intend to explore schemes that track predicates on heap
structure in future work.

\subsection{Comparison with Logistic Regression}
\label{sec:comparison}

\begin{table}
\nocaptionrule
\caption{Results of logistic regression for \moss}
\label{tab:logregression}
\centering
\small
\begin{tabular}{ll}
  \toprule
  Coefficient & Predicate \\
  \midrule
  0.769379 & \verb|(p + passage_index)->last_line < 4| \\
  0.686149 & \verb|(p + passage_index)->first_line < i| \\
  0.675982 & \verb|i > 20| \\
  0.671991 & \verb|i > 26| \\
  0.619479 & \verb|(p + passage_index)->last_line < i| \\
  0.600712 & \verb|i > 23| \\
  0.591044 & \verb|(p + passage_index)->last_line == next| \\
  0.567753 & \verb|i > 22| \\
  0.544829 & \verb|i > 25| \\
  0.536122 & \verb|i > 28| \\
  \bottomrule
\end{tabular}
\end{table}

In earlier work
we used $\ell_1$-regularized logistic regression
to rank the predicates by their
failure-prediction strength \cite{PLDI`03*141,NIPS2003_AP05}.
Logistic regression uses linearly weighted
combinations of predicates to classify a trial run as successful or
failed.  Regularized logistic regression incorporates a penalty
forcing most coefficients to be set to zero, thereby
selecting only the most important predicates.  The output is a set of
coefficients for predicates giving the best overall prediction.

A weakness of logistic regression for our application is that it seeks
to cover the set of failing runs without regard to the orthogonality
of the selected predicates (i.e., whether they represent distinct
bugs).  This problem can be seen in \autoref{tab:logregression},
which gives the top ten predicates selected by logistic regression
for \moss.  The striking fact is that all selected predicates are
either sub-bug or super-bug predictors.  The predicates beginning with
\texttt{p + \ldots} are all sub-bug predictors of bug \#1 (see
\autoref{tab:mossdilute}).  The predicates \texttt{i > \ldots} are
super-bug predictors: \texttt{i} is the length of the command line and
the predicates say program crashes are more likely for long command
lines (recall \autoref{sec:introduction}).

The prevalence of super-bug predictors on the list shows the
difficulty of making use of the penalty term.  Limiting the number of
predicates that can be selected via a penalty has the effect of
encouraging regularized logistic regression to choose super-bug predictors, as
these cover more failing runs at the expense of poorer predictive
power compared to predictors of individual bugs.  On the other hand,
the sub-bug predictors are chosen based on their excellent prediction
power of those small subsets of failed runs.
%%Relaxing the penalty
%%allows logistic regression to add more predicates to improve its
%%prediction, but the sub-bug predictors apparently are favored.

%% LocalWords:  exps mossdilute ccrypt bc exif buf mnote rhythmbox
%% LocalWords:  logregression


\section{Related Work}
\label{sec:related-work}

The Daikon project \cite{ernst2001} monitors instrumented applications
to discover likely program invariants.  It collects extensive trace
information at run time and uses this offline to accept or reject any
of a wide variety of guessed candidate predicates.  The DIDUCE project
\cite{ICSE02*291} tests a more restricted set of predicates within the
client program, and attempts to relate state changes in candidate
predicates to manifestation of bugs.  Both projects assume complete
monitoring, such as within a controlled test environment.  Our goal is
to use lightweight partial monitoring, suitable for deployment to end
users.  We never have complete information, and therefore must use a
more probabilistic approach: we wish to identify broad trends over
time that correlate predicate violations with increased likelihood of
failure.

\termdef{Software tomography} as realized through the GAMMA system
\cite{PASTE'02*2,Orso:2003:LFDIART} shares our goal of low-overhead
distributed monitoring of deployed code.  GAMMA collects code coverage
data to support a variety of code evolution tasks.  Our
instrumentation exposes a broader family of data- and
control-dependent predicates on program behavior and uses randomized
sparse sampling to control overhead.  We observe, however, that the
predicates injected by our instrumentor can approximate coverage: over
many runs, the sum of all predicate counters at a site converges on
the relative coverage of that site.

Efforts to directly apply statistical modeling principles to debugging
have met with mixed results.  Early work in this area by Burnell and
Horvitz \cite{Burnell:1995:SCM} uses program slicing in conjunction
with Bayesian belief networks to filter and rank the possible causes
for a given bug.  Empirical evaluation show that the slicing component
alone finds 65\% of bug causes, while the probabilistic model
correctly identifies another 10\%.  This additional payoff may seem
small, especially in light of the effort, measured in multiple
man-years, required to distill experts' often tacit knowledge into a
formal belief network.  However, the approach does illustrate one
strategy for integrating information about program structure into the
statistical modeling process.

In more recent work, Podgurski et al.\ \cite{ICSE`03*465} apply
statistical feature selection, clustering, and multivariate
visualization techniques to the task of classifying software failure
reports.  The intent is to bucket each report into an equivalence
group believed to share the same underlying cause.  Features are
derived offline from fine-grained execution traces without sampling;
this reduces the noise level of the data but may be impractical to
deploy outside of a controlled testing environment.  As in our own
earlier work, Podgurski uses logistic regression to select features
which are highly predictive of failure.  \comment{Is it worth noting
  that we use different strategies for limiting the size of the set of
  selected features?  We use regularized logistic regression whereas
  Podgurski applies standard logistic regression to randomly selected
  subsets of the complete feature set and keeps the best-performing
  subset.}  Clustering tends to identify small, tight groups of runs
which do share a single cause but which are not always maximal.  (That
is, one cause may be split across several clusters.)

In contrast with Podgurski's statistical approach, widespread
industrial practice uses stack traces to cluster failure reports into
equivalence classes.  Two crash reports showing the same stack trace,
or perhaps only the same top-of-stack function, are presumed to be two
reports of the same failure.  This works to the extent that a single
cause corresponds to a single point of failure, but our experience
with \moss suggests that this assumption may not often hold.  We find
that only bugs \#2 and \#5 have truly unique ``signature'' stacks: a
crash location which is present if and only if the corresponding bug
was actually triggered.  These bugs are also our most deterministic.
Bugs \#4 and \#6 have nearly unique stack signatures, modulo small
changes to frames several levels removed from the point of failure.
The remaining bugs are much less consistent: each stack signature is
observed after a variety of different bugs, and each triggered bug
causes failure in a variety of different stack states.  Stack-based
clustering provides little insight for temporally extended bugs that
do not crash until well after the bad behavior.  A special case of
this is a program which produces incorrect output but exits normally
without crashing and therefore without any crash stack on which to
cluster.

Studies that attempt real-world deployment of monitored software must
address a host of practical engineering concerns, from distribution to
installation to user support to data collection and warehousing.
Elbaum and Hardojo \cite{Elbaum:2003:DISATA} have reported on a
limited deployment of instrumented Pine binaries.  Their experiences
have helped to guide our own design of a wide public deployment of
applications with sampled instrumentation, presently underway
\cite{Liblit:2003:CBIP}.

\comment{More recent work of interest:

  \begin{itemize}
  \item S.\ Elbaum, S.\ Kanduri and A.\ Andrews, ``Anomalies as
    Precursors of Field Failures, International Symposium of Software
    Reliability Engineering'', IEEE, November 2003.

  \item S.\ Elbaum and M.\ Hardojo, ``An Empirical Study on Profiling
    Strategies for Released Software and Their Application to QA
    Activities'', Technical Report TR03-09-01, University of Nebraska
    - Lincoln, September 2003.
  \end{itemize}

  Cannot find either online, but have written to Elbaum asking him for
  copies.}

For some highly available systems, even a single failure must be
avoided.  Once the behaviors that predict imminent failure are known,
automatic corrective measures may be able to prevent the failure from
occurring at all.  The Software Dependability Framework (SDF)
\cite{Gross:2003:PSMUST} uses the multivariate state estimation
technique to model and thereby predict impending system failures.
Instrumentation is assumed to be complete and is typically
domain-specific, whereas our sampled predicates cast a wider, less
specialized net.  \comment{We understand through informal
  communication that the SDF is able to anticipate when a player is
  about to lose an instrumented game of Tetris, and can intervene by
  removing rows to allow the game to continue.  But maybe I shouldn't
  say that, as it kind of makes their system sound like a joke.}

\comment{Cannot find any more recent work by these people in this
  area.  Where did they all go?  Porter has plenty of other recent
  work, but apparently nothing related.  Gross and McMaster have zero
  publication information on their home pages, while Umranov and Votta
  seem to have vanished entirely.  Have written to Gross and Porter
  asking if they have anything more recent I should look at.}

\section{Conclusions and Future Work}
\label{sec:conclusions}

When hunting for bugs, the first thing an engineer wants to know is
under what circumstances the failure occurs.  However, any complex
application is likely to contain not just one but many bugs, any of
which may have affected any single execution.  Isolating bug causes in
such an environment is a difficult process.

We have demonstrated an algorithm which turns multiple-bug isolation
into a numbers game.  Given feedback profiles of enough runs, overall
trends emerge which can help to guide an engineer to the most likely
causes of the most common bugs.  Our approach uses lightweight,
sampled instrumentation suitable for wide scale deployment to real end
users, which means that the system also performs implicit triage: it
learns the most, most quickly, about the bugs that happen most often.
The key property of our approach is that it filters and orders
potential causes based on the degree to which they \emph{increase} the
likelihood of failure.  This allows us to capture both deterministic
and non-deterministic bugs, and gives engineers a mechanism to gauge
the strength of the analysis results.

Our experience with \moss highlights the importance of selecting basic
instrumentation schemes that capture behavior commonly associated with
failure.  The branches and returns schemes yield a small number of
high quality bug causes.  Scalar-pairs instrumentation, while
potentially able to capture very subtle errors, suffers from an excess
of noise predicates from which the few ``smoking guns'' are difficult
to extract.  A more selective approach to injecting scalar-pairs
instrumentation warrants study.

Our \moss case study suggests several additional areas for future
development.  In the experiments reported here, sampling is
dynamically uniform: each crossing of each instrumentation site has
the same fixed chance of being sampled or skipped.  However not all
code is equally interesting.  Our approach can be generalized to
support non-uniform sampling.  By observing rare events more often, we
increase the odds of spotting key failure causes while ensuring that
the bulk of execution is spent in fast, sparsely-sampled code.

\placeholder{Yet to be addressed: need for better statistical models.
  This is a bit vague, but should at least discuss problems we
  encountered using logistic regression in the Moss experiment.}

The sampling instrumentor uses several program analyses to boost the
performance of instrumented code.  However, the post-run data analysis
described in \Autoref{sec:algorithm} is largely ignorant of program
structure.  These and other, deeper structural properties of the code
should be to help guide the filtering and ranking process.  For
example, the control flow graph can be used to direct engineers'
attention to likely first causes that appear earlier in the flow of
execution.

\bibliography{cacm1990,icse02,icse03,misc,paste02,pldi03,pods,ramss,refs}

\end{document}

%% LocalWords:  DIDUCE Burnell Horvitz Podgurski Elbaum Hardojo SDF
%% LocalWords:  cacm icse ramss pldi
