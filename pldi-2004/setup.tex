This section discusses background for the experimental results
reported in \Autoref{sec:experiments:results}.  While all software
experiments are difficult to do well, we have learned the hard way
that there are some particular problems that must be addressed to do
our experiment well.  Thus, this section discusses the set-up for our
experiment in some detail, especially how we have compensated for
potential sources of bias. 

The basic framework of our experiment is a straightforward five step
process: select an existing software application, modify the source
code to inject bugs into the program, instrument the modified program,
gather results from a large number of runs performed with
automatically generated data, and then apply our algorithm to the
results.  We discuss each step; the reader who is not interested in
these details may wish to proceed to \Autoref{sec:experiments:results} and
use this section only for reference.

We chose \moss\ \cite{Schleimer:2003:WLA} as our benchmark program.  \moss\ is a
software plagiarism detection service\footnote{That is,
\moss\ detects copying in large sets of programs.  The typical \moss
user is a professor teaching assistant in a programming course.}  late
'90's and has several thousand users worldwide.  As such, \moss\ has
many of the characteristics of real software: it has users who depend
on it, it is constantly undergoing revision as its purpose and the
environment in which it runs evolves, and it is complex enough to be
composed of several interacting subsystems.  The last point, in
particular, means that it is reasonable that \moss could have multiple
bugs simultaneously.  From our point of view, \moss\ has the
additional advantage that it was written and is maintained by one of
the authors.

The next step, injecting bugs into the software, is problematic, as
the choice of bugs to include or exclude can dramatically affect the
results.  In our case, we sought to use ``real'' bugs as much as
possible, while also seeking to have a number of different bugs and
some variety among the bugs we included.  Nearly all of the bugs were
taken directly from the bug logs for \moss.  In some cases the code
had evolved since the original bug had been fixed, in which case we
had to judge how to modify the bug to inject it into the code.  We
feel the modifications, where needed at all, were straightforward.  We
also included three bugs that were not \moss bugs.  One of these is a
known bug from another system where there is an obviously analogous
place to add that bug to \moss\ (see below). The other two are
duplicates of two different buffer overrun bugs in \moss.  In each
case, we restored the original bug, and then added a second, very
similar buffer overrun in a different place, the purpose being to see
if our algorithm could not only detect the overruns, but also
distinguish between them.

We briefly describe the nine bugs we added to \moss:
\begin{enumerate}
\item To correctly report the location of duplicate code \moss\ must
track the range of line numbers involved in each match.  We
introduced a bug that causes the number of lines in C-style multi-line
comments to be counted incorrectly.  The bug only occurs under a
special set of circumstances: the option to match comments must be on
(normally \moss\ ignores comments completely, and that is a separate
code path with no bug), the programs involved must have C multi-line
comments, and in addition the position of these comments must
ultimately affect the position of reported matches.  Note that this
bug is not only non-deterministic, it does not cause the program to
crash; it just generates incorrect output.

\item \moss\ has the option to dump its internal data structures in a
binary file format that can be reloaded quickly; these external files
are called databases.  We modified the code to remove the check for a
null {\tt FILE} pointer in the case that the database cannot be opened
for writing.  This bug is analogous to one reported in {\tt ccrypt}
\cite{Selinger:2003:cqual}.  This is a deterministic bug, and in fact the
program crashes almost immediately after failing to open the file.

\item Loading a \moss\ database is fairly complex, as a number of data
structures must be kept in sync.  We removed an array bounds update
in the database loading routine, so that even though a database was
loaded, the pointer to the end of one array {\tt A} was not moved to
reflect that new data had been added to the end of {\tt A}. The
program behaves normally unless a second database is loaded, at which
point the second database at least partially overwrites that portion
of the first database stored in {\tt A}.  This bug has unpredictable
effects.  Depending on what files are compared and the contents of the
databases loaded, the result might be that the program terminates with
correct output, that it terminates with incorrect output, or that it
crashes.  This was a particularly difficult bug to find originally.

\item We removed a size check that prevented users from supplying command-line arguments
that could cause the program to overrun the bounds of an array.  When
this bug is triggered the program may terminate with correct output,
terminate with incorrect results, or crash.


\item \moss\ handles Lisp programs differently from all other languages;
at one time all languages where handled in the same manner, but the
others have been gradually ported to an improved algorithm.  The Lisp
processing involves a standard hash table; we removed one of the
end-of-bucket checks, which causes a crash when the program scans to
the end of a hash bucket and tries to dereference a NULL pointer.  This
bug only occurs for Lisp programs, but does reliably crash the program
when it is touched.

\item For efficiency \moss\ preallocates a large area of memory for its primary data structure.
When this area of memory is filled, the program should fail
gracefully.  We removed the out-of-memory check.  The original bug
was a bit more complex, but cannot be reproduced exactly because this
portion of the code has been substantially revised.  

\item \moss\ has a routine that scans an array for multiple copies of the same passage of code within
a single file; duplicate code within a file is treated differently than duplicate code across files.
We removed the limit check that prevents the code from searching past the end of the array.  This is another
buffer overrun of a different kind.  First, whether it will occur is very data dependent and in fact it is
difficult to construct a test case by hand that triggers the bug.  Second, the routine in question only reads
past the end of the array (no memory locations are written), so it is quite likely that the program will
succeed in spite of the error.  This bug is synthetic (it never occurred in \moss) but is derived from bug \#8.

\item This is a variant on bug \#7, in another routine that deals with duplicates.
Again, the bug allows the program to read past the end of an array
while searching for particular data values.  In contrast to bug \#7,
bug \#8 occurs under an even rarer set of circumstances.  In
fact, this bug was never known to have caused a failure in \moss; it
was discovered by a code review.

\item This bug is a variant of bug \#4, but involves a different command-line argument and
a different array.
\end{enumerate}

In summary, the nine bugs are all either real bugs in \moss\ or bugs
closely related to real bugs in \moss\ or other programs.  The bugs
range from typical C coding errors (e.g., NULL pointer dereferences
and array overruns) to high-level violations of a system's internal
invariants (e.g., bugs \#1 and \#4).

To allow us to measure the accuracy of our techniques we also added code to \moss\ to 
log when each bug was triggered.  For example, for bug \#1 we added:
\begin{verbatim}
void check_bug_1(const char *yytext, int yyleng)
{
  static int seen;

  if (memchr(yytext, '\n', yyleng))
    reportOnce(&seen, 1, "... log message ...");
}
\end{verbatim}
The purpose of the static variable {\tt seen} is to ensure that each bug is logged only the first time
it occurs in a run (some of the bugs can occur multiple times).  

In the next step we ran our source-to-source instrumentor on \moss,
modified to include the bugs and the logging code.  Recall that our
system correlates predicates derived from the program source with the
success or failure of an execution to isolate bugs.  Given the log
function for bug \#1 above, it is very likely that our system would
report that whenever the predicate of the conditional {\tt
memchr(\ldots)} is true, the program is likely to fail.  To prevent
our logging code from ``giving away'' the causes of the bugs to our
bug isolation algorithm, the logging functions were isolated in a
separate file that was not processed by our instrumentor.  We also
excluded from instrumentation code automatically generated by the tool
{\tt flex}.  While examining predicates on {\tt flex}'s internal state
would quite possibly yield some useful clues about the sources of
bugs, it is very unlikely that any programmer working on \moss\ would
be able to interpret them, and so it is better to exclude them from
consideration.

After instrumenting \moss\ we ran both the buggy version and the
original version on random inputs. We recorded whether the buggy
version succeeded or failed by examining its exit code and, in the
case that it terminated normally, by checking whether the output of
both versions matched.  In practice, we envision that users would have
a way to give simple feedback on program executions in the case that a
program terminates normally but produces incorrect output.  We have
modeled that in our experiment by recording one bit (success or
failure) for each run based on whether its output matches that of the
unmodified \moss\ on the same inputs.

The randomly generated inputs are produced according to the following
scheme.  A different probability distribution is associated with each
command line option.  For the numeric options, the distribution gives
the probability that the option takes on a small, medium, or large
value, or is absent altogether.  For options that are essentially
enumerated types (such as the programming language used) the
distribution just includes the probability of each element of the
enumeration as well as the probability that the option is absent.
Depending on the programming language chosen, programs to submit to
\moss\ are chosen randomly from a collection of thousands of C, Lisp,
and Java files.  The number of programs to submit with each run is
another probability distribution, which guarantees that there are at
least some runs with a very small number of files and some with a very
large number of files submitted.  In the process of performing the
experiment we discovered several ways in which our inputs were not as
random as they could be---we had accidentally coupled two or more
choices that could be independent.  We removed all such sources of
coupling among input parameters that we discovered.  Note that having
more variation in the inputs helps our techniques, as
increased variation reduces the chances that a program predicate will
coincidentally appear to be correlated with failure.

The infrastructure for this experiment was sufficiently complex
that we found it necessary to add separate ``sanity checking'' of the results
to be sure that we had correctly captured the results of both the buggy and
reference versions of \moss.  Any execution that did not pass these sanity
checks is discarded.  \placeholder{INSERT NUMBERS HERE.}

Finally, the analysis of the results of applying our algorithm is the
subject of \Autoref{sec:experiments:results}.

%% LocalWords:  ccrypt
