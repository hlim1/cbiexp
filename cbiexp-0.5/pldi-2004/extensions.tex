While we have targeted our algorithm at finding bugs, there are other
possible applications, and there are variations of the basic approach
we have presented that may prove useful.  In this section we briefly
discuss some of these possibilities.

While we have focused on bug finding, the same ideas
can be used to isolate predictors of any program event.  For example,
we could potentially look for early predictors of when the program
will raise an exception, send a message on the network, write to disk,
or suspend itself.  Furthermore, it is interesting to consider
applications in which the predictors are used on-line by the running
program; for example, knowing that a strong predictor of program
failure has become true may enable preemptive action (see
\autoref{sec:related-work}).

There are also variations on the specific algorithm we have proposed
that are worth investigating.  For example, we have chosen to discard
all the runs where $R(P) = 1$ when $P$ is selected by the
elimination algorithm, but there are at least three natural choices:
\begin{enumerate}
\item When $P$ is selected, discard all runs where $R(P) = 1$.

\item When $P$ is selected, discard only failing runs where $R(P) = 1$.

\item When $P$ is selected, relabel all failing runs where $R(P) = 1$ as successful runs.
\end{enumerate}

We have already given the intuition for (1), our current choice.  For
(2), the idea is that whatever the bug is, it is not manifested in the
successful runs and thus retaining all successful runs is more
representative of correct program behavior.  Proposal (3) goes one step
further, asserting that even the failing runs will look mostly the same
once the bug is fixed, and the best approximation to a program without the
bug is simply that the failing runs are now successful runs.

On a more technical level, the three proposals differ in how much code
coverage they preserve.  By discarding no runs, proposal (3) preserves
all the code paths that were executed in the original
runs, while proposal (1) discards the most runs and so
potentially renders more paths unreached in the runs that
remain.  This difference in paths preserved translates into differences
in the \crash\ and \context\ scores of predicates under the different proposals.
In fact, for a predicate $P$ and its complement $\neg P$, it is
possible to prove that just after predicate $P$ is selected by the
elimination algorithm, then 
\[ \increase_3(\neg P) \geq \increase_2(\neg P) \geq \increase_1(\neg P) = 0 \]
where the subscripts indicate which proposal for discarding runs is used
and assuming all the quantities are defined.  Thus, proposal (1) is the
most conservative, in the sense that only one of $P$ or $\neg P$ can
have positive predictive power, while proposal (3) potentially allows
more predictors to have positive \increase\ scores.  

This analysis reveals that a predicate $P$ with a negative \increase\
score is not necessarily useless---the score may be negative only
because it is temporarily overshadowed by stronger predictors of
different bugs that are anti-correlated with $P$.  It is possible to
construct examples where both $P$ and $\neg P$ are the best predictors
of different bugs, but the result mentioned above assures us that once
one is selected by the elimination algorithm, the other's
\increase\ score is non-negative if it is defined.  This line of
reasoning also suggests that when using proposal (2) or (3) for discarding runs,
a predicate $P$ with $\increase(P) \leq 0$ should not be
discarded in a preprocessing step, as $P$ may become a positive
predictor once $\neg P$ is selected by the elimination algorithm.  In
the case of proposal (1), only one of $P$ or $\neg P$ can ever have a
non-negative \increase\ score, and so it seems reasonable to eliminate
predicates with negative scores early.

