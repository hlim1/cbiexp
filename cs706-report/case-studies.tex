% -*- TeX-master: "master" -*-

\section{Case Studies}
\label{sec-qual}
This section discusses two cases where complex predicates prove to be useful.  The first study is about a memory access bug in \prog{exif} 0.6.9, an open source image manipulation program.  A complex predicate is useful in increasing the score of an extremely useful bug predictor.  The second study uses an input validation bug in \prog{ccrypt} 1.2 to explain how complex predicates can be used to identify partial predictors automatically.

\subsection{\large\textbf{\prog{exif}}}
\label{sec-exif}

\prog{exif} 0.6.9 crashes while manipulating a thumbnail in a Canon image.  The bug is in function \texttt{exif\_mnote\_data\_canon\_load} in the module handling Canon images.  The following is a snippet from said function:
\begin{quote}
\small
\begin{verbatim}
for (i = 0; i < c; i++) {
    ...
    n->count = i + 1;
    ...
    if (o + s > buf_size) return;    // (a)
    ...
    n->entries[i].data = malloc(s);  // (b)
    ...
}
\end{verbatim}
\end{quote}

If the condition \texttt{o + s > buf\_size} is true on line (a), then the allocation of memory to the pointer \texttt{n->entries[i].data} on line (b) is skipped.  The program crashes when other code reads from \texttt{n->entries[i].data} without checking if the pointer is valid.  This is an example of a non-deterministic bug as the program succeeds as long as the uninitialized pointer is not accessed somewhere else.

We generated 1,000 runs of the program using randomly generated command line arguments and input images randomly selected from a set of Canon and non-Canon images.  There are 934 successful executions and 66 crashes.  Applying the redundancy elimination algorithm with only simple predicates produces two predicates that account for all failed runs as shown in \autoref{tab:tbl1}.  Studying the source code of the program does not show any obvious relation between the two predictors and the cause of failure.  Even though the second predictor is present in the crashing function it is a comparison between two unrelated variables: the loop iterator \texttt{i} and the size of the data stored in the traversed array \texttt{s}.  Also it is $\true$ in only 31 of the 66 failures.

\begin{table*}
\caption{Results for \prog{exif} with only simple predicates}
\label{tab:tbl1}
\centering
\scriptsize
\begin{tabular}{lllll}
\toprule
Score & Predicate & Function & File:Line \\
\midrule
0.704974 & $\text{new value of len} == \text{old value of len}$ & \func{jpeg\_data\_load\_data} & jpeg-data.c:224 \\
0.395001 & $\text{i} == \text{s}$ & \func{exif\_mnote\_data\_canon\_save} & exif-mnote-data-canon.c:176 \\
\bottomrule
\end{tabular}
\end{table*}

The analysis assigns a very low score of 0.0191528 to the predicate $p_1$: \texttt{o + s > buf\_size} despite the fact that it captures the exact source of the uninitialized pointer.  Because the bug is non-deterministic, $p_1$ is also $\true$ in 335 runs that succeeded, making $p_1$ a partial predictor.  Including complex predicates in the analysis produces one complex predicate shown in \autoref{tab:tbl2}.  (The second row is the second component of a complex predicate, which is a conjunction as indicated by the keyword \emph{and} at the start.)  Conjunction of $p_1$ with the second predicate $p_2$: \texttt{offset < len} eliminates all false positives and thereby earns a very high score.  This is an example of how a conjunction can improve the score of a partial predictor.  $p_2$ is in function \texttt{exif\_data\_load\_data} that calls \texttt{exif\_mnote\_data\_canon\_load} indirectly.  It is possible that $p_2$ is another partial predictor, capturing another condition that drives the bug to cause a crash.  If it does, it has to be a deep relationship as we could not find such a relation even after spending a couple of hours trying to understand the source code.  However this does not reduce the importance of this result as the conjunction has a very high score compared to $p_1$ and $p_2$ individually.

\begin{table*}
\caption{Results for \prog{exif} with complex predicates}
\label{tab:tbl2}
\centering
\scriptsize
\begin{tabular}{lllll}
\toprule
Score & Predicate & Function & File:Line \\
\midrule
0.941385 & $\text{o} + \text{s} > \text{buf\_size}$ is TRUE & \func{exif\_mnote\_data\_canon\_load} & exif-mnote-data-canon.c:237 \\
         & \emph{and} $\text{offset} < \text{len}$ & \func{exif\_data\_load\_data} & exif-data.c:644 \\
\bottomrule
\end{tabular}
\end{table*}

At the point where the uninitialized pointer is actually used, a hypothetical predicate $p_3$: \texttt{n->entries[i].data == 0} ought to be a perfect bug predictor.  However, the CBI instrumenting compiler does not actually instrument this condition or any direct equivalent.  Furthermore, this assumes that \texttt{n->entries[i].data} is zero-initialized even when \texttt{exif\_mnote\_data\_canon\_load} returns early without filling in this field.  Predicate $p_1$ provides critical additional information, as it identifies the initial trigger (skipping the \texttt{malloc}) that sets the stage for eventual failure (use of an uninitialized pointer).  Thus one role for complex predicates is to capture those program behaviors, like $p_1$, that are necessary but not sufficient preconditions for failure.

% Threat taxonomy discussion:
% Placing the threats listed below in the taxonomy is difficult because the case studies are more anecdotal than experimental.  IV and DVs can be identified, but there are many possible DV definitions, which potentially shift threats from one tier to another.
% The IV is obvious - whether analysis included complex predicates.
% DV is harder to identify.  Possibilities include: (1) Whether p_1 was top-ranked (or tied for top rank).  (2) The score of the best predicate involving p_1.  (3) The score of the top-ranked predicate.  (4) How useful the top-ranked predicate is in identifying the bug (hard to quantify).
% All of the above are at some point relevant in the above discussion (and there are many more possibilities).  Depending on which is regarded as the 'true' DV the below are threats to a different tier of validity.  What we can do is determine which tiers they definitely don't belong to.
% Conclusion validity: Obviously the analysis gave different results when complex preds were considered.  No threats to conclusion validity.
% Internal validity: threats would have to interfere with the conclusion that the differing results were caused by complex pred. analysis and not some other factor.  Simple and complex pred analyses were both run on the exact same data set.  Assuming there weren't any result-altering bugs in our code there are no threats to internal validity, since the IV was literally the only thing that differed between cases.  Altering the test suite to produce the desired bug doesn't affect internal validity because both the simple analysis and the complex pred analysis got the exact same fixed input.
% Construct validity: this one has some merit.  Input data was fixed to produce the particular bug being investigated.  We didn't try different sets of input to make sure p_1 can be identified under different input conditions.  It's possible that the fixed input altered the results we would have seen.  (Construct validity is ~'can the results you saw be generalized to an identical construct,' e.g. the same analysis run on the same program)
% External validity: This is probably the most likely.  The experimental input was fixed to emphasize the predicate we wanted high-ranked.  End users won't do this.  Additionally, searched the predicate list with a particular pred in mind.  To perform the same analysis on a similar but not identical program we would need to preidentify another predicate to examine (as opposed to one being immediately noticeable in the analysis).
% I'm of the opinion that most of the threats below are threats to external validity, but I'm not sure.  A compelling argument can be made for construct validity.  I'm splitting the difference and just calling them 'threats.'

\paragraph{Threats to validity}
There are a few threats to the validity of the above experiment.  Firstly, \prog{exif} 0.6.9 had two other bugs and we had to manually remove command line arguments that trigger those bugs.  Secondly, the bug studied here is very rare.  In order to get sufficient failed executions, we downscaled the input images by selecting many Canon images (that cause the bug) and some other images (both Canon and non-Canon) that do not trigger the bug.  These two changes introduce some bias into the scores of some predicates.  For example, a high score is assigned to the predicate \func{remove\_thumbnail} \texttt{is true} that corresponds to the \texttt{---remove-thumbnail} command-line flag.  However a subjective evaluation of the predicates in \autoref{tab:tbl1} and \autoref{tab:tbl2} shows that their scores are not affected by any bias introduced by the test suite.  Another threat is that the analysis with complex predicates produces many of other predictors tied with the predicate listed in \autoref{tab:tbl2} for the highest score (0.941385).  Because of this the predicate is not selected by the redundancy elimination algorithm, and we had to scan the list of all predicates to identify it.  This is not a real threat but is an instance of the numerous complex predicates problem discussed in \autoref{sec-metrics}.

\subsection{\large\textbf{\prog{ccrypt}}}
\label{sec-ccrypt}
\prog{ccrypt} 1.2 contains a known bug that can cause a crash on certain user-input - when an \texttt{EOF} is entered at the confirmation prompt when overwriting an existing file.  Entering \texttt{EOF} in other contexts does not cause failure, however, and an examination of the source code quickly reveals why:
\begin{quote}
\small
\begin{verbatim}
/* read a yes/no response from the user */
int prompt(void) {
  ...
  line = xreadline(fin, cmd.name);    // (a)
  return (!strcmp(line, "y") ||
     !strcmp(line, "yes"));
}

char *xreadline(FILE *fin, char *myname) {
  ...
  res = fgets(buf, INITSIZE, fin);
  if (res==NULL) {                    // (b)
    free(buf);
    return NULL;
  }
  ...
  return buf;
}
\end{verbatim}
\end{quote}

Calls to \func{xreadline}, the function used to get user-input, can return \texttt{NULL} under some circumstances.  In most cases the value is checked before being dereferenced; in \func{prompt} however it is used immediately after the call on line (a).  \func{xreadline} returning \texttt{NULL} in \func{prompt} should thus be a perfect predictor of failure, occurring in no successful runs and in every failure related to this bug.  The branch taken on line (b) in \prog{xreadline} is important as well, serving as the moment failure in \prog{prompt} becomes inevitable.  This branch is only taken when the user enters \texttt{EOF} on the command line.  In mapping the cause of failure, a programmer without a clear understanding of the code is likely to spend time tracking the user-entered \texttt{EOF} through \prog{xreadline} to the \texttt{NULL} dereference in \prog{prompt}, requiring either a visual inspection of the source or use of an interactive debugger.  Knowledge of the connection between program events such as these is necessary to make good debugging decisions, e.g., adding a \texttt{NULL} check to \prog{prompt} versus ensuring \prog{xreadline} always returns a valid pointer.  Automated bug analysis should ideally reveal as much of this chain of causation to the programmer as possible.

We generated 1,000 runs of \prog{ccrypt}, again using randomly selected command line arguments.  Input files include images and text archived from the online documentation of a remote desktop display system.  There are 658 successful executions and 342 crashes.  All failing runs crash due to the \texttt{NULL} dereference described above - no other bugs were visible to our test suite.

\begin{table*}
\caption{Results for \prog{ccrypt} with only simple predicates}
\label{tab:tbl3}
\centering
\scriptsize
\begin{tabular}{lllllll}
\toprule
Score & True Successes & False Successes & Predicate & Function & File:Line \\
\midrule
0.431678 & 0 & 342 & $\text{xreadline} == \text{0}$ & \func{prompt} & traverse.c:122 \\
0.385597 & 200 & 342 & $\text{res} == \text{(char *)0}$ & \func{xreadline} & xalloc.c:43 \\
\bottomrule
\end{tabular}
\end{table*}

An initial analysis of only simple predicates (\autoref{tab:tbl3}) finds $p_1$: \texttt{xreadline == 0} as the top predictor of failure: true in no successes and all 342 failed runs, verifying our assumptions.  The related predicate $p_2$: \texttt{res == (char *)0} scores substantially lower, appearing in all failures but a large number of successes.  $p_2$'s reported score is low enough that without knowledge of the nature of the bug a programmer would be likely to overlook its significance, and because of its relationship to $p_1$ it is removed by the redundancy elimination algorithm.  More importantly, traditional CBI analysis reveals no connection between the two predictors to the programmer, despite the fact that $p_2$, a necessary but not sufficient condition for failure, is subordinate to $p_1$ in predicting a crash.

\begin{table*}
\caption{Results for \prog{ccrypt} with complex predicates}
\label{tab:tbl4}
\centering
\scriptsize
\begin{tabular}{lllllll}
\toprule
Score & True Successes & False Successes & Predicate & Function & File:Line \\
\midrule
0.72814 & 0 & 342 & $\text{xreadline} == \text{0}$ & \func{prompt} & traverse.c:12 \\
	&   &     & \emph{and} $\text{res} == \text{(char *)0}$ & \func{xreadline} & xalloc.c:43 \\
\bottomrule
\end{tabular}
\end{table*}

When complex predicates are included in the analysis (\autoref{tab:tbl4}), a conjunction of $p_1$ and $p_2$ is among the top predictors.  This provides little help in finding the bug, which is easily identified by traditional CBI analysis, but it does reveal the nature of $p_2$ as a partial predictor.  The conjunction $p_1 \wedge p_2$ is observed in more successful runs than $p_1$ alone, but is true in the same number of successes and failures.  That $p_1$ can be conjoined with $p_2$ without affecting $p_1$'s predictive power demonstrates a connection between the two predicates, in this case suggesting that $p_1 \implies p_2$.

This implication is detectable because the experiment is run using complete data collection.  Results taken using sparse sampling rates would have made this detection impossible, given the likelihood of $p_2$ being unobserved in a run where $p_1$ was true.

This result provides evidence that complex predicate analysis can automatically group related predicates in ways traditional CBI analysis does not, including the discovery partial, sub-bug, and perfect predictor hierarchies and implications.  Grouping related predictors statistically provides insight into program structure and execution features that can be used in debugging.  This example reiterates that complex predicates can collaborate with tools like \textsc{BTrace} that produce an execution trace from a set of predicates.  Cooperative Bug Isolation can therefore utilize techniques that previously required detailed execution information by generating a facsimile from statistical data.

\paragraph{Threats to validity}

The version of \prog{ccrypt} used in this experiment had only one bug visible to our test suite.  The statistically demonstrated relationship between $p_1$ and $p_2$ is discovered in the absence of predictors for other bugs, which may affect the results.  Intuitively an unrelated bug would cause faults in different program runs, allowing the analysis to distinguish between unrelated sets of predictors, but we have not demonstrated this.  Further experimentation is needed to determine if this analysis retains this power in the face of multiple bugs.  Additionally the predictor $p_1 \wedge p_2$, though it scores highly, is not top-ranked, and in fact is discarded by the redundancy elimination algorithm due to the large number of similar predicates.  Knowledge of the code and the component predicates is necessary to distinguish it as important.  This once again demonstrates the need for techniques to effectively filter through large numbers of complex predicates, as discussed in \autoref{sec-metrics}.

% LocalWords:  ccrypt mnote buf tbl lllll len jpeg pred preidentify xreadline
% LocalWords:  lllllll src BTrace
