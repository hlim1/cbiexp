\section{Case Studies}
\label{sec-qual}
This section discusses two cases where complex predicates prove to be useful.  The first study is about a memory access bug in Exif 0.6.9, an open source image manipulation program.  A complex predicate was useful in increasing the score of an extremely useful bug predictor.  The second study uses an input validation bug in \texttt{ccrypt} 1.2 to explain how complex predicates can be used to detect $super$ bug predictors automatically.

\subsection{exif}
\texttt{exif} 0.6.9 crashes while manipulating a thumbnail in a Canon image.  The bug is in function \texttt{exif\_mnote\_data\_canon\_load} in the module handling Canon images.  The following is a snippet from said function:
\begin{quote}
\begin{verbatim}
for (i = 0; i < c; i++) {
    ...
    n->count = i + 1;
    ...
    if (o + s > buf_size) return;    (a)
    ...
    n->entries[i].data = malloc(s);  (b)
    ...
}
\end{verbatim}
\end{quote}

The function skips the call to \texttt{malloc} that allocates memory to the pointer \texttt{n->entries[i].data} when \texttt{o + s > buf\_size}.  The program crashes when another function \texttt{exif\_\-mnote\_\-canon\_save} reads from \texttt{n->entries[i].data} without checking if the pointer is valid.  This is an example of a non-deterministic bug as the program succeeds as long as the uninitialized pointer is not accesed somewhere else.

We generated 1000 runs of the program using randomly generated command line arguments and input images randomly selected from a set of Canon and non-Canon images.  There were 934 successful executions and 66 crashes.  Applying the redundancy elimination algorithm with only simple predicates produced two predicates that account for all failed runs as shown in ~\autoref{tab:tbl1}.  Studying the source code of the program did not show any obvious relation between the two predictors and the cause of failure.  Even though the second predictor is present in the crashing function it was a comparison between two unrelated variables: the loop iterator \texttt{i} and the size of the data stored in the traversed array \texttt{s}.  Also it was $true$ in only 31 of the 66 failures.

\begin{table*}
\nocaptionrule
\caption{Results for Exif with only simple predicates}
\label{tab:tbl1}
\centering
\scriptsize
\begin{tabular}{lllll}
\toprule
initial & effective & Predicate & Function & File:line \\
\midrule
0.704974 & 0.704974 & new value of len == old value of len & jpeg\_data\_load\_data() & exif-0.6.9/libjpeg/jpeg-data.c:224 \\
0.395001 & 0.589484 & i == s & exif\_mnote\_data\_canon\_save() & libexif-0.6.10/libexif/canon/exif-mnote-data-canon.c:176 \\
\bottomrule
\end{tabular}
\end{table*}

The analysis had assigned a very low score of 0.0191528 to the predicate $P$: \texttt{o + s > buf\_size} despite the fact that it captures the exact source of the uninitialized pointer.  Because the bug was non-deterministic, $P$ was also $true$ in 335 runs that succeeded.  Including complex predicates in the analysis produced one complex predicate shown in ~\autoref{tab:tbl2}{\footnote{the second row is the second component of a complex predicate, which is a conjunction as indicated by the keyword $and$ at the start}}.  Conjunction of $P$ with the second predicate $P'$: \texttt{offset < len} eliminated all the false positives and thereby assigning a very high score.  This is an example of how a conjunction can improve the score of a $super$ bug predictor.  $P'$ is in function \texttt{exif\_data\_load\_data} that calls exif\_mnote\_data\_canon\_load indirectly.  It is possible that it captures another condition that drives the bug to cause a crash.  If it does, it has to be a deep relationship as we could not find such a relation even after spending a couple of hours trying to understand the source code.  However this does not reduce the importance of this result as the conjunction has a very high score compared to $P$ and $P'$.

\begin{table*}
\nocaptionrule
\caption{Results for Exif with complex predicates}
\label{tab:tbl2}
\centering
\scriptsize
\begin{tabular}{lllll}
\toprule
initial & effective & Predicate & Function & File:line \\
\midrule
0.941385 & 0.941385 & o + s $>$ buf\_size is TRUE & exif\_mnote\_data\_canon\_load &
 libexif-0.6.10/libexif/canon/exif-mnote-data-canon.c:237 \\
 
         &          & $and$ offset $<$ len & exif\_data\_load\_data & libexif-0.6.10/libexif/exif-data.c:644 \\
\bottomrule
\end{tabular}
\end{table*}

A good predictor that could be found using a CBI style analysis is $Q$: \texttt{n->entries[i].data == 0} in function \texttt{exif\_\-mnote\_\-canon\_save} or some predicate equivalent to it.  Scanning the source file shows that there is no such predicate that is currently instrumented by CBI.  If a future instrumentation scheme{\footnote{one suggestion is a predicate on scalar parameters to functions}} instruments $Q$, then even an analysis with simple predicates will find $Q$ as the top bug predictor.  However, $P$ is still more useful than $Q$ in identifying the actual source of the $null$ pointer.  In fact, $P \wedge Q$ will be the perfect result as it captures the bug (skipping the \texttt{malloc}) and the trigger (the point where the illegal memory access is performed).  Thus, this bug in Exif presents a compelling evidence that complex predicates can be better bug predictors.

\vspace{4pt} \noindent
% If there are threats in other places, put this in a separate section
{\bf Threats to validity:}  There are some internal threats to the validity of the above experiment.  Firstly, \texttt{exif} 0.6.9 had two other bugs and we had to manually remove command line arguments that trigger those bugs.  Secondly, the bug studied here was very rare.  In order to get sufficient failed executions, we downscaled the input images by selecting many Canon images (that cause the bug) and some other images (both Canon and non-Canon) that do not trigger the bug.  These two changes introduced some bias into the scores of some predicates.  For example, our analysis found \texttt{remove\_thumbnail} in function \texttt{main} as a good bug predictor due to the bias introduced by our test suite.  However a subjective evaluation of the predicates in ~\autoref{tab:tbl1} and ~\autoref{tab:tbl2} showed that their scores were not affected by any bias introduced by the test suite.  Another threat is that the analysis with complex predicates produced a lot of other predicates with the highest score (0.941385) and we had to scan this list to identify the predicate listed in ~\autoref{tab:tbl2}.  This is not a real threat but is an instance of the numerous complex predicates problem discussed in section ~\ref{sec-metrics}.

\subsection{ccrypt}
\texttt{ccrypt} 1.2 contains a known bug which can cause a crash on certain user-input - when an \texttt{EOF} is entered at the confirmation prompt when overwriting an existing file.  Entering \texttt{EOF} in other contexts does not cause failure, however, and an examination of the source code can quickly reveal why:
\begin{quote}
\begin{verbatim}
/* read a yes/no response from the user */
int prompt(void) {
  ...
  line = xreadline(fin, cmd.name);    (a)
  return (!strcmp(line, "y") ||
     !strcmp(line, "yes"));
}

char *xreadline(FILE *fin, char *myname) {
  ...
  res = fgets(buf, INITSIZE, fin);
  if (res==NULL) {                    (b)
    free(buf);
    return NULL;
  }
  ...
  return buf;
}

\end{verbatim}
\end{quote}
Calls to \texttt{xreadline()}, the function used to get user-input, can return \texttt{NULL} under some circumstances.  In most cases the value is checked before being derefenced; in \texttt{prompt()} however it is used immediately.  \texttt{xreadline()} returning \texttt{NULL} in \texttt{prompt()} should thus be a perfect predictor of failure, occurring in no successful runs and in every failure related to this bug.  The branch taken in \texttt{xreadline()} is important as well, serving as the moment failure in \texttt{prompt()} becomes inevitable.  This branch is only taken when the user enters \texttt{EOF} on the command line.  In mapping the cause of failure, a programmer without a clear understanding of the code is likely to spend time tracking the user-entered \texttt{EOF} through \texttt{xreadline()} to the \texttt{NULL} dereference in \texttt{prompt()}, requiring either a visual inspection of the source or use of an interactive debugger.  Knowledge of the connection between program events such as these is necessary to make good debugging decisions, e.g. adding a \texttt{NULL} check to \texttt{prompt()} versus ensuring \texttt{xreadline()} always returns a valid pointer.  Automated bug analysis should ideally reveal as much of this chain of causation to the programmer as possible.

We generated 1000 runs of \texttt{ccrypt}, again using randomly selected command line arguments.  Input files included images and text archived from the online documentation of a remote desktop display system.  There were 658 successful executions and 342 crashes.  All failed runs crashed due to the \texttt{NULL} dereference described above - no other bugs were visible to our test suite.

\begin{table*}
\nocaptionrule
\caption{Results for \texttt{ccrypt} with only simple predicates}
\label{tab:tbl3}
\centering
\scriptsize
\begin{tabular}{lllllll}
\toprule
initial & effective & true successes & false successes & Predicate & Function & File\:line \\
\midrule
0.431678 & 0.431678 & 0 & 342 & xreadline == 0 & prompt() & src/traverse.c:122 \\
0.385597 & 0 & 200 & 342 & res == (char *)0 & xreadline() & src/xalloc.c:43 \\
\bottomrule
\end{tabular}
\end{table*}

An initial analysis involving only simple predicates found $P:$\texttt{xreadline() == 0} as the top predictor of failure: true in no successes and all 342 failed runs, verifying our assumptions.  The related predicate $Q$:\texttt{res == (char *)0} scored substantially lower, appearing in all failures but a large number of successes.  $Q$'s reported score was low enough that without knowledge of the nature of the bug a programmer would be likely to overlook its significance, and because of its relationship to $P$ it is removed by the redundancy elimination algorithm (see ~\autoref{tab:tbl3}).  More importantly, traditional CBI analysis reveals no connection between the two predictors to the programmer, despite the fact that $Q$, a necessary but not sufficient condition for failure, is subordinate to $P$ in predicting a crash.

\begin{table*}
\nocaptionrule
\caption{Results for \texttt{ccrypt} with complex predicates}
\label{tab:tbl4}
\centering
\scriptsize
\begin{tabular}{lllllll}
\toprule
initial & effective & true successes & false successes & Predicate & Function & File\:line \\
\midrule
0.72814 & 0 & 0 & 342 & xreadline == 0 & prompt() & src/traverse.c:12 \\
 
        &   &   &     & $and$ res == (char *)0 & xreadline() & src/xalloc.c:43 \\
\bottomrule
\end{tabular}
\end{table*}

When complex predicates are included in the analysis, a conjunction of $P$ and $Q$ is among the top predictors.  This provides little help in finding the bug, which is easily identified by traditional CBI analysis, but it does reveal the nature of $Q$ as a \textit{super} bug predictor.  The conjunction $P \wedge Q$ was observed in more successful runs than $P$ alone, but was true in the same number of successes and failures.  That $P$ can be conjoined with $Q$ without affecting $P$'s predictive power demonstrates a connection between the two predicates - in this case suggesting that $P \implies Q$.

This result provides evidence that complex predicate analysis can automatically group related predicates in ways traditional CBI analysis does not, including the discovery of \textit{super}, \textit{sub} and \textit{perfect} predictor heirarchies.  Grouping related predictors statistically generates information very similar to an execution trace, which can be used in the same way.  This example reiterates that complex predicates can collaborate with tools like BTRACE that produce an execution trace from a set of predicates.  Cooperative Bug Isolation can therefore utilize techniques which previously required detailed execution information by generating a facsimile from statistical data.

\vspace{4pt} \noindent
{\bf Threats to validity:}
The version of \texttt{ccrypt} used in this experiment had only one bug visible to our test suite.  The statistically demonstrated relationship between $P$ and $Q$ was discovered in the absence of predictors for other bugs, which may have affected the results.  Intuitively an unrelated bug would have caused faults in different program runs, allowing the analysis to distinguish between unrelated sets of predictors, but we have not demonstrated this.  Further experimentation is needed to determine if this analysis retains this power in the face of multiple bugs.  Additionally the predictor $P \wedge Q$, though it scored highly, was not top-ranked, and was in fact discarded by the redundancy elimination algorithm (see ~\autoref{tab:tbl4}).  Knowledge of the code and the component predicates was necessary to distinguish it as important.  This once again demonstrates the need for techniques to effectively filter through the large numbers of complex predicates, as discussed in section ~\ref{sec-metrics}.
