% -*- TeX-master: "report" -*-

\section{Background}
\label{sec-bground}
CBI uses lightweight instrumentation to collect feedback reports that contain truth values of predicates in an execution as well as the outcome (e.g., crash or non-crash) of the execution.  A large number of these reports are collected and analyzed using statistical debugging techniques.  Liblit et al. \cite{Liblit:2005:SSBI} compute a numeric $\Importance$ score corresponding to each predicate, define as follows.

The truth values of a predicate $p$ from all the runs can be aggregated into four values:

\begin{enumerate}
\item $\obs{S}{p}$ and $\obs{F}{p}$, the number of successful and failed runs respectively, in which the value of $p$ was evaluated.
\item $S(p)$ and $F(p)$, the number of successful and failed runs respectively, in which the value of $p$ was evaluated and was found to be true.
\end{enumerate}

Using these values, two scores of bug relevance are calculated.  They are:
\begin{enumerate}
\item $F(p)$.  A good predictor must predict a large number of failed runs.
\item $\Increase(p)$, the amount by which $p$ being true increases the probability of failure over simply reaching the line where p is defined.  It is computed as follows:
\end{enumerate}

\begin{equation}
\label{eqn1}
\Increase(p) \equiv
\frac{F(p)}{S(p) + F(p)}
-
\frac{\obs{F}{p}}{\obs{S}{p} + \obs{F}{p}}
\end{equation}

Both of these scores are independent but good dimensions of a bug predictor.  These dimensions are combined into a single value by taking their harmonic mean.  Since $\Increase(p)$ is bounded by 1, the $F(p)$ component in $\Importance$ is normalized over the total number of failed runs $\NumF$ after a logarithmic transformation.  The overall metric is:
\begin{equation}
\label{eqn2}
\Importance(p) \equiv
\frac{2}{%
  \frac{1}{\Increase(p)}
  +
  \frac{1}{log(F(p)) / log(\NumF)}}
\end{equation}

To eliminate different predicates that predict the same bug, only the top ranked predictor is presented to the user.  To handle the case where multiple bugs are present, all the failed runs in which the top predicate is true are eliminated and the remaining predicates are ranked by recomputing their scores in the remaining set of runs.  This process of eliminating runs continues until there are no remaining failed runs or no remaining predicates.

The output of the analysis will contain the list of predicates that had the highest score during any iteration of the redundancy elimination algorithm.  This list can be used by the programmer to track down bugs, or as input to other automated analysis tools.

\subsection{Expected Benefits}

Complex predicates can improve the analysis described above in three ways:
\begin{inparaenum}[(1)]
  \item provide a richer diagnostic language for complex bugs,
  \item better recall and precision for both complex and simple bugs, and
  \item enhanced support of automated tools that further process CBI output.
  \end{inparaenum}

\paragraph{Richer Diagnostic Language for Complex Bugs}

A single predicate can be thought of as partitioning the space of all runs into two subspaces: those satisfying the predicate and those not. Bugs with easily-described causes partition cleanly using a single bug predictor, with (all or most) successes in one subspace and (all or most) failures in the other.  If a bug has a simple cause, and this cause corresponds well to a simple predicate, then a simple analysis is sufficient.

However, the success/failure boundary may be considerably more irregular.  A richer language of candidate bug predictors describes more complex shapes within the space of runs, and thereby diagnoses the causes of more complex failures.  Of course, a fiendishly convoluted (but hopefully rare) bug may require a predicate as complex as the entire original program to describe its behavior.  Thus, there is a balance to be struck: the language of complex predicates should be expressive enough to describe important bugs, but not so expressive as to become unwieldy in practice.  The present research explores one possible point on this complexity spectrum.

\paragraph{Improved Precision and Recall}

We hypothesize that complex predicates will be especially helpful in coping with \emph{super-} and \emph{sub-bug predictors}, a challenging aspect of statistical debugging first identified by Liblit et al. \cite{Liblit:2005:SSBI}.

A \emph{super-bug predictor} is one that is true in a large number of failures but also true in a large number of successful runs.  In other words, it is not a specific enough predictor of failure.  In information-retrieval terms, a super-bug predictor has high recall but low precision.  A possible (but not the only) cause for super-bug predictors is a non-deterministic bug that does not necessarily cause a failure when it is triggered.  In the most extreme case, a super-bug predictor may be complete (predicts all failures) but not sound (predicts some successes also).  False positives could be reduced or eliminated by taking a conjunction of this predictor with another predicate that captures another aspect of the bug.

A \emph{sub-bug predictor} is one that is true in a small number of failed executions.  In other words, it is not sensitive enough to predict all instances this failure.  In information-retrieval terms, a sub-bug predictor has high precision but low recall.  In the most extreme case, a sub-bug predictor may be sound (predicts only failures) but not complete (does not predict all failures).  False negatives could be reduced or eliminated by taking a disjunction with another predictor can captures other instances of the bug.  It is important to note that in software with multiple bugs, the analysis may find a disjunction of predictors of individual bugs as a predictor for the whole set of failures.  The user should keep this in mind while using a disjunction predictor to track down a bug.

A \emph{perfect bug predictor} is one that is true in all failed executions and false in all successful ones: in other words, it is both sound and complete.  Such predictors indicate a bug which is completely deterministic given the involved predicates.  Generally speaking, the closer a predictor is to perfect the more informative it is to a programmer.  By eliminating false positives and negatives conjunctions and disjunctions can move super- and sub-bug predictors closer to perfection.

\paragraph{Enhanced Support of Automated Tools}

Bug predictors identified by statistical debugging can be useful to humans directly, or can serve as input to other automated tools and analyses.  In the later case, it may be feasible to use even richer, more complex, or more numerous predictors that a human would find overwhelming if examined directly.  \textsc{BTrace} \cite{Lal:2006:POPAD} is one such tool. \textsc{BTrace} finds the shortest control- and dataflow-feasible path in the program that visits a given set of bug predictors.  More predictors, and predictors drawn from a more expressive language, can better guide \textsc{BTrace}'s path exploration toward paths which accurately reflect the sequence of events leading to failure.
