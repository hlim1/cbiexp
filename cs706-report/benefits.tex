\section{Background}
\label{sec-bground}
CBI uses lightweight instrumentation to collect feedback reports that contain truth values of predicates in an execution as well as the outcome{\footnote{crash and non-crash or some other binary classification}} of the execution.  A large number of these reports are collected and analyzed using statistical debugging techniques.  The analysis described in~\cite{Liblit:2005:SSBI} computes a numeric score corresponding to each predicate.  The score is called $Importance$ and is computed as follows:

The truth values of a predicate $P$ from all the runs can be aggregated into four values:

\begin{enumerate}
\item $S$($P$ observed) and $F$($P$ observed), the number of successful and failed runs respectively, in which the value of $P$ was evaluated.
\item $S$($P$) and $F$($P$), the number of successful and failed runs respectively, in which the value of $P$ was evaluated and was found to be $true$.
\end{enumerate}

Using these values, two scores of bug relevance are calculated.  They are:
\begin{enumerate}
\item $F$($P$).  A good predictor must predict a large number of failed runs.
\item $Increase(P)$, the amount by which $P$ being true increases the probability of failure over simply reaching the line where P is defined.  It is computed as follows:

\begin{equation}
\label{eqn1}
Increase(P) \equiv 
\frac{F(P)}{S(P) + F(P)}
-
\frac{F(\text{$P$ observed})}{S(\text{$P$ observed}) +
  F(\text{$P$ observed})}
\end{equation}
\end{enumerate}

Both of these scores are independent but good dimensions of a bug predictor.  These dimensions are combined into a single value by taking their harmonic mean.  Since $Increase(P)$ is bounded by 1, the $F(P)$ component in $Importance$ is normalized over the total number of failed runs $NumF$ after a logarithmic transformation.  The overall metric is:
\begin{equation}
\label{eqn2}
Importance(P) =
\frac{2}{
  \frac{1}{Increase(P)}
  +
  \frac{1}{log(F(P)) / log(NumF)}}
\end{equation}

To eliminate different predicates that predict the same bug, only the top ranked predictor is presented to the user.  To handle the case where multiple bugs are present, all the failed runs in which the top predicate is $true$ are eliminated and the remaining predicates are ranked by recomputing their scores in the remaining set of runs.  This process of eliminating runs continues until there are no remaining failed runs or no remaining predicates.  

The output of the analysis will contain the list of predicates that had the highest score during any iteration of the redundancy elimination algorithm.  This list can be used by the programmer to track down bugs, or as input to other automated analysis tools.
 
Complex predicates can improve the analysis described above in three ways: high scoring predicates, better bug predictors and as a better complement to automated tools.

\vspace{4pt} \noindent
{\bf Predicates with High Scores:} A statistical analysis with a larger input set will find more predicates with high scores.  Moreover, complex predicates can track deep properties about program behavior.  So they are more likely to differentiate faulty runs from successful ones and thus have higher scores.  Predicates with high scores are useful because they capture a program behavior that is highly correlated with failure.  Even if they do not directly point at the faulty location, they can help identify use cases or modules in which the failure occurs.  Though a high scoring predicate that has no relation to the bug can be a hindrance to the user, a programmer with sufficient knowledge about the software can easily filter out such predicates.

\vspace{4pt} \noindent
{\bf Better Bug Predictors:} A predicate obtained by combining simple predicates using logical operators can yield better bug predictors.  These predicates can improve the analysis of $super$ and $sub$ bug predictors.  A $super$ bug predictor is one that is true in a large number of failures but also true in a large number of successful runs.  In other words, it is not a {\em specific} enough predictor of failure.  A possible (but not the only) cause for $super$ predictors is a non-deterministic bug that does not necessarily cause a failure when it is triggered.  A $super$ predictor is complete (i.e. predicts all failures) but it is not sound (predicts some successes also).  False positives could be eliminated by taking a conjunction of this predictor with another predicate that captures another aspect of the failure.

A $sub$ bug predictor is one that is true in a small number of failed executions.  In other words, it is not {\em sensitive} enough to predict this failure.  A $sub$ bug predictor is sound (predicts only failures) but is not complete (contains false negatives).  It means that this predicate is not the only cause of failures.  False negatives could be eliminated by taking a disjunction with another predictor.  It is important to note that in software with multiple bugs, the analysis may find a disjunction of predictors of individual bugs as a predictor for the whole set of failures.  The user should keep this in mind while using a disjunction predictor to track down a bug.

A $perfect$ bug predictor is one that is true in all failed executions and false in all successful ones - in other words, it is both sound and complete.  Such predictors indicate a bug which is completely deterministic given the involved predicates.  Generally speaking the closer a predictor is to $perfect$ the more useful it is to a programmer.  By eliminating false positives and negatives conjunctions and disjunctions can move $super$ and $sub$ predictors closer to $perfect$ ones.

\vspace{4pt} \noindent
{\bf Automated Tools:}  Unlike programmers, automated tools are scalable and can benefit from more data points.  BTRACE~\cite{Lal:2006:POPAD} is one such tool that finds the shortest feasible path in the program that visits a given set of predicates.  Since complex predicates increase the input to BTRACE, BTRACE can making precise decisions at branch-merge points, where previously it did not have enough information to choose the right branch direction to include in its output.
