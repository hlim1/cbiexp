\section{Conclusion}
\label{sec-conc}
We have demonstrated that complex predicates are useful predictors of bugs.  Our experiments show qualitative and quantitative evidence that complex predicates can improve the current statistical analysis used by CBI.  We describe two optimizations that make the task of computing complex predicates feasible.  First is a numeric estimate on the upper bound of the score of a complex predicate.  The second is a metric that quantifies the usefulness of a complex predicate.  Even after these, the computational complexity is still high and requires further optimizations.  The metrics described in section ~\ref{sec-metrics} help reduce the number of spurious predicates.  But the metrics are not perfect as good bug predictors are still swamped by less useful ones.  The algorithm described in ~\cite{Zheng:2006:SDSIMB} may solve this problem as it was designed with the goal of handling multiple predictors for the same bug.

