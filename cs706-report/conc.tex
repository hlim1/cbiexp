% -*- TeX-master: "master" -*-
%       *         *         *         *         *         *         *         *

\section{Conclusion}
\label{sec-conc}
We have demonstrated that complex predicates are useful predictors of bugs.  
Our experiments show qualitative and quantitative evidence that the statistical 
analysis used by CBI can be effectively applied to compound predicates, and that
the resulting analysis provides improved results.  We describe
two methods of eliminating predicate combinations from consideration, making
the task of computing complex predicates more feasible.  First is a numeric 
estimate on the upper bound of the score of a complex predicate.  The second is 
a metric that quantifies the usefulness of a complex predicate.  Even after these, 
the computational complexity is still high and requires further improvements.  
The metrics described in \autoref{sec-metrics} help select the most useful predicates
from those generated.  But the metrics are not perfect as finding a good candidate
for use in debugging requires sifting through a large number of less useful predicates
that also passed automated inspection.  Most such predicates are redundant predictors
for the same set of program failures, and so the bi-clustering algorithm of Zheng et al. 
\cite{Zheng:2006:SDSIMB} may solve this problem as it was designed with the goal 
of handling multiple predictors for the same bug.
