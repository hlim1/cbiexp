\begin{abstract}
Cooperative Bug Isolation (CBI) is a technique to find bugs in programs that analyzes data collected from program executions.  At each program point, CBI identifies boolean expressions, called as predicates, to be instrumented.  We augment CBI's bug predictive ability by combining these predicates using logical operators (conjunction and disjunction).  The motivation is that a complex predicate will provide more information to the programmer by narrowing down possible program states.  Our experimental results show that the scores of complex predicates are usually higher or atleast comparable to single predicates.  This makes them good indicators of bugs.  We discuss a new metric that uses program structure to quantify the usefulness of complex predicates.  Using this metric, we could eliminate a large number of spurious predicates from consideration.  Finally we discuss the effect of sparse random sampling on the usefulness of complex predicates.

\end{abstract}