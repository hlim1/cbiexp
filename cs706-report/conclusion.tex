% -*- TeX-master: "master" -*-
%       *         *         *         *         *         *         *         *

\section{Conclusions and Future Work}
\label{sec-conclusion}
We have demonstrated that compound Boolean predicates are useful
predictors of bugs.  Our experiments show qualitative and quantitative
evidence that statistical debugging techniques can be effectively
applied to complex predicates, and that the resulting analysis
provides improved results.  We describe two methods of eliminating
predicate combinations from consideration, making the task of
computing complex predicates more feasible.  The first employs
three-valued logic to estimate set sizes and thereby estimate the
upper bound of the score of a complex predicate.  The second uses
distances in program dependence graphs to quantify the programmer
effort involved in understanding complex predicates.

These techniques help the statistical debugging analysis scale up to
handle the large number of candidate predicates we consider.  However,
using the analysis results in debugging can require sifting through a
large number of less useful predicates that also pass automated
inspection.  Further shrinking this list while retaining useful
predictors remains an important open problem.  Most identified bug
predictors redundantly describe the same small set of program
failures.  Thus the bi-clustering algorithm of Zheng et al.\
\cite{Zheng:2006:SDSIMB} may be promising as it was designed to handle
multiple predictors for the same bug.  Automated analyses which
further process predictor lists, such as \textsc{BTrace}
\cite{Lal:2006:POPAD}, may also benefit from the richer diagnostic
language offered by the work presented here.
