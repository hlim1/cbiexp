% -*- TeX-master: "report" -*-

\section{Related Work}
\label{sec-rw}
Daikon~\cite{ErnstPGMPTX2006} detects invariants in a program by observing values computed by it.  It can compute complex invariants by combining program variables and operators like sum, max etc on collection (e.g. array) objects.  Invariants are predicates that must be true in correct executions.  ~\cite{ErnstDRAFT} extends the work to compute implications of the form a $\implies$ b.  Our project is different from this in two ways.  First, the data we have (bit vector of predicate counts) is different from what Daikon uses (values of variables at different points).  So the fundamental techniques in our approach and ~\cite{ErnstDRAFT} are different.  Secondly, our project also aims at processing a sparse random sample of predicate values, whereas ~\cite{ErnstPGMPTX2006} requires complete execution traces.  Diduce~\cite{581377} is inspired by Daikon and identifies predicates that are true in failed runs.  ~\cite{1081753} is a statistical debugging tool similar to CBI.  Neither of these approaches try to construct complex predicates.
